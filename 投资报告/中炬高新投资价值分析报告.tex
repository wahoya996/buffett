%\documentclass[UTF8,a4paper,zihao=-4]{ctexart} %设置了A4纸张和小四字号,这个mac也可以用
\documentclass[UTF8,a4paper,zihao=-4,fontset = windows]{ctexart} %设置了A4纸张和小四字号和windows字体
\usepackage{graphicx}
\setmainfont{Times New Roman} %设置英文字体为Times New Roman
\title{\textbf{600872中炬高新投资价值分析报告}} %标题加粗
\author{王琛}
\date{\today}
\begin{document}
\maketitle
\tableofcontents
\part{行业分析}
\section{行业概况及发展趋势}
\subsection{调味品行业概况}
调味品是指能增加菜肴的色、香、味,促进食欲,有益于人体健康的辅助食品。其主要功能是增进菜品质量,满足消费者的感官需要,从而刺激食欲,增进人体健康。一般意义的调味品主要是指酱油、蚝油、醋等,也包括由多种成分组成的复合调味料。而中式餐饮口味较多,制作过程工业化程度低,导致标准化难度较大。调味品作为中国美食的重要组成部分,是中国餐饮文化的代表。而酱油是单味调味品市场中的最大子品类占比最高约为15\%,随着我国经济的稳步发展,居民收入和消费水平不断提升,酱油行业呈现出产品升级和行业集中度提升的趋势,龙头酱油企业的盈利水平不断提升。

据中国调味食品著名品牌企业 100 强数据统计(100 家)显示,2019 年百强企业生产总量为1428.9 万吨,销售收入为 1051.1 亿元,2017 年-2019 年百强企业生产产量与销售收入平稳逐年增长,百强企业销售收入总量首次过千亿。2017-2019 年百强企业产量增长率分别为 9.4\%、7.5\%和10.1\%;2017-2019 年百强企业销售收入增长率分别为 9.5\%、10.8\%和 10.7\%。百强企业总体销售均价分别为 6577 元/吨、7098 元/吨、7356 元/吨,调味品产品总体价格逐年提升。

\subsection{调味品行业产量稳中有升,总量或于2030年见顶}
中国酱油行业仍然处于较快增长阶段,但增速已有所放缓。中国酱油行业 2019/2020 年出厂口径收入为 568/624亿元,同时也看到2015年后,我国酱油行业增速趋于放缓,近年来行业增速稳定在8-10\%左右,面对人口红利渐渐消失的现实情况,酱油行业未来需要通过增加品类,品质升级来推动行业增长。\textbf{(慢慢有点类似白酒行业的现状)}按照券商的研究,根据我国人口总数和结构预计,酱油消费量在2030年前后达到峰值,然后像日本一样酱油消费总量将慢慢下降。%字体加粗命令

2006-2019 年期间零售渠道占比从 46\%缩减至 42\%,餐饮端占比从 54\%逐步扩大至 58\%。由于餐饮渠道占比大,且相对零售渠道粘性更强,因此餐饮渠道成为众多调味品企业重点发力之地,措施包括对分销网点进行密集覆盖,与学校、厨师等意见领袖进行合作推广、开展定制化服务等。目前传统调味品龙头企业如海天和李锦记的餐饮渠道占比收入已超过 60\%,中炬高新与天味食品的餐饮收入占比约 20\%-30\%。\textbf{餐饮端主要受厨师影响,渠道粘性高,导致餐饮市场壁垒较高。而零售渠道由消费者自己选择,专业知识薄弱,企业进入市场壁垒低,市场集中度低。}

\includegraphics[scale=0.4]{fg1.png} 

\subsection{行业集中度较低}
与日本酱油行业相比,中国酱油行业整体集中度较低。中国酱油行业呈现一超多强的竞争格局:海天味业已成为全国性品牌,李锦记主要布局高端餐饮,而中炬高新旗下的厨邦、千禾味业仍偏区域性,全国的渗透率仍需进一步提高,行业中仍存在较多中小规模工厂及小作坊。2019 年海天味业及中炬高新的酱油销量分别占行业总销量的 19.9\%、4.1\%,位居行业第一、二位。李锦记由于未上市,我们估算其销量占比约在 3-4\%,\textbf{中国酱油行业整体 CR3 约 28\%}。而日本酱油行业企业数量在1955 年-2018 年期间从 6000家减少至1169家,集中度不断提升,2018 年 CR3 已达到 53\%,其中龙头企业龟甲万市占率达 34\%。\textbf{对标日本,中国酱油行业市场集中度还有较大提升空间。}

\includegraphics[scale=0.4]{fg2.png} 

\section{行业商业模式优劣}
\subsection{调味品是弱周期性行业}
“民以食为天”酱油是生活的必需品,老百姓的开门七件事之一,因此酱油属于弱周期性行业,从上图国内酱油出厂收入稳步增长就能看出,酱油属于慢增长但是可持续很强的行业,短期内也没有其他替代品可以替代酱油的作用,不像味精受到了鸡精的挑战。尤其在疫情下,相较于其他产业表现出较强韧性。
\subsection{调味品具有品牌历史和消费黏性}
调味品作为菜肴味道提供者和消耗品,单价低、作用大、重复购买率高,消费者的口味黏性大,存在习惯性消费,优质品牌均是有百年历史以上的,消费者的品牌忠诚度较高。
\subsection{上下游竞争力强,享受高毛利率和净利率}
行业上游对接大宗农产品,大量采购议价力较强;下游口味粘性大,存在习惯性消费,优质品牌提价能力较强。传统调味
品公司一般采用经销商模式,先款后货,厂商相对强势。根据2020年年报,海天销售毛利率为42.17\%,销售净利润为28.12\%;中炬高新销售毛利率为41.56\%,销售净利润为18.96\%。

\includegraphics[scale=0.45]{fg3.png} 

\subsection{渠道为企业重要壁垒}
渠道为企业重要壁垒\footnote{实际上渠道壁垒是属于弱壁垒,因为渠道是容易模仿复制的,并不牢靠,可以参考白酒中的洋河一直以渠道和管理作为自己的壁垒,但事实证明了这两者都是属于容易被攻克的弱壁垒。},率先抢占餐饮渠道的企业具有先发优势。餐饮渠道较家庭渠道易守难攻主要原因为:1、餐饮渠道占比达40\%,具有高频、高粘性、价格敏感度低等特点,厂商及经销商愿意将主要精力放至餐饮渠道;2、家庭渠道投入费用高、毛利率低、客户购买粘性低、竞争激烈,同时渠道收入占比较低,投入费用转化率低。李锦记、海天味业等餐饮企业由于开发较早,已拥有大量高粘性客户,因此持续获得可观销量,受益于渠道优势带来的规模优势,调味品企业强者恒强,而其他区域性调味品企业在传统或现代等渠道上也具有将强竞争优势,各大调味品企业为实现长足发展,渠道多元化成为必经之路,主要企业已通过强大的渠道力开启全国化进程。
\section{行业竞争格局呈现集中度不断上升、需求高端化}
从行业消费角度看,随着中国人均可支配收入的提升,消费者对健康营养方面越来越关注,对价格敏感程度有所下降,中国调味品消费升级趋势已初步明显,向健康、美味、安全、方便快捷等消费观念转变。头部企业由于其品牌、渠道、规模化等竞争优势,抗风险能力强,市场份额进一步提升。


按照中国酱油行业标准,酱油根据含氮量可以分为四个级别(特级、一级、二级、三级),目前中国市场的特级酱油占比达到 42\%,但同时仍存在43\%的三级酱油,预示酱油行业仍存在较大的升级空间。12 元以上的 500ml 酱油一般被认作高端酱油。2018 年中国高端酱油市场规模达到 202 亿元,预计 2025 年规模可达 360 亿元左右。近年来中国高端酱油销量保持增长,2018 年中国高端酱油市场销量达 189.3 万吨,同比增长 5.9\%。除高端酱油外,优秀调味品龙头企业为迎合消费者更加注重健康的需求,也推出了零添加酱油及减盐酱油。

%\includegraphics[scale=0.5]{fg4.png} 

同时,外卖市场规模的快速扩大同样加快了快餐企业对调味品的需求,2019 年中国餐饮外卖产业规模达6536亿元,同比增长39.3\%,2015-2019年CAGR达90\%,外卖消费者规模达 4.6 亿人,同比增长 13\%。电商新零售的发展加速线上购物习惯的养成,疫情引起的居家消费占比大幅提升,这些变化在后疫情时代将深远影响消费者的消费行为,美团发布的《2020 年春节宅经济大数据》显示,春节期间,酱油醋、十三香等调味品的销量增长了 8 倍多。并且近年来直播、短视频等社交媒体平台的兴起也进一步加速推动了调味品线上化的进程。线上销售的数据也有助于调味品企业对客户消费行为及需求进一步分析,推动产品的创新。

\part{公司分析}
\section{公司市场地位 — 位于行业第二,但与海天差距较大}
2019 年海天味业及中炬高新的酱油销量分别占行业总销量的 19.9\%、4.1\%,位居行业第一、二位。李锦记由于未上市,根据券商估算其销量占比约在 3-4\%。2020年海天味业营收228亿元,净利润64亿元;中炬高新营收51亿元,净利润8.9亿元,海天净利润约为中炬高新7倍,但当前海天味业市值5521亿元,中炬高新市值421亿元,中炬高新作为酱油行业的老二,市值约为海天的1/10。

\includegraphics[scale=0.6]{fg5.png} 

综合来说,海天具备非常强的龙头优势,从规模上看,海天领先中炬十年的时间,不仅仅在于龙头的体量遥遥领先,更是在于强者恒强的行业逻辑、海天对于餐饮渠道的卡位、以及领先同行的公司治理。从发展历史来看,海天在快速增长时期,就采用了较为先进的渠道管理模式,扁平化程度较高。因此,海天已经在企业规模、品牌认知度、产品线、管理能力和销售渠道上形成了压倒性的优势,这就是海天所具备的护城河,也是同行难以企及的地方。


海天中炬间存在竞争,但也在共同做大市场,本质上是竞合关系。由于天然的口味壁垒较高,某一品牌单打独斗,较难打开局面,调味品营销中往往存在联盟效应和流量带动效应。调味品行业具有较强的口味壁垒、习惯消费壁垒,因此竞争格局演变较为特殊,孤立单品难在规模上形成气候,适度竞争带来的“联盟效应”反而带来共同成长;同时,当前品牌酱油共同负担着消费者教育的任务,可以通过广告共同做大品牌酱油的市场,例如今年海天不涨价、厨邦也不涨价,损失的是中小作坊区域性酱油厂商,在这个过程中行业前几位的厂家处于一条船上,能够通过在通过收割小厂家实现共同成长,直到龙头占有率很高的情况下,才会出现此消彼长的情形。\textbf{但是从目前的角度来看,海天龙头的地位未来已不可撼动,中炬高新无法超越他,同时后面的千禾也在不停的追赶,留给中炬高新的时间并不是很多,且看看它能否突围而出站牢自己的地位}。

\section{公司管理能力 — 国企体制正向民企体制改革}
中炬高新成立于 1993 年,以高新产业园区开发起步,1995 年在上交所挂牌上市,1999 年收购美味鲜食品总厂(前身为中华老字号香山酱园)后进军健康食品产业,成功从以园区开发为核心业务的企业转型成多元发展的投资控股型集团。公司原实控人及第一大股东为中山火炬集团,中山火炬集团为中山火炬高新区管委会全资控股子公司。2015 年“宝能系”旗下险资机构前海人寿通过二级市场多次增持后成为第一大股东;2018年 9月,前海人寿将其持有中炬高新 24.92\% 股份全部转让给同受“宝能系”控制的中山润田投资有限公司;2018 年 11 月,董事会换届完成,宝能系在 6 个常务董事席位中占 4 席,由此取得中炬高实际控制权;2019 年 3 月,公司实际控制人变更为宝能集团董事长姚振华,标志着宝能集团入主中炬高新一事尘埃落定,公司正式转为民营体制,原国资股东持股10.72\%是公司第二大股东。

董事长陈琳、总经理李翠旭、董秘邹卫东由宝能公司派驻,副总经理朱洪斌、吴剑为中炬高新老员工,为了提高管理层积极性。2019 年出台经营绩效激励办法,2019年中炬高新人均薪酬达 12.5 亿元,同比增长15\%,位居调味品行业上市公司第三名;人均创收 100.2 万元,同比增长 12\%,位居调味品行业上市公司第五名。2021年4月1日公司公告以不低于3亿元不高于6亿元回购股权用于股权激励,实现管理层持股,在高层激励方面已经逐步和业内先进的同行海天和千禾看齐了。

宝能入股近五年来,公司ROE由 13.64\%增长到 20.96\%,但近2年来营收和净利润增速有所下降。2019 年初,公司制订了五年“双百”发展计划,即从 2019年到 2023年,实现健康食品产业年营业收入过百亿,年产销量过百万吨的双百目标,目前来看营收2年后要翻翻,存在较大的困难。2020年年报中,管理层提出要进行营销上对营销人员适度授权、发挥好人才和资源的作用和加强决策效率的改革,提升企业经营效能,未来这些改革能否落地提升企业的运营效率,向业界优秀的海天团队看齐。

\section{公司竞争优势 — 产品高端、产能释放、营销下沉}
\subsection{品牌产品高端、消费升级空间较大}
“厨邦酱油美味鲜,晒足 180 天”的口号深入人心,产品为大豆酿造,质量上乘。公司有厨邦和美味鲜两大品牌,其中厨邦品牌主打“高鲜”的中高端产品,而美味鲜主要以性价比取胜。2018 年厨邦产品约占公司全部调味品销量的 92.9\%,与海天的主打中低端的大众化路线形成错位竞争,在消费升级的背景下具备提价能力。中炬高除了以酱油为主力、鸡精和鸡粉为辅,并不断向蚝油、酱类、食醋、料酒等多个品类延伸,公司已逐渐成长为多元化的调味品平台。

人均可支配收入较高的北京、天津、上海属于公司的四级市场,而这些城市的城镇家庭人均调味品消费支出均超过公司的一至三级市场,该类城市消费升级趋势明显,对中高端调味品需求旺盛。调查数据显示我国一线城市中高端酱油消费需求相对较高,占比达 64\%,远高于三四线城市的 15\%。中炬高新定位中高端,有利于公司在四级市场的发展。此外公司五级市场里的重庆、四川等城市人均调味品消费支出也明显高出其他城市,公司在非成熟区域仍有较大增长空间。
\subsection{产能提升释放}
产能方面,公司拥有 50 万吨左右产能(中山基地产能约 31 万吨、阳西生产基地 19 万吨)。面对中山厂区产能基本饱和的情况,公司于 2020 年 3 月开始积极对其进行技改拓建,预计 2022 年中山基地产能有望扩充至 58 万吨。同时阳西厨邦公司三期酱油扩产项目计划在今年下半年带料投产,产能有望达 47 万吨。阳西美味鲜基地于 2017 年动工,目前蚝油、料酒、食醋项目扩产顺利,预计 2020 年投产,2023 年达产,规划产能约 65 万吨。\textbf{预计2023年公司产能将达到170万吨,是在当前70万吨\footnote{来源于年报产销情况分析表,2020年产量为69.7万吨,销量为69.6万吨}的2.4倍}。

\includegraphics[scale=0.6]{fg6.png} 

\subsection{渠道下沉、增量提价}
海天、李锦记已经为全国品牌,而其他品牌多为区域品牌,厨邦目前类似“半全国化品牌”,在区域品牌全国化中步伐最为领先,2020年中炬高新在地级市覆盖率为89\%、区县市场覆盖率51\%,渠道覆盖深度与海天差距还很大\footnote{海天销售网络基本完成对地级市以上城市的布局,覆盖90\%的县级市场,并通过深度分销体系下沉至乡镇村。}。公司餐饮渠道销量增速远高于公司整体销量(疫情期间的短期下降是暂时性因素),得益于公司对餐饮渠道市场的积极推广,利用厨邦顶级厨师俱乐部平台,推出超级凉菜万元大奖赛等专业活动,实现 B 端渠道的快速增长。同时公司积极开拓线上渠道,开展网络营销推广,并逐步进军海外市场,推动厨邦酱油国际化,公司产品目前已出口到 7 国。在跑马圈地过程中,产品、市场培育为关键课题,市场占有率提升为第一要诣。因此,未来以量增为主,提价主要为转移成本压力,或在龙头大幅提价时进行跟随,保障其产品高端定位。

\section{公司近五年财务数据分析}
\subsection{营收和净利润分析}
营收和净利润方面,2015-2020年中炬高新营收从27.59亿元增长到51.23亿元,净利润从2.47亿元增长到8.9亿元,6年间净利润增长了260\%,但2016年增速高峰后\footnote{当年年房地产实现收入 9,255 万元,同
比增长近 30 倍,因此公司房地产及服务业的主营业务收入、毛利率、主营业务利润等指标均大幅上升。}处于下降趋势,2021年按年报预计大约为10\%处于底部阶段,能否触底回升值得期待。销售毛利率从35.07\%增加到41.56\%,与海天的42.17\%已接近。

{
\centering       %居中放置表格
    \begin{tabular}{lllllll}
                     & \textbf{2015} & \textbf{2016} & \textbf{2017} & \textbf{2018} & \textbf{2019} & \textbf{2020} \\
                     \hline
    \textbf{营业收入}    & 27.59         & 31.58         & 36.09         & 41.66         & 46.75         & 51.23         \\
    \hline
    \textbf{营收同比增速}  & 4.42\%        & 14.48\%       & 14.29\%       & 15.43\%       & 12.20\%       & 9.59\%        \\
    \hline
    \textbf{净利润}     & 2.47          & 3.62          & 4.53          & 6.07          & 7.18          & 8.9           \\
    \hline
    \textbf{净利润同比增速} & -13.78\%      & 46.55\%       & 25.08\%       & 34.01\%       & 18.19\%       & 23.96\%       \\
    \hline
    \textbf{销售毛利率}   & 35.07\%       & 36.69\%       & 39.27\%       & 39.12\%       & 39.55\%       & 41.56\%       \\
    \hline
    \textbf{ROE}     & 10.16\%       & 13.64\%       & 15.21\%       & 18.07\%       & 19.42\%       & 20.96\%   \\
    \hline   
    \end{tabular}
    }
%\includegraphics[scale=0.8]{fg8.png}  直接用表格舍弃图片

\includegraphics[scale=0.9]{fg9.png} 

\subsection{费用分析}
公司三项费用方面,占比从2015年的24\%逐年下降至17\%,在销售费用翻倍的情况下,同时管理费用稳步下降,财务费用基本大幅下降,总体控制的非常好,预计未来一段时间销售费用会持续增长,三项费用相较海天5.8\%的水平还有较大的下降空间。

\begin{tabular}{lllllll}
    \textbf{}                & \textbf{2015} & \textbf{2016} & \textbf{2017} & \textbf{2018} & \textbf{2019} & \textbf{2020} \\
    \hline
    \textbf{管理三费}            & \textbf{6.62} & \textbf{6.87} & \textbf{7.27} & \textbf{7.61} & \textbf{7.99} & \textbf{8.6}  \\
    \hline
    \multicolumn{1}{r}{销售费用} & 2.7           & 2.75          & 4.26          & 4.31          & 4.56          & 5.66          \\
    \multicolumn{1}{r}{管理费用} & 3.31          & 3.49          & 2.41          & 2.76          & 2.95          & 2.81          \\
    \multicolumn{1}{r}{财务费用} & 0.61          & 0.63          & 0.6           & 0.54          & 0.48          & 0.13          \\
    \hline
    \textbf{费用率}             & \textbf{24\%}          & \textbf{22\%}          & \textbf{20\%}          & \textbf{18\%}          & \textbf{17\%}          & \textbf{17\%}         \\
    \hline
    \end{tabular}

\subsection{ROE拆解及趋势分析}
净资产收益率(ROE)方面,一方面从纵向比较公司从15年的10.16\%提高到2020年20.96\%,提升了9个百分点;另一方面行业横向比较,2020年海天36.13\%、千和13.90\%(2019年)公司处于中上水平,离海天差距还较大,主要的差异还在于销售净利润方面海天高出中炬10\%,两者毛利率接近情况下,造成净利率差别这么大主要还是海天规模效应更佳费用占比更小。

\includegraphics[scale=0.9]{fg10.png} 

同时也指出,未来中炬高新ROE提升的关键在于提高销售净利润率,次之为提高权益乘数,而酱油属于发酵产品需要一定周期,因此总资产周转率提升空间不大。

\begin{tabular}{rllll}
    \multicolumn{1}{l}{\textbf{}} & \textbf{销售净利润率} & \textbf{权益乘数} & \textbf{总资产周转率} & \textbf{ROE}     \\
    \hline
    中炬高新                          & 18.96\%         & 1.33          & 0.81            & \textbf{20.96\%} \\
    海天味业                          & 28.12\%         & 1.46          & 0.84            & \textbf{36.13\%} \\
    \hline
    \end{tabular}

\part{估值分析}
\section{历史估值区间分析}
中炬高新历史最高PE-TTM估值在2007年曾经是191倍,极低估值发生在2008年为21倍,当时08年疯牛造成的极度不理性情况(想想当时价格跌的多么惨烈!)。近年五年来一直在30-60倍之间运行(中枢在40倍左右),当前股价52.89元,估值为47.35倍,较20年高点87倍下降了约50\%,处于合理估值附近区域\footnote{08年大牛市估值跌了90\%,20年抱团牛市估值跌了50\%,即便是优秀的企业跌起来也是很狠的,特别是食品饮料这些业绩相对比较稳定,弹性空间不大的股票一定要特别注意买入要是合理或者低估的价格。}。

\includegraphics[scale=0.45]{fg7.png} 

\section{未来三年营收及估值结论}
按照公司“双百”规划,产销过百万吨,营收过百亿,正好是2020年营收51亿的2倍,原计划在2023年完成,但是由于疫情原因2020年增速下降,又由于21年公司主动调低了增速,那么待2023年170万吨产能完全释放出来后,预计2024年可以完成目标,而按照20\%的销售净利润率,则对应净利润为20亿元,按照估值中枢40倍PE,市值为800亿元(100元/股),大约为当前420亿的一倍,满足了3年一倍的投资要求,即便保守预测在2025年完成营收百亿,也就是4年一倍,完全可以接受。

按照公司年报给出2021年保底净利润9.85亿元,EPS为1.2元,根据30倍PE为低估40倍PE为合理,那么大致可以入手的价格区间为36-48元(上涨的空间为2-3倍),目前我建仓的价格大概在48元左右,未来如果价格可以继续下跌那机会则更大。
%\includegraphics[scale=0.5]{fg11.png} 

\part{存在的主要风险和措施}
\section{食品安全风险}
食品行业特别关注食品卫生风险,如奶粉的三聚氰胺和白酒的塑化剂事件,均对行业造成了巨大的打击,但是食品行业的特点也是在于嘴巴总是要吃的,如果发生全行业的事件但是对龙头企业是好事情,只要龙头企业没有出现食品卫生事件,反而是加速行业集中度的好机会。

\section{股东风险}
中山润田作为公司第一大股东持有24.92\%的股份,而中山润田背后的实际控制人是姚振华,姚振华在资本市场上的名誉不佳,曾经作为“野蛮人”入主万科失败,然后被禁止进入资本市场。这段时间中炬高新大幅下跌,跌幅近50\%远超同行业的海天,主要是市场有传言大股东质押的股份有平仓风险,甚至有认为宝能将中炬高新作为旗下公司意图实现资本腾挪;同时宝能入住后曾试图收购厨邦另外20\%的股权,但因内部权力斗争导致收购失败,这两件事情都表面大股东造成了负面影响。此外,原国资背景的中山火炬公司仍持有公司10.72\%的股份,近期公司公告公司意图更名但是遭到了国资股东的反对,也侧面说明了公司股东之间存在较大分歧,但是公司不更名其实对希望购买更多份额的投资者是好事情,国旅更名后股价直接就上涨了。

\section{增速下降风险}
中炬高新在2020年年报中提及:2021 年实现营业收入确保目标 61 亿元,同比增幅 19.06\%;实现归属于母公司的净利润 9.85亿元(其中扣除非经常性损益后归母净利润 9.65 亿元),同比增幅 10.68\%。这样的增速目标在行业内部属于较低的增速,也是比过去几年低,让人怀疑公司能够在未来保持较高增速,主要是几个方面原因:1、因为通胀原因,黄豆等原材料价格上涨,但是酱油行业龙头为了保住市场份额挤压对手纷纷表示2021年不涨价,因此成本上涨带来必然毛利率下降;2、公司处于改革阵痛期,2020年年报中公司提出要经营体制改革,加强市场营销,扩大经营渠道,这些费用均吞噬了利润;3、公司管理层大换血,目前高管大部分是宝能派过来的,经验有所欠缺,同时股权激励手段刚刚才开始,管理能力还有待验证。

\section{高估值和通胀风险}
当前,海天味业PE-TTM估值为86倍,中炬高新为47倍,而两者近年的增速在20-30\%之间,历史估值中枢大约是40倍,从PEG的角度来说已经明显得过高了,这么高估值主要还是去年疫情影响下,大量资金集中抱团在盈利稳定同时相对增速较高的食品饮料行业(白酒、调味品),资金为了避险去了更有确定性的食品饮料行业,但是今年明显的通胀会侵蚀调味品行业的利润,行业的毛利率会下降,而随着经济回升和行业复苏,食品行业相对不高的增速变得更加没有吸引力,因此市场在当前明显对海天存在过于乐观的情绪,而中炬高新在受到股东负面情况影响下,估值基本上接近于历史中枢,这也是我在海天基本面更好的情况下,回避海天选择中炬高新的原因,若如果食品行业的估值未来受到压制情况下,那么中炬高新也不可避免受到同样影响,届时如出现远低于历史估值区间下限例如30倍PE,那就构成了一个绝好的投资机会,当然如果海天也被错杀那就是一个更棒的选择了\footnote{食品饮料行业盈利增速相对稳定的情况下,估值变化是超额收益的关键,为了获得高安全边际必须买的够便宜,行业最不利的时刻一般就是在通胀高企的时候。}。

\part{投资建议}
调味品行业总体在未来是一个10\%稳定增速的“慢行业”,目前离行业天花板还有近十年左右的时间,投资的逻辑必须基于“快公司”上——也就是说主要将基于既有强大竞争优势的公司整合市场份额,在此期间具备全国影响力的品牌企业将受益于行业集中度提升和消费升级的双重红利,海天无疑依然是行业中最具管理能力和产品力的企业,也是行业其他品牌的模范。\textbf{而中炬高新目前依然处于一个重要的发展机遇期内,可以复制海天过去在渠道、管理方面的成功要素,来共同提升调味品行业集中度,通过挤压中小企业在全国性调味品品牌大门关闭之前获得一席之位}。这一点很像当前白酒行业发生情况,行业萎缩但高端品牌茅五泸三家活的很滋润,也没有其他品牌能够成为全国性的高端白酒。

在一个技术变化不大的调味品行业,无法通过革命性技术进步击倒海天的龙头位置,市场空间的占领主要依靠市场需求、产能扩张和营销突破\textbf{三个方面的红利:}
\begin{itemize}
    \item \noindent \textbf{消费升级红利},厨邦酱油中高端的品质和深入人心的品牌,在消费升级的市场需求扩大及主要原材料成本上升情况下,可以让企业在激烈的竞争中保持优势和提价权,突破区域的限制覆盖到全国;
    
    \item \noindent \textbf{产能释放红利},未来3年左右调味品产能可以增加一倍,在单价保持不变情况下营收可以翻一倍,如果提价那营收和利润提升幅度更高,这也是为什么公司规划2023年营收百亿的依据;
    
    \item \noindent \textbf{营销突破和管理提升红利},海天曾经也是由国企改制而来,而当前中炬高新股权激励和经营改革可以在学习领先者基础上进行改进,特别当前营销渠道扩面和下沉上下功夫可以快速占领市场空间,叠加公司的管理效能提升,降低成本费用提高产品净利率,提高ROE水平。
    \end{itemize}

中炬高新当前因为宝能的关系导致股价大幅下跌,已经让公司具备了初步的安全边际,公司属于好行业和好公司,但是公司的管理能力和市场营销与先进龙头还有较大的差距,未来如果能够有效提升管理能力和营销水平,那么中炬高新在当前的价格下具备了3年一倍的空间,如果价格继续跌到36元则赢面更大,则值得重仓。总之,公司具备投资价值,仍需保持观察,目前仓位不宜超过15\%。
\end{document}