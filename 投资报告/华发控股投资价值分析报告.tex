%!TEX program = xelatex
\special{pdf:minorversion 7} %使用1.7版本PDF,适配pdf编辑器版本的需要
\documentclass[a4paper,12pt,lang=cn,fontset = windows]{elegantpaper} %使用了elegantpaper模板,更加简单美观
%设置了A4纸张和小四字号和windows字体
\title{600325—华发股份投资价值分析报告} %标题加粗
\author{雕弓满月@ 雪球}
\date{\zhtoday}
\begin{document}
\maketitle
\section{前言}

各位大湾汇的朋友们大家好,我是雪球网友雕弓满月,受地产群管理员(金鹏鸟)的邀请,给大家分享一下华发股份的投资逻辑。

华发股份是在A股上市的房地产企业,总部在珠海,大股东华发集团是目前珠海最大的地方国企,主营业务涵盖:金融、实业投资、城市运营以及房地产开发等。华发股份是华发集团旗下的地产上市平台。在当下这个时间点,为什么要去谈一下华发股份的投资逻辑?

我认为房地产行业在当下面临集中供地、三道红线、银行贷款集中度等政策限制,毛利率面临不断被压缩的不利的局面,这是行业整体的情况。\emph{在当下我们就要去找毛利率能够企稳,甚至可以反弹,就是企稳回升的房地产企业,这是我们当下要抓的主要矛盾}。这样的企业具备什么样的特点呢?下面以华发为例一一解析。



\section{华发的竞争优势}

\subsection{拿地能力}

\subsubsection{拿地有抓手,总体成本较低}

华发股份拿地是比较多元化的,它可以借助华发集团的优势去做产业购地,也可以通过旗下的华发商都做商业勾地,还包括TOD、收并购、旧改等等,华发在珠海、中山拥有很多旧改项目。公司有非公开市场拿地的优势,据披露,\emph{2020年华发股份一共获取了27个项目,其中23个是通过非公开市场获取的}。

有些项目可能是通过招拍挂市场拍出来的,但之前都已经跟当地政府谈好了,这种方式叫做定向招拍挂,一般这种项目溢价率都比较低,很多是底价拿地的,所以华发在这方面还是比较有优势。

在拿地这块可以通过比较低的溢价率,甚至是底价拿地,这样就能保证未来几年毛利率能够保持在一定的水平之上。我个人测算华发股份2020年的毛利率已经是见底了,从2021年一季度的毛利率来看,它已经开始企稳回升了,2020年的毛利率大概是25\%,今年一季度是26\%,我认为后续毛利率还会进一步提升。



\subsubsection{善抓窗口期,逆周期拿优质地}

2019到2020年最近这两年,在别的企业整体资产开始降杠杆的过程中,华发其实是在加杠杆的,也是得益于它先天性的国企优势,所以公司2019年、2020年有息负债规模增长比较快。截止到2020年末,公司有息负债规模在1500亿左右,但它2020年全口径销售额也就1200多亿,相比于销售额来说,它的负债规模应该是偏高的。但是华发为什么要这么做呢?我认为这就是华发的一个优势,它能够精准判断出接下来拿地的形势可能会进一步的不利,或者说进一步的内卷,所以抢在集中供地之前,通过加大杠杆,从二级市场获取了大量廉价的土储。

这里有个数据,据不完全统计,华发2019年到2020年,不包括旧改及收并购项目,拿地总金额超过1200亿(全口径),其中它的权益占比大概是70\%左右。单2020年它获取的全口径的土地就达到900多亿,权益大概65-70\%。这在行业内是一个什么水平呢?大概能排进前10左右。公司2020年权益销售额大概700多亿,而它全口径拿地900多亿,乘以权益占比,基本上回款与拿地是1:1的关系。大家都知道主流房企一般来说拿地金额占当年回款40\%左右,可以看到华发在2019-2020年加大杠杆,拿了很多土地,而且这些土地都是非常优质的。

我们举个例子,比如说华发在2019年的最后一天,在绍兴拍了一块地,大概是66个亿,楼面价6000多元/平米,现在这个盘已经开了,叫绍兴华发活力金融城,大概已经卖了40多亿了,销售价格是24,000元/平米。可以想象一下,6000多拿地卖24,000,毛利和净利有多高?今年万科在华发边上拿两块地,楼面价如果算上配建基本上逼近27,000元/平米了。

龙头房企在拿地这块并不一定比中小型房企有多大的优势,拿地讲究的是什么呢?第一,勾地,就是你跟地方政府的谈判能力;第二,你能不能逆周期拿地。什么叫逆周期?土地市场会有一定的窗口期,可能每一两年都会出现一个拿地的窗口期,在这个时间点土地溢价率会比较低,整体价格会比较低,如果你能够在窗口期获取大量的土地,当土地溢价率升高的时候,可以相对从容很多。

比如像现在这种集中供地政策下,华发完全可以不拿地了,公司目前的土储,官方口径大概是5700亿左右(不包括旧改项目),2020年销售额是1200亿(全口径),我们假设它未来三年年均销售额1800亿左右,三年下来就是5400亿,也就是说即便未来三年不拿地,它的货值也可以满足高速发展的需要。

所以我认为华发这一轮精准拿地是非常体现管理层水平的,华发在拿地这块的独特优势,使得它毛利率一直能维持在一个比较高的水平,不会随着行业整体趋势下滑。



\subsection{公司成长性}

在当前监管政策下,很多房企销售目标从前几年的高增长慢慢已经开始向低增长甚至不增长转变了。比如说像龙头房企(Top20、Top10),2021年的销售增速目标基本上都是在10\%以内,高的可能20\%左右。华发未来两三年销售增速大概能到多少呢?据我们测算,应该能保持在25-30\%的复合增速,主要依据:
\begin{itemize}
    \item 公司19、20年拿的地都是像上海、深圳、武汉这种位于大湾区、长三角,还有长江经济带等核心一二线城市都市圈里面,而且都是在城市的主城区核心区不是郊区,去化完全不成问题。
    \item 公司近两年周转速度有了明显提升,基本上拿地9个月左右都可以开盘。比如像去年在杭州临安拿的地,今年大概八九个月的时间就已经开盘,首开之后就已经开始摇号了,是今年杭州临安第一个摇号的盘,去化非常好。大家可以看到各地的销售情况都非常火爆。
\end{itemize}

我认为华发前两年拿的优质土储是可以确保未来两三年比较高的销售增长态势的。这是华发非常大的一个优点,因为投资地产股,\emph{最好的安全边际就是成长,就是说哪怕利润再高,分红在高,如果企业不成长,可以认为企业的内在价值是没有提升的}。如果内在价值不提升,那么股价为什么会涨对不对?所以应该从成长这个角度去分析,我认为华发是当前A股+港股里面非常稀缺的仍然能够保持高成长的一个房企。



\subsection{公司产品力}

华发主要做的是中高端改善型楼盘,销售均价基本上都在25,000元/平米以上(这里指的是全国的销售均价)。比如上海的华润华发静安府18年、19年都是上海区域的销售冠军,都是单盘过百亿而且蝉联两年。这可以反映华发产品力这块非常强大的优势。我认为接下来在存量博弈的时候,公司产品力是非常能够体现公司核心竞争优势的。

我们举个简单例子,比如同样一个区域的两个楼盘,假设一个是品质房企做的,另一个是比如碧某园做的,大家可以想象一下,价格都一样,很有可能是品质房企的楼盘已经销售一空了,而碧某园的楼盘卖不动。长期下来的话,公司的销售回款会有巨大差异,进而影响到两家公司的长期竞争力。



\subsection{公司估值}

华发目前市值是140多亿,2020年净利润是29亿,静态PE大概5倍左右,我们测算未来三年它的净利润能保持20\%以上的复合增速,20\%其实是比较快的速度了,基本上是3-4年可以翻一番。我认为未来3-4年华发净利润做到50-60亿以上的规模是不成问题的,它前瞻PE还是非常低。未来三年来看的话,如果按照券商的预测,我的预测可能比券商还要乐观一点,2023年前瞻PE也就两倍多。

公司分红这块相对来说是比较厚道的,2020年分红是每股0.45元,2019年是0.4元,2018年是0.35元,2017年是0.3元,基本上每年都有增长。而且公司也在非公开场合表示过,保持每年30\%以上的分红率基本上没有多大问题的,我认为这也是一个国企的担当。

未来几年大家可以测算一下,华发的净利润还能保持一个比较快的增长,分红也能跟随净利润保持一定速度的增长。这家公司的成长性,包括它的销售增长,毛利企稳回升,最后会让华发迎来一个什么?\emph{就是类似于戴维斯双击的比较好的一个局面,就是它的销售额是在不断增长的,结算利润也是在不断增长,毛利率也能在不断提升,再叠加公司在一二线核心城市布局的优势,以及它强大的产品力,我认为华发在当下这个时间点还是具备非常高的投资价值}。



\subsection{华发的旧改}

大家都知道大湾区的房企都有一个特点,就是旧改的项目会比较多一点,华发之前旧改这块还是比较低调的,不像很多内房股,对这块有比较大的宣传,它其实是比较低调的,但是它的项目一点都不低调。公司今年在业绩沟通会上首次披露了在珠海、中山,包括像广州、上海等地拥有的旧改项目大概是50个左右,但没有披露更多的细节。据我们一个个项目数过来之后,我认为华发在旧改这块潜在的货值应该是非常大的,丝毫不输于当前所拥有的确权货值,大家可以想象一下。这些旧改项目基本上在未来3-5年陆续开始贡献销售。

华发的旧改项目布局相对来说还是比较好的,比如说像珠海、中山这块,都是销售价格比较高的地方,而且大家都知道未来大湾区房价也会有进一步上涨的空间,所以我认为未来旧改这块的项目可以对华发起到一个锦上添花的作用。



\subsection{华发的商业}

华发之前因为公司规模比较小,在商业这块的布局还是比较晚的,主要也是因为资产规模比较小,不太适合大规模的去做商业地产,但是大家可以看到华发在珠海南湾的华发商都,大概是18万平米,是做的非常成功的,它开业以后到现在每年的出租率都是100\%,而且如果大家真的去现场体验过的话,可以看到华发的招商运营能力是非常强的。

华发2021年还会有6个商业项目开业,未来在全国形成20~30个的以华发商都为主品牌的商业地产布局,租金收入达到30亿左右,我认为在未来三年内实现是大概率的。大家也可以看一下华发商业地产的规划情况,我这块有一些资料后续也可以分享。公司在商业地产未来会持续发力,我认为也是一个看点。



\subsection{股东背景}

华发是一个地方国企,在当下我认为相对来说安全边际是比较高的。它的大股东是华发集团,持股比例大概30\%,不算很高。华发集团对华发股份基本处于放手的状态,应该说是任由华发股份自己去发展,当然在一些财务融资方面还是给华发股份提供了非常大的支持,比如华发集团旗下的华发财务公司,每年给华发股份提供270亿的授信,这块是循环使用的,而且不高于银行同期的贷款利率,其实支持力度是非常大的。

除此之外,华发集团拥有金融、实业投资、城市运营等业务,相当于是珠海的一个城投公司,在这方面也可以给华发股份提供很多的支持,比如它有很多土地一级开发项目,在珠海的横琴岛十字门,还有富山工业区等,基本上珠海大大小小的一级开发项目都是华发集团做的。这些土地后续也存在向上市公司打包注入的可能性,这个可能性还是存在的,因为要规避同业竞争对吧?这是一方面。

另一方面,在除了珠海之外的地方,华发集团还可以帮助华发股份去做产业勾地,比如像跟绍兴当地的城投合作,像跟成都的锦江区,武汉城投等等,跟当地地方政府的城投公司合作,华发集团都扮演了一个相当于牵头人的角色,我认为这个也是华发股份比较独特的优势,就是地方国企之间的合作会相对来说比较多一点。



\section{华发的风险点}

说了这么多之后,我们现在再来谈谈华发股份可能面临的风险,投资光讲好的地方不讲风险肯定是不对的,客观的说,华发股份有以下几点风险:



\subsection{有息负债}

它的有息负债规模还是比较高,截止目前是1500亿左右,三道红线应该是除了现金短债这块是达标的,其他两道还是踩线的。华发股份明确表示过,从2021年开始有息负债规模应该是不增长的,存量的高息非标融资会逐渐被低息贷款融资替换掉。从2021年开始,它的有息负债规模应该处在一个稳步微降的过程。

它的综合融资成本2020年末应该是6.17\%,因为它是地方国企,这个还是偏高,主要也是因为它负债规模比较高,后续随着高息非标融资项目被替换后,综合融资成本也会进一步下降。大家可以看一下华发在2020年的公司债发行情况,还有中票,利率基本上都是3-4\%左右,整体利率还是边际往下走的,所以我认为公司虽然整体有息负债规模比较高,但是风险还是不大的。



\subsection{公司治理}

因为是地方国企,它的大股东华发集团没有太多的市值管理的动力。虽然说管理层有一定的持股股权激励,但是持股比例还是比较低的,管理层没有特别大的把公司市值做高的动力。但是我们换个角度去想这个问题,因为他没有刻意的去把公司市值做高,我们现在能买到的华发股份相对来说市值是比较合理的,没有虚高的成分。我们知道有些民营企业,通过把利润去有意的做高来做市值管理,导致你看到的财务报表其实是含有一定水分的,这里就不点名了,比如通过公允价值的变动、销售费用资本化等等,最后你看到的报表利润,其实我也不说是造假,只是它相对于一些偏保守的房企来说是偏高的,水分比较高。

华发是地方国企,它的大股东包括管理层没有市值管理的动力,所以它就没有必要去做这种财务造假或者是销售额的注水,所以你看到的数据相对来说是比较真实的,应该是不含太多水分的比较真实的财务和销售数据,这个其实也是一体两面的关系,看你怎么去看待。



\subsection{成本管控}

这个也是历史遗留问题,公司在成本管控精细化管理方面之前还是存在一定的问题的。这个问题你要怎么看?比如说他对一些建筑的要求,包括像水泥的要求是非常高的,因为它是国企,不可能为了一点利润去做粗制滥造的房子,所以老百姓买地方国企的房子是非常好的,但是对于我们股东来说,它的成本一定会比较高的,所以你看到它的成本管控这块总有点搂不住的感觉。

但是从去年开始,上市公司意识到了这个问题,一直在做精细化管理,从目前来看,成效还是比较显著的,这个问题我相信陆续会得到改善。我认为产品质量和成本管控这块是可以找到一个比较好的平衡的,而且未来产品力在公司的核心竞争力这块占有越来越高的比重,所以我认为舍得去花成本打造产品的比较优质的房企未来应该会迎来一个更大的发展空间,这是我的一个判断。



\subsection{回款能力}

华发之前的回款还是有一定问题的,大家可以看到主流房企的回款率基本在90\%以上,华发之前是做不到这一点,基本上80\%左右,有的时候还不到80\%。自2020年以后,公司加大回款的考核,可以看到从今年一季度开始,公司回款率应该是上升非常明显的,这个问题我认为也会逐步得到改善。



\section{华发未来的想象空间}

介绍完公司之后,最后再谈一点公司之外的事情。现在都比较流行叫房地产+,就是说房地产转型,比如说像新城的双轮驱动、华侨城的文旅,华发带什么呢?华发独特的区位优势,总部在珠海,是珠海的龙头国企,而珠海是粤港澳大湾区西部的核心城市,珠海下面的横琴接下来可能会成为粤澳合作的深度示范区。我个人判断,横琴未来会成为一个“一国两制”的融合区,就是“一国两制”慢慢向“一国一制”慢慢的转变。那么怎么转变?我认为横琴就是一个试验田,比如说横琴当地居民和澳门居民不断的生活上的融合,工作的融合,交通的融合等,慢慢导致横琴、澳门、珠海,从之前的“一国两制”慢慢融为一体,逐渐模糊“两制”的界限,包括民生制度方面,都可能会慢慢的统一化。对于接下来解决香港和台湾问题,我认为都会成为一个非常大的样板。我认为这是华发一个非常大的想象空间,这是第一。

第二,华发集团的业务相对比较广泛,除了房地产开发以外,旗下还有很多创投项目,前段时间传出他跟贾跃亭的FF91在谈投资入股合作。此外,华发集团跟中国平安一起收购了方正集团,它旗下有很多地产金融产业项目,未来会不会跟华发集团旗下公司去整合,存在一定可能性。

第三,华发集团计划十四五期间冲击世界500强,未来会不会整体上市,像招商局旗下的招商蛇口一样吸收合并招商地产,也有一定可能性。

我认为包括华发集团、珠海横琴等给了华发股份非常大的想象空间,这在房地产开发以外相当于一个锦上添花的亮点。最近股价的上涨,我认为除了华发股份本身基本面比较优异以外,跟这些亮点不无关系。

\maketitle
\end{document}