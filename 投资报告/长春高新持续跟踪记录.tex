\documentclass[a4paper,12pt,lang=cn,fontset = windows]{elegantpaper} %使用了elegantpaper模板,更加简单美观
%设置了A4纸张和小四字号和windows字体
\title{长春高新持续跟踪记录} %标题加粗
\author{}
\date{}
\begin{document}
\maketitle
%\tableofcontents
\section*{2021年5月27日,金磊减持及集采冲击事件记录}
2021年5月21日,长春高新股价创新高522.2元后,市场传出了2个消息:金磊减持18亿元、广东牵头进行集采涉及生长激素。随后,长春高新当天股价跌停、后面连续下跌至今天(5月27日)股价416.51元,跌幅超过了20\%,连带安科生物也是持续大跌,股价从18元跌到15元。为了稳定股价,公司董事长进行增持股份。
\paragraph{事件点评}
\begin{itemize}
    \item 关于金磊减持,上次我在330元买入的时候,正好就是金磊减持造成股价下跌,上次金磊减持了16.8亿元,说是交税用。这次没有明确说原因,只是表示短期内不再减持,但是他还持有8.5\%的股份,我认为减持还是会持续的,每个人对自己的安排是不同的,减持不一定就代表对公司没有信心,只是恰好这次减持和集采结合在一起,造成了轩然大波。
    \item 关于生长激素集采,其实不是完全意料之外的事情,之前公司也对此有所表态,认为“粉针的集采可能性比较大,水针可能性较小,因为市场竞争不充分”。我个人也认为当前水针厂家不到三家无法进行集采,而且集采针对的是矮小症,其实实际上来看大部分用药的并非矮小,而是希望增高。从渠道来说,实际上都是公立医院开药然后到私立门诊拿药,走公立渠道的量非常少。因此,集采可能性较小,就算有影响范围不大,按照狮子皇@雪球计算大概影响额度在1.68亿元。
\end{itemize}
\paragraph{投资感悟}
\begin{itemize}
\item 投资过程中,即便是伟大的公司也会发生意外事件的冲击,年前的上海机场和现在的长春高新一样,上海机场意外事件打破了原有的逻辑,看看金赛这次会不会打破逻辑?事件考验最大的还是投资者的内心,考验投资者对公司的认识深刻不深刻,说实话我的内心也出现了波动,但是我依然还是相信长春高新可以过去,拭目以待吧。
\item 好公司受到意外事件冲击时,是一个买入的好时点,平时深研,关键时候出手是长期致胜之道!
\end{itemize}
\paragraph{后续持续关注的要点}
\begin{itemize}
    \item 关注集采后续的进展情况及实际影响是否在可控范围内?
    \item 关注安科的水针产量释放后对金赛的市场占有率是不是造成了影响?
    \item 生长激素行业竞争格局会出现什么变化,处于双寡头垄断还是会造成价格战?
    \item 关注金赛生长激素的增速是否出现疲软下降的趋势?
    \end{itemize}
\end{document}