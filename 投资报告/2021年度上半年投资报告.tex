%\documentclass[UTF8,a4paper,zihao=-4]{ctexart} %设置了A4纸张和小四字号,这个mac也可以用
\documentclass[UTF8,a4paper,12pt,lang=cn,fontset = windows]{elegantpaper} %使用了elegantpaper模板,更加简单美观
%设置了A4纸张和小四字号和windows字体
\usepackage{graphicx}
\title{2021年度上半年投资报告} %标题加粗
\author{王琛}
\date{\zhtoday}
\begin{document}
\maketitle
%\tableofcontents
\section{前言}

2008年是我股票投资的元年,但是真正能够明白和清晰的开始投资是2021年,自从我2020年8月离职之后,我认真学习了“价值投资”的理念和方法,投资不再是随随便便凭感觉买股票了,在投资股票的迷雾世界中,我找到了地图和坐标,可以利用“价值投资”的方法在一个小小的范围内安全的行走,让自己能够按意愿游刃有余。

今天是五一假期的第三天,在家闲来无事,正好看到网上有人分享投资年报的写法,正好前两天是巴菲特的股东会,因此效仿巴菲特开始写年度投资报告,虽然报告本应该是在年末撰写,今天先算做一个开头吧,以后每年年末的时候写一篇投资报告,放在“家庭财务会议”上来进行汇报。

\section{投资业绩}

截止2021年5月3日,依据同花顺上的数据(以2020年为基准年)上半年投资收益率为-1.34\%,累计收益率为18.13\%,跑输上证指数0.59\%,总体情况不算理想,但是这是在2019年和2020年连续大涨之后所致,对于2021年的收益不应抱有过高的期待。而年度收益率为负的主要原因是持有60\%仓位的上海机场在年后因为与中免签订补充协议的原因而遭遇连续跌停,造成了3万元的亏损,资产回撤5万元,对投资收益造成了巨大的打击,之后持有的长春高新和洋河大幅上涨,才把总市值追回来。这个事情给了我3个教训:1、严格按照最大个股仓位限制进行持仓;2、无论多么确定性的股票也要持续进行关注经营变化;3、保住本金是第一要务,大幅亏损伤害极大。

\begin{table}[htbp]
    %\small
    \centering
	\caption{历史收益率}
\begin{tabular}{rccccc}
    \toprule
    \multicolumn{1}{l}{\textbf{}} & \multicolumn{1}{l}{\textbf{上证指数}} & \multicolumn{1}{l}{\textbf{深圳成指}} & \multicolumn{1}{l}{\textbf{创业版}} & \multicolumn{1}{l}{\textbf{净收益}} & \multicolumn{1}{l}{\textbf{累计收益}} \\
    \midrule
    2020                          & 13.87\%                           & 38.73\%                           & 64.96\%                          & \textbf{15.99\%}                 & -                                 \\
    2021                          & -0.75\%                           & -0.22\%                           & 4.22\%                           & \textbf{-1.34\%}                 & 18.13\%                          \\
    \bottomrule
    \end{tabular}
\end{table}

\section{我们拥有的生意}
\subsection{持仓情况}

截止2021年5月3日,持有股票4支,其中:仓位大小依次排序为长春高新、中炬高新、洋河股份、中概互联网。仓位集中在生物医药、食品饮料和互联网三个好赛道,这几家企业未来年均增长均在20\%以上,以合理的价格买入优质企业,伴随企业增长是我投资收益的源泉。

\begin{table}[htbp]
    %\small
    \centering
	\caption{持仓明细表}
\begin{tabular}{lcccccc}
    \toprule
    \textbf{} & \multicolumn{1}{l}{\textbf{持有数量}} & \multicolumn{1}{l}{\textbf{收盘股价}} & \multicolumn{1}{l}{\textbf{账面市值}} & \multicolumn{1}{l}{\textbf{归属净利润}} & \multicolumn{1}{l}{\textbf{股息}} & \multicolumn{1}{l}{\textbf{仓位比例}} \\
    \midrule
    长春高新      & 400                               & 498                               & 199040                            & 3012                               & 320                             & 58.99\%                           \\
    中炬高新      & 1400                              & 47                                & 65320                             & 1568                               & 952                             & 19.37\%                           \\
    洋河股份      & 300                               & 193                               & 57750                             & 1494                               & 900                             & 17.11\%                           \\
    中概互联网     & 7500                              & 2.0                               & 15255                             & -                                  & -                               & 4.52\%                            \\
    现金及其它     & -                                 & -                                 & 34                             & -                                  & -                               & -                                 \\
    \textbf{合计}        & -                                 & -                                 & \textbf{337431}                            & \textbf{8246}                               & \textbf{2172}                            & \textbf{100\%}                               \\
    \bottomrule
    \end{tabular}
\end{table}

\subsection{账面资产和实际收益}

账面市值合计为33.7万元,归属净利润为8264元\footnote{以持股数量$\times$ 每股盈利,此处数字为2020年年报数。},持仓市盈率为40.8倍\footnote{即以总市值$\div$ 归属净利润算出。},总体属于合理区间,预计未来主要将以利润增长为主,不应对PE上升过于期待。2020年获得分红2172元,分红率不到1\%,考虑到这些公司均为优质成长性公司,无法与平安和地产类股票媲美,总体仍在可接受范围之内。

\section{投资学习思考}

虽然今年上半年投资收益只能说差强人意,但是最大的收获是自己真正领悟了“价值投资”的内容,自离职后专心在家看投资方面的书籍,其中对我收益最大的书籍是《巴菲特致股东信》、《如何选择成长股》,这两本书特别是费雪的书让我从买低估值的烟蒂投资转变为买好公司的成长型价值投资,我也把这两本书做成了电子版放在Github上与大家分享。

我认为在当前的时代更适合费雪成长股的投资理念,因为低估值的硬资产企业无法长期持有,而且大部分低估值企业其实本身存在很多硬伤,也就是估值陷阱。而通过认真分析可以理解的成长型企业,通过长期持有可以获得更加的回报。通过学习也有以下几点思考:
\begin{itemize}
    \item \noindent 价值投资理念不复杂,其主要核心支柱:安全边际、能力圈、内在价值、市场先生,这些概念不难理解,但是难在坚持并实践;
    
    \item \noindent 投资其实就是预测未来,对于未来股价影响最大的是未来的营收和净利润,而不是过去的,所以准确预测企业的未来是关键,而我能够做出有效预测的企业是很少的,所以严守能力圈对企业进行深入研究;
    
    \item \noindent 投资过程中一定会犯错,但是要控制风险每次犯错不能致命,要从错误中学习改进自己的系统。
    \end{itemize}
\section{未来展望和计划}
关于2021年的投资业绩个人认为不宜过于期待,因为一方面前两年涨的多了,另一方面今年通胀形势可能会严峻,美联储和中国央行的宽松货币政策退出,有可能利率会暂时上升,总体对于股市是有压制的作用。今年重要的事情,还是集中在修炼内功上,还有几点需要继续完善:
\begin{itemize}
    \item \noindent 继续完善投资系统,在去年的投资系统之上,按照自己的认识更新完善投资系统并简化;
    
    \item \noindent 坚持运用价值投资的理念,特别是严守安全边际和能力圈,对投资企业进行深入研究后才行动;
    
    \item \noindent 坚持有空就看招股书和研报,让自己对行业和公司的理解更加深入,提高对企业经营未来的判断能力;
    \item \noindent 坚持有空就写投资方面的札记,记录自己投资的历程和想法,发布到网上让更多人看到;
    \item \noindent 年底到了写年度投资报告,进行复盘和回顾。
    \end{itemize}
\end{document}