\documentclass[UTF8,a4paper,12pt,lang=cn,fontset = windows]{elegantpaper} %使用了elegantpaper模板,更加简单美观
%设置了A4纸张和小四字号和windows字体
\usepackage{graphicx}
\title{03690.HK—美团投资价值分析报告} %标题加粗
\author{王琛}
\date{\zhtoday}
\begin{document}
\maketitle
%\tableofcontents
\section{行业概况及发展趋势}
\subsection{行业市场空间及发展趋势}
生活服务行业\footnote{生活服务包括餐饮外卖、到店餐饮、食品零售、当地交通、机票预订、酒店预订、火车票预 订、度假、美容服务、卡拉ok、婚庆服务、亲子服务、洗衣服务、家政服务、汽车售后服务、 房屋装修、电影票购买和其它现场娱乐服务。}主要集中在人口稠密的城市地区。中国人口众多,城市化进程迅速,2020年中国的城镇化率已经达到63.89\%,催生出了一批大城市,进一步加快了全国生活服务行业的发展。根据艾瑞报告,截至2016年底,中国人口超过100万的城市已有156个,而美国人口超过100万的城市有10个。根据艾瑞报告,2017年中国城市的人口密度为每平方千米2,426人,是美国同期的7倍以上。
\begin{figure}[htbp]
    \centering
    \includegraphics[width=1\textwidth]{figure/fg1.png}
    \end{figure}

根据艾瑞报告,2017年至2023年,预计个人消费将以8.0\%的复合年增长率增长。中国经济结构正在从投资驱动型转型为内需消费驱动型。中国消费者生活水平的提升已经导致消费者的消费行为发生了显著变化,即从基本需求转向更为自主型的支出,从实物商品转向生活服务和其它服务及体验导向。由此,旨在提升人们生活质量的消费和其它服务大量涌现并得到广泛运用。
\begin{figure}[htbp]
    \centering
    \includegraphics[width=0.9\textwidth]{figure/fg2.png}
    \end{figure}

生活服务行业预计将从2017 年的人民币18.4万亿元增长为2023年的约人民币33.1万亿元,复合年增长率达10.2\%。 由于上述因素,相信越来越多的消费者将使用电子商务,并在更多的服务品类中更频繁、广泛的使用。根据艾瑞报告,生活服务电子商务行业的规模在2017年已达到人民币 27,050亿元,预计到2023年将增至人民币80,110亿元,复合年增长率达19.8\%。
\begin{figure}[htbp]
    \centering
    \includegraphics[width=1\textwidth]{figure/fg3.png}
    \end{figure}

\subsection{行业竞争格局}
从业务发生场景和业务特点来看,目前全渠道下的生活服务业务可以分为到店\footnote{指从消费者角度而言,在商户开展业务的实体地点进行线下消费服务的特定应用场景。}、到家、出行三个主要板块,分别代表了消费者到店消费、商品服务上门以及消费者的出行需求。由于各类业务不同的业务特征,三大板块各具团点,市场的发展阶段和市场参与者都存在明显的差异。
\subsubsection{餐饮外卖}
近两年,中国外卖行业集中度愈趋增加,2019年外卖市场规模突破6500亿元。2020年Q2(按交易额统计)美团外卖、饿了么、饿了么星选及其它市场占有率分别为68.2\%、25.4\%和6.4\%,其中,美团外卖市场份额逐年提升已成为行业绝对龙头。随着美团对下沉用户的持续培育,预计美团在外卖行业的市场份额仍将继续增加。

从区域来看,2020年上半年,中国一、 二线城市仍是外卖的核心区域,外卖用户占比超过60\%,但三线及以下城市的外卖用户占比逐步提升,由2020年1月的30.7\%上升至 2020年6月的32.2\%。未来,三线及以下城市将是外卖平台用户增长的关键。
\begin{figure}[htbp]
  \centering
  \includegraphics[width=0.9\textwidth]{figure/fg5.png}
  \end{figure}

\subsubsection{到店酒旅}
2015 年美团与大众点评合并,成为中国最大的到店餐饮服务平台,早期团购业务积累的商户资源、大众点评 UGC 内容资产、以及高频外卖带来的客群,为美团交叉销售休闲娱乐、美业、亲子服务等高毛利业务提供了有力支持。
中国在线酒店行业较为集中,2020下半年美团的在线酒店市场份额(按间夜量计)约占全行业的51\%,位列第一,其次为携程市场份额约26\%。美团低线客群占比高、休闲旅游特点突出,但携程在高端商务方面依然占优。
\subsubsection{新业务及其它}
\paragraph{共享单车} 中国共享交通出行市场规模已趋于饱和,2017年是共享单车市场的爆发阶段,交通出行领域迅速成为投资热点,2017年市场规模达1072亿元,2019年市场规模达到高峰2700亿元,2020年下降至2276亿元。经过资本大战后,原来共享单车的两大巨头 OFO 和摩拜单车,一个已经陨落另外一个卖身美团。共享单车业务在风口过去之后,回归平静,目前在上海街头能见到的共享单车,主要是哈罗单车和美团,但是美团的单车规模目测远不及原来摩拜,也远少于哈罗单车的数量。
\paragraph{美团打车} 美团打车业务自2017年上线,2018年3月登陆上海,但是目前这块业务已经停滞不前,存在感比较低,目前打车行业的巨头依然是滴滴。
\paragraph{生鲜购物} 美团曾经推出过生鲜购物品牌—小象生鲜和美团买菜,目前小象生鲜基本上已经关店了,而美团买菜在短期高调推广之后现在陷入低迷目前仅一线城市还有,根据我从资料的了解是这两个生鲜业务本身成本过高、对供应链要求苛刻,可以从生鲜超市领域的盒马和超级物种扩张不力就明白这个模式存在较大困难。

可以看出,在美团的新业务及其它领域中,无论是共享单车还是打车及生鲜领域竞争都是比较激烈的,而美团并没有建立起足够的竞争优势。因此,2020年疫情爆发情况下,让社区团购大火,美团也顺势进军社区团购业务(具体社区团购业务在3.4节介绍)。
\section{公司基本概况}
美团网于 2010 年由王兴、穆荣均、王慧文等人创立。几人曾是清华大学时期的同学及校友。在创立美团之前,几人曾联合创办过“多多友”、“校内网”、“饭否网”、“海内网”等为人熟知的互联网社交网站。

美团拥有美团、美团外卖、大众点评等APP,服务范围涉及餐饮、外卖、酒店、旅游、电影、共享单车
等,业务已覆盖中国2,800余个市县区,逐步成为中国领先的本地服务电商平台。

2018 年 10 月 30 日,公司在完成港交所上市后对组织架构进行了升级。此后两
年间陆续进行了几次业务单元的优化,但基本组织框架在 2018 年调整后得到沿用。
在内部架构设计上公司采用相对扁平的业务单元架构,注重效率及各业务间的协同。
从企业发展战略角度看,本轮调整相较前轮更加突出优势强化、流量打通以及新业
务加速。公司提出“Food+Platform”战略,以“帮大家吃的更好、生活更好”为使命,建设生活服务业从需求
侧到供给侧的多层次科技服务平台。

\begin{figure}[htbp]
  \small
  \centering
  \includegraphics[width=1\textwidth]{figure/fg4.png}
  \end{figure}
\newpage
\section{主要业务板块}
美团主要为消费者提供了多样的日常生活服务选择,如餐饮外卖、到店、酒店及旅游服务,及新业务及其他服务。具体包含三大块内容:1、餐饮外卖主要包括通过平台提供订餐及配送服务,主要为美团贡献营收和用户;2、到店酒旅主要包括通过平台提供销售代金券、优惠券、订票和预定服务,这个板块毛利率稳定在88\%,主要为美团贡献毛利;3、新业务及其他业务主要包括社区团购及其他非餐饮外卖服务、交通票务、共享单车及试点网约车服务,主要为美团提供未来增长的空间,目前为亏损。

% Table generated by Excel2LaTeX from sheet '营收分析表'
\begin{table}[htbp]
  \centering
  \caption{美团2015-2020年主要经营数据汇总表\label{tab:huizong}}
    \begin{tabular}{lrrrrrr}
      \toprule
    单位为:百万元 & 2015  & 2016  & 2017  & 2018  & 2019  & 2020 \\
    \midrule
    总交易金额 & 16,100  & 236,600  & 357,200  & 515,600  & 682,100  &            -    \\
    整体变现率(\%) & 25.0\%       & 5.5\%       & 9.5\% & 12.6\% & 14.3\% &            -    \\
    收入    & 4,019  & 12,988  & 33,928  & 65,227  & 97,529  & 114,795  \\
    \multicolumn{1}{r}{佣金} & 3,601  & 10,230  & 28,009  & 47,012  & 65,526  & 74,213  \\
    \multicolumn{1}{r}{在线营销服务} & 377   & 2,465  & 4,702  & 9,391  & 15,840  & 18,908  \\
    \multicolumn{1}{r}{其它服务及销售} & 41    & 292   & 1,217  & 8,823  & 16,162  & 21,673  \\
    \multicolumn{1}{r}{餐饮外卖收入占比(\%)} & 4.3\% & 40.8\% & 62.0\% & 58.5\% & 56.2\% & 57.7\% \\
    \multicolumn{1}{r}{到店酒旅收入占比(\%)} & 93.9\% & 54.0\% & 32.0\% & 22.8\% & 24.3\% & 18.5\% \\
    \multicolumn{1}{r}{新业务及其它收入占比(\%)} & 1.8\% & 5.2\% & 6.0\% & 17.2\% & 21.0\% & 23.8\% \\
    销售成本  & 1,240  & 7,047  & 21,708  & 50,122  & 65,208  & 80,744  \\
    毛利    & 2,779  & 5,941  & 12,220  & 15,105  & 32,320  & 34,051  \\
    毛利率(\%) & 69.2\% & 45.7\% & 36.0\% & 23.2\% & 33.1\% & 29.7\% \\
    经营利润  & \textcolor[rgb]{ 1,  0,  0}{(8,474)} & \textcolor[rgb]{ 1,  0,  0}{(6,255)} & \textcolor[rgb]{ 1,  0,  0}{(3,829)} & \textcolor[rgb]{ 1,  0,  0}{(11,086)} & 2,680  & 4,330 \\
    经营利润率(\%) & -210.8\% & -48.2\% & -11.3\% &           -    & 2.7\% & 3.8\% \\
    股东净利润 & \textcolor[rgb]{ 1,  0,  0}{(10,519)} & \textcolor[rgb]{ 1,  0,  0}{(5,790)} & \textcolor[rgb]{ 1,  0,  0}{(18,917)} & 115   & 2,239  & 4,708  \\
    净利润率(\%) & -261.7\% & -44.6\% & -55.8\% & 0.2\% & 2.3\% & 4.1\% \\
    经营现金流 & \textcolor[rgb]{ 1,  0,  0}{(4,004)} & \textcolor[rgb]{ 1,  0,  0}{(1,918)} & \textcolor[rgb]{ 1,  0,  0}{(310)} & \textcolor[rgb]{ 1,  0,  0}{(9,179)} & 5,574  & 8,475  \\
    活跃商家(百万) & 2.0   & 3.0   & 4.4   & 5.8   & 6.2   & 6.8  \\
    年度交易用户数(百万) & 206   & 259   & 310   & 400   & 451   & 511 \\
    用户平均交易笔数 & 10.4  & 12.9  & 18.8  & 23.8  & 27.4  & 28.1 \\
    \bottomrule
    \end{tabular}%
\end{table}%

根据美团2015年-2020年的年报数据表~\ref{tab:huizong},分析如下:
\paragraph{总收入分析}
总收入从40亿增长到1148亿元,年复合增长96\%但增速在不断下降,主要来源于总交易金额的快速上涨和整体变现率的提高,佣金是收入的大头2020年占比65\%。具体板块来看:外卖板块是绝对主力,贡献了一半以上的收入;到店酒旅业务总体收入占比在下降;新业务占比在不断提升。
\begin{figure}[htbp]
  \centering
  \begin{minipage}[t]{0.48\textwidth}
  \centering
  \includegraphics[width=7cm]{figure/fg6.png}
  %\caption{World Map}
  \end{minipage}
  \begin{minipage}[t]{0.48\textwidth}
  \centering
  \includegraphics[width=7cm]{figure/fg7.png}
  %\caption{Concrete and Constructions}
  \end{minipage}
  \end{figure}

\paragraph{毛利和营业利润分析}
毛利由27亿元增长到340亿元,年复合增长65\%;毛利率由69.2\%下降到29.7\%。在毛利率下降的同时,毛利总额的增长主要来源于营收的快速增长。从板块来看,2018年前毛利基本上都是到店酒旅贡献的,之后外卖板块毛利率提高开始贡献毛利,2019年到店酒旅毛利占比为44\%、餐饮外卖毛利占比32\%。营业利率由亏损84.7亿元成长为盈利43.3亿元。
\paragraph{归属股东净利润分析} 
美团2015年、2016年及2017年,分别录得亏损105亿元、58亿元及190亿元,经调整EBITDA分别为净亏损59亿元、54亿元、29亿元。2018年实现盈利,2020年盈利47亿元同比增长了75\%。经营现金流由-40亿元逐步增长至85亿元。
\paragraph{商家和用户数据分析}
活跃商家从200万增长至680万,但近期增速有所下降;年度交易用户数从2.06亿增长到5.11亿,近期增速也有所下降;用户平均交易笔数由10.4增长到28.1,也趋于平稳。从这几个业务指标来看,美团现有业务的增长趋势在放缓,急需要开拓新业务来确保用户、商家及交易数量的增长。
\begin{figure}[htbp]
  \centering
  \begin{minipage}[t]{0.48\textwidth}
  \centering
  \includegraphics[width=7cm]{figure/hoyue.png}
  %\caption{World Map}
  \end{minipage}
  \begin{minipage}[t]{0.48\textwidth}
  \centering
  \includegraphics[width=7cm]{figure/yonghu.png}
  %\caption{Concrete and Constructions}
  \end{minipage}
  \end{figure}

\subsection{餐饮外卖增长趋于饱和}
2015年—2020年,餐饮外卖交易金额从156亿增长到4889亿元,年复合增长99\%;交易笔数从6.4亿笔增长到101亿笔,年复合增长74\%;平均交易金额从24元增长到48元,年复合增长15\%,已基本趋于稳定,与个人经常订外卖的数额保持一致。餐饮外卖收入从1.8亿元增长到586亿元,年复合增长228\%,速度要快于交易金额的涨幅,主要得益于变现率从1.1\%提高到13.6\%。2015年—2019年,毛利\footnote{2020年开始美团不再公布毛利数字,改为经营利润}从-2.2亿元增长到102亿元,主要来源于收入的快速增长和毛利率的提高,毛利率从-123.7\%提高到18.7\%。2019年—2020年,餐饮外卖经营利润从14.2亿元提高到28.3亿元,接近增长了一倍,经营利润率从2.6\%提高到4.3\%。

% Table generated by Excel2LaTeX from sheet '餐饮外卖'
\begin{table}[htbp]
    \small
    \centering
    \caption{餐饮外卖主要数据指标\label{tab:waimai}}
    \begin{tabular}{lrrrrrr}
      \toprule
      单位:百万元 & 2015  & 2016  & 2017  & 2018  & 2019  & 2020 \\
      \midrule
      交易金额  & 15,557  & 58,718  & 171,088  & 282,800  & 392,722  & 488,851  \\
      \multicolumn{1}{r}{交易笔数(百万)} & 637   & 1,585  & 4,090  & 6,393  & 8,722  & 10,147  \\
      \multicolumn{1}{r}{平均交易金额(元)} & 24    & 37    & 42    & 44    & 45    & 48 \\
      变现率(\%) & 1.1\% & 9.0\% & 12.3\% & 13.5\% & 14.0\% & 13.6\% \\
      活跃商家(百万) & 0.5   & 1.4   & 2.8   &           -    &           -    &             -    \\
      收入    & 175   & 5,301  & 21,032  & 38,143  & 54,843  & 66,265  \\
      \multicolumn{1}{r}{佣金} & 175   & 5,209  & 20,284  & 35,719  & 49,646  & 58,592  \\
      \multicolumn{1}{r}{在线营销服务} &            -    & 83    & 710   & 2,335  & 5,103  & 7,565  \\
      \multicolumn{1}{r}{其它服务及销售} &            -    & 9     & 38    & 89    & 93    & 108  \\
      占美团总收入(\%) & 4.3\% & 40.8\% & 62.0\% & 58.5\% & 56.2\% & 57.7\% \\
      销售成本  & 391   & 5,707  & 19,333  & 32,875  & 44,610  &             -    \\
      毛利    & (216) & (406) & 1,699  & 5,268  & 10,233  &             -    \\
      毛利率(\%) & -123.7\% & -7.7\% & 8.1\% & 13.8\% & 18.7\% &             -    \\
      经营利润  &       &       &       &       & 1,416  & 2,833  \\
      经营利润率(\%) &       &       &       &       & 2.6\% & 4.3\% \\
      \bottomrule
      \end{tabular}% 
  \end{table}%

餐饮外卖板块收入的主要来源:1、商家就在平台上产生的订单所支付的佣金(通常按已完成交易金额的百分比确定);2、以各种形式向商家提供的在线营销服务;3、按完成的配送服务而向交易用户及商家收取的配送费。餐饮外卖的收入成长取决于两方面:\textbf{一方面}受交易金额驱动,而 交易金额=交易笔数$\times$平均交易金额;\textbf{另外一方面},由美团提高变现率的能力决定。

餐饮外卖的销售成本主要包括:1、餐饮外卖骑手成本(占90\%以上);2、支付处理成本;3、客户服务及其他人员的雇员福利开支;4、交易用户激励;5、推广及广告。

公司对于餐饮外卖板块的战略重点是:业务扩展、提升市场份额和变现率\footnote{变现率等于年/期内收入$\div$年/期内交易金额。},但是从表~\ref{tab:waimai}可以看出,变现率已经趋于稳定在14\%左右,而且近来由于舆论质疑抽成太高美团调整了外卖算法,所以长期来看这方面很难再提升\footnote{2020年业绩会上公司高管表述:“公司截至2020年年底有680万个活跃商家,其中大部分都是小微商户,他们对于本地经济和公司平台的贡献都很大,我们视这些商户为商业上的合作伙伴,而非帮助平台变现的工具。公司评价外卖递送业务的关键指标一直是商家数字化比例,而非营收转化率”。};同时,美团外卖市场份额约占70\%,提高市场份额的空间也不是很大,外卖的单笔交易金额未来大概率随物价指数上涨,近两年也趋于收敛在50元左右,用户平均交易笔数近两年也稳定在28左右。当前,公司主要通过采取会员制和低线城市扩张来推动交易金额增长。\textbf{因此,无论从交易金额、还是变现率及市场率方面来看,餐饮外卖业务的增长趋于饱和,未来几年的增速会进一步下降,其主要的作用应该是贡献经营规模和用户}。(需要持续观察跟踪)
\subsection{到店酒旅进入成熟阶段}
2015年—2020年,到店酒旅收入从37.7亿增长到213亿元,年复合增长41\%,收入的增长主要来源于佣金和在线营销服务,两者目前各占一半;国内酒店间夜量从0.82亿增长到3.55亿,年复合增长34\%;2015年—2019年,毛利从30亿增长到141亿元,2020年公司不再公布毛利数据,毛利率稳定提高到89\%;2020年受到疫情营销,经营利润较2019年出现下降,但是经营利润率从37.7\%提高到38.5\%。可见到店酒旅板块非常赚钱,一直以来均是美团利润的主要贡献者。
% Table generated by Excel2LaTeX from sheet '到店酒旅'
\begin{table}[htbp]
  \centering
  \caption{到店酒旅主要数据指标}
    \begin{tabular}{lrrrrrr}
      \toprule
    单位:百万元 & 2015  & 2016  & 2017  & 2018  & 2019  & 2020 \\
    \midrule
    交易金额  & 127,500  & 158,400  & 158,100  & 176,800  & 222,100  &           -    \\
    \multicolumn{1}{r}{交易笔数(百万)} & 1,290  & 1,439  & 1,394  &           -    &           -    &           -    \\
    \multicolumn{1}{r}{平均交易金额(元)} & 99    & 110   & 113   &           -    &           -    &           -    \\
    变现率(\%) & 3.8\% & 4.4\% & 6.9\% & 9.0\% & 10.0\% &           -    \\
    国内酒店间夜量(百万) & 82    & 132   & 205   &          284  &          392  &          355  \\
    收入    & 3,774  & 7,020  & 10,853  & 15,840  & 22,275  & 21,252  \\
    \multicolumn{1}{r}{佣金} & 3,426  & 4,870  & 7,135  & 9,042  & 11,679  & 10,193  \\
    \multicolumn{1}{r}{在线营销服务} & 345   & 2,113  & 3,650  & 6,735  & 10,516  & 11,018  \\
    \multicolumn{1}{r}{其它服务及销售} & 2     & 37    & 67    & 63    & 80    & 41  \\
    占美团总收入(\%) & 93.9\% & 54.0\% & 32.0\% & 22.8\% & 24.3\% & 18.5\% \\
    销售成本  & 741   & 1,081  & 1,273  & 1,745  & 19,746  &           -    \\
    毛利    & 3,033  & 5,939  & 9,579  & 19,746  & 14,095  &           -    \\
    毛利率(\%) & 80.4\% & 84.6\% & 88.3\% & 88.6\% & 89.0\% &           -    \\
    经营利润  &       &       &       &       & 8,403  & 8,181  \\
    经营利润率(\%) &       &       &       &       & 37.7\% & 38.5\% \\
    \bottomrule
    \end{tabular}%
  \label{tab:jiulv}%
\end{table}%

到店、酒店及旅游板块收入的主要来源:1、商家在平台上出售代金券、优惠券、订票和预定支付的佣金;2、为商家提供在线营销服务和根据年度套餐提供的营销服务。该板块的销售成本和运营费用主要包括:1、雇员福利开支;2、交易用户激励;3、推广及广告;4、物业、厂房及设备折旧;5、其它外包劳务成本。
\textbf{从销售规模来看,到店、酒店及旅游业务已进入成熟阶段,到店、酒店及旅游业务的战略重点是优化向商家提供的服务及提高变现率,进军高端商务领域。}在美团“酒店+X”项目持续扩张的背景下,美团与高星级酒店的合作持续加强,2020年第四季度五星级酒店的间夜量同比增长超过110\%。

\subsection{新业务及其它—社区团购如火如荼}
2020年7月美团成立优选事业部推出社区团购业务——美团优选,重点针对低线下沉市场;2021年1月,美团优选的业务模式正式更名为“社区电商”,意图切入社区零售领域,而不仅仅限于生鲜零售板块。截至2020末,在业务上线还不到一年的时间里,美团的社区团购业务范围已覆盖中国90\%
的市县

\subsubsection{社区团购的基本概况}
社区团购是依托社区和团长社交关系,以微信群运营为传播手段,以实现生鲜商品流通的新零售模式。从用户端的视角来看,其运营模式为:社区居民加入由团长建的社区群,根据群中信息指引在微信小程序或者 APP 上下单,平台根据需求清单向供应商(多数为产地直发)采购生鲜产品,将其发至城市中心仓,再分发至社区网格仓,最后配送时由用户去自取点或团长家自取,也可由团长配送。按标普估计社区团购的 EBITDA 利润率在 2\%到 5\%之间,好于传统电商-9\%至-6\%的 EBITDA 利润率,这门生意没有像看起来那么赚钱,这是需要警惕的地方。
\begin{figure}[htbp]
  \centering
  \includegraphics[width=1\textwidth]{figure/shequ.png}
  \end{figure}

社区团购的核心优势在于\textbf{低价和高效}:
\begin{itemize}
  \item \noindent 社区团购能够达到低价的原因在于其运营成本的降低,包括履约成本、获客成本以及配送成本等。与传统商超生鲜零售模式相比,由于传统商超运营成本、门店租金、人员工资相对较高,其毛利盈亏平衡点为 35\%-40\%,而社区团购的毛利盈亏平衡点为 25\%,其中团长佣金 10\%,仓储配送成本占10\%-12\%,运营费用约占 3\%,成本大大降低。与每日优鲜等半小时达的前置仓模式相比,前置仓模式的客单价 60 元也无法覆盖配送或仓储成本,而社区团购下 20 元即可达到一个较好的平衡点.
  
  \item \noindent 社区团购高效的原因在于:1)社区团购省去了中间不必要的环节,供应链缩短;
  2)借助微信等社交流量入口,更接近用户,用户扩张性强;3)运营模式本质上是预售模式,有效实现“按需定采”,降低损耗。

  \end{itemize}

\textbf{社区团购的市场规模将超万亿}。根据网上收集到的招商证券和兴业证券近期的两份研报,分别预测社区团购的市场规模为2-2.8万亿和1.3万亿,预计占整体零售市场规模的3.3\%。据标普估计2020年社区团购交易额为1300亿元,并预计2021年交易额将增长50\%-60\%。从当前的情况来看,各家预测未来市场空间差异还是很大的,招商和兴业差了快一倍了。但是从本质上来说,社区团购业务其实就是革了乡镇市场那些小卖部的命,平台企业可以供应有限的货品来确保低价和高效,然后由用户或者团长来负责配送,减轻送达的成本,其实可以看成是低配版的“开市客”,我认为未来的前景还是很不错的,比起6500亿规模的外卖市场空间还是要大的多。
\subsubsection{社区团购当前处于军阀混战时期}
当前,社区团购领域竞争异常激烈,多多买菜、美团优选、橙心优选等全国性平台在2020年三季度集中性杀入社区团购市场,依托流量、资本等优势在全国快速扩张。美团入局后,在较长的时间保持领先优势,但2021年4月以来多多买菜增长迅猛,快速缩短了与美团优选的差距。截止2021年4月底,拼多多与美团单日GMV大致相当。

\begin{figure}[htbp]
  \centering
  \includegraphics[width=1\textwidth]{figure/fg9.png}
  \end{figure}

\textbf{社区团购竞争短期看资源、中期看履约和物流能力、长期看平台生态能力}。社区团购的竞争,短期来看,企业是以价格优势换流量,裂变拉新与高额补贴相结合,拥有丰富地推资源和强大资金实力的企业更占优势;中期来看,了解下沉用户需求,运营能力强的平台,能提高留存率,实现流量沉淀;长期来看,供应链能力建设保障履约时效和商品品质,保障用户体验,是形成品牌护城河的关键,同时平台需充分发挥内部生态协同效应,巩固市场地位。按此看目前头部的几家:

\begin{itemize}
  \item 美团集合了资源、运营和生态三个维度的综合优势;
  \item 橙心优选只有物流配送相对突出,缺乏平台优势,资金上也较为欠缺;
  \item 多多买菜在运营的供应链服务上最为出色,现金流较为充足,但新业务的发展易对传统业务造成替代;
  \item 兴盛优选的优势在于其运营方面的便利店线上下单和线下经营及提货的复合优势,但缺乏平台优势。
  \end{itemize}
  
  \begin{figure}[htbp]
    \centering
    \includegraphics[width=1\textwidth]{figure/fg11.png}
    \end{figure}
\textbf{预计社区团购未来将形成寡头垄断的格局,最终将会形成2-3家全国性的社区团购平台,拼多多和美团有望笑到最后}。主要依据为:1、社区团购具备规模效应,头部企业竞争力将随着规模的扩大而持续增强;2、社区团购因为上游生鲜供应链无法垄断,因此无法形成完全垄断,只能形成寡头垄断局面;3、当前大部分参与者没有资金实力要面临短期持续的亏损而无法持续,按招商证券预计2021年社区团购业务亏算200亿元以上。拼多多和美团市值超万亿账面现金充裕,背靠上市平台融资渠道通畅,可以坚持持续作战,有望能够脱颖而出,但这需要持续的观察。

\begin{figure}[htbp]
  \centering
  \includegraphics[width=1\textwidth]{figure/fg10.png}
  \end{figure}

\subsubsection{社区团购对于美团的意义}

社区团购是一场生死大战,要付出巨额经济代价,2020年第四季度公司新业务营业亏损为60亿,其中一半来自美团优选\footnote{其他营业亏损扩大的业务还有美团打车,美团买菜,商家进货平台美团快驴。}。王兴在业绩发布上表述:\textit{“社区团购业务美团优选是五年,或者十年才有一次的优质机会。对于电商企业而言,得到建立新基础设施的机会并不是一件普通的事情,回顾中国电商发展史,无论是美团还是京东,可能都会承认建立新的基础设施需要巨大投入,但是一旦拥有了完备的基础设施,就可以覆盖更大的用户群,获得更大的市场,重构价值链,也为社会创造巨大价值”}。具体意义有三个方面:

\textbf{1、社区团购能够挖掘下沉用户,增大用户规模}。美团用户增长率从 2018 年底的接近 30\%的高位逐步降低,在 2020 年 3 月首次下降至个位数。由于社区团购商品为食品生鲜及日用杂货等,用户目的是在家做饭,这些需求是以往的业务现有核心业务餐饮外卖和到店酒旅业务无法触达的,而社区团购业务能够有效拓展这些未曾开发过的市场。2020 年 9 月,在美团新业务美团优选的推出下,交易用户同比增长率首次出现正增长,表明美团布局社区团购的有效性;
\begin{figure}[htbp]
  \centering
  \includegraphics[width=1\textwidth]{figure/fg12.png}
  \end{figure}

\textbf{2、促进业务板块协同效应。将吸纳的新客户引流至其他业务,给原有业务带来新增长动力}。从企业战略角度来看,美团可以将通过社区团购模式获得的新用户引流至到店业务和其他服务(如金融服务、交通服务等),能够有效地提高用户 LTV 和公司的盈利能力;

\textbf{3、社区团购市场规模成长空间巨大,成熟后大于外卖市场规模}。按兴业证券预计到 2025年,中国的社区团购市场规模将达到 1.3 万亿元,将大于外卖市场规模。基于美团在社区团购板块的一系列优势,能够占据一定优势地位的市场份额,对于美团来说这是有效切入到实物零售行业的历史机遇。


\section{竞争优势和未来增长潜力}
\subsection{主要竞争优势}
\subsubsection{具备丰富的行业经验}
美团自身具有多年的团购经验和送餐的 O2O 经验,使得其在争夺市场、运营管
理、制定佣金制度方面有较为丰富的经验。在 2010 年开始的团购大战中,美团凭
借自身强大的体系化管理、组织能力以及对行业清醒的认识,成功从千团中脱颖
而出成为最终赢家,积累了丰富的本地生活服务大战经验,因而在社区团购制定
团长佣金制度方面也拥有丰富的经验和战斗力。此外,多年的送餐 O2O 运营经验
使得其在用户反馈及售后等方面有足够的处理能力,有利于客户关系的良好维系。
\subsubsection{平台优势流量巨大}
美团主营业务包括外卖、到店酒旅业务以及包括食杂零售业务等的新业务,属于综合服务平台,用户只要有众多业务中的其中某一项需求,就会点开软件。美团拥有大众点评(独特优势)、微信小程序等引流手段,因而用户覆盖群体较广,2021 年 1 月美团的月活跃用户数达 3.06 亿。鉴于所提供服务的规模和全面性,一站式服务平台可通过跨市场的营销能力进行交叉销售,通过服务质量和深度来建立品牌和认可度,同时可以通过口碑效应有效降低用户获取成本。此外,一站式平台还可以通过其便利性,提升用户的留存率和黏性。由于规模径济,具有规模的公司在获得用户和扩展新服务品类方面也具有显著优势。
\subsubsection{强大配送能力}
美团当前骑手数约为400万,饿了么大约为300万。消费者对更高服务质量和便利性的需求日益增长,使配送能力在如今激烈的服务类电子商务的竞争环境中成为一项关键的差异化要素。配送能力一般取决于基本技术基础设施、覆盖范围、覆盖深度、配送速度和服务质量。强大的配送能力有助于明显提高消费者服务质量,能吸引及留住客户,增加平台的销售和收入,从而形成良性循环。此外,强大的技术基础设施、规模化和高订单密度有助于通过优化资源分配和规模经济效应提高效率,从而改善运营的成本效益。
\subsubsection{优秀的管理团队}
以王兴为首的美团管理团队,具备丰富的互联网创业经验,美团从起初的团购网站起家,一步步通过外卖、酒旅到现在综合性生活服务平台,能够在互联网生死存亡大战中脱颖而出,充分说明了团队的优秀。而且,在网上视频可以看到,王兴本人没有独立办公室,仅仅拥有一个普通的工位,说明企业管理团队作风优良,值得信赖。
\subsection{未来增长潜力}
通过对美团几大业务板块进行分析后,可以对美团未来的增长潜力做一个分解:
\begin{itemize}
  \item 餐饮外卖业务,作为当前的主力,要继续加强市场开拓,提高交易金额、稳定变现率和毛利率,在总体增速下降的情况下提高板块的盈利能力为美团贡献利润,同时作为现金奶牛支撑新业务发展;
  \item 到店酒旅业务,作为最赚钱的业务已进入成熟阶段,要继续稳定收入规模和毛利率,通过进入高端商旅市场实现扩张,保障板块在美团营收比例不要下滑过快,对美团的利润贡献要保持基本稳定;
  \item 社区团购业务,这是美团未来5年的最大增长空间,当前近两年处于大量投入时期,关键是要迅速占领市场,保持竞争优势,同时熬死竞争对手来实现对市场的垄断,使之成为另外一个“餐饮外卖业务”;
  \item 其它业务如美团打车、闪购、买菜业务等拖累业务,预计在2021-2022年由于社区团购业务会出现巨大亏损,如果能够趁机剥离这些不赚钱也没有竞争优势的业务,收缩战线,加大业务聚焦,会提升美团的整体业务表现。
  \end{itemize}
\section{估值分析}
\subsection{历史估值区间分析}
美团自2018年上市以来,股价最低是2019年1月份40元,最高点是2021年2月460元(PE-TTM为644倍)。美团股价在2020年4月后经历了快速上涨,从70元上涨到460元,接近涨了6.6倍,当前股价244元接近高点的一半,但是PE-TTM 依然是264倍,而同样在港股上市的腾讯控股目前的PE-TTM为29倍,虽然企业所处的阶段不太一样,美团成长性更好,但从差了一个数量级的PE而言美团确实很贵,存在高估的风险。

\subsection{未来三年营收及估值结论}
在美团发力社区团购的情况下,预测其未来盈利情况存在很大的难度,因此在此个人无法给出预测,也没有合适券商的报告可以参考。但2021年4月标普对美团发布了负面信用展望,指出:\textit{“2021 年 4 月 1 日,因美团扩张社区团购业务所需投资规模庞大,由此面临较高的执行风险,我们将其评级展望从稳定调整至负面。我们估算,该公司 2021 和 2022 年的 EBITDA 亏损将达到 150 亿元至 200 亿元人民币。此外,我们预计其 2021 年的自由经营性现金流净流出也将达到 230 亿元至 280亿元;2022 年或可改善至净流出 90 亿元至 140 亿元”}。我认为标普的报告更具有参考意义,如果标普的意见正确,那么很明确美团2021年和2022年都将会巨亏(2019年-2020年EBITDA分别为72.5亿元、47.4亿元),用PE法估值也没有意义。因此,就目前情况来看,由于不确定性很大,当前无法对其进行有效的估值。
\section{主要的风险及应对措施}
\subsection{反垄断政策风险}
2021年4月26日,国家市场监管总局发布公告称,根据举报,依法对美团实施“二选一”等涉嫌垄断行为立案调查。按照监管部门此前的处罚先例,美团或面临超百亿罚款。这是阿里在被处罚182亿元后,又一进行的反垄断调查。由于目前依然还处于调查期间,还未看到具体监管部门的结论。5月10日晚,上海市消保委表示在5月10日下午约谈了美团,指出了美团在消费者权益保护方面存在的突出问题,据悉美团的主要问题有以下三点:一是取消订单引发的退款问题;二是订送餐、生鲜蔬菜配送不履约问题;三是页面误导消费者的问题。资本市场对此做出了反应:从4月28日下跌至5月11日,美团股价已经经历了十连跌,股价距离今年最高点已经下滑了46.5\%,市值距离今年最高点已经蒸发超过1.2万亿港元。
\subsection{社区团购九不准}
2020年12月22日,国家市场监管总局联合商务部组织召开规范社区团购秩序行政指导会,阿里巴巴、腾讯、京东、美团、拼多多、滴滴等6家互联网平台企业参加。会议指出当前社区团购存在的低价倾销及由此引起的挤压就业等突出问题,希望互联网平台企业切实践行以人民为中心的发展理念,主动承担更大的社会责任,在增创经济发展新动能、促进科技创新、维护公共利益、保障和改善民生等方面体现更多作为、更多担当。会议提出“九不准”,并要求各地市场监管部门要积极回应社会关切,加强调查研究,研判掌握社区团购市场动态,针对低价倾销、不正当竞争等问题,创新监管方式,加大执法办案力度,依法维护社区团购市场秩序。

虽然在短期内,由于监管机构的介入,并出台了“九不得”会对社区团购过热得情况进行压制,特别对恶行和不正当竞争进行监管,但往往规范化受益的是龙头企业,小企业往往无法承受合规的成本导致竞争失败,因此对于美团和多多这样的龙头,长远来看影响应该是正面的。
\subsection{非正式雇佣风险}
五一前北京市人社局劳动关系处副处长王林体验了一天做外卖小哥的感觉,送单12小时只赚了41块,感慨员工生存不易。美团公司代表表示,在册外卖员有将近1000万人,但和公司属于外包关系,员工所有保障就是由每天从他们佣金中扣除的3元钱的商业保险来承担,引发了社会舆论对于美团等互联网企业应该给员工上“社保”的议论。随之,美团、饿了么两大平台近日悄然调整了商家佣金。以美团外卖为例,新的佣金费率将不再是固定的,而是根据配送距离、订单价格、配送时段等因素进行浮动。这也是美团高管表态不愿过多提高佣金比例和变现率所在。
\subsection{社区团购持续亏损风险}
按标普预计 2021 和 2022 年,与美团优选相关的资本支出将达到人民币 10-30 亿元,主要用于升级供
应链。除基础设施外,美团还提供高佣金给社区团购的团长,并增加向终端用户发放优惠。特别是在
美团积攒业务量的初始阶段,此类支出将持续较高。在美团优选达到一定的用户数量级之前,经营亏
损的状况可能将持续,而且标普认为持续时间可能长达至少两年。

\section{投资建议}
如果说阿里和京东是上一个商品消费时代的超级平台,那么美团就是下一个服务消费时代的超级平台,而且看起来现在没有对手,从这一点来说美团具备良好投资基础。目前的美团已经不是那个外卖时代快速成长的小孩了,而是已经到了青春期尴尬的年龄了,当前的三大业务板块中,外卖和酒旅发展趋于缓慢,而新业务板块中众多业务仍处于亏损阶段,社区团购就是这颗看起来最耀眼的明星,引来无数的互联网大佬进入纷纷厮杀。

我虽然有理由相信社区团购会继续发展,但也知道这一趋势尚未经受检验。当前的风口能不能持续得到验证以证明是一个成功的商业模式?还是会像2018年流行的共享单车一样一地鸡毛,然后走向没落,只能说目前存在着巨大的不确定性,远远还未到分出胜负和证明可行的时候。

因此,对于经历疯涨和腰斩之后,当前估值仍高达264倍的美团目前还不具备投资价值,因为缺乏几个方面的要素:1、美团能否成为社区团购的最后胜利者存在极大的不确定性;2、美团当前的股价透支了过多的乐观情绪,而不具备安全边际。我觉得似乎在2022年团购市场大面积亏损,企业开始悲观纷纷退出时,到时候可能是一个更好机会,在此期间需要保持继续观察以下几个内容:
\begin{itemize}
  \item 外卖业务营收的增长是否还有动力,变现率和毛利能否再提升;
  \item 社区团购业务市场竞争格局的变化;
  \item 国家对于平台企业的监管政策和走向;
  \item 未来2年美团亏损的情况及持续的期间;
  \item 美团是否会关闭打车、买菜等亏损业务。
  \end{itemize}

\textbf{综上所述,美团是一家优秀的互联网平台企业,但目前不具备投资价值,需要耐心持续的观察和验证观点,到了一个更加确定的时点或许美团就是一个很好的投资对象}。
\end{document}