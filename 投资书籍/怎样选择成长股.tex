%这个模板在自己写的基础上参考了李杰的分析模板,做了结构的调整,优化了公司竞争优势,增加了公司成长能力分析的部分,比较具备逻辑性
%\documentclass[UTF8,a4paper,zihao=-4]{ctexbook} %设置了A4纸张和小四字号,这个mac也可以用
\documentclass[UTF8,a4paper,zihao=-4,fontset = windows]{ctexart} %设置了A4纸张和小四字号和windows字体
\usepackage{graphicx,epigraph}
\setmainfont{Times New Roman} %设置英文字体为Times New Roman
\CTEXsetup[name={第,篇}]{part}
\CTEXsetup[name = {第,章}]{section} %使用第x章
\CTEXsetup[number={\chinese{section}}]{section} %使用中文的数字作为章节
\title{\textbf{怎样选择成长股 \\
                Common Stocks and Uncommon Profits}} %标题加粗
\author{菲利普·A·费雪  \\   
        罗耀宗(译)    \\
        王琛(整理)}
\date{\today}
\begin{document}
\maketitle
\newpage
\tableofcontents
\newpage
\part{怎样选择成长股}

\textbf{序}
\\

在投资领域出版一本新书,很可能需要作者有所说明。因此以下的文字是个人的感想,用以说明为什么要再就这一个主题,写书供投资大众参考。

在那时新设的史丹福大学企业管理研究所待了一年之后,一九二八年五月我踏进商业世界。我到现在的旧金山国安盎格鲁国民银行
(Crocker-Anglo National Bank)一个重要单位的统计部门做事,廿个月后当上那个部门的主管。以今天的用语来说,我应叫做证券分析师。

我在这里就近看到难以置信的金融纵欲游戏,高潮止于一九二九年秋,以及接下来的困顿期。根据个人的观察,我相信西岸有大好机会,可以经营专业投资顾问公司,与古老但无人尊敬的若干证券经纪商──晓得每一样东西的价格,但对价值一无所知──截然有别。

一九三一年三月一日,我创立费雪公司(Fisher &Co.),那时是个投资顾问咨询事业,服务一般大众,但注意重点主要放在几家成长型公司。业务蒸蒸日上。接下来碰到二次世界大战。前后三年半,我在陆军航空兵团做各式各样的工作,利用公余之暇,检讨个人曾经做过的成功投资行动,特别是不成功的投资行动,以及以往十年我见到别人成功和不成功的投资行动。检讨过程中,若干投资原则开始浮现,而这些投资原则和金融圈普遍奉为圭臬的一些原则有异。

复员后,我决定把这些原则实际应用到尽可能不受周遭问题干扰的企业环境中。费雪公司并没有服务一般大众,因为十一多年来,费雪公司从来不曾同时拥有十来位客户。这段期间内,这些客户大多保持原状。费雪公司以前主要的兴趣,放在资本大幅增值上,现在所有的作为,则把这件事当作唯一的目标。我注意到过去十一年,股价普遍上扬,任何人只要这么做,都能赚到钱。不过,虽然在某种程度内,这些资金获得的报酬一直领先投资人普遍观察的整体市场指数,我发现战后时期遵循这些原则,比战前十年我局部运用这些原则更有所获。或许更重要的是,在整体市场静止不动或下跌的年头,运用这些原则,所获报酬不比市场急剧上扬时差。

研究我自己和别人的投资纪录之后,两件重要的事促成本书完成。其中之一我在其他地方提过几次,也就是投资想赚大钱,必须有耐性。换句话说,预测股价会到达什么水平,往往比预测多久才会到达那种水平容易。另一件事是股票市场本质上具有欺骗投资人的特性。跟随其他每个人当时在做的事去做,或者自己内心不可抗拒的吶喊去做,事后往往证明是错的。

这些年来,我发现自己必须不厌其详地向个人所管理的基金投资人,解释我采取的某种行动背后的原则。只有如此,他们才能了解何以我要买一些他们完全没听过的证券,因此不致一时冲动弃基金而去,好让我有充裕的时间让它在市场的报价上开始证明买得有理。

慢慢的,我兴起了一个念头,想把这些投资原则汇集成篇,印成文字,留下纪录,以便查考。于是我开始摸索这本书的组织架构。接下来我想到许多人,其中大部分都买了基金,但规模远不如我管理,只属于少数人的基金。这些年来,他们来找我,问到身为小额投资人的他们,如何踏出正确的一步。

我想到无数小额投资人处境之艰困,因为他们无意中吸收了各式各样的想法和投资观念,但多年下来,已证明代价太过昂贵,可能的原因是他们从来没有面对更根本性的观念挑战。最后我想到和另一群人的许多对话,他们对这些事情也很感兴趣,只是出发点不同。他们是股票公开发行公司的总裁、财务副总裁、出纳员,许多人渴望尽可能了解这些事情。

我得到的结论是,有必要写一本这样的书。我希望这种书能以非正式的写法,把我想讲的事情以第一人称的方式,告诉身为读者的你。书内用到的大部分语言、许多例子和比喻,和我把这些观念亲口告诉买我基金的人一样。希望我的坦诚,有时是直言无隐,不致冒犯任何人。特别希望书内所提观念的价值,或能掩盖文笔之拙劣。

\rightline{Philip A.Fisher}

加州圣马特奥(San Mateo)

一九五七年九月

\section{过去提供的线索}

你在银行存了点钱,现在想买些普通股。会有这个决定,可能是因为你希望以别的方式运用这笔钱,多一点收入,也可能因为你想和美国这个国家一起成长。你也许想起亨利·福特(Henry Ford)创立的福特汽车公司(Ford Motor Company),或安德鲁·梅隆(Andrew Mellon)创办的美国铝业公司(Aluminum Company of America),想象自己能不能也找到一些年轻公司,可能今天就为你奠下雄厚财富的基础。你也有可能害怕甚于期待,希望攒些老本,以备不时之需。因此,听过愈来愈多有关通货膨胀的事情之后,你渴望找到既安全,又能防止购买力减退的某种东西。

或许你真正的动机,是许多这类事情的综合体,原因是你晓得某位邻居在市场赚了一些钱,也有可能是你接到一份宣传邮件,说明为什么中西全麦面包公司(Midwestern Pumpernickel)的股票现在很便宜。但是,背后基本的动机只有一个。不管基于什么样的理由,或者用什么样的方法,你买普通股是为了获取利润。

所以说,似乎有个合乎逻辑的做法,也就是想到买普通股之前,第一步是看看过去以什么方法最能赚到钱。即使随意浏览美国的股票市场史,也可以看出人们使用两种很不相同的方法,累聚可观的财富。十九世纪和廿世纪上半叶,许多巨额财富和不少小额财富,主要是靠预测企业景气周期而赚到的。在银行体系不稳,导致景气荣枯循环相生的期间,景气坏时买进股票,景气好时卖出,则投资增值的可能性很高。和金融界有良好关系的人尤占上风,因为金融界可能事先晓得银行体系何时会呈现紧张状态。

但应了解的最重要事实,或许在于一九一三年联邦准备制度(Federal Reserve System)建立后,那种股市时代已经结束,并于罗斯福总统任内初期通过证券交易管理法后,成了历史,使用另一种方法的人,赚了远比以往多的钱,承担的风险远低于从前。即使早年,找到真正杰出的公司,抱牢它们的股票,度过市场的波动起伏,不为所动,也远比买低卖高的做法赚得多,而且赚到钱的人数远多于往日。

如果这段话令你惊异不置,扩大而言可能更难叫人相信。它也能提供一把钥匙,打开投资成功的第一道大门。今天美国各证券交易所挂牌交易的股票,不是只有几家公司,而是很多公司。廿五到五十年前,投资一万美元,今天有可能成长为二十五万美元或此数的几倍之多。换句话说,大部分投资人终其一生,以及他们的父母亲可以为他们几乎所有人打算的期间内,有无数的机会,为自己或子女奠下成为巨富的基础。这些机会存在的地方,不见得必须在大恐慌底部的特定一天买股票。这些公司的股票价格年复一年都能让人赚到很高的利润。投资人需要具备的能力,是区辨提供绝佳投资机会的少数公司,以及为数远多于此,但未来只能略为成功或彻底失败的公司。

今天是不是有这样的投资机会,未来几年能给我们等量齐观的获利率?这个问题的答案值得注意。如果答案是肯定的,则投资普通股为致富之道便不言可喻。幸好,有强烈的证据显示,今天的机会不只和本世纪头廿五年相当,更且远优于当年。

个中理由之一,在于这段期间内,企业管理的基本观念有所变化,处理企业事务的方法也随之更动。一个世代以前,大公司的负责人通常是拥有公司的家族成员。他们视公司为私人财物。外部股东的利益多遭忽视。他们如考虑到经营管理延续性的问题──也就是,训练年轻人接替年迈无法视事的老人──主要动机一定是为儿子或侄子着想,要他们继承掌舵者的职位。管理阶层很少想到进用贤才,以保护一般持股人的投资。在个人独裁主宰一切的那个时代,马齿渐增的管理阶层往往抗拒创新或改善,甚至不愿倾听建言或批评。这与今天企业界不断竞相寻找各种方法,把事情做得更美好的现象大相径庭。今天的企业高阶主管往往持续自我分析,而且马不停蹄地找寻改善方法,同时经常跨出本身的组织,就教于各方面的专家,以求金玉良言。

以前的日子中,总是存在一个很大的危险,也就是当时最吸引人的公司,不会继续在它的领域保持领先地位,或者内部人会攫取所有的利益。今天,这样的投资危险虽没有完全消除,小心谨慎的投资人所冒风险远低于以往。

企业管理阶层的一个变化面向值得留意。企业的研究和工程实验室不断成长──企业管理阶层如果没有学习相对应的技巧,这件事对持股人没有好处;企业管理阶层如有相关技巧,研究发展可以成为一种工具,开启黄金收获大门,让持股人的利润节节上升。即使今天,许多投资人似乎只是略微晓得这方面的发展变得多快、这事肯定进一步强化,以及对基本投资政策的冲击。

一九二〇年代末,只有约六家制造业公司有象样的研究组织。以今天的标准来说,它们的规模很小。直到人们担忧希特勒(Adolf Hitler)加速这方面的活动,用在军事上,工业研究才真正开始成长。

此后不断成长。一九五六年春《商业周刊》(Business Week)发表一份报告,以及麦格罗·希尔公司(McGraw-Hill)其他很多专业刊物指出,一九五三年民间企业研究发展支出约三十七亿美元,一九五六年成长为五十五亿美元,而依目前的企业经营计划,一九五九年将有六十三亿美元以上。同样叫人称奇的是,调查指出,到一九五九年,也就是仅仅三年后,许多知名企业预料,总营业额中来自一九五六年不存在的产品比率,将从十五%提高到二十%以上。

一九五七年春,同一份杂志做了类似的调查。如果一九五六年发表的总支出数字之大令人惊讶的话,则仅仅一年之后揭露的数字或可称为爆炸性成长。研究支出比前一年的总额增加二十%,升抵七十三亿美元!四年内约成长一百%。这表示,十二个月内实际增加的研究支出,比一年前预测的卅六个月总增加金额还多十亿美元。在此同时,预估一九六〇年的研究支出为九十亿美元!此外,所有的制造业预期一九六〇年的营业额中,将有十%来自三年前还不存在的产品,而前一年的调查中,只有少数几种制造业有这种预期。若干制造业的这个比率──只是推出新机型和风格上的转变,不计入其内──为数倍之高。

这种事情对投资的影响,不可能高估。研究成本变得很大,没有从商业观点善加处理的公司,可能在营运费用不胜负荷的情形下,步履蹒跚。此外,管理阶层或投资人找不到唾手可得的简单量尺,衡量研究的获利性。即使最出色的职业棒球选手,也无法预期每上场打击三次,会有一次以上击出安打。同样的,数目庞大的研究项目,受制于平均数法则,根本无法创造利润。而且,纯因机率作祟,连经营管理最好的商业实验室,也可能出现异常状况,不少无利可图的项目全部集中在某段期间出现。最后,一件项目从首次构思,到对公司的盈余带来显著有利的影响,动辄需要七到十一年的时间。因此,连利润最丰厚的研究项目,在财务上也是不小的重担,直到有一天,才能增添股东的利润。

但如管理不良的研究成本既高且难发觉,则研究做得太少的成本可能更高。未来数年,随着许多新型原物料和新型机械的引进,成千上万公司,甚至整个行业,如没跟上时代脚步,则它们的市场会日渐萎缩。计算机用于追踪纪录,以及放射用于制造加工,也将使企业的基本经营方式发生重大变化。有些公司会留意这些趋势,同时根据自己的观察,设法大幅提升营业额。这类公司中,可能有一些公司的管理阶层,继续在日常营运工作上维持最高的效率标准,并运用同样良好的判断力,在影响长期未来的事务上保持领先地位。幸运的持股人很可能发大财。

除了企业管理阶层眼光改变和研究崛起等影响力量,还有第三个因素同样能给今天的投资人多于过去数十年的机会。本书稍后──谈何时应买卖股票──似乎比较适合讨论景气周期对投资政策有什么样的影响。但此时似应讨论这个主题的一部分。这是持有若干类别的普通股占有较大优势的原因,因为美国联邦政府的基本政策有所转变,主要是一九三二年后的事。

在那年之前和之后,不管他们做了多少事,两大党执政时经济若有荣面,总往自己脸上贴金,别人也歌功颂德。同样的,景气不振时,他们通常同时受到反对党和一般大众挞伐。不过,一九三二年以前,刻意创造庞大的预算赤字,以支撑疲弱不振的工商业,在道德上是否合理,以及是否具备政治智慧一事,两党负责任的领导人总是慎重以对。除了设立面包配给站和免费餐厅,凡是对抗失业成本远高于此的办法,不管哪个政党执政,都不会认真考虑。

一九三二年以后,政策一百八十度转弯。民主党对平衡联邦预算的关心,可能不如共和党,也可能不亚于共和党。但艾森豪威尔总统以降,前财政部长韩弗莱(Humphrey)可能除外,负责任的共和党领导人一再指出,要是企业景气大幅转差,他们毫不犹豫会降低税负,或者做其他必要,但会使赤字增加的事,以恢复景气荣面和消除失业。这和经济大萧条之前奉行的教条大相径庭。

即使这种政策上的转变没有为人普遍接受,其他一些转变却带来大致相同的结果,但速度可能没那么快。威尔逊(Wilson)总统任内,所得税的征收才合法。直到一九三〇年代,此事才对经济发生重大影响。早年时,联邦岁入多来自关税和类似的货物税。这些税收随着景气的荣枯而温和波动,但大体上相当平稳。相反的,今天约八十%的联邦岁入来自企业和个人所得税。这表示工商业景气若普遍大幅衰退,联邦税收也会相对减少。

在此同时,农产品价格支撑和失业补助等各种措施,立法通过。在企业景气下降会使联邦政府的税收大幅减少之际,法律强制政府在这些地方提高支出,政府的花费势必急剧提高。除此之外,为了扭转不利的企业景气趋势而减税,增加公共建设和借钱给各种艰困行业,情势变得日益明显,也就是如果经济萧条果真发生,联邦赤字轻而易举便会高达每年二百五十亿到三百亿美元。这种赤字会导致通货膨胀率上扬,一如战争支出造成的赤字,于战后导致物价窜升。

这表示,经济萧条真的发生时,为期可望比以前的一些严重萧条短。继之而来的,几乎肯定是通货膨胀率进一步上扬,导致物价普遍上涨,而这在过去,对某些行业有帮助,但伤害其他一些行业。在这种一般性的经济背景之下,企业景气循环的威胁,可能和过去财务疲弱或边际公司的持股人受到的威胁一样严重。但对财力雄厚或有借款能力以度过一两年艰困期的成长型公司的持股人来说,在今天的经济环境下,业绩即使下挫,也只是持股市值暂时萎缩,不像一九三二年以前那样,必须深思投资本身受到根本性的威胁。

另一个来自这种内在通货膨胀倾向的基本金融趋势,已经根深蒂固于美国的法律,以及大家普遍接受的政府经济责任观念中。对严格遵守长期抱牢准则的一般投资散户来说,债券成了不理想的投资对象。数年来利率上升的趋势,到了一九五六年秋,涨势更为激烈。高评等债券的价格因此跌到廿五年来最低,金融圈内许多人高声疾呼,认为应从价格处于历史性高档的股票,转而投资这些固定收益证券。债券极高的收益率相对于股票的股利报酬率──和正常状况下的比率相较──似乎强烈支持这种做法的正确性。短期内,这种做法迟早可能证明有利可图。因此中短期投资人──也就是进出时机触觉敏锐,善于判断何时做必要的买进和卖出动作的「操作者」──可能大受吸引。这是因为如果景气大幅衰退,几乎肯定会导致货币市场利率下降,债券价格相对上涨,而股价很难上扬。这样的说法引导我们做成结论,认为高评等债券可能有利于投机客,不利于长期投资人。这似乎正和一般人在这个问题上的想法相抵触。不过,了解了通货膨胀的影响之后,为什么这种事可能发生,便显得相当清楚。

纽约第一国民银行(First National City Bank of New York)一九五六年十二月在一封信中,列出一张表,说明一九四六到一九五六年十年内全球性货币购买力贬值的现象。表内包括自由世界十六个主要国家,每个国家的货币价值都显著萎缩,从瑞士程度轻微,到智利极其严重;前者十年期间结束时,能买到十年前八十五%的东西,后者十年内则丧失九十五%的价值。美国的跌幅是二十九%,加拿大三十五%。也就是,这段期间内,美国每年的货币贬值率是三.四%,加拿大是四.二%。相对的,这段期间之初,利率相当低,美国政府公债提供的收益率,只有二.一九%。这表示,如果考虑货币的实质价值,持有这种高评等固定收益证券的投资人,实际上每年承受一%以上的负利率(或损失)。

但假使投资人不是在这段期间之初利率相当低时买进债券,而是十年后利率相当高时买进。纽约第一国民市银行也在同一篇文章中,针对此事提供数字。他们估计,十年期间结束时,美国政府公债的报酬率是三.二七%,投资不但还是没有报酬,甚至略微亏损。可是这篇文章发表后六个月,利率急剧上升到三.五%以上。投资人如有机会在这段期间之初,获得四分之一世纪以来最高的投资报酬率,最后情形如何?绝大部分的例子中,他还是无法获得实质投资报酬。许多例子中,他实际上发生亏损。这是因为几乎所有的这种债券买主,必须就领取的利息缴交最低二十%的所得税,才能计算真正的投资报酬率。许多例子中,债券持有人的税率很高,因为只有最初二千到四千美元的应税所得适用二十%的税率。同样的,如果投资人在这个历史性最高报酬率的水平购买免税地方公债(municipal bonds),由于这些免税证券的利率较低,依然无法提供任何实质投资报酬率。

当然了,这些数字只适用于这一段十年的期间。但它们的确指出,这是全球性的现象,任何一国不太可能藉政治趋势加以扭转。债券当做长期投资工具的吸引力,真正重要的地方在于能否期待未来出现类似的趋势。在我看来,仔细研究整个通货膨胀的机制,可以很清楚地看出,通货膨胀大幅攀升源于总体信用扩增,而此事又是政府庞大的赤字使得信用体系的货币供给大增造成的。赢得二次世界大战带来的庞大赤字,种下了恶因。结果是:战前的债券持有人如维持当时的固定收益证券部位,则投资的实质价值已损失逾半。

我们法律,以及更重要的,大家普遍认为经济萧条时期应做的事,导致两种情况里面的一种不可避免。这事前面解释过。这两种情况,一是企业营运保持不错,出色的股票表现继续优于债券,另一是经济严重衰退。如是后者,债券的表现会暂时优于最好的股票,但接下来政府大幅制造赤字的行动,导致债券投资真正的购买力再次大跌。经济萧条几乎肯定会制造另一次通货膨胀急升;这种令人心慌意乱的时期中,决定何时应该卖出债券极其困难,我因此相信,在我们复杂的经济中,这种证券主要适合银行、保险公司、其他机构投资,因为它们有资金上的义务,必须加以对冲,或者,适合抱持短期目标的投资人投资。对长期投资人来说,它们无法提供足够的利益,抵消购买力进一步减退的可能性。

继续讨论之前,宜先简短汇总研究过去、从投资观点比较过去与现在的主要差异得出的各种投资线索。这样的研究指出,运气特别好,或者观察力特别敏锐的人,偶尔能找到一家公司,多年来营业额和盈余成长率远超过整体行业,而能获得很高的投资报酬。研究进一步指出,当我们相信自己已找到这样一家公司时,最好长期抱牢不放。它强烈暗示我们,这样的公司不见得必须年轻,规模小。相反的,不管规模如何,真正重要的是管理阶层不但有决心推动营运再次大幅成长,也有能力完成他们的计划。过去给了我们另一个线索,也就是这样的成长往往和他们晓得如何在各个自然科学领域组织研究工作有关,好在市场上推出经济上有价值,而且通常相互关联的产品线。我们可以清楚地看出,这种公司的共同特性,是管理阶层不因重视长期规划,而在日常任务的执行上稍有松懈;他们仍会把平常的营运工作做得很好。最后,我们觉得十分放心,因为廿五或五十年前存在很多绝佳的投资机会,今天,这样的机会可能更多。

\section{“闲聊”有妙处}

我们应该注意什么事情,上一章所谈种种,就一般性的描述来说,有其帮助。但要当做实务上的指引,藉以找到杰出的投资对象,显然帮助很小。它从大方向勾勒投资人应买哪种证券,但投资人如何才能找到特定的公司,开启大幅增值之门?

有个方法,马上可以看出它本身合乎逻辑,却欠缺实用性。假设某人具有充分的才能,擅长于各个管理面向,能检视一公司组织中的每个单位,并详细调查高阶主管的素质、它的生产作业、销售组织、研究活动,以及其他每一个重要的职能,形成有价值的结论,晓得这家公司有无很好的成长和发展潜力。

这种方法看起来似乎很有道理。遗憾的是,有几个理由可以说明为什么它对一般投资人通常没有太大的用处。首先,只有少数人具备必要的高阶管理技能,能做这样的事。可是这类人士大多忙于高阶和高薪的管理职务,既没时间也没意愿,以这种方式占用自己的时间和精力。此外,即使他们有意愿,则美国真正有成长实力的公司,到底有多少家愿意让外人获得所有必要的数据,以做成信息充分的决定,值得存疑。以这种方式取得的一些知识,对现有和潜在的竞争对手十分宝贵,不容流落到对数据提供公司不负责任的某人手中。

幸好,投资人可以走另一条路。运用得当的话,这个方法能提供线索,让投资人找到十分出色的投资对象。由于找不到更好的词汇,我姑且称这种做法为「闲聊」法。

以下详细介绍这个方法的过程中,一般投资人会有一个重要的反应。也就是说,不管这种「闲聊」法可能对别人多有用处,对他绝对不管用,因为他根本没有太多的运用机会。我晓得大部分投资人没办法为自己做太多必做的事,好从投资资金中获得最高的报酬率。不过,我还是认为他们应彻底了解需要做什么事,以及为什么要做。只有这么做,他们才能选择专业顾问,提供最好的帮助。只有这么做,他们才能正确地评估顾问所做事情的质量。此外,一旦他们不只了解能做到什么事,也了解如何做到,则投资顾问已经为他们做的一些有价值的事情,有时他们能够锦上添花,获得更多利润,而叫他们惊异不置。

企业界的「耳语网」是件很奇妙的事。熟悉一家公司特定面向的人,你可以从他们具有代表性的意见切面,获知每一家公司在业内的相对强弱势,而且信息之准确,令人咋舌。大部分人,特别是如果他们肯定自己不致祸从口出时,喜欢谈论他们从事的工作领域,并且畅谈竞争对手。你不妨找一个行业的五家公司,问每一家公司一些聪明的问题,如另外四家公司强在哪里,弱在哪里。全部五家公司极其详尽和准确的画面,十之八九可因此获得。

不过竞争对手只是其中一个信息来源,不见得是最好的信息来源。从供货商和客户口中,也能打听到他们来往的对象,到底是什么样的人,而且所获信息之丰富,一样叫人称奇。大学、政府和竞争公司的研究科学家,也能从他们身上获得很有价值的信息。同业公会组织的高阶主管是另一个信息来源。

尤其是同业公会组织的高阶主管,但在相当大的程度内,其他群体也一样,有两件事十分重要。到处打听消息的投资人,必须能够十分确定,他的信息来源绝不会曝光。此后他必须严守这个政策,否则提供信息惹来麻烦的顾忌,将使别人不敢表达不利的意见。

潜在的投资人寻找高获利公司时,还有另一群人能提供很大的帮助。但如投资人不善用判断力,而且没和别人做很多交互查证的工作,以确认自己听到的事的确可靠,则这群人可能害处多于益处。这群人包括以前的员工。这些人对前雇主的强弱势,往往拥有一针见血的观点。同样重要的是,他们通常乐于一谈。但不管对错,这些以前的员工可能觉得他们没来由便遭解雇,或者因为言之成理的不满因素而离开原来的公司,所以务必仔细探讨为什么那些员工离开你所研究的公司。只有这么做,才有可能确定他们内心的偏见有多深,并在听取以前的员工所说的话时,考虑这件事。

研究一家公司时,如果不同的信息来源很多,就没理由相信获得的每一份数据彼此相互吻合。实际上,你根本不必指望会有这样的事。真正出色的公司,绝大多数的信息十分清楚,连经验不多、但晓得自己正寻找什么的投资人也能区辨哪些公司值得进一步调查。下一步是接触该公司的高阶主管,设法填补整个画面仍存在的空白。

\section{买进哪只股票——寻找优良普通股的十五要点}

投资人如想找到一种股票,几年内可能增值几倍,或在更长的期间内涨得更高,则应晓得哪些事情?换句话说,一家公司应具备什么特质,才最有可能为它的股票创造这种成绩?

我相信,投资人应关心十五个要点。一家公司未能完全符合的要点如果很少,则有可能是很好的投资对象。未能符合的要点如果很多,我不认为吻合理想中值得投资的定义。有些要点和公司政策有关;其他一些则和政策的执行效率有关。有些要点涉及的事项,主要应从公司的外部信息来源加以确定,其他一些最好直接询问公司内部人士。这十五要点是:
\\

\textbf{要点一:这家公司的产品或服务有没有充分的市场潜力,至少几年内营业额能够大幅成长?}


一公司的营业额静止不动,甚至每下愈况时,并非不可能获取仅此一次的不错利润。成本控制得当带来营运上的经济效益,有时能够提升纯益,推升公司股票的市场价格上扬。许多投机客和逢低承接者渴望寻求这种仅此一次的利润。但希望从投资资金获取最大利得的投资人,对这种机会的兴趣不大。

另一种情况有时能提供高出许多的利润,但一样引不起后者的兴趣。这种情况发生于环境改变后,短短几年内营业额大幅提高,但之后停止成长。电视机商业化后,许多制造商便出现这种显著的现象。几年内,营业额大幅成长。现在,有电力可用的美国家庭中,约九十%都有电视机,营业额曲线再呈平疲。就这个行业中的许多公司来说,很早就买进股票的人赚到很高的利润。接下来,随着营业曲线止涨回软,许多这类股票的吸引力亦然。

即使最出色的成长型公司,也不能期望每年的营业额都高于前一年。在另一章,我会试着说明,为什么商业研究正常的错综复杂性和新产品营销的问题,往往导致营业额成长趋势出现不规则的忽起忽落现象,而非年复一年平滑顺畅地提高。工商业景气循环反复无常,也严重影响逐年的比较。所以说,不应以年为基础,判断营业额有无成长,而应以好几年为一个单位。有些公司不只未来几年的成长可望高于正常水平,更长的期间内也可望如此。

数十年来始终如一,不断有突出成长率的公司,可以分成两类。由于没有更好的用语,我称其中一类「幸运且能干」,另一类「因为能干所以幸运」。两类公司的管理阶层都必须很能干才行。没有一家公司光因运气不错,而能长期成长。它必须拥有,而且继续拥有杰出的经营才能,否则将无法妥善掌握好运气,并保卫自己的优势竞争地位不受他人侵蚀。

美国铝业公司(Aluminum Company of America)是「幸运且能干」的例子。这家公司的创办人怀有远大的梦想。他们正确地预见到他们的新产品将有重要的商业用途。不过他们和其他任何人,当年都没看到接下来七十年铝制品形成的整个市场规模。该公司是技术发展和经济情势的受益者,而非开创者。这才是它经营成功的主要因素。美国铝业公司拥有且继续展现高超的才能,鼓励和掌握这些趋势。不过即使环境背景,如空中运输臻于完美之境,带来的影响没有完全超乎美国铝业开启广泛新市场的掌控能力,该公司还是会成长──但速度较慢。

美国铝业公司很幸运,发现自己置身的行业,比管理阶层当年构思的富于魅力的行业还好。这家公司许多早期的股东,因为抱牢持股而赚到不少钱,当然人尽皆知。连后来才新加入股东名单的一些人,也赚了不少,只是知道的人没那么多。撰写本书第一版时,美国铝业公司的股价比一九五六年创下的历史性高价低约四十%。不过,就算这个「低」价,仍比十年前,也就是一九四七年能够买到的平均中价高约五百%。

杜邦公司(Du Pont)为另一类成长型股票的例子──这类公司我称之为「因为能干所以幸运」。杜邦公司本来不生产尼龙(nylon)、赛璐仿(cellphane)、人造荧光树脂(Lucite)、氯丁橡胶(neoprene)、奥龙合成纤维(orlon)、米拉(milar),或其他多种令人瞩目的产品。多年来,杜邦生产的是爆破药粉。和平时期,公司的成长主要和采矿业的成长并肩齐步。最近几年,它的成长率可能略高于此,因为铁路建设增加,为它带来额外的营业收入。这家公司优异的商业和财务判断力,加上出色的技术能力,目前每年的营业额超过二十亿美元。以上所说产品本来不可能在这么巨大的业绩中占有显著地位。杜邦运用原来的药粉业务学得的技能和知识,不断推陈出新,产品源源不绝上市成功,成了美国企业伟大的成功故事之一。

投资新手乍看一眼化学业,可能认为那实在是幸运的巧合,因为业务上其他许多层面投资评等通常最高的公司,也在业内生产那么多很吸引人的成长型产品。这样的投资人未免倒果为因,就像没见过世面的年轻女子,第一次欧洲行回来,告诉朋友说,就那么凑巧,大河往往流经那么多大城市的心脏地带。研究杜邦、道氏化学(Dow Chemical)、永备(Union Carbide)等公司的历史,可以很清楚地看出,就营业曲线来说,这些公司属于「因为能干所以幸运」一类。

通用美国运输公司(General America Transportation)可能是「因为能干所以幸运」类公司中最显著的例子之一。五十多年前,这家公司成立时,铁路设备业似乎是成长空间宽广的好行业。但最近几年,很难找到一些行业,持续成长的前景比它差。可是当铁路业的展望改变,货车厢制造业的前景日益转淡之际,过人的创新能力和足智多谋,维持这家公司的收益稳定攀升。管理阶层不以此为满足,开始善用从基本业务学得的一些才能和知识,踏进其他不相关的产品线,提供进一步的成长潜力。

一家公司如果未来几年的营业额可望急剧成长,则不管它比较像是「幸运且能干」的公司,或像「因为能干所以幸运」的公司,都可能给投资人带来财运。不过,从通用美国运输公司等例子,可以清楚地看出一件事。不管是何者,投资人都必须时时留意,观察管理阶层目前以及未来是不是一直很能干;若非如此,营业额将无法继续成长。

对投资人来说,正确分析一家公司长期的营业额曲线,极其重要。肤浅的判断会导向错误的结论。例如,我提过收音机─电视机股价没有持续长期上升,只在美国家庭购买电视机时,营业额突然大幅成长。不过近年来,若干收音机─电视机公司出现了一个新趋势。它们运用自己在电子业的长才,进入其他电子领域建立起庞大的事业,如通讯和自动化设备。这些工业电子产品,以及一些军事电子产品,可望稳定成长很多年。在一些公司,如摩托罗拉(Motorola),这些产品的重要性已超过电视机。同时,若干新技术发展带来新的可能性,到一九六〇年代初,现有的电视机机型将不但难看且落伍,一如原来挂在墙壁、以曲杠操作的手摇式电话,于今已不入流。

一个潜在的发展,也就是彩色电视机,可能为一般大众过度期待而视为已经成真。另一个是晶体管开发和印刷电路带来的直接冲击。(译注:本书一九五八年出版)那将是种屏幕式的电视机,大小和形状与我们现在挂在墙上的大型图画几无两样。目前笨重庞大的外壳将成过去。这些发展如果大量商业化成功,现有电视机生产业者中,一些技术能力最强的公司,营业额可能再次突飞猛进,增幅和维持的时间可能高于它们几年前所经历者。这些公司将发现,除了工业和军事电子产品业务稳定成长外,这方面营业额的冲刺会有锦上添花的效果。它们的营业额将大幅成长,凡是希望获得最高投资利润的人,都应该优先考虑这一点。

提这个例子,不只用以说明肯定将发生的事,更且用以指出哪些事情能够轻而易举发生。这么做,是因为我相信谈到一家公司未来的营业额曲线时,有一点应时时牢记在心。如果一家公司的管理阶层十分出色,而且整个行业将有技术上的变迁,开发研究进步神速,则精明的投资人应提高警觉,留意管理阶层有没有能力妥善处理公司事务,于将来创造理想中的营业额曲线。这是选择出色投资对象应考虑的第一步。

第一版写下这些话以来,就摩托罗拉公司而言,有趣的可能不是什么事情「肯定发生」或「可能发生」,而是已经发生。那时我们还不到一九六〇年代初,也就是最接近我所说,有可能发展出新的电视机机型,淘汰一九五〇年代的旧机型。这事还没发生,近期的未来也不可能发生。但此时让我们看看机敏的管理阶层做了哪些事,掌握技术上的变迁,创造出往上攀升的营业额曲线;前面我说过,营业额曲线上升,是出色投资的先决条件。

摩托罗拉公司让自己成为双向电子通讯领域十分杰出的领导者,而且现在似有几无止尽的成长率。双向电子通讯器材起初是做为警车和出租汽车的专用品,后来货车运输公司、各种送货车队的业主、公用事业公司、大型营造计划和管线也竞相采用这种多用途的设备。在此同时,经过几年所费不赀的开发努力,这家公司建立起获有利润的半导体(晶体管)事业部,似乎将在这个行业急剧成长的趋势中取得一席之地。它在立体音响唱机的新领域中成为要角,而且这个新销售收入来源的重要性和金额日增。但和全国首屈一指的家具制造商(醉客舍〔Drexel〕)形成相当独特的风格结盟,高价位电视机的销售额突飞猛进。最后,它以小钱并购另一家公司,踏进助听器材领域,而且可能开发其他新型的专用器材。简言之,下个年代某个时候,重大的刺激因素可能促使它原来的收音机─电视机产品线再次大幅成长。这事还没发生,短期内也不可能发生。可是管理阶层已再度掌握利用组织内部的资源和才能,公司的成长蓄势待发。股票市场对这件事有所反应吗?初版写完时,摩托罗拉的股价是四五.五美元,今天是一百二十二美元。

投资人注意到这种机会时,利润可能有多少?我们从刚谈过的这个行业举实例来说明。一九四七年,华尔街一位朋友正调查萌芽中的电视机工业。他研究了一年中大部分时候十来家主要电视机制造商的情形,结论是这个行业的竞争将很激烈,主要公司的地位将大幅变动,而且这个行业中一些股票具有投机性魅力。不过,调查过程中,映像管使用的玻璃真空管严重缺货。经营最成功的制造商似乎是康宁玻璃厂(Corning Glass Works)。进一步探讨康宁玻璃厂的技术面和研究面之后,可以清楚地看出,这家公司非常有资格,为电视机工业生产玻璃真空管。估计可能的市场规模后发现,这将是康宁公司主要的新业务来源。由于其他产品线的展望普遍不错,这位分析师建议散户和机构投资人买进这支股票。这支股票那时的价格约二十美元,后来一股分割成两股半。在他买进之后十年,股价上涨到一百美元以上,等于老股二百五十美元以上。
\\

\textbf{要点二:管理阶层是不是决心继续开发产品或制程,在目前富有吸引力的产品线成长潜力利用殆尽之际,进一步提高总销售潜力?}


有些公司因为目前的产品线有新需求,未来几年的成长展望很好,但依公司的政策和经营计划,产品线不再进一步开发,则优渥的利润可能昙花一现。它们不可能在十年或廿五年内,源源不断带进利润,而这正是公司财务成功最稳当之路。到了这个时点,科学研究和发展工程开始进入整张画面。企业界主要必须靠这些方法,才能改善旧产品和开发新产品。管理阶层如不满意成长昙花一现,而希望成长不绝如缕时,通常会这么做。

企业界的工程或研究努力,如能在相当大的程度内,投入和公司目前营运范畴有若干关系的产品,则投资人的收获通常最大。这个意思不是说,理想的公司可能没有很多事业部门,而且产品线相当不同。它的意思是,一公司的研究如能围绕每一个事业部,如很多树木各从自己的树干长出树枝,成果通常比一家公司从事许多不相干的新产品好得多;后者的新产品研制成功后,公司势将踏入与现有事业无关的几个新行业。

乍看之下,要点二似乎只是要点一的重复。其实不然。要点一讲的是事实,用以评估一公司的产品目前存在的销售成长潜力。要点二谈的是管理阶层的态度。这家公司是否体认到,总有一天,公司几乎肯定会成长到目前市场的潜力极限,如要继续成长,未来某个时候可能必须另行开发新市场?一家公司必须在要点一有好评等,同时在要点二有正面积极的态度,才有可能吸引投资人最大的兴趣。
\\

\textbf{要点三:和公司的规模相比,这家公司的研究发展努力,有多大的效果?}


对很多股票公开上市公司来说,取得数字,了解每年花在研究发展支出上的金额有多少,不是很困难。这些公司几乎都会报告每年的营业总额,只要运用最简单的算术,把研究金额除以总销售额,就能晓得一公司的营业额中有多少百分率用于研究发展。许多专业投资分析师喜欢比较一公司和同业的研究支出。有时他们会拿这个数字和业界平均值比较,方法是把许多同类公司的数字加总起来,取其平均值。从这个数字,就能得出结论,晓得一公司的研究努力相对于竞争对手的多寡,以及投资人买一家公司的股票时,每股研究支出金额是多少。

这类数字可以当做粗略的量尺,找到有用的线索,晓得一家公司的研究支出是不是高得异常,或者另一家公司的研究支出不够。但除非进一步取得很多数据,否则这种数字容易产生误导。其中一个理由是,哪些项目应列为研究发展费用,哪些项目则排除在外,各公司的做法差距很大。一公司可能把某种工程费用列为研究发展支出,大部份主管官员却期期以为不可,不觉得那真能算是研究费用,因为该公司不过把现有产品略加修改,以因应某特定订单的要求──换句话说,那只是销售工程费用。相反的,另一家公司可能把某种全新产品试作工厂的运转费用当做生产成本,而非研究支出。大部分专家会称此为纯正的研究功能,因为它和新产品生产知识的取得有直接关系。如果所有的公司都以类似的会计基础,报告研究发展支出,则各知名公司相对的研究发展支出数字,看起来会和金融圈常用的数字很不一样。

企业各项主要营运活动中,以研究发展领域的成本效益差距最大。即使管理最优良的公司,彼此的差异,比率可高达二比一。这个意思是说,有些经营良好的公司,花在研究上的每一块钱,最终获得的效益,是其他公司的两倍之多。把经营普通的公司纳入,则最佳和普通公司间的差异更大。个中原因主要在于新产品和制程能够突飞猛进,不能再只靠一位天才,而必须结合受过高度训练的工作团队,而且人人各有所长。其中一人可能是化学家,另一人是固态物理学家,第三人是冶金学家,第四个人是数学家。每一位专家的技能,只是产生优异结果的一部分。这里也需要领导人,协调背景那么不同的许多人群策群力,为共同的目标努力。因此,某公司研究人员的数目或声望,和另一家公司以工作团队方式运作而获得的效果比起来,可能相形失色。

协调技术能力各有所长的研究人员,结合成紧密的工作团队,并激励工作团队中的每位专家发挥最大的生产力,以取得最佳的研究成果,这事做起来十分复杂,但不是管理阶层唯一需要的能力。每个开发项目,在研究人员和十分熟悉生产、销售问题的人员间,做密切和详尽的协调,也一样重要。对管理阶层来说,如何让研究、生产和销售人员建立起紧密的关系,不是容易的事。若非如此,最后构思出来的新产品往往不是无法低价生产,便是在设计时欠缺最迷人的销售魅力。如此研究开发出来的产品,通常禁不起更有效率的竞争对手一击。

研究支出要获得最大的成效,最后需要另一种协调。这是和高阶管理人员的协调。或许这么说比较好:高阶管理人员应了解商业研究的基本特质。景气好的年头,开发计划没办法扩张,景气差的年头,则遭大幅删减,因为公司不肯急剧提高总成本,以达成应有的目标。一些高阶管理人员喜爱的「紧急」(crash)计划,偶尔可能有其必要,但往往过于昂贵。紧急计划所以发生,在于研究人员一直着手的项目突然叫停,转而集中心力在新的任务上。就那时候来说,新任务可能比较重要,但往往不值得因为它们,使原来的项目受到干扰而中断。成功的商业研究的精髓,是只选报酬金额可望达研究成本好几倍的任务。不过,一旦某个项目开始着手,基于预算上的考虑和项目本身以外的其他因素,而加以缩减或加速,难免导致总成本相对于能够获得的利益上升。

有些高阶管理人员似乎不了解这一点。我曾见过成功的小型电子公司高阶主管对业内一家巨擘的竞争不以为意。这种态度叫人惊讶。他们不担心规模大得多的公司有能力生产竞争性产品,并非起于他们不尊敬大公司个别研究人员的能力,或者不晓得大公司不惜耗费巨资研究发展可能获得什么成果。相反的,他们知道大公司一向经常喊停,搁置正常的研究项目,插入紧急计划,以完成高阶管理人员因急迫感而订定的立即性目标。同样的,几年前,我听说一位优秀的技术性同事私底下劝告毕业班学生不要到某家石油公司找工作,原因不言可喻,但不希望有人到处张扬此事。这是因为那家公司的高阶管理人员,喜欢雇用技术熟练的人,做正常情况下需要五年才能完成的项目。而后到了约三年,公司对某特定项目失去兴趣,将之束诸高阁,结果不但浪费公司的金钱,也害员工一事无成,在技术成就上难着有声誉。

国防合约的庞大研究开销,要如何评估,使得研究开发的投资评估更显复杂。很多这方面的研究支出,往往不是执行研究工作的公司费用,而是挂在联邦政府帐上。有些国防业的转包商也为承包商做很多研究。这些转包商供应产品给承包商。投资人应视这方面所有的研究支出,和公司自行支出做研究一样重要?如果不然,如何和公司本身的研究相互比较?和投资领域中其他许多层面一样,这些问题没办法用数学公式回答。每个个案都不一样。

国防合约的利润率低于政府部门以外的商业活动,而且往往有这样的特性:某种新武器的合约必须根据政府的蓝图竞标。这表示,政府主办的研究工作,发展出来的产品,有时不可能带来稳定、重复性的业务,但民间的研究做得到这一点,因为专利和顾客的口碑通常能源源不绝带进收入。基于这些理由,从投资人的观点来说,政府主办的各种研究项目,经济价值差别很大,虽然就国防努力的效益而言,这些项目的重要性大致相当。从下面所举理论上的例子,或许可以看出为什么在投资人眼中,三个项目的价值大异其趣:

有个项目可能研制出重要的新武器,但不具军事以外的用途。这种武器的权利全为政府拥有,而且一旦研制出来,生产过程十分简单,原先做研究的公司投标承揽生产合约时,相对于其他公司,不具竞争优势。在投资人眼里,这种研究努力几乎没有任何价值。

另一个项目可以生产相同的武器,但制造技术相当复杂,没有参与原始开发工作的公司,将很难生产。对投资人来说,这种研究项目具有中等价值,因为能从政府部门持续不断取得业务,只是利润可能不高。

另一家公司可能执行这种武器的工程研发工作,并从中学得一些原理和新技术,可直接应用在利润较高的经常性商业产品在线。投资人可能认为这种研究项目有很高的价值。近年来,一些公司展现高度的才华,找到复杂和具技术性的国防工作,经营十分成功。政府花钱,它们却从研究中取得技术知识,而且能够合法运用到利润较高、与现有商业活动有关的非国防领域上。这些公司把国防单位亟需的研究成果呈交政府,但同时以很低的成本,或者根本不需要成本,取得相关的非国防研究利益。它们本来必须自己花钱才能取得这些利益。投资德州仪器公司(Texas Instruments, Inc)股票赚大钱的一个原因,很可能和这个因素有关。一九五三年,德州仪器公司的股票在纽约证券交易所首次挂牌交易,价格是五.二五美元,四年内涨了约五百%;同一年,安培斯公司(Ampex)的股票也首次公开发行,持股人同期内赚了约七百%,涨幅高于德州仪器,也可能和这个因素有关。

最后,判断企业的研究组织相对的投资价值时,还有另一种活动必须评估。正常情况下,这种活动根本不被视为开发研究──似乎和开发研究沾不上边的市场调查。市场调查被视为开发研究和销售的桥梁。高阶管理人员必须提高警觉,慎防花大钱,研究发展炫丽的产品或制程,一旦臻于完美境地,的确有市场存在,可惜市场规模太小,难有利润。所谓市场规模太小而无利润,我的意思是说,这样的市场中,销售额不够大,无法回收研究成本,投资人去赚那种蝇头小利划不来。市场调查研究组织如能把公司的重大研究项目,从技术上即使成功,也难以回收成本,转为迎合更广大的市场,获得三倍大的报酬,则可大幅提高持股人投资该公司科技人力的价值。

如果计量量数──如每年的研究支出或拥有工科学位的员工人数──只是粗略的指南,而非判断一公司是否为优秀研究组织的最后依据,则谨慎的投资人如何取得这方面的信息?同样的,「闲聊」法能够发挥神奇的作用。一般投资人除非去尝试,否则不相信提出一些聪明的问题,到处询问研究人员,包括公司内部人士,以及同业、大学、政府相关领域的人士,谈某公司的研究活动,能拼凑出一幅完整的画面。一个比较简单,但往往有效的方法,是仔细探讨一段期间内,例如过去十年,研究单位的成果对一公司的营业额或净利有多大的贡献。这样一段期间内,和活动规模相比,研究单位如能源源不断推出高利润的新产品,则只要根据同样的一般性方法继续运作,将来可能仍有等量齐观的生产力。
\\

\textbf{要点四:这家公司有没有高人一等的销售组织?}


在这个竞争激烈的年代,即使公司的产品或服务十分出色,但如果不善于营销,销路终有极限。没有销路,企业不可能生存。顾客因为满意而为公司带来重复性的营业收入,是经营成功的第一个判断准绳。不过,企业的销售、广告和配销组织的相对效率,大部分投资人对它们的重视程度,远不如对生产、研究、财务或企业活动其他主要部门的注意,连小心谨慎的投资人也不例外。

这种现象的存在,可能有个原因。比较一公司的生产成本、研究活动或财务结构与竞争对手的优劣时,我们很容易建构简单的数学比率,藉以提供某种指引。但是谈到销售和配销的效率,即使意义雷同,计算比率却困难得多。至于研究,我们已知道,这种简单的比率太过粗浅,只能做为第一个线索,告诉我们应该观察什么。不久我们就会讨论它们相对于生产和财务结构的价值。但是不管这种比率是否真如金融圈所认为的那么有价值,投资人的确喜欢利用这些比率。由于销售努力没有那么容易公式化,许多投资人并未加以正视,可是决定投资是否真有价值时,它具有基本上的重要性。

同样的,我们可以利用「闲聊」法解决这个困难。一公司营运活动的所有面向中,从公司外部打听销售组织的相对效率,最容易做到。竞争对手和顾客知道答案。同样重要的是,他们很少不敢表达自己的看法。小心谨慎的投资人花时间探讨这个问题,通常可以满载而归。

关于相对销售能力,我给的篇幅少于相对研究能力。这不表示我觉得它比较不重要。以今天激烈竞争的世界来说,很多重要的事情攸关企业经营成功。但是优异的生产、销售、研究或可视为成功的三大支柱。说其中之一比另一重要,就像说心、肺、消化道里面的一个是维持人体正常运转最重要的器官。人要生存,所有的器官都不可或缺,而且所有的器官都必须健康,才会身强体壮。不信,不妨看看你身边已证明是杰出投资对象的公司。你能找到有些公司不是积极的配销努力和不断改善的销售组织两者兼具吗?

我已提过道氏化学公司,而且可能一提再提,因为这家公司多年来给予股东很高的报酬,我相信它是理想的保守型长线投资对象。在大众心目中,这家公司和研究成果突出几乎划上等号。但我们不知道的是,这家公司甄选、训练销售人员时,和甄选、训练研究化学家一样小心翼翼。年轻的大学毕业生成为道氏的业务员之前,可能受邀到密德兰(Midland)数趟,好让他和公司双方尽可能确定他拥有的背景和个性,适合待在公司的销售组织。接下来,在他拜访第一位潜在顾客之前,必须接受专业训练,为期短则几个星期,长则持续年余,以便做好准备,面对更为复杂的销售工作。这只是他接受训练的开端;公司投注很大的心力,不断寻找更有效率的方式,争取、服务以及交货给顾客。

道氏和化学业其他杰出公司是否与众不同,那么重视销售和配销?当然不是。在另一个相当不同的行业中,国际商业机器公司(Internation Business Machines;IBM)给予股票持有人优渥的报酬(讲得保守一点)。IBM一位高阶主管最近告诉我,一般业务员全部的受训时间,三分之一待在公司赞助的学校中!比率如此之高,主要原因在于公司希望业务员随时了解一日千里的科技最新动态。我相信,这是另一个证据,显示经营最成功的公司,十分重视必须不断改善业务人员的素质。一公司的制造或研究技能强,能够取得若干赚钱的业务,但配销组织能力弱,则高利润将如昙花一现。这样的公司相当脆弱。一公司如要长期稳定成长,强大的销售人力不可或缺。
\\

\textbf{要点五:这家公司的利润率高不高?}


我们终于谈到一件重要的东西,本身是种数学分析,许多金融圈人士认为它是良好投资决策的骨干。从投资人的观点来说,营业收入导致利润增加才有价值。如果多年来利润一直不见相对增加,则营业额再怎么成长,也无法创造合适的投资对象。检视一公司利润的第一步是探讨它的利润率(profit margin),也就是说,算出每一元的营业额获有多少分的营业利润(operating profit)。数字算出来,马上可以看出不同的公司差别很大,即使同一行业的公司也不例外。投资人不应只探讨一年的利润率,而应探讨好几年的利润率。整个行业欣欣向荣之际,几乎所有的公司都有高利润率──以及高利润金额。不过,我们也能明显看出,景气好的年头,边际公司──也就是利润率较低的公司──利润率成长的幅度几乎总是远高于成本较低的公司;后者的利润率也提高,只是提高的幅度没那么大。因此,景气非常好的年头中,体质疲弱公司的盈余成长率往往高于同行中体质强健的公司。但是我们也应记住,一旦景气转差,前者的盈余也会下降得比后者快。

由于这个理由,我相信投资边际公司,绝对无法获得最高的长期利润。一家公司的利润率极低,但考虑长期投资的唯一理由,在于可能有强烈的迹象,显示该公司正从根本发生变化。例如,利润率正在改善,但和业务量暂时扩增无关。换句话说,平心而论,这家公司不能算是边际公司,因为购买股票的真正原因,是该公司经营效率高,或开发出新产品,已使它脱离边际公司之林。一家公司如出现这样的内部变化,而且其他方面也值得长期投资,则可能是非常理想的购买对象。

至于历史较悠久和规模较大的公司,真正能够让你投资赚大钱的公司,大部分都有相对偏高的利润率。通常它们在业内有最高的利润率。至于年轻的公司,有时一些老公司也一样,有一件重要的事偏离这个准则──不过一般来说,那只是表面上偏离,实质上没有偏离。这种公司有时刻意动用所有的利润,或者一大部分利润,以加速成长。这些本来可以放进口袋的利润,用于进一步加强研究或促销。这种情况中,重要的是百分之百确定它真的花钱促进研究、推动促销,或者加强其他任何活动,好为将来打下基础。这是利润率缩水或不存在的真正理由。

投资人最要注意的地方,是确定导致利润率下降的活动量,不只是为取得高成长率所需的活动量,实际上还要做更多的研究、促销等。果真如此,则利润率明显欠佳的公司,反而可能是绝佳的投资对象。但除了刻意拉低利润率,以进一步加速成长率的这些公司,希望长期大赚的投资人,最好远离利润率低的公司或边际公司。
\\

\textbf{要点六:这家公司做了什么事,以维持或改善利润率?}


购买股票要赚钱,不是看购买当时这家公司有哪些事情普遍为人所知。相反的,能不能赚钱,要看买进股票之后需要知道的事情。因此,对投资人来说,重要的不是过去的利润率,而是将来的利润率。

我们生存的这个年代中,利润率似乎不断受到威胁。工资和薪水成本年年上涨。许多公司现在订有长期劳动合约,未来几年的薪资涨幅都已确定。劳动成本上扬,导致原物料和进货价格对应上涨。税率趋势,特别是不动产和地方税率,也似乎稳定攀升。

在这种背景下,各公司的利润率趋势将有不同的结果。有些公司似乎站到幸运的位置,只要提高价格,就能维持利润率。它们所处的行业,产品需求通常很强,或者因为竞争性产品的售价涨幅高于它们的产品。不过在我们的经济中,以这种方式维持或改善利润率,通常只能保持相当短暂的时间。这是因为额外的竞争性产能会创造出来。这些新产能足以抵消增加的利益,随着时间的流逝,成本增幅不再能够转嫁到价格涨幅上。接着利润率开始下滑。

一九五六年秋便有一个急转弯的显著例子。当时几个星期内铝市场从供不应求,转为厂商竞相抛售。在那之前,铝价随着成本而上扬。除非产品的需求成长得比产能快,否则价格涨幅就不会再那么快速。同样的,若干大钢厂一直不愿提高一些稀有钢品的价格到「市场能够忍受的极限」,部分原因反映了它们长期以来的想法,也就是因为有能力把成本的涨幅转嫁到价格的涨幅上,而致利润率上升的现象,都很短暂,除非有其他的理由。

同样在一九五六年下半年,大炼铜厂采取的措施或许最能说明这种做法的长期危险性。这些公司十分自制,甚至于将价格压低在全球性的水平以下,以防价格涨得太高。不过,铜价还是涨到够高的水平,仰制了需求,并吸引新的供应产能加入。苏伊士运河关闭使得西欧的消费益形不振,供需情势失衡相当严重。要是一九五六年的利润率没有那么好,或许一九五七年的利润率就不会那么糟糕。整个行业的利润率因为价格一再上涨而升高时,对长线投资人来说,不是好兆头。

相反的,其他一些公司,包括这些行业中的若干公司,不是靠提高价格,而是藉远富创意的方法,提升了利润率。有些公司因为维持资本改善或产品工程部门,做得很成功。这些部门唯一的职能,是设计新的设备,以降低成本,抵消或部分抵消工资日渐上升的趋势。很多公司不断检讨作业程序和方法,研究哪些地方可以提高经济效益。就这种活动而言,会计职能和纪录的处理,一直是特别有收获的地方。运输方面也是一样。运输成本的涨幅高于大部分的费用,因为和大多数的制造活动比起来,大部分运输形式中,劳工成本所占比率较高。惊觉心高的公司使用新型货柜、采用以前没用过的运输方法,甚至把货品置于分厂,以免交互送货,因而降低了成本。

这些事情无法一日之间完成。它们都需要仔细研究和事前详加规划。想要投资的人应注意企业所采降低成本和提升利润率的新观念是否富有创意。在这里,「闲聊」法可能有若干价值,但远不如直接询问公司内部人士。幸好,大部分高阶主管都乐于详谈这方面的事情。这方面做得最成功的公司,很有可能是以同样的知识建立起公司,将来可望继续以建设性的态度做事。他们极有可能为股东创造最高的长期报酬。
\\

\textbf{要点七:这家公司的劳资和人事关系是不是很好?}


大部分投资人可能没有充分体认良好的劳资关系能带来利润。很少人看不出恶劣的劳资关系造成的冲击。任何人只要稍微浏览财务报表,经常性、久悬不决的罢工对生产造成的影响,便跃然纸上。

不过,人事关系良好和人事关系乏善可陈的两种公司间,获利力差异的程度,远大于罢工的直接成本。如果员工觉得受到雇主公平对待,整个工作环境便大不相同,高效率的领导阶层可以大幅提高单位员工的生产力。此外,训练每位新进员工需要相当高的成本。因此,员工流动率过高的公司,必须负担这方面不必要的成本,而管理良好的企业不必为这种事烦恼。

但是投资人如何判断一公司劳资关系和人事关系的良窳?这个问题没有简单的答案,找不到一套通则适用于所有的情况。我们能做的事,是观察很多因素,然后从拼凑起来的综合画面加以判断。

目前工会势力普遍存在,公司内部尚无工会组织者,劳资和人事关系可能优于一般水平。若非如此,则工会很早以前便组织起来。例如,投资人可以相当肯定,在工会势力庞大的芝加哥,摩托罗拉公司(Motorola)至少说服很多员工,相信公司真的有意愿和能力,善待员工。在工会势力逐渐抬头的达拉斯,德州仪器公司也是如此。公司员工没和国际性工会挂钩,唯一的理由是公司的人事政策执行得很成功。

相反的,企业内部有工会组织,无疑是劳资关系不睦的征兆。有些公司的员工全部参加工会,但劳资关系很好,因为它们晓得必须和工会互敬互信。同样的,罢工接连不断和拖延不决,正是劳资关系恶劣的明证,但完全没有罢工,不见得表示劳资关系本质上良好。有些时候,没有发生罢工的公司,很像惧内的老公。没有冲突,婚姻生活不见得幸福美满,因为当事人只是害怕冲突。

为什么有些员工对某些雇主异常忠诚,对其他一些雇主则痛恨有加?个中理由往往一言难尽,不容易厘清头绪,投资人最好观察显示员工整体感觉的比较性资料,不必注意导致他们出现某种感觉的每一种背景因素。不少数字可以显示基本员工素质和人事政策的好坏,其中之一是一家公司相对于同一地区另一家公司的员工流动率高低。同样重要的是应征某家公司工作的人数相对于同一地区其他公司应征人数的多寡。在劳工未见供过于求的地区,如有很多人希望到某家公司工作,则从劳资和人事关系良好的角度来说,这样的公司通常值得投资。

不过,除了这些一般性的数字,投资人还可以注意一些明确的细节。劳资关系良好的公司,通常尽一切努力尽速化解员工的不满。管理阶层拖很久才处理员工的小抱怨,而且不认为那有什么重要,则星星之火恐怕终有燎原之虞。除了评估解决怨诉的方法,投资人可能也需要密注意工资级距。在公司所在地支付的工资高于平均水平,但盈余也高于平均水平的公司,劳资关系可能不错。如果公司的盈余有很大一部分源于支付低于所在地标准水平的工资,投资人如买它的股票,迟早可能尝到严重的苦果。

最后,投资人应了解高阶管理人员对待基层员工的态度。有些管理人员嘴里讲得天花乱坠,实际上不认为对普通员工必须负起责任,也不关心他们。他们最关心的是,营业收入流入低阶员工的比率,不能高于强悍的工会施压强求的水平。他们根据公司营业收入或盈余展望的略微变化,随意大量雇用和解雇员工。员工眷属可能受影响,生活困难,但他们不觉得自己有责任。他们没做什么事,让一般员工觉得公司需要他们,才能经营下去。他们没做什么事,让一般员工觉得有尊严。管理阶层抱持这种态度的公司,通常不是最理想的投资对象。
\\

\textbf{要点八:这家公司的高阶主管关系很好吗?}


如果和低阶员工关系良好很重要,则在高阶人员之间创造正确的气氛也十分要紧。这些人的判断、创造力和群策群力,能够成事,也能败事。由于他们举足轻重,工作压力往往很大。所以有些时候,高阶主管人才因为摩擦或彼此怀恨,而挂冠离去,或者并未卯足全力做事。

高阶主管气氛良好的公司,能提供最佳的投资机会。这样的公司中,高阶主管对总裁和董事长有信心。这表示,从最低阶层往上,每位员工都感受到,公司的升迁是以能力为依归,不能靠结党成群。拥有控制权的家族成员,不会升迁到更有能力的人才头上。公司会定期检讨调整薪水,高阶主管因此主动更加努力。薪水至少和业界、当地的标准看齐。除了最基层的工作,只有在组织内部找不到适合升迁的人才时,管理阶层才会引用外人。高阶管理人员了解,只要人们在一起工作,难免结党成群和发生人际间的摩擦,但不能忍受有人不肯在团队中携手合作,好把这种磨擦和结党成群的现象降到最低。只要和公司不同责任阶层的高阶主管稍微闲聊,直接问几个问题,投资人通常就会知道高阶主管间的气氛是不是融洽。企业偏离这些标准愈远,愈不可能成为绝佳的投资对象。
\\

\textbf{要点九:公司管理阶层的深度够吗?}


小公司可以做得非常好,而且如果其他因素都合适,则在真正能干的一人管理领导之下,多年内这家公司有可能是很好的投资对象。不过,人的能力毕竟有其极限,即使是规模较小公司的投资人,也应该预防关键人物不在其位可能带来灾难。现今杰出小公司的投资风险没有表面上看起来那么大,因为最近有个趋势,也就是拥有许多管理人才的大公司常会买下规模较小的公司。

但是值得投资的公司,必须能够继续成长。一家公司迟早会到达某种规模,除非开始在某种深度内培养高阶主管人才,否则将没有能力掌握进一步的机会。这一点,因不同的公司而异,视它们从事何种行业以及一人管理公司的才干而定。这事通常发生在每年总营业额升抵一千五百万到四千万美元之际。如要点八所述,这时投资某家公司的股票时,高阶主管间的气氛良好格外重要。

当然了,要达成要点八所谈的事情,必须深入培养合适的管理阶层才行。但是除非另外实施若干政策,否则没办法培养出这样的管理人才。其中最重要的是授权。如果从最高阶层到基层,每个层级的主管没有以别出心裁和有效率的方式,配合个人的能力,获得实权,以执行指派的任务,优秀的主管人才便有如身强体壮的动物被关在牢笼里,无法舒活筋骨,尽情挥洒。他们没办法发挥长才,因为根本没有充分的机会去运用。

高阶管理人员如事必躬亲,插手日常的营运事务,这样的组织很难成为极富吸引力的投资对象。高阶主管虽然本意良善,但跨越自己授予部属的权限,将使他们经营的公司严重偏离优良投资对象之林。不管一两个主管处理所有琐碎事务多能干,一旦公司到达某种规模,这样的高阶主管会在两方面碰到难题。公司一大,太多的琐碎事务将令他们分身乏术,无法一一处理。公司也没办法培养干才,处理仍在成长中的业务。

判断一公司的管理深度是否合适时,还有另一件事值得投资人注意。高阶管理人员是否虚心欢迎并乐于评估员工提出的建议,即使这些建议有时严厉批判目前的管理实务?今天的企业环境竞争如此激烈,改善和变革的需求如此强烈,要是高阶管理人员因为骄矜自满或无动于衷,未能探索值得开采的新点子金矿,这样的公司可能不适合投资人垂青。公司亟需的年轻主管也不可能培养出来。
\\

\textbf{要点十:这家公司的成本分析和会计纪录做得多好?}


如果不能够准确和详尽地细分总成本,显示每一小步营运活动的成本,没有一家公司有办法长期经营得十分成功。只有这么做,管理阶层才晓得什么事最需要注意。只有这么做,管理阶层才能判断他们有没有适当地解决需要注意的每一个问题。此外,最成功的公司不只生产一种产品,而是生产很多产品。如果管理阶层无法确切知道每种产品相对于其他产品的真正成本,势必束手无策。他们几乎不可能订定价格政策,确保获得最高的总利润,同时制止过度的竞争。他们将无从得知哪种产品值得特别着力推广和促销。最糟的是,有些表面上成功的活动,其实可能正在赔钱,使得整体利润每下愈况,而非节节上升,但管理阶层不知道这件事。这种情况下,几乎不可能做出聪明的规划。

虽然投资的时候,企业的会计控制十分重要,但小心谨慎的投资人通常很少看清他想投资的公司全貌,晓得成本会计和相关活动的真实面貌。在这一方面,「闲聊」法有时能指出做事掉以轻心的公司。除此之外,能告诉我们的东西不多。直接询问公司里面的人,对方通常回答得十分真诚,相信成本数据非常适当。他们往往提出详细的成本数据表,用以证明他们所说不假。但是重要的不是详细的数字,而是它们之间的相对准确性。就此而言,小心谨慎的投资人通常最好同时承认这个课题很重要,以及他本身能力有限,没办法给予适当的评估。在这些限制之下,他通常只能依赖一般性的结论,也就是如果一公司经营能力的大部分层面远高于一般水平,这方面的表现也可能远高于一般水平,但前提是高阶管理人员体认到专业会计控制和成本分析很重要。
\\

\textbf{要点十一:是不是有其他的经营层面,尤其是本行业较为独特的地方,投资人能够得到重要的线索,晓得一公司相对于竞争同业,可能多突出?}


依定义,这个问题有不分青红皂白的味道。这种事情势必每家公司差别很大──在某些行业显得十分重要的东西,在其他行业,则不怎么要紧,或者根本不值一提。例如,零售业最重要的工作,也就是公司处理不动产事务的能力──如承租物的品质──至关紧要。但在其他很多行业,拥有这方面高超的能力,则没那么重要。同样的,对某些公司来说,处理信用的能力很重要,但其他公司没那么重要,或者不必理会。这两件事,我们的老朋友,亦即「闲聊」法,通常能让投资人把画面看得更清楚。如果探讨的问题很重要,值得深入研究,则他获得的结论,往往能用数学比率加以验证,如单位销售额的相对承租成本,或信用损失比率。

很多行业中,总保险成本相对于销售额的比率很重要。有些时候,一家公司的总保险成本比同等规模的竞争对手低三十五%,利润率将高出不少。有些行业中,保险是相当大的因素,足以影响盈余,研究这些比率,并和知识丰富的保险业人士讨论,投资人将受益良多。要晓得某家公司的管理阶层表现多出色,这些数据虽属辅助性质,却能透露很多内情。单单因为比较擅长于处理保险事务,不能降低保险成本,这和擅长于处理不动产事务而降低平均租金不同。相反的,它们主要反映了处理人事、存货和固定财产的整体能力,因而减少发生意外、毁损和浪费的整体数量,从而能够降低保险成本。从保险成本的高低,可以明显看出某个行业中哪家公司经营得不错。

专利权也因不同的公司而有很大的差异。对大公司而言,专利权多通常有额外的好处,但不表示它有基本上的强势。专利权多,通常可以防止公司若干部门的营运活动遭遇激烈的竞争。正常情况下,公司相关部门的产品线将因此享有较高的利润率。这又进一步提升所有产品线的平均利润率。同样的,专利权强大,有时能让一公司享有独家权利,以最简单或最便宜的方式,生产某种产品。竞争对手必须走更远的路,才能到达相同的地步,使得专利权拥有人占有明显的竞争优势,但这种优势通常不大。

在专业技术知识普及的这个年代,大公司受专利权保护的领域,绝大部分的情况下,只及于公司一小部分的营运活动。专利权通常只能阻止少数竞争对手获得同样的成果,但无法阻止所有的竞争对手。基于这个原因,许多大公司根本不想透过专利权结构将竞争对手关在门外,反而收取相当低廉的费用,授权竞争对手使用它们的专利,并希望别人也以同样的态度对待它们,允许它们使用别人的专利。在制造技术、销售和服务组织、顾客的口碑,以及对顾客问题的了解等方面,要维持竞争优势,需要着力的地方,远多于专利权的保护。其实,大公司维持利润率的主要手段如果是靠专利保护,通常是投资弱势而非强势的表征。专利权没办法无限期提供保护。专利保护不再存在时,公司的获利可能大打折扣。

年轻的公司刚开始建立生产、销售和服务组织,而且处于拓展客户口碑的初期阶段,情况大不相同。要是没有专利,它的产品可能遭根基稳固的大公司抄袭;大公司可能运用它们既有的客户关系通路,置年轻的小型竞争对手于死地。因此对于刚在营销独特产品或服务的小型公司,投资人应密切检视它们的专利状况。专利权保护的范围到底有多广,应从足资信赖的来源取得信息。获得某种产品的专利是一回事,得到保障,阻止他人以略微不同的方式生产又是另一回事。不过就这一方面而言,从工程研究下手,不断改善产品,远比静态的专利权保护占优势。

比方说,几年前,西岸一家年轻电子制造商和今天比起来,规模小得多时,推出一种很成功的新产品。有人向我说,业内一家大公司却以「依样画葫芦的抄袭手法」,用自己的知名品牌营销。依这家年轻公司设计师的看法,那家大型竞争对手矫正了小公司工程设计上的所有错误,再结合原有产品的优点,推出自己的产品。小型制造商消除原有产品的缺点,推出改良型产品之际,大公司正好也推出产品。大公司的产品卖得不好,于是从那个领域撤退。我们见过无数这样的例子,也就是最有效的根本保障方式,来自工程设计保持领先地位,不是靠专利权。投资人至少应十分小心谨慎,不要太强调专利权保护的重要性,但也要晓得,评估某项投资是否理想时,专利权保护偶尔是个重要因素。
\\

\textbf{要点十二:这家公司有没有短期或长期的盈余展望?}


有些公司的经营方式是追求眼前最大的利润,有些则刻意仰制近利,以建立良好的口碑,因而获得较高的长期整体利润。这方面常见的例子,是对待客户和供货商的态度。一家公司可能老是以最严苛的态度对待供货商,另一家则可能在供货商为确保可靠的原物料来源,或在市况转变、供给十分紧俏时,为确保获得高质量的零组件,以致交货意外多出费用,而乐于支付比合约高的价格。对待客户的差异也同样显著。有些公司愿意在老客户出乎意料碰到困难时,不厌其烦和多花钱照顾它们的需求,因此在某些交易上获得较低的利润,但长期可望得到远高于以往的利润。

「闲聊」法通常可以相当清楚地反映这些政策上的差异。想要获得最高利润的投资人,应留意在盈余上眼光放远的公司。
\\

\textbf{要点十三:在可预见的将来,这家公司是否会因为成长而必须发行股票,以取得足够的资金,使得发行在外股数增加,现有持股人的利益将因预期中的成长而大幅受损?}


一般谈投资的书,都花很大的篇幅,探讨公司的现金存量、企业组织架构、发行各种证券所占资本比率等,所以读者很可能会问,为什么谈财务面的这个要点,在十五个要点中,所占篇幅不多于十五分之一?个中理由在于本书的基本信念是,聪明的投资人不应光因价格便宜就买普通股,而必须在有大赚的可能性时才买。

本章所谈其他十四个要点,很少公司能在全部十四个要点,或几乎全部的要点上,获得很高的评价。符合这个标准的任何公司,很容易按当时适合本身规模的利率水平借到钱,而且借到本行业最高百分率的债务。这样的公司一旦借钱到举债上限──当然是根据它未来的营业收入成长、利润率、管理阶层素质、研究发展以及本章讨论的其他各个要点,有资格借到上限或接近上限──需要更多资金时,还是能以某种价格,发行股票,筹措资金,因为投资人乐于参与这种企业。

因此,如果投资人只找杰出的公司投资,则真正要紧的是这家公司的现金加上进一步借款的能力,是否足以应付未来几年的需求,以掌握美好的前景。果真如此,而且如果这家公司愿意借钱到上限,则普通股投资人不用担心较久以后的事。假使投资人已经对当时的情势做过适当的评估,则未来几年如果公司发行股票,筹措资金,价格会远高于目前的水平,投资人根本不必担心此事。这是因为短期融资会使盈余增加,几年后需要进一步筹措资金时,盈余增加会推升股价到比目前高出很多的水平。

但如目前的借款能力不足,发行股票筹措资金便有必要。这种情况下,投资对象是不是有吸引力,必须仔细计算。也就是,投资人应计算:筹措资金后,盈余可能增加,目前的普通股持有人将受益,但因发行在外股数也增加,股权稀释,利益将受损。和发行普通股一样,发行可转换优先证券的股权稀释效果,计算得出来。这是因为可转换优先证券订有条款,允许将来行使转换权利的价格,通常比发行时的市价高一些──从十%到二十%不等。由于投资人不应对十%到二十%的小涨幅有兴趣,而应寻求几年内十倍或百倍于此数的涨幅,所以转换价格通常可以不予理会,并以新发行优先证券完全转换为基础,计算稀释效果。换句话说,计算普通股发行在外真正的股数时,最好假定所有的优先可转换证券都已经转换,而且所有的可转换权证、选择权等都已行使。

如果买进普通股之后几年内,公司将发行股票筹措资金,而且如果发行新股之后,普通股持有人的每股盈余只会小幅增加,则我们只能有一个结论,也就是管理阶层的财务判断能力相当差,因此该公司的普通股不值得投资。除非这种现象很严重,否则投资人不应单因财务因素上的考虑而却步,因为一公司如果仍在其他十四个要点上获得很高的评价,将来可望有杰出的表现。相反的,为了从长期投资获得最高的利润,即使财务面很强或者现金很多,投资人也不应选择其他十四个要点中任何一点评价不佳的公司。
\\

\textbf{要点十四:管理阶层是不是只向投资人报喜不报忧?诸事顺畅时口沫横飞,有问题或叫人失望的事情发生时,则「三缄其口」?}


即使是经营管理最好的公司,有时也会出乎意料碰到困难、盈余萎缩、产品需求转向别处。另外,年复一年不断透过技术研究,设法产销新产品和新制程的公司,可望让投资人获得极高的利润,投资人应买进这样的公司。依平均数法则,有些新产品或新制程势将惨败,所费不赀。有些则会在试车工厂最后测试阶段的早期意外延误,花费不少冤枉钱。连续好几个月,这些预算外的成本相当沉重,即使原本审慎规划的整体盈余预测值,也终告无效。连最成功的企业,也无法避免这种叫人失望的事情。坦诚面对,加上良好的判断力,会知道它们只是最后成功的成本之一。它们往往是公司强势的迹象,而非弱势的征兆。

管理阶层面对这些事情的态度,是投资人十分宝贵的线索。碰到坏事,管理阶层不像碰到好事那样侃侃而谈,「三缄其口」的重要理由有好几种。他们可能没有锦囊妙计,解决出乎意料的难题;管理阶层可能已经心生恐慌;他们可能不觉得对持股人有责任,不认为一时的横逆有必要向持股人报告。不管是什么样的理由,凡是保留坏消息或设法隐匿坏消息的公司,投资人最好不要纳入选股考虑对象中。
\\

\textbf{要点十五:这家公司管理阶层的诚信正直态度是否无庸置疑?}


公司的管理阶层远比持股人更容易接触公司的资产。控制公司经营大权的人,有无数方法,能在不违法的情形下,假公济私,牺牲一般股东的利益,图利自己和家族。其中一个方法,是给自己──更不要说是亲戚──远高于正常水平的薪水。另一个方法是用高于市价的价格,把自己拥有的财产出售或租给公司。规模较小的公司中,这种做法有时难以察觉,因为掌握经营权的家族或重要干部,有时买进不动产出租给公司的目的,不是为了获取不当的利益,而是真心希望公司有限的营运资金能用在其他目的上。

公司内部人图利自己的另一个方法,是要求公司的供货商透过某些经纪商出售产品给公司。这些经纪商的股东是这些内部人或他们的亲友,没提供什么服务,但收取一定的经纪手续费。伤害投资人最深的做法,或许当属内部人滥用职权,发行普通股认股权。这个合法的方法,本来是用以酬庸能干的管理人员,但他们可加以滥用,自己发给自己很多股票;在立场公正的外部人眼里,奖酬数量已远超过他们的贡献。

面对这种滥用手法,只有一种方法能够保护自己。也就是,投资对象限于管理阶层对股东有强烈受托感和道德责任感的公司。这一点,「闲聊」法很管用。走笔至此,十五个要点已谈完。一家公司在这十五个要点中,有任何一点不如人意,但其他要点得到很高的评价,则仍可视为理想的投资对象。不过,不管其他所有的事务得到多高的评价,如果管理阶层对股东有无强烈的受托感一事,令人深感怀疑的话,投资人绝不要认真考虑投资这样一家公司。

\section{要买什么——应用所学各取所需}

一般投资人在投资领域并非专家。如果是男士,和本身的工作相比,他只腾出少许时间或精力处理投资事务。如果是女士,则和平常料理家务相比,花在投资上的时间和精力同样很少。结果,典型的投资人通常慢慢吸收到很多如真似假的知识和错误的看法,以及一些胡言乱语,对投资成功的真谛一知半解。

这样的观念中,最普遍和最不真确的一个,和一般人心目中,投资奇才必须具备的特质有关。如果就此事进行民意调查,我想,这样一位专家的综合形象,是个深思熟虑的书呆子,擅长于处理会计数字。这位学者状的专家整天孤坐独处,没人打扰,钻研资产负债表、企业的盈余报表以及交易统计数字。从这些资料中,他以卓越的智能和对数字的深入了解,取得一般人无法获得的讯息。埋首研读的结果,将得到宝贵的知识,晓得出色的投资对象在哪里。

和其他很多常见的误解一样,这幅心理上的画像不够准确,使它变得很危险,对想从普通股获得最大长期利益的人不利。

前面一章谈过,如果不想纯靠运气,选到投资大赢家,则应探索十五个要点;其中一些要点得靠埋首做数学运算才能确定。此外,如同本书开始时提过的,投资人如果拥有充分的技能,则长期投资获得若干利润──偶尔甚至赚到大钱──的方法不只一种。本书目的不在点出每一种赚钱方法,而在于指出赚钱的最好方法。所谓最好的方法,是指以最低的风险获得最高的总利润。一般大众心理的成功投资人形象,是很懂会计和统计数字的人。如果他们够努力的话,会找到某些显然是便宜货的股票。其中一些可能真的很便宜,但其它一些股票,公司未来经营上可能陷入困境,光从统计数字看不出来,因此不但不能算是便宜货,和几年后的价格比起来,目前的价格其实太高。

在此同时,即使是真正的便宜货,便宜的程度毕竟有其极限,往往需要很长的时间,价格才能调整到反映真实的价值。就我的观察来说,这表示在一段长到足以做公平比较的时期内──如五年──技巧最纯熟的统计数字逢低承接者最后获得的利润,和运用普通智慧,买进管理优异的成长型公司股票的人比起来,实在是小巫见大巫。当然,这考虑了成长型股票投资人所买股票未如预期理想而发生损失的情况,以及逢低承接者同比例未如理想的便宜货造成的损失。

成长型股票所得利润高出许多的原因,在于它们似乎每十年就能增值好几倍。相反的,我们很难看到便宜货的价值低估达五十%。这个简单算术的累积效果很明显。

走笔至此,想要投资的人可能必须开始修正他的看法,不能对找到合适投资对象所需的时间抱有错误的见解,更不用说找到它们必须具备若干个人特质。也许他认为每个星期花几个小时,在舒适的家里,大量研究文字数据,便能打开利润之门。他就是抽不出时间去寻找、耕耘合适的人脉,找一些聪明人聊聊,好在普通股的投资上获得最适当的利润。或许他有时间,但个性上还是不愿找人一谈,因为以前和这些人不是很熟。此外,和他们谈话还不够;必须引起他们的兴趣和信心到某种程度,才会把自己知道的事情告诉你。成功的投资人本性上通常对企业经营问题很感兴趣。因此,如果他想向某人索取数据,讨论问题的方式,很容易引起对方发生兴趣。当然了,他必须有相当不错的判断力,否则搜集到的全部资料可能形同废纸。

投资人可能有时间、意愿和判断力,投资普通股仍无法获得最大的成果。地理位置也是个因素。比方说,投资人如住在底特律市内或附近,将有机会了解汽车零配件公司,而住在奥勒冈州的投资人,即使同样勤奋或能干,却没有相同的好机会。目前许多大公司和行业的配销组织都属全国性质(即使生产活动不见得如此),散布在大部分主要城市,住在大工业中心或近郊的投资人,通常有很多机会,练习寻找杰出长期投资对象的艺术。相当遗憾,投资人如住在偏远地区,远离这些产销中心,则不然。

不过,偏远地区的投资人,或其他可能没时间、意愿、能力,自行寻找杰出投资对象的大多数投资人,绝对不可能因为这一点而无法投资。其实,投资事务非常专业、错综复杂,没有理由非要个人处理自己的投资不可,一如没有理由强迫个人当自己的律师、医生、建筑师或汽车技工。如果一个人对以上所说的特定领域有兴趣,就应该去执业,否则,根本不必成为专家。

重要的是,他必须相当了解有关的原则,如此才能找到真正的专家,而非找到一窍不通或冒充内行的人。从某些方面来说,认真仔细的外行人,选到杰出的投资顾问,比选到同样优秀的医生或律师容易。但从其他方面来说,则困难得多,因为近来投资这个领域发展得比其他专业领域快很多。因此,无数观念尚未具体成形,画出一条大家能够接受的界线,区分真正的知识和故弄玄虚的胡言乱语。投资理财这个领域,还没有一道门坎,用以筛除不学无术和能力不够的人,一如法律或医学的领域。连一些所谓的投资主管机关,也还没就基本原则取得共识,所以不可能设立学校,训练投资专家,就像传授法律或医学知识的名校那样。政府主管单位因此更不可能发放执照给具有必要知识背景的人,由他们指导别人投资,正如各州发放执照给合格者执业当律师或医生的做法。没错,美国很多州的确有发放投资顾问执照。不过这种情形中,我们只听过,不给执照的理由是背信诈欺或未能履行债务,而非欠缺必要的训练或技能。

所有这些事情,产生的不合格财务顾问比率,可能高于法律或医疗等领域。但由于某些补偿性因素存在,本身不善于投资理财的个人,选到能干的财务顾问,可能比选到同样出色的医生或律师容易。观察哪位医生执业时造成的死亡率最低,不是选择优秀医生的办法。从辩护案件胜诉和败诉的纪录,也不能看出律师的相对能力。幸好大部分医疗过程并非立即攸关生死,而且好律师可能根本不想对簿公堂。

投资顾问的情形则相当不同。经过一段够长的时间,便有相当多的纪录反映投资顾问的投资才能。偶尔可能需要五年的时间,才能看出他们的真正价值。通常不需要这么长的时间。经验不到五年的所谓顾问,可能自己当老板,也可能在别人那里当伙计。投资人把储蓄托付这样的人,一般情况下,未免愚不可及。所以说,谈到投资,想选专业顾问的人,没理由不要求阅读别人也能取得、内容翔实的投资纪录。拿这些投资纪录和同期内的证券价格比较,便能了解投资顾问的能力好坏。

投资人最后选定某些个人或组织,托付投资理财重责大任之前,还有两件事要做。其中一件很明显非做不可,也就是必须确定投资顾问的诚信正直毫无问题和瑕疵。另一件事则比较复杂。市场价格下跌期间内,某位财务顾问可能有远高于平均水平的操作纪录,但这也许不是因为他能干,而是他把个人管理的一大部分资金,拿去投资高评等债券。有些时候,价格长期上涨,另一位顾问可能因为喜欢买风险高的边际公司,而有高于平均水平的操作成绩。前面讨论利润率时说过,这种公司通常只在这种时期有好表现,此后表现便相当差劲。第三位顾问可能在两种时期都有好表现,因为他总是设法分析证券市场未来的走势。这么做,可能一段时间内有很好的成绩,但几乎不可能永远如此。

投资人挑选顾问之前,应设法了解他的基本理财观念为何。投资人应只挑基本观念和自己相同的顾问。当然了,我相信本书阐述的观念,基本上应遵守。从理财旧时代走过来的许多人,信奉「买低卖高」的做法,一定很不同意这个结论。

投资人如希望获得长期厚利,则不管请投资顾问代劳,或者自己操作,有件事情必须自己决定。我认为,几乎所有的普通股投资,目标都应该放在长期赚得厚利上。投资人必须做这个决定,因为最能符合前章所述十五要点的股票类别,投资特性可能有很大的差异。

天平的一边是大公司,进一步大幅成长的前景十分亮丽,财务状况非常良好,根基深植于经济沃土中。它们属于「机构型股票」的大类,也就是保险公司、专业受托人,以及类似的机构型买主会买这种股票,因为它们觉得自己可能误判市场价格,万一被迫在价格走低之际卖出股票,将损失一部份原始投资资金。买进这种股票,因公司竞争地位从目前的水平下滑,致蒙受损失的风险较低。

道氏化学公司(Dow Chemical Company)、杜邦(Du Pont)和国际商业机器公司(International Business Machines)是这类成长型股票(growth stock)的好例子。我在第一章提过,一九四六到一九五六年十年内,高评等债券的投资报酬率乏善可陈。这段时间结束时,三支股票──道氏、杜邦和IBM──价值都是期初售价的五倍左右。从当期收益的观点来说,十年内,这些股票的持有人都未受损伤。例如,就当期市价而言,道氏向以报酬率偏低著称,但在这段期间之初买进道氏股票的投资人,到了期末,从当期收益的角度来看,表现很好。虽然买进时,道氏的报酬率只有约二.五%(这段期间内,所有股票的收益率都很高),仅仅十年后,它的股利增加,或将股票分割好几次,以十年前的投资价格计算,投资人享有的股利报酬率介于八%到九%间。更重要的是,对类似这三支股票的杰出公司来说,这段十年期间没有什么特别不一样的地方。好几十年中,除了一九二九到一九三二年的大空头市场,或二次世界大战等偶尔出现的暂时性影响因素之外,这些股票一直有极为出色的表现。

天平的另一端也非常值得长期投资。它们往往是相当年轻的小型公司,每年总营业额可能只有一百万到六百万或七百万美元,但它们拥有展望可能十分美好的产品。为符合前面所说十五个要点,这些公司通常同时拥有杰出的经营管理人员和同样能干的科技人才,抢先进军前景看好的新领域或经济效益高的领域。一九五三年股票首次公开上市的安培斯公司(Ampex Corporation)可能是这类公司很好的例子。四年内这支股票的价值上涨了七倍以上。

这两个极端之间,有许多未来看好的其他成长型公司,介于一九五三年年轻、高风险的安培斯公司,和今天根基稳固的道氏、杜邦和IBM之间。假使现在要选购股票(参考下一章),投资人应买哪一种?

到目前为止,年轻的成长型股票最有可能上涨。有些时候,十年内可以上涨数十倍之多。但技巧纯熟的投资人,也难免偶尔犯错。投资人千万不能忘记,要是投资这类普通股犯下错误,丢出去的每一块钱可能消失不见。相反的,如果根据下章所述的原则买股票,则投资历史较悠久、根基较稳固的成长型股票,纵使因为整体股市出乎意料下跌而有损失,也属暂时性质。这类大型成长股的长期增值潜力,远低于小型、年轻的公司,但整体而言,是非常值得投资的对象。即使最保守的成长股,至少也会增值到原始投资的数倍之多。

因此,任何人拿一笔对自己和家人攸关重大的钱去冒险时,应遵守的原则相当明显。这个原则是,「大部分」资金投入的公司,即使不像道氏、杜邦或IBM那么大,至少应该比较接近这类型的公司,而不是年轻的小型公司。所谓「大部分」资金,到底是占总投资资金的六十%或一百%,要看每个人的需求或需要而定。膝下无子的某位寡妇,如果总资产有五十万美元,或许可以把全部资金拿去买较保守的成长股。另一位寡妇的投资资金是一百万美元,但有三个子女,则为了替子女着想,希望资产能够增值──可是不希望危及目前的生活水平──则可能拿十五%的资金,投入精挑细选的年轻小型公司。一位企业人士如有妻子、两个小孩,目前可以投资的资金是四十万美元,收入够多,缴纳所得税后,每年能存一万美元,有可能拿现有四十万美元全部投资较为保守的成长股,但每年一万美元的新储蓄用于购买风险较高的股票。

不过所有这些例子中,投资较为保守的股票,长期增值幅度必须够大,足以弥补投资风险较高的股票,万一血本无归的损失。在此同时,审慎选择的话,风险较高的股票可能大幅提高总资本利得。如果这样的事发生,同样重要的是,年轻高风险公司由于本身不断发展成熟,可能终有一天成长到股票不再带有以前那么高的风险,甚至可能进步到机构投资人开始购买的地步。

小额投资人的问题有点麻烦。投资大户往往能够完全漠视股利报酬率,动用所有的资金以获得最高的成长潜力。以这种方式投资之后,他还是能从那些股票获得足够的股利,支应理想中的生活水平,或者股利收入和其他正常收入加起来,能让他过那种生活水平。但是不管收益率多高,大部分小额投资人没办法靠投资报酬过活,因为他们持有的股票总值不够多。所以对小额投资人来说,面对当期股利报酬的问题时,通常必须二选一:现在开始每年有几百美元的收入,或者未来某日,获得数倍于此的收入。

就这件很重要的事做成决定之前,有个问题,小额投资人应正面面对:用于投资普通股的资金,只能是真正多余的资金。这不表示,超过日常生活开销所需的资金,都应用于投资。除了十分异常的状况,他应有数千美元以备不时之需,足以支应急病或其他意外紧急事故的花费,之后才考虑购买普通股等具有内在风险的东西。同样的,打算花在某种特定用途的资金,如送孩子上大学,绝对不该拿到股票市场冒险。考虑了这类事情之后,才应投资普通股。

接下来,小额投资人多余资金的运用目标,必须考虑个人的选择和本身特殊的处境,包括其他收入的多寡和性质。年轻男士或女士、有特别值得关爱的子女或其他继承人的中老年投资人,可能愿意牺牲每个月三十美元或四十美元的股利收入,好在十五年后取得十倍于此的收入。相反的,没有亲近继承人的老年人,当然希望马上获得较多的收入。另外,平常收入相当少,而且财务负担沉重的人,除了好好应付眼前迫切的需要之外,可能别无选择。

但是对绝大部分的小额投资人来说,立即获得收入是否重要,纯属私人问题,可能主要要看每位个别投资人的心理动机。依我个人之见,眼前少量的额外收入(税后)和未来几年有高收入,而且可能让子女富有比起来,食之无味。其他人对这件事的看法也许大不相同。本书所述投资程序,是针对投资大户,以及在这个问题上看法和我相同的小额投资人;他们希望根据所说的原则,找到聪明的方法,实现前述的可能成果。

任何人投资时运用这些原则能否成功,取决于两件事。其一是运用这些原则时的技巧良窳,另一当然是运气好不好。目前这个时代中,某个研究实验室明天可能有始料未及的发现,但这个实验室和你已经投资的公司无关。这个时代中,五年后毫无关联的研究发展,可能使你投资的公司盈余增为三倍或减半。所以就任何一笔投资来说,运气好坏显然扮演吃重的角色。资金规模中等的投资人,相较于资金很少的投资人占有优势,道理便在这里。如果你能精挑细选几样投资,则运气好坏大致可以相互抵消。

不过,对偏爱几年后获得高额收入,而不求今天拥有最高报酬的投资大户和小额投资人而言,最好记住,过去卅五年,各金融机构做过无数研究,比较了两种做法的成果:所买普通股提供高股利收益率,以及所买股票的收益率低,公司着眼于未来的成长和资产再投资。就我所知,每一份研究都指出相同的趋势:五年或十年期内,成长型股票的资本增值幅度远高于另一种股票。

更叫人惊讶的是,同一时段内,这类股票通常会提高股利,虽然和那时已经上涨的股价比起来,报酬仍然偏低,但和当初只看收益率挑选出来的股票相比,这时它们的原始投资股利报酬比较高。换句话说,成长股不只在资本增值方面表现出色,一段合理的时间内,因为公司不断成长,股利报酬也有同样不俗的演出。

\section{何时买进}

前面几章试图指出投资成功的核心,在于找到未来几年每股盈余将大幅成长的少数股票。因此,到底有没有理由挪出时间和精力偏离这个主题?何时买进的问题相较之下,不是没那么重要吗?一旦投资人肯定他已找到一支好股,任何时候不都是很好的买进时机吗?这些问题的答案,部分取决于投资人的目标,也和他的个性有关。

举个例子来说明。由于事后检讨一向不费吹灰之力,所以我们拿近代金融史上的极端实例来解释。一九二九年夏,也就是美国有史以来最严重的股市崩盘之前不久,精挑细选买进几家公司的股票。假以时日,这些股票终会带来不错的报酬。但廿五年后,投资人使出浑身解数,挑出合适的公司,之后再多花点力气,了解成长股买进时机的少数简单原则,则这些股票的涨幅,将远高于一九二九年夏买进股票的涨幅。

换句话说,买到正确的股票,并抱牢够长的时间,总会带来一些利润。通常它们会创造可观的利润。不过,要获得最高的利润,也就是如前所述的那种惊人利润,则应考虑进出时机的问题。

我觉得,传统上选取买进股票时机的方法,表面上说得头头是道,其实十分愚蠢。依传统的方法,投资人必须先搜集一大堆经济数据,从这些数据,得出一般企业中短期景气状况的结论。比较老练的投资人,除了分析景气状况,通常还会预测未来的资金利率走向。接着,如果所有的预测都指出背景状况不会大幅恶化,则结论是想买的股票或许可以放手去买。有些时候,地平在线似乎乌云涌现,运用这种通用方法的人,会推迟或取消买进想买的股票。

我反对使用这种方法,不是因为它在理论上不合理,而是目前人类预测未来景气趋势的经济学知识尚嫌不足,实务上不可能应用这种方法。预测正确的机率不够高,实在不适合以这种方法为基础,拿储蓄去冒险投资。或许情况不见得永远如此,五或十年内可能改观也说不定。目前,能力强的人试着利用计算机,建立错综复杂的「输出─输入」模式,也许将来某一天,有可能相当精确地预知未来的景气趋势。

这样的事情出现时,投资普通股的艺术或许必须改头换面。不过,这事没出现前,我相信预测景气趋势的经济学,不妨视为有如中世纪的炼金术,不能和今天的化学相提并论。那时的炼金术和今天的景气预测一样,基本原理刚从一堆神秘的符咒中现身。但是炼金术的这些原理,没有进步到可以做为安全的基础,据以采取行动。

如同一九二九年,经济偶尔会脱离常态,投机歪风盛行。即使以我们目前对经济无知的状态来说,也有可能相当准确地猜出将发生什么事。不过我怀疑,猜对的年头,平均十年会超过一年。未来的比率可能更低。

典型的投资人已经习于听信经济预测,可能开始过分信任这些预测的可靠性。果真如此,我建议他去找二次世界大战结束后任一年的《商业金融年鉴》(Commercial &Financial Chronicle)过期刊物档案。事实上,即使他晓得这些预测难免犯错,去查这些档案对他或许仍有帮助。不管选看哪一年,他会找到很多文章,里面有知名经济和金融权威人士对未来展望的看法。由于这份刊物的编辑似乎刻意平衡内容,让乐观和悲观的意见并陈,所以在过期刊物中找到相反的预测不足为奇。叫人称奇的是这些专家看法分歧的程度。更令人惊讶的是,有些论点强而有力,条理分明,叫人折服,后来却证明是错的。

金融圈不断尝试根据随机和可能不完整的事实资料,臆测未来的经济情势,而且所花心血之多,令人不禁想问:要是只花少数心血,用在可能更有用的事情上,会有什么样的成就?我老爱拿经济预测和炼金术时代的化学相互比较。显然没办法做好的事,却有那么多人耽溺其中,这种行为或许也可拿来和中世纪比较。

那段期间内,大部分西方世界人士不必有那么多欲望,也不用承受那么多苦难,主要原因出在人们把相当多的精力投入徒劳无功的事物上。人们花了很多时间辩论有多少天使能停留在一根针头上。试想:要是腾出一半的思虑,探索如何消弭饥饿、疾病、贪婪,情况会如何?今天投资圈花了很多心力,试图预测未来的景气循环趋势,如果腾出其中一部分心力,用在更有生产力的目的上,或许能产生惊人的成就。

那么,如果传统上有关近期经济展望的研究,不能提供正确的方法,让我们确定合适的买进时机,有什么办法能告诉我们这件事?答案和成长股本身的特性有关。

虽嫌重复,我们还是要挪出一点时间,复习前一章所说,十分理想的投资对象具备的特质。这类公司通常在某些技术方面走在非常前端。它们正在实验室开发各种新产品或制程,并透过试车工厂,初步商业化生产。这些事情必须花钱,金额不一,难免消耗其他业务的利润。商业生产的初步阶段,为了生产足够的新产品数量,以获取理想的利润率,必须额外增加销售费用。这个开发阶段立即出现的损失,甚至可能高于试车工厂的花费。

对投资人而言,这些事情有两个层面格外重要。其一是新产品开发周期的时间表不可能十分确定,另一是即使经营管理绝佳的企业,也有开发失败的时候,而且这是经营事业难以避免的成本。拿运动比赛来说,连棒球联盟中最出色的冠军队伍,也有打输球的时候。

谈到普通股的买进时机,开发新制程最值得密切注意的地方,或许在于全面性的首座商业化工厂何时开始生产。即使使用旧制程或生产旧产品的新厂房,试产期可能还是要六到八个星期,而且相当花钱。时间必须这么长,才能调整设备达到理想的运转效率,并消除无法避免的「臭虫」;这些臭虫常会侵入错综复杂的现代化设备中。如果制程属于革命性的发展,则昂贵的试产期所需时间,可能远超过公司中看法最悲观的工程师的估计值。等到问题终于解决的时候,筋疲力尽的持股人还是没办法期待立即获得利润。他们仍须再等好几个月,资金进一步耗损,因为公司必须挪用旧产品线更多的盈余,发动特别的销售和广告攻势,好让新产品为消费者接受。

做这些努力的公司,其他旧产品的营业收入可能继续成长,因此一般持股人不会看到利润有所流失。不过实际的情形往往相反。经营管理良善的公司中,实验室正在研发绝佳新产品的消息一传出,买主便会蜂拥而来,推升该公司的股票价格。试车工厂运转成功的消息传出,股价涨得更高。很少人想到一个旧比喻,说试车工厂的运转,有如在崎岖的乡间小路上,以十哩的时速开车。商业化量产工厂的运转,则有如在同样的道路上以一百哩的时速开车。

接下来,月复一月,商业化量产工厂的困难慢慢浮现之后,始料未及的费用支出导致每股盈余重挫。厂房营运陷入困境的消息传开。没人能保证问题何时可以解决。本来积极买进的人,失望之余,大量卖出,股价随之下跌。困顿期拖得愈久,市价跌得愈深。最后,厂房终于运转顺畅的好消息出现,股价连续弹升两天。不过接下来一季,特别的销售费用导致纯益进一步下降,股价跌到多年来的最低水平。整个金融圈都晓得该公司的管理阶层犯下大错。

这个时候,这支股票也许很值得买进。一旦额外的销售努力创造出足够的销售数量,第一座量产工厂终于获有利润之后,正常的销售努力往往足以继续推动营业收入上升好几年。由于使用的技术相同,第二座、第三座、第四座、第五座厂房建厂运转,几乎没有延误,而且不必负担第一座工厂冗长的试车期间发生的特殊费用。到了第五座工厂全能运转之际,公司规模已成长得相当大,业务蒸蒸日上。这时,另一种全新产品的整个周期重新上演一次,但公司整体的盈余不致受到拖累,股票价格也不会出现同样的下跌走势。投资人已在正确的时机买到合适的股票,能够成长好几年。

当年的第一版中,我用以下的文字描述这种机会。我用到的例子,那时还算相当新颖。我说:


「一九五四年国会选举前不久,一些投资基金掌握了当时的情势。在那之前好几年内,美国氰胺公司(American Cyanamid)在市场上的本益比远低于其他大部分主要化学公司。我相信这是因为金融圈普遍认为,虽然该公司的Lederle事业部是全球最杰出的制药组织之一,但规模相对较大的工业和农业化学部门,构成一个大杂烩,工厂费用支出沉重,效率低落;这些厂房是百业欣欣向荣的一九二〇年代,典型的『股市』合并热潮期间拼凑起来的。一般人普遍认为这不是理想投资对象具备的特质。

「但是大多数人没有注意到,新上任的管理阶层坚定果断地缩减生产成本、裁汰冗员和冗物、精简组织,只是没有大张旗鼓。大家只注意到这家公司『正下豪赌』──以它的规模来说,竟在刘易斯安纳州霍提尔(Fortier)斥巨资盖了一座很大的有机化学新工厂。这座厂房的工程设计很复杂,预估的损益平衡点日期落后好几个月才出现,自然不足为奇。但霍提尔厂的问题继续存在,美国氰胺的股票雪上加霜。这个时候,前面所提的那些投资基金见到买点已到,以平均四五.七五
美元的价格买进。由于一九五七年股票一股分割成两股,等于目前的股票价格每股为二二.八七五美元。

「此后情况如何?这家公司已经历一段充分的时间,开始从一九五四年产生异常成本的一些管理活动中获益。霍提尔厂已有获利。(目前的)普通股每股盈余从一九五四年的一.四八美元,提高为一九五六年的每股二.一○美元,一九五七年可望再略微提升,而这一年大部分化学业(不是制药业)的利润都不如前一年。至少同样重要的是,『华尔街』业已认清美国氰胺的工业和农业化学事业值得机构投资人投资。结果,这支股票的本益比大幅改观。不到三年的时间内,盈余成长三十七%,市值增加约八十五%。」

写下这些文字之后,金融圈对美国氰胺公司的评价不断提高,而且这种情况似乎会持续下去。一九五九年的盈余可望超过一九五七年二.四二美元的前一历史最高纪录,所以股票市价稳定攀高。目前价格约为六十美元,自本书第一版推荐买进这支股票以来,五年内盈余增加约七十%,市值上升一百六十三%。

很高兴能以这种快乐的笔调结束讨论美国氰胺公司。不过在本修正版的序中,我曾提到,本修正版要做忠实的纪录,而不是只纪录到目前为止看起来最准确的纪录。你可能注意到,本书第一版中,我提到一九五四年「某些基金」买进美国氰胺公司的股票;这些基金已经不再持有这支股票,一九五九年春卖出持股的平均价格是四十九美元左右。当然了,这个价格远低于目前的市价,但仍有约一百一十%的获利。

这种获利率水平和卖出决定毫无关联。卖出决定的背后有两个动机。其一是另一家公司的长期展望似乎更好。下一章会讨论卖出有理的这个原因。虽然时间还不够长,足以提供证据,做成结论,但到目前为止,比较两支股票的市场价格,卖出行动似乎正确。

不过,改变投资决定的第二个动机,事后看来比较欠缺说服力。那是相对于最出色的竞争同业,美国氰胺的前景令人忧虑。美国氰胺的化学(不是制药)事业未如预期,利润率大幅提升,并建立赚钱的新产品线。除了这些因素叫人关切,该公司企图在竞争极为激烈的纺织业打进亚克力纤维市场,可能增加不少成本,前景相当不明确。这种想法或许正确,但事后来看可能仍是错误的投资决定,因为Lederle制药事业部的远景明亮。股票卖出之后,它的远景更为清楚。中期来看,Lederle的获利能力有可能进一步跃升,主要是因为(一)推出展望相当不错的新抗生素,以及(二)一种口服「天然」小儿麻痹疫苗可望出现庞大的市场;在这个领域,这家公司一直是领导厂商。这些发展使得情势变得有问题,只有将来才能证明脱售美国氰胺股票的决定可能没有犯下投资错误。由于研究可能犯下的错误,比回顾过去的成功,获益更多,我建议──虽有狂妄之虞──任何想精进投资技巧的人,不妨把后面几段画起来,看完下一章「何时卖出」之后,再重看这几段。

接下来容我再谈另一个最近的例子,用以说明本书第一版提过的这种买进机会。我说:


「一九五七年下半年,食品机械化学公司(Food Machinery and Chemical Corporation)的情况有点类似。一些大型机构买主喜欢这支股票已有一段时间。不过,更多人似乎觉得,虽然他们有兴趣,但一些事情,需要看到有利的证据,才会去买股票。要了解为什么有这种态度,必须看一下当年的若干背景环境。

「二次世界大战以前,这家公司只生产各式机械。由于经营管理优异,开发工程设计也一样优异,食品机械公司成了战前获利十分可观的投资对象之一。战争期间,除了制造公司相当拿手的军需品,还建立起多元化的化学事业。这么做,目的是希望透过消费性产品的生产,稳定机械事业周期性荣枯的冲击。消费性产品的销售能够持续扩张好几年,因为可以利用公司做得十分成功的机械和军需品事业部类似的研究方式。

「到了一九五二年,食品机械公司收购另外四家公司,改成四个(现在是五个)事业部。把军需品部门加进去,合起来的销售额略低于总销售额的一半,如果只考虑正常的非国防事业,则略高于一半。收购之前和收购之后几个年头内,这些化学单位的变动很大。其中之一在某个快速成长的领域居龙头老大地位,利润率高,技术能力在业内无人能出其右。另一个单位则苦于厂房老旧、利润率低、士气差。所有的化学单位平均而言,和真正的化学业领导公司还差上一大截。有些化学单位生产中间产品,但不生产基本原物料。有些化学单位有很多低利润原物料,但几乎不从这些原物料生产利润率较高的产品。

「从所有这些事实,金融圈做成若干相当肯定的结论。机械事业部──每年的内部成长率是九%到十%(和整体化学业相当),年复一年,展现他们有能力设计和销售富有创意和商业上有价值的新产品,而且在各自的领域盖了成本最低的工厂──是评价最高的投资对象。但在化学事业部有更高的整体利润率和展现其他内在价值之前,很少人愿意投资整家公司。

「在此同时,管理阶层积极设法解决这个问题。他们做了什么事?他们采取的第一个行动,是经由内部升迁和对外招募人才,建立高阶管理团队。新的团队花钱把旧厂现代化、筹建新厂以及推动研究发展。如果完全不谈正常情况下资本化的厂房支出,根本不可能在不提高目前费用水平的情况下,进行大规模的现代化和扩厂计划。一九五五、一九五六、一九五七年异常的费用支出,并没使那段期间的化学业务盈余下降,颇令人称奇。盈余持稳的事实,强烈显示过去所做的事有其价值。

「无论如何,如果项目计划审慎规画,则已经完成的工作带来的累积效果,迟早会超过仍在花费的异常支出。如果一九五六年的研究支出没有比一九五五年的水平高约五十%,则早在一九五六年,这样的事情或许不会发生。即令一九五五年,化学事业的这些研究支出也没有远低于业界的平均水平,机械事业的研究支出则远高于大部分同业。虽然研究支出持续处于偏高的水平,一九五七年下半年,盈余可望跃增。年中时,该公司设于西弗吉尼亚州南查尔斯顿(South Charleston)的现代化氯制品工厂,本来预计上线生产,但是发生化学业常见的意外问题,幸好这家公司的其他现代化和扩张计划多已完成,一九五八年第一季盈余才会跃增。

「我猜想,在盈余改善、化学事业利润率成长,并持续上扬一段期间之前,机构投资人通常目光如豆,未能见及表面底下发生的事情,大多会远离这支股票。要是真如我所想,一九五八和一九五九年情况明朗之后,这段期间某个时候,金融圈的看法会改变,体认到几年前基本面已经开始改善的事实。到那时候,股票价格会上涨,部分原因是每股盈余已经上升,但更重要的是投资人普遍重新评估这家公司的内在素质,使得本益比改变。股价可能持续上涨好几年。」

我相信,过去两年的纪录强烈显示上面的说法正确。投资人首次普遍体认到表面底下发生的事情,也许是在景气近于萧条的一九五八年。这一年,几乎所有的化学和机械公司的获利都大幅衰退,食品机械公司报告的每股盈余却达二.三九美元,创历史新高纪录。前几年,整体经济景气较好时,盈余仍略低于一九五八年。人们认为,化学事业部门终于到了能与机械事业部门平起平坐的地步,不再是边际投资,而是很理想的投资。写下这段文字时,一九五九年全年的盈余数字还没发表,但已报告的头九个月获利比一九五八年同期激增,进一步确认化学事业部长期的组织结构调整努力,终于开花结果,而且收获丰硕。同时,军需品事业部的主要产品,从以前使用钢铁生产人员装甲车和轻装备两栖坦克式车辆,转为可以空投的铝制产品,因此一九五九年盈余增幅显得格外突出。就最近的过去或可预见的将来而言,军需品事业部对一九五九年的总获利没有重大的贡献,可是一个重要的获利新高峰已经出现。

市场对所有这些事情有什么样的反应?一九五七年九月底,本书第一版完成时,这支股票的价格是二五.二五美元。今天,股价升抵五十一美元,涨幅达一百零二%。整个情况看起来,我在第一版所提,金融圈「体认到几年前基本面已经开始改善的事实」一事,似乎开始应验。

其他事情也证实这个趋势,而且可能多加一把劲。一九五九年,麦格罗·希尔出版公司(McGraw-Hill Publications)采用一种新做法,决定每年给化学业一个杰出管理成就奖。为决定这个荣耀的第一年得主,他们选了十分有名和知识渊博的十位成员,组成一个小组。其中四人代表知名大学企业管理研究所,三人来自持有大量化学类股的大型投资机构,三人是著名化学业顾问公司的高阶主管。廿二家公司获得提名,十四家到场说明。这个管理成就奖没有落到业内巨擘头上;这些公司中,有几家的管理阶层极受华尔街推崇。相反的,食品机械公司的化学事业部获此殊荣,而仅仅两年前,大部分机构投资人还认为这家公司是相当不理想的投资对象,许多机构投资人后来仍这么认为!

为什么这种事情对长期投资人那么重要?首先,不管整体工商业景气趋势如何,它能强烈保证这样一家公司未来的盈余会成长好几年。消息灵通的化学业人士不会把业内这种奖项授给没有研究部门,以继续开发高价值新产品的公司,和能在获有利润的情况下生产这些产品的化学工程师。第二,这种奖项会在投资圈留下好印象。正如我在本书第一版针对这家公司做结时提到的,盈余的上升趋势对股价产生影响,加上每一块钱的盈余受市场重视的程度同样与日俱增,再也没有其他事情比这件事对持股人更有利了。

除了推出新产品和复杂的工厂开始运转出现问题,其他事情发生时,也有可能是买进杰出公司股票的大好良机。举例来说,中西部一家电子公司以劳资关系十分融洽著称,可是单单由于规模成长,公司就不得不调整对待员工的方式。很不幸的,个人相互影响下,导致劳资摩擦、怠工式的罢工、生产力低落,而这家公司不久前才因劳资关系良好和劳工生产力高而普受好评。该公司本来很少犯错,偏偏在这个时候犯下错误,误判某种新产品的市场潜力。结果盈余急剧下挫,股价也一样。

非常能干和足智多谋的管理阶层马上拟定计划,矫正这种状况。虽然计划几个星期内就能做成,但付诸实施产生效果所需的时间,远长于此。这些计划的成果开始反映到盈余上时,股价到达或可称之为买点A的价位。但所有的利益充分实现在盈余报表上,花了约一年半的时间。这段期间快结束时,第二次罢工发生。解决此事,是公司重振效率,恢复竞争力,最后一件该做的事。这次罢工没有拖得很久。不过,短暂且损失不大的罢工发生,消息传到金融圈,说该公司的劳资争议愈演愈烈。虽然公司高阶主管大量买进,股价还是下跌。可是价格没有跌太久。从进出时机的观点来说,这是另一个正确的买进机会,或可称之为买点B。愿意深入表层去观察真正发生什么事的人,能以便宜的价格,买到价格会上涨好几年的股票。

我们来看看,投资人如在买点A或买点B买进这支股票,可望获得多大的利润。我不打算拿最低价格来计算;只要一张每个月的高低价格表,就可看出在这两个买点,股价曾跌到多低的价位。这是因为在最低点,换手的股票只有几百股。投资人如能买到最低点,则运气的成分居多。相反的,我要在一个例子中使用略高于最低点的价格,另一个例子中使用高于最低点数美元的价格。两个价位都有数千股可买和换手。任何脚踏实地审情度势的人,能够轻易买到的价格,我才用来说明。

在买点A,这支股票仅仅几个月内便从前一个高点下跌二十四%左右。约一年内,在这个价位买进的投资人,市值增加五十五%到六十%间。接着罢工潮带来买点B。股价回跌约二十%。很奇怪,罢工结束后,它还处在那个价位数星期之久。这时,某大投资信托一位很聪明的员工向我说明,他觉得当时的情势再好不过,也十拿九稳,晓得将发生什么事。不过,他不会向公司的财务委员会推荐买进这支股票。他说,一些委员一定会向华尔街的朋友查证,不只驳回他的建议,还会指责他害他们挪出时间,注意一家管理懒散、劳资关系问题没希望解决的公司!

几个月后,我写这段文字时,这支股票的价格已比买点B上涨五十%,也就是比买点A涨了九十%以上。更重要的是,这家公司的前景十分明亮,从每个角度看,未来几年将有很高的成长率,一如异常且暂时性的不幸遭遇带来买点A和B之前几年的情况。在这两个时点买进股票的人,都在正确的时机买到正确的公司。

简言之,投资人应买进的公司,是在非常能干的管理阶层领导下做事的公司。他们所做的一些事情势将失败,其他一些事情偶尔遭遇始料未及的问题,之后才否极泰来。投资人心里应该十分清楚,晓得这些问题都属暂时性质,不会永远存在。接着,如果这些问题导致股价重跌,但可望在几个月内解决问题,而非拖上好几年,则考虑在这个时候买进股票可能相当安全。

并不是公司经营出问题才有买点。有些行业,如化学生产,需要投入大量的资金才能创造一美元的营业收入,有时会有另一种买进良机。这种情况的数学运算方式如下所述:一座新厂或好几座工厂需要一千万美元才能盖好。这些厂房全面运转之后一两年内,公司的工程师前往现场详细检讨,建议再支出一百五十万美元。他们指出,总资本投资增加十五%,工厂的产量将比以前的产能提高四十%。

很明显可以看出,由于这些厂房已获有利润,而且只要多花十五%的资金成本,便能增加四十%的产销数量,同时一般性的间接费用几乎没有增加,额外四十%产量的利润率将很高。如果这项计划的规模很大,足以影响公司整体的盈余,则在获利能力提升反映到市场价格之前不久,买进这家公司的股票,也一样是在正确的时机买到正确的公司。

以上所举例子有什么共同点?那就是适合投资的公司,盈余即将大幅改善,但盈余增加的展望还没有推升该公司的股票价格。我相信,这样的情况出现时,适合投资的公司便处于合适的买点。相反的,如果这样的事没发生,只要买进的是出色的公司,长期而言投资人仍能获利。不过这时最好多点耐性,因为需要较长的时间才能获利,而且和原始投资金额比较,获利率远不如另一种情况。

这是不是表示一个人如果有点钱可以投资,一发现如第三章所定义的正确股票,以及如本章指出的好买点,则应完全忽视未来可能出现的景气循环趋势,把所有的钱投资下去?在他投资之后不久,经济萧条可能来袭。即使最好的股票,不巧碰到正常的景气萧条期,股票价格从高峰跌落四十%到五十%也相当常见。那么,完全忽视景气周期不是很冒险?

我觉得,投资人对这种风险可以泰然处之,因为他在相当长的时期内,手中一堆持股都是精挑细选出来的。挑选得当的话,这些股票现在应已有相当大的资本利得。但现在,可能因为他相信手上的某种证券应该卖出,或者由于某些新资金流进手中,有钱买新的股票。除非碰到十分罕见的年头,也就是股市的投机性买气炽热,而且重大的经济风暴讯号响个不停(如一九二八和一九二九年间的情形),我相信这种投资人应该不理会整体景气或股市趋势的臆测之词。反之,合适的买进机会一出现,就应投入适当的金额买进股票。

他不该去猜测整体景气或股市可能往哪个方向走,而应有能力判断他想买进的公司相对于整体景气会有什么样的表现,同时,判断错误的机率很低。这一来,一起步他便占了两项优势。首先,他把赌注放在他十分肯定的事情上,不是放在只凭猜测的事情上。此外,根据定义,他只买基于某种理由,中短期获利能力将大幅提高的公司,所以得到第二股支撑力量。如果景气状况持续良好,则新现的获利能力终于为市场肯定时,他持有的股票会涨得比一般股票多;万一不幸在大盘下跌之前不久买进股票,则同样新现的盈余,应能阻止所买股票跌得和同类其他股票一样重。

但是许多投资人的处境都不快乐,因为手中持股不是在精挑细选的情况下,以低于目前的价格买得十分安心。或许这是他们第一次有钱投资。或许他们的投资组合中有债券和相当静态的非成长型股票,但愿历经波折,将来终能转换成赚到更多钱的股票。如果这类投资人有了新资金,或在长期的景气荣面和股价上涨多年之后,想要转为投资成长股,他们可以忽视景气可能萧条的风险?要是后来他发现全部或大部分的资金套牢在长期涨势的高点,或在大跌之前不久买进,一定难展欢颜。

这真的会制造问题。但是解决这个问题的方法,不是特别困难──和股市有关的其他很多事情一样,只需多点耐心便可以解决。我相信,这类投资人一发觉自己确实找到一或多支合适的普通股,就应开始买进。不过,买进之后,进一步加码的时机应慎思。他们应做好计划,几年之后,才把最后一部分可用的资金投资下去。这么做,万一这段期间内市场重挫,他们仍有购买能力,可以掌握跌势,趁机买进。如果股价没有下跌,而且早先买进的股票选得很适当,则至少手中有一些涨幅不错的股票。这么做有缓冲作用,要是在他们处境最糟的时候──也就是在最后一部分资金完全投资下去之后──股价碰巧重跌,则早先买进的股票的涨幅,即使不能完全抵消新买股票的跌幅,也应能抵消一大部分。这么一来,原始投资资金不致严重耗损。

投资纪录尚难满意的投资人,以及有足够资金可以再买进的投资人应该这么做,理由同样重要。这类投资人用光所有资金之前,有机会以务实的作风,展现他们或他们的顾问擅长于运用各种投资技巧,以合理的效率运作。要是没取得这样的纪录,则至少在投资人获得警讯,修改他的投资技巧,或者找到别人替他处理这些事情之前,所有的资产不致一头栽入。

所有类型的普通股投资人可能应该谨记一件事。目前金融圈一直忧虑,而且念兹在兹的是景气下挫有使正确的投资行动化为乌有之虞。这件事是廿世纪中叶的此刻,景气现状只是至少五股强大力量中的一股。所有这些力量,不是影响群众心理,便是透过经济体系的直接运作,可以对整体股价水平产生极强的影响。

另外四股影响力量是利率趋势、政府对投资和私人企业的整体态度、通货膨胀的长期趋势,以及──可能是所有力量中最强的一种──新发明和新技术影响旧行业。这些力量很少在同一时间把股价拉往同一个方向,而且没有任何一股力量的重要性必然长期远高于其他任何一股力量。这些影响力量十分复杂和多样化,乍看之下风险最高的做法反而最安全:确定某家公司值得投资时,放手去投资便是。因为推测而产生的恐惧或希望,或者起于揣测而获得的结论,不应令你却步。

\section{何时卖出——以及何时不要卖出}

投资人决定卖出普通股,有许多好理由。他可能要盖栋新房子,或筹钱供孩子创业。从享受生活的观点来说,类似的理由不胜枚举,卖出普通股合情合理。在这些情况下卖出持股,动机出于私人因素,和财务上的考虑无关,超出本书的范畴,不予讨论。本书只谈因为单一目标──从可用的投资资金中获得最大的利益──而出售股票的情形。

我相信有三个理由,而且只有三个理由,才会出售根据前面讨论过的投资原则,精挑细选买进的普通股。对任何人来说,第一个理由很明显,就是原始买进动作犯下错误,而且情况愈来愈清楚,某特定公司的实际状况显著不如原先所想那么美好。这种情况要妥善处理,主要得靠情绪上的自制。在某种程度内,也要看投资人能不能坦诚面对自己。

普通股的投资有两个重要特性,一是妥善处理能带来厚利,二是要能妥善处理,必须有高超的技能、知识和判断力。由于获取这些近乎梦幻般利润的过程相当复杂,所以买进股票难免犯下若干比率的错误。幸好,真正出色的普通股,长期利润应足以弥补正常比率的错误造成的损失仍有余。它们也应会留下很大的涨幅。如果尽早认清和接受所犯的错误,尤其如此。能够这么做,如有任何损失,应会远低于买进错误的股票之后,长期抱牢产生的损失。更重要的是,套牢在不利状况中的资金,可以释放出来,用于购买其他精挑细选的好股,而带来相当大的利得。

不过,有个复杂的因素,使得投资错误的处理更为棘手。这事和我们每个人的自尊心有关。没有一个人喜欢自承犯错。如果我们买进股票犯下错误,但卖出时能获得些许利润,就不会觉得自己当初做得很蠢。相反的,卖出时如发生小损失,我们对整件事会觉得相当不高兴。这种反应十分自然和正常,却可能很危险,会让我们在整个投资过程中放任自我。投资人死抱很不想要的股票,寄望有一天能够「至少打平」,因此损失的金钱,可能多于其他任何单一的理由。除了这部分实际上的损失,如果考虑发生错误时能够当机立断,认赔卖出,释出资金,转投资于合适的股票,并获有利润,则放纵自我的成本很高。

此外,连小损失也不愿认赔卖出的行为,虽自然却不合理性。如果投资普通股的真正目标,是几年内赚数倍的利润,则二十%的损失和五%的利润,两者间的差距便微不足道。重要的不是损失会不会偶尔发生,而是可观的利润是不是经常没有实现。使得投资人或他的顾问的投资理财能力备受质疑。

虽然损失不应导致投资人强烈自责或情绪失控,但也不应淡然处之。投资发生亏损,应仔细检讨,好从每个错误中学得教训。导致买进普通股判断错误的特定因素,如能彻底了解,就不可能因为误判相同的投资因素,再犯另一次的买进错误。

现在来谈根据第二章和第三章所说的投资原则买进的普通股,应该卖出的第二个理由。如果随着时间的流逝,一家公司发生变化,符合第三章所述十五要点的程度,不再和当初买进时相近,就应卖出它的股票。投资人必须一直提高警觉的道里便在这里。这可以说明为什么持有某家公司的股票时,时时密切注意该公司相关事情的发展,十分要紧。

一家公司经营每下愈况,通常出于两种原因:不是管理阶层退步,就是该公司的产品不再像以前那样,市场可望逐渐扩大。有些时候,管理阶层退步,是因为经营成功影响了居于关键地位的一位或数字高阶主管。耽于逸乐、洋洋自得和怠惰松懈,取代了以前的干劲和才气。新的高阶主管班底不能向以前的管理阶层的绩效标准看齐时,常发生这样的事。以前让公司经营十分出色的政策,他们不再奉行,或者没有能力持续执行。这些事情一发生,则不管股市大盘看起来多好,或资本利得税多高,都应立即卖出受影响的股票。

同样的,有些时候,多年来公司成长惊人之后,终于到达某个阶段,市场成长展望耗竭。此后,它只能做得和整体业界差不多一样好,成长率只能和全国的经济成长率不相上下。这种变化可能不是源于管理阶层退化。很多管理阶层擅长于开发相关或相近的产品,掌握利用各有关领域的成长机会。不过他们很清楚,踏进不相干的业务范畴,根本没有任何特殊优势。因此,如果一公司多年来在年轻的成长性行业中表现突出,但由于时移势转,市场成长潜力耗竭殆尽,股票和我们常提的十五要点标准严重脱节时,便应该卖出。

这种情况和管理阶层退步比较,卖出动作或许可以不必那么急迫。也许一部分持股可以保留到找着更合适的投资对象再脱手。但是无论如何,这家公司都不应再被视为里想的投资对象。至于资本利得税,不管金额多高,很少应成为换股的障碍,因为换股之后,未来几年的成长状况,可能和被换掉的股票以前的表现一样好。

谈到未来会不会进一步成长的问题时,有个好方法,可用以检定一公司是否不再适合投资。投资人不妨问自己:下一次景气周期的高峰到来时,不管这之前可能发生什么事,这家公司的每股盈余(必须考虑股利和股票分割,但不考虑发行新股以筹措额外的资金)和目前的水平比较,增幅是不是至少和上次已知的景气高峰期到目前水平的增幅一样大?如果答案是肯定的,也许这支股票应抱牢不放。如果答案为否定,或许应该卖出。

当初如依照正确的原则买进股票,则卖出股票的第三个理由很少出现,而且只应在投资人十分有把握的情况下,才能有所行动。这个理由起于一项事实,也就是大好投资良机千载难逢。从进出时机的观点来说,它们很少在刚好有资金可以投资时找到。如果投资人有一笔能够投资相当长的时间,并且找到难得一见的好股,可以投入资金,那么他很有可能把一部分资金或全部资金,投入他相信成长前景美好、管理良善的公司。不过这家公司的平均年成长率,和后来发现,看起来更有吸引力的另一家公司预期中的成长率比较,相形失色。在其他一些重要层面上,已经持有股票的那家公司,看起来也望尘莫及。

如果证据很明显,而且投资人对自己的判断相当有把握,则换股买进远景似乎更美好的股票,即使扣除资本利得税,投资人所获利润可能仍然很可观。一公司长期内平均每年成长十二%,对持股人的财富应该很有帮助。但是这种成果和平均每年增值二十%的公司比起来,值得不怕麻烦换股操作,也不必担心资本利得税的问题。

但是,随时准备卖出某支普通股,希望把资金转入更好的股票的做法,有待商榷。整幅画面中,某些重要因素遭误判的风险永远存在。万一发生这样的事,换股操作可能不如原先预期那么好。相反的,机敏的投资人如果抱牢某支股票一段时间,通常会知道它比较讨人厌的特质和比较令人喜爱的特质。所以说,卖出相当令人满意的持股,转进更好的股票之前,有必要十分小心谨慎,设法准确评估整个情势中的所有因素。

走笔至此,聪明的读者可能已看出一个基本的投资原则。大体而言,似乎只有少数成功的投资人才了解这个原则:一旦某支股票审慎挑选出来,而且历经时间的考验,则很难找到理由去卖它。可是金融圈仍不断提出建议和评论,列举卖出优异普通股的其他种种理由。这些理由站得住脚吗?

最常见的理由,是相信整体股市某种幅度的跌势呼之欲出。前面一章,我曾说过,由于担心大盘可能如何而延后买进值得投资的股票,长期而言,是损失惨重的做法,因为投资人忽视了他相当肯定的强大影响力量,反而担心没那么强的力量。但以目前人类的知识而言,他和其他每一个人对于后者,大致上只能依赖猜测。买进值得投资的普通股时,不应因为忧虑普通的空头市场来袭,而受到过度影响,如果这个论点有理,则不应因为这种忧虑而出售优异股票的论点,更有道理。前一章提到的所有论点,同样适用于此。此外,考虑资本利得税的因素之后,投资人卖出这样的股票卖得正确的机率,更为减弱。出色股票持有好几年,应该会有很大的利润,资本利得税将使出售持股的成本更显沉重。

还有另一个损失更大的理由,可以说明为什么投资人绝对不要因为忧虑普通的空头市场可能来袭,而脱售出色的股票。如果所选的股票的确是好股,那么下一次多头市场来临时,应会看到这支股票创下的新高价,远高于迄今缔造的最高价。投资人如何晓得何时应买回股票?理论上,应在即将出现的跌势之后买回。但是这无异于事先假定投资人知道跌势何时结束。我看过很多投资人,因为忧虑空头市场将来,而脱售未来几年将有庞大涨幅的持股。结果,空头市场往往未现身,股市一路扶摇直上。空头市场果真来临时,我从没看过买回相同股票的投资人,能在当初的卖价以下买得。通常股价实际上没跌那么多,他却还在苦苦等候股价跌得更深,或者,股价的确一路下挫,他们却因忧虑别的事情发生,一直没有买回。

这又带领我们到另一种常见的推理方式,使得善良但涉世未深的投资人错失未来庞大的利润。这个论点是,某支出色的股票价格涨得过高,因此应该卖出。还有什么比这种说法更有道理?要是一支股票的价格过高,为何不卖出,而要死抱不放?

仓促做成结论之前,我们稍微深入表层探讨这种说法对不对。价格到底多高才算过高?我们到底想要什么?任何真正出色的股票,本益比(价格相对于当期盈余的比率)会比获利能力稳定但未见提升的股票高,而且理应如此。毕竟,要参与分享企业持续成长的利益,显然值得多付出一些。我们说某支股票的价格过高时,意思可能是指它的售价相对于预估获利能力的比率,高于我们相信应有的水平。或许我们的意思是,它的本益比高于前景类似的其他公司;这些公司未来的盈余也有可能大幅提高。

这些说法,都试图以很高的准确度衡量某些东西,但实际上我们不可能办到。投资人没办法准确地指出某家公司两年后的每股盈余将有多少。他顶多只能用一般性和非数学式的语汇,判断这种事情,如「大致相同」、「略微上升」、「上升很多」、「大幅上升」。其实,公司高阶管理人员也没办法做得比他们好很多。要判断几年后平均盈余有没有可能大幅上升,他们和投资人的能力相当接近。但到底增加多少,或者到底哪一年会有这种增幅,通常必须猜测很多变量,根本不可能预测得很准确。

在这些情况下,一个人怎能判断成长异常迅速的杰出公司股价过高?不要说他无法预测得很准,连勉强可以接受的准确度也难以达到。假设某支股票现在的价格不是常见的盈余的廿五倍,而是盈余的卅五倍。或许这家公司近期内将有新产品推出,但金融圈还没能了解新产品真正的经济价值。或许根本没有任何这样的产品。如果成长率很高,再等个十年,这家公司的规模将成长为四倍,则目前的股价可能高估三十五%,或者不可能高估三十五%,有那么叫人提心吊胆吗?真正要紧的是,未来价值将很高的部位,不要随便搅动。

我们的老朋友资本利得税再次在这个结论中插上一脚。成长股被人认为价格过高而建议卖出,则于出售时,持有人几乎都必须缴纳很高的资本利得税。因此,除了冒险永远失去一家公司的部位,没办法在未来好几年继续分享很高的成长率,我们还必须承担可观的税负。下定决心相信那支股票的价格可能暂时涨过头,不是比较安全和省钱吗?投资那支股票,我们已有可观的利润。假使股价有一阵子跌得比目前的市价低三十五%,真的有那么严重吗?同样的,保住部位不是比暂时失去一小部分资本利得的可能性重要吗?

投资人有时还用另一种论点,而与本该上门的利润错失交臂。这个论点是所有不当看法中最荒谬的一个。他们认为,手中持股已有很大的涨幅。因此,单单因为股价已经上涨,大部分的上涨潜力可能已经耗尽,所以应该卖出,转而买进还没上涨的其他股票。我相信,投资人只应购买杰出的公司,而杰出的公司股票根本不会以这种方式运作。它们的运作方式,用下述假设性的例子,或许说明起来最清楚明白:

设想你大学毕业那一天。如果你没上大学,则不妨想成高中毕业那一天。就我们举例说明的目的而言,大学或高中毕业都无关紧要。这一天,班上每位男同学亟需马上用钱,每个人都向你提出相同的交易方式。如果你肯给他们一笔钱,金额相当于他们开始工作后十二个月内总收入的十倍,则在他们的余生,每年收入的四分之一会交给你!最后我们假设,虽然你认为这种交易方式很棒,但你手上的钱只够和三位同学达成交易。

这时,你的推理过程会很像投资人运用良好的投资原则选择普通股。你会马上开始分析你的同学,但不看他们与人相处多融洽,或者多有才气,只看他们可能赚多少钱。如果班上同学很多,可能有相当多人一开始便被剔除,因为你对他们认识不够清楚,没办法判断他们的赚钱能力将有多高明。同样的,这种比拟和运用智慧买进普通股很接近。

最后,你会选出三位同学和他们达成交易,因为你觉得他们未来的赚钱能力最强。十年过去了,三位同学中有一位表现十分突出。他在一家大公司工作,一再获得擢升。公司内部已有传闻,说总裁属意于他,再等个十年,很有可能接任总裁的宝座。到时,这个职位将带来很高的薪酬、认股权、养老福利给付。

这种情况下,一向建议人们卖出「涨过头」的超级好股,获利了结、落袋为安的股市报导作者,看到别人同意给你六倍于原始投资金额的价码,买你和同学的合约,会怎么说?如有人建议你卖出这份合约,转而和另一位同学缔约,而他的年收入仍和十年前刚踏出校门时大致相同,你一定觉得,提出这种建议的人该去做脑部检查。(财务上)成功的同学收入已经增加,而不成功的同学未来增加的空间仍大,这样的说法可能相当愚蠢。如果你对自己的普通股认识一样清楚,则我们常听到卖出好股的许多论点,也一样愚蠢。

你可能认为,这些话听起来的确有道理,但毕竟同学不是普通股。没错,两者间有一个很大的不同点,但这个不同点只能强化,而非减弱绝对不要因为价格涨幅已大,而且可能暂时过高,而卖出优异普通股的理由。这个不同点在于同学的寿命有限,可上马上死亡,而且最后难免一死。普通股则没有类似的寿龄限制。发行普通股的公司,可以甄选优秀的管理人才,训练这些人才了解公司的政策、方法和技术,以保留和传承公司的活力好几个世代之久。杜邦这家公司已经迈进第二个世纪。道氏公司在才华洋溢的创办人去世多年,依然屹立不摇。目前这个时代,人的欲望无穷,市场潜力惊人,企业的成长毫无限制,不像人有一定的天年。

本章背后的想法或许可以归纳成一句话:如果当初买进普通股时,事情做得很正确,则卖出时机是──几乎永远不会来到。

\section{股利杂音}

关于普通股的投资,在许多层面上,有不少牵强附会的想法和普遍接受的似是而非的说法,但每次谈到股利的重要性,一般投资人混淆得更为严重。

这种混淆和似是而非的说法,甚至蔓延到平常和股利有关的各种习惯性遣词用字上。有家公司一向不发放股利,或者发放很低的股利,后来总裁要求董事会开始大幅发放股利。该公司这么做了。谈到这件事时,总裁或董事会常说,现在该为持股人「做些事情」了。他们的看法是,不支付或不提高股利,等于没为股东做什么事。或许这种说法正确。但不是单单因为没在股利的发放上采取行动,就没替股东做什么事。盈余不以股利的形式发放,而用在建造新厂房、推出新产品线,或在老厂房装设大幅节省成本的设备,管理阶层为股东创造的利益,还是有可能远高于从盈余中提拨股利。投资人不管未发放股利的盈余用途为何,只要股利率能够增加,都视之为「受欢迎的」股利行动。基于更重要的原因,而减少发放股利或不发放股利,几乎总是被投资人看成「不受欢迎的」行动。

投资大众常把股利一事搞混,有个主要的原因,就是每次盈余没有发放给股东,而保留在企业中时,股东获得的利益差异很大。有些时候,保留盈余对他毫无好处。有些时候,得到的好处是负值。如果盈余没有保留下来,他的持股价值会下降。但是保留盈余感觉上无法提高他的持股价值,因此看起来对他没有好处。最后,持股人从保留盈余受益匪浅的很多例子中,同一公司不同类别的股东所得利益殊不一致,令投资人更为混淆。换句话说,每次盈余未以股利的形式发放时,应检讨这种行动的用意,了解到底发生了什么事。在这里,稍微深入表层去观察,并详细讨论其中一些差异,可能有所帮助。

持股人何时无法从保留盈余获得好处?一种情形是管理阶层累积的现金和流动资产,远超过目前和未来经营所需。管理阶层这么做,可能没有不良的动机。有些高阶主管因为不必要的流动性准备稳定增加,而产生信心和安全感。他们似乎不了解自己的安全感,是建立在没有给予股东财富上;股东应该有权利以他认为合适的方式,运用这笔财富。今天的税法倾向于仰制这种罪恶,因此即使仍然发生,已不像以前那么严重。

还有另一种更严重的状况:盈余保留在企业中,但持股人往往没有得到重大的利益。由于管理阶层的素质有欠理想,留在企业中的资金获得的报酬率低于正常水平,保留盈余只好用于扩张欠缺效率的营运活动,而不是设法改善营运活动,便会发生这样的事。通常管理阶层迟早会建立起更大但无效率的领域,而且成功地给自己提高薪水,理由是他们做的事情多于以往。结果,持股人最后得到的利润微乎其微,或者根本没有利润。

按照本书所说观念去做的投资人,不可能受到这些情况影响。他会买进股票,是因为那些股票十分出色,不是只因它们很便宜。一家企业的营运活动欠缺效率,而且管理阶层的素质低于标准,就没办法符合我们所说的十五要点。在此同时,符合前述要点的管理阶层,几乎肯定会为多余现金寻找出路,不是只顾着累聚现金。

保留盈余为企业亟需,为什么有可能无法提高股东的持股价值?原因有二。其一是经常性或大众的需求改变,迫使每一家竞争公司非得花钱在某些资产上不可,但这些资产没办法提高业务量,可是不花这些钱,生意却可能流失。零售店装设昂贵的空调系统,便是典型的例子。每一家相互竞争的商店都装了这种设备之后,净营业收入不会增加,但如某家商店不向竞争对手看齐,没装空调系统,可能发现炎热的夏日里门可罗雀。基于某些奇怪的理由,我们公认的会计制度和税法,根本不区分这种「资产」和真能提高业务价值的资产,所以股东往往觉得受到不公平待遇,因为盈余没有转移到他们手中,而且看不出保留在企业中的盈余,使他的持股价值增加。

保留盈余未能促使利润提高,另一个更重要的原因,起于公认会计方法更严重的缺陷。处在我们这个货币购买力变动既快且巨的世界中,标准的会计处理方式却似乎视货币价值为固定不变。会计师说,会计处理本来就是要做这种事。这种说法很有可能是对的,但假使资产负债表和上面所说资产的实质价值有任何关系,则因此产生的混淆,似乎和工程师、科学家只用两度空间的平面几何,在我们的三度空间世界做运算一样。

现有的资产不再具有经济实用价值时,折旧摊提额理论上应足以置换现有的资产。把折旧率仔细计算出来,而且资产的重置成本在它的可用寿期内保持不变,就会有这样的事。但由于成本节节上升,总累积折旧额很少足以置换过时的资产。因此,如果公司希望继续拥有以前拥有的东西,就必须从盈余中多保留一些资金,补足其间的差额。

这类事情虽然影响所有的投资人,但对成长公司持股人的影响,通常低于其他任何类别的投资人。这是因为购买新资本资产(有别于仅仅置换现有和即将淘汰的资产)的速度通常很快,有比较多的折旧属于最近购置的资产,而这些资产比较接近目前的价值。折旧摊提额中,属于几年前购置的资产比率较低;它们的成本远低于今日。

详述用于建置新厂和推出新产品的保留盈余,在哪些地方对投资人有巨大的利益,未免有重复之嫌,但是某类投资人相对于另一类投资人获益多少,则有仔细探讨的价值,理由有二。整个金融圈老是误解这件事。正确了解这件事,便能轻而易举评估股利的实质意义。

我们假想一个例子,用以探讨人们对谁从股利中获益最多的一些错误观念。管理良好的XYZ公司过去几年的盈余稳定成长,股利率维持相同。四年前,它拿出盈余的五十%发放股利,四年来,获利提高不少,现在支付相同的股利,只需动用今年盈余的二十五%。有些董事要求提高股利。其他一些董事则指出,公司从来不曾见过那么多大好机会,可以拿保留盈余去投资。他们进一步表示,只有维持原来的股利率,而非提高,才可能好好把握所有美好的机会。只有这么做,公司才能取得最高的成长率。到底应该怎么做,双方爆发激烈的争辩。

这个时候,一定有位董事讲出金融圈最常见的似是而非论调:XYZ公司不提高股利,就是牺牲小股东,图利大股东。这句话背后的理论是,大股东的税率级距理当较高,缴纳税款之后,余款占股利的百分率会远低于小股东。因此,大股东不喜欢提高股利,但小股东很希望股利增加。

事实上,XYZ公司的某些个人于股利提高后比较有利,还是应该留下更多资金以挹注未来的成长,和他的所得多寡不太有关系。这事取决于每位股东是不是会挪出部分所得,增加投资。所得税率级距较低的数百万持股人,每年都会挪出一些钱,不管多少,以增加投资。如果他们这么做,而且如果必须缴税(情形很可能如此),那么用小学算术算得出来,运用公司的保留盈余,可以掌握所有美好的机会时,董事会提高股利反而有损他们的利益。相对的,股利提高可能对某位大股东有利,因为他急着用钱;税率级距高的人,也难免有急需。

现在来说明何以如此。任何人只要有足够的多余资金投资普通股,所得几乎肯定相当高,至少必须按最低的所得税率纳税。因此,个人股利免税额五十美元扣除后,即使持股最低的投资人,也必须就其他股利所得缴纳最低二十%的税率。此外,买进股票必须付给经纪商手续费。由于零股收取的费率较高,以及有最低手续费规定等不利因素,小额买进时,这些成本所占比率,远比大量买进时高。这一来,可用于再投资的实际资金,会远低于所领股利的八十%。如果持股人适用的税率级距较高,股利增加部分可用于再投资的百分率会减低。

当然了,有些特殊股东类别,如大学和养老基金,不必缴纳所得税。也有一些个人,股利所得低于五十美元的免税额,但是这些持股人的总股数很低。对这些特殊持股人而言,上面所说的情况有点不一样。至于绝大多数持股人,不管持股多寡,都没办法避免股利必须缴税的基本事实。如果他们的所得都储蓄下来,而非花掉,同时资金能用于投资正确的普通股,则所投资企业的管理阶层把增加的盈余拿去再投资,而不是提高发放股利,对他们比较有利。

股东获得的好处──资金百分之百留在公司为他们效力,而不是领得股利,缴交所得税和经纪商的各项费用之后,大打折扣──不限于此。选到正确的普通股,不是简单容易的事。如果公司认为增加发放股利为好事,则投资人当初选股时一定做得很聪明。所以说,请这批优秀的管理人员运用保留下来的额外盈余,另做其他投资,风险比较低,以免自己必须再冒犯下严重错误的风险,另寻同样突出的新投资对象。考虑要保留或发放盈余增加部分的公司愈是优秀,这个因素愈重要。连不用缴纳所得税和没有把全部所得花掉的持股人也发现,公司保留资金以掌握值得投资的新机会,对他们带来的利益,不亚于必须缴纳所得税的持股人获得的利益。

根据以上所说,股利的真正意义开始浮现。对那些希望善用资金以获得最大利益的人来说,股利不像金融圈很多人所说的那么重要。不管是买进机构型成长股的保守型投资人,或是愿意而且有能力冒更高风险以获得更高报酬的人,情况都是如此。有时人们会说,高股利报酬率是种安全因子。这种说法背后的理论是,高收益股票已提供高于平均水平的报酬,所以价格不可能过高,也不可能跌很多。没有什么事情能悖离真理。关于这个主题,我看过的每一份研究报告都指出,发放高股利的股票中,价格表现差劲的数目,远多于发放低股利的股票价格表现差劲者。本来相当优秀的管理阶层,如果选择增加发放股利,不把增加的盈余再投资于企业中,而牺牲美好的机会,就像农场管理人急着把能够卖掉的肥壮牲畜送到市场,不肯继续养到能卖到最高价的时候。

前面所谈是提高股利的公司,不是针对配发任何股利水平的公司。我晓得,偶尔有些投资人不需要任何所得,但几乎所有的人都偶尔有需要。杰出的公司碰到大好成长良机时,很少见到管理阶层没办法在发放若干股利之后,仍有能力──保留其余的盈余,以及透过发行优先证券筹措资金──取得足够的现金,以把握美好的成长机会。每位投资人都必须决定,相对于本身的需求,有多少资金能投入成长很高,但未发放股利的公司。但最重要的是,所买的股票,不能只强调发放股利,以致于限制了成长实现的机会。

这件事带领我们触及可能最重要,但很少人讨论的股利层面。这是规律性或可靠性的问题。聪明的投资人必须自己做好计划。他会往前看。观察自己有没有能力拿所得资金去做什么事。他或许不在意所得会不会马上增加,但希望获得保障,确保所得不致减少,并因此出乎意料扰乱他的计划。此外,他必须自己做成决定,在会把大部分或全部盈余再投资的公司,以及成长不错,但速度较慢,盈余再投资比率较低的公司间,有所选择。

由于这些理由,股东关系政策订得高明,而且因为这种政策以致股价本益比偏高的公司,思考方向通常能避免把财务人员和财务副总裁搞得昏头转向。他们订好股利政策之后,就不会改变。他们会让股东知道这个政策。他们可能大幅调整股利,但很少调整政策。

这个政策是以盈余应保留多少比率,才能获得最高的成长率为基础。较年轻和成长迅速的公司,可能很多年都不发放股利。接着,当资产折旧回流金额较高时,盈余中的二十五%到四十%会发放给股东。历史较悠久的公司,股利配发比率因不同的公司而异。但是上述两个例子都不以大致相同的比率左右实际发放的金额。因此,每一年的股利金额都和前一年不同。这正是股东所不乐见,因为如此一来,他们很难做长期规划。他们希望能大致依某个比率订定一个固定数字,并且定期发放──每季、半年或一年一次。随着盈余成长,配发金额有时会提高到以前的比率。但是只有在下述情况下才这么做:(a)有足够的资金,可以善用管理阶层发现到的所有美好的成长机会,以及(b)考虑了将来景气转差的所有合理可能性,或者其他成长机会出现之后,仍有充分的理由相信这种新的定期性股利率可以维持下去。

管理阶层如坚持应以十分审慎的态度提高股利,而且只在很有可能继续维持下去的时候,才提高股利,这样的股利政策,最受有眼光的投资人激赏。同样的,只有在最糟的紧急情况中,才能降低股利。很多企业财务主管同意偶尔一次大幅提高股利,这种做法很叫人惊讶。这种出乎意料加发股利的行动,几乎未能对股票市价产生永久性的影响,他们还是执意这么做──由此应可看出,这种政策和大部分长期投资人的愿望背道而驰。

不管股利政策订得聪明还是愚蠢,只要公司始终如一维持原来的政策,通常迟早能引来一群喜欢那种政策的投资人。许多投资人仍然喜欢高报酬率,不管这种政策是否对自己最有利。有些人喜欢低报酬率,有些人则喜欢不发股利。有些人喜欢很低的报酬率,加上每年定期少量发给一些股票股利。有些人不喜欢股票股利,只喜欢低报酬率。如果管理阶层依照自己的自然需求,选定某种政策,通常会吸引到一群股东,喜欢并期望这种政策维续下去。聪明的管理阶层如希望公司的股票营造出投资声望,则会尊重投资人期望政策持续下去的心声。

订定股利政策和餐馆经营政策很像。优秀的餐馆业业主有可能以高价政策把餐馆经营得很好,也有可能以最低廉的价格供应最美味的菜色,把餐馆经营得有声有色。不管是卖匈牙利菜、中国菜、意大利菜,他都有可能经营得很成功。每一种菜都有可能引来一批顾客群。顾客上门之际,总是期望吃到某种菜色。不过,即使他的才能很强,要是某天卖昂贵的菜,隔日出便宜的菜,后来在没有预警的情况下,又卖外国菜,便很难建立顾客群。一家公司的股利政策如变动个不停,也很难吸引到长久的股东群。它的股票不是最好的长期投资对象。

只要股利政策始终如一,投资人便能在获有若干保障的情况下规划未来。整个投资过程中,股利一事便没有那么重要,不必费尽唇舌探讨各种股利政策的相对好坏。金融圈内一定有很多人不赞成这种看法,但他们没办法解释为什么不少股票展望未来,只能提供低于平均水平的收益,持股人却大尝甜头。前面已经提过几支这样的股票,洛姆哈斯(Rohm-Haas)是另一家典型的公司。这支股票一九四九年首次公开上市,一群投资银行家大量买下外国财产托管(Alien Property Custodian)持有的股票,并公开转售。公开上市价格是四一.二五美元,现金股利只有一美元,但另有股票股利。许多投资人觉得,由于收益低,这支股票不是理想的保守型投资对象。但是这一天以后,该公司继续配发股票股利,并经常每隔一段时间就提高现金股利,但收益仍然很低,而且股价远高于四百美元。一九四九到一九五五年,洛姆哈斯公司原来的持股人每年领得四%的股票股利,一九五六年是三%,因此资本利得高于十倍。

其实,想要挑到出色股票的人,对股利一事的关心,应降到最低,不要花太多心思。股利这件人们讨论甚多的问题,最奇特的地方,或许在于最少去烦恼它的人,最后却得到最高的股利报酬。这里值得再说一次:五到十年的一段期间内,最高的股利不是来自高收益股票,而是来自收益相当低的股票。能力突出的管理阶层经营的事业获利可观,虽然继续实施原来的政策,只从当期盈余配发低比率的股利,股利金额却比高收益股票愈来愈高。这种合理和自然的趋势将来为什么不能持续下去?

\section{投资人“五不”原则}

\textbf{一、不买处于创业阶段的公司}


投资成功的准要诀,是找到正在开发新产品和制程,或者正在开拓新市场的公司。刚创立或即将起步的公司,往往试着做这些事。许多这类公司成立的目的,是开发多采多姿的新发明。很多公司创立,是为了参与成长潜力雄厚的行业,如电子业。另一大群新公司,则为了开采矿物或其他天然财富──这个领域的成功果实可能十分甜美。由于这些理由,营运尚未获利的年轻公司,乍看之下似乎具有投资价值。

还有一种论调往往能够提高投资人的兴趣:现在就买进首次公开上市股,才有机会「进入一楼」。某家经营成功的公司,目前股票首次公开上市的本益比只有几倍,因此,为什么要坐等别人把钱赚走?相反的,为什么不用寻找出色老公司同样的研究和判断方法,去寻找正在推销股票的杰出新企业?

从投资的观点而言,我相信,有个基本问题,使得尚未商业营运至少两三年以及一年获有营业利润的公司,和根基稳固的老公司──即使是规模很小的公司,年营业额不超过一百万美元──分属完全不同的类别。就老公司而言,经营事业所有的主要职能都已在运作。投资人能够观察这家公司的生产、销售、成本会计作业程序、管理团队的运作情形,以及营运上其他所有的层面。或许更重要的是,他能取得其他高明观察者的意见,因为他们定期观察该公司一些或全部的相对优点或缺点。相对的,一家公司如仍处于创业阶段,投资人或其他任何人只能看它的运作蓝图,并猜测它可能出现什么问题,或可能拥有什么优点。这事做起来困难得多,做出错误结论的机率也高出许多。

事实上,不管投资人的能力多强,这事做起来极其困难。和判断对象限于老公司相比,从创业阶段的公司中挑选理想投资目标的「安打率」很低。年轻创业公司往往由一两个人主导,他们在若干经营事务上才华洋溢,但欠缺其他同样不可或缺的才干。他们可能是卓越的推销员,但欠缺其他经营能力。他们有可能是发明家或生产高手,却完全不知道即使最好的产品也需要高明的营销技巧搭配。投资人很少能够说服这些人,相信他们本身或他们的年轻组织缺少其他的技能。投资人更难向他们指出,哪里能找到这些人才。

由于这些理由,不管创业公司乍看之下多吸引人,我相信这些公司应留给专业团体去投资。专业团体有优秀的管理人才,在创业公司营运开展之际,一发现弱点,便能提供支持。一般投资人没办法供应这种人才,并说服新的管理阶层相信有必要善用这种协助,如果硬要投资创业公司,到头来往往发现梦想破灭。老公司里面多的是绝佳的投资机会,一般投资散户应严守原则,绝不要买进创业公司,不管它看起来多有魅力。
\\

\textbf{二、不要因为一支好股票在「店头市场」交易,就弃之不顾。}


未上市股票相对于证券交易所上市股票的魅力,和一类股票相对于另一类股票的市场性(marketability)有很密切的关系。每个人都应该确实了解市场性的重要性。正常情况下,买进的股票大部分应限于卖出的理由──可能起于财务上的考虑,也可能起于私人因素的考虑──出现便能脱手。但在这一方面,何者能提供适当的保障,何者不能,投资人似乎混淆不清。未在证券交易所上市的股票是否适合投资,更令他们困惑。这些股票通常称为「店头」(over-the-counter)股。

他们混淆不堪的理由,在于廿五年来普通股投资从根本发生变化──这些变化使得一九五〇年代的市场,和叫人难忘的一九二〇年代有很大的不同。一九二〇年代大部分时候,以及以前的所有期间,股票营业员服务的客户,是数目相当少的有钱人。大部份买盘的数量都很大,往往一买就是几千股。买主的动机通常是以更高的价格卖给另一人。当时的风气可说是赌博,不是投资。融资买进──也就是借钱买股票──是当时人们接受的做法。今天所有买盘中,现款买进的比率很高。

这些年来发生了很多变化,改变了过去多彩多姿的市场。高所得税和遗产税是其一。更重要的影响力量,是美国各个阶层的所得年复一年持续拉近。非常富有和非常贫穷的人,每一年都愈来愈少。每一年,中产阶级人数不断增加。这使得大量股票的买盘稳定萎缩,少量股票买主则成长得很快。此外,另一类股票买主,也就是机构买主,急剧增加。投资信托、养老基金和分红信托,甚至某个程度内的大银行信托部门,并非只代表少数大买主。相反的,他们是专业经理人,受托处理无数小额买主汇聚一处的储蓄。

因果相生之下,我们的法律和机构有了基本上的变化,对股市产生影响。政府创设证券管理委员会(Securities and Exchange Commission),以防堵过去赌风甚炽的股票市场产生的炒作和股友社集团操纵股票的歪风。各项法律规定实施之后,融资买进萎缩,远不如过去视为常规时的盛况。但最重要的一件事,已在前面某章讨论过,亦即今天的公司和以前大为不同。由于前面已经解释过的理由,今天的企业组织,设计得远比以前适合做为投资管道,追求长期的成长性,而非抢进杀出的工具。

所有这些事情,深深改变了市场。毫无疑问,情况大有改善──却是牺牲市场性之后的改善。一般股票的流动性(liquidity)降低而非提高。虽然经济急速成长,以及股票分割的动作似无休止,纽约证券交易所(New York Stock Exchange)的成交量却见减少。至于规模较小的交易所,成交量则几乎消失不见。赌徒、抢进杀出的买主,甚至于容易受骗上当的人,试着赢过集体炒作的股友社,这种情形对经济健全毫无益处。但他们有助于市场买卖容易。

我不想玩文字游戏。不过,我们必须晓得,这些事情导致「股票营业员」(stock broker)日趋没落,或可称之为「股票推销员」(stock salesman)的一群人则崛起。就股票来说,营业员是在标售市场中工作。他必须从某人那里拿一张买进委托单;这个人已经决定好采取哪种投资行动。营业员将这张委托单和他自己或其他营业员的卖出委托单撮合。这个过程不会太花时间。如果拿到的委托单进出的股数很多,虽然营业员处理每一张股票收取很低的手续费,全年下来还是有可观的利润。

推销员则与他不同,必须花费长时间,说服客户采取某些投资行动。一天能够运用的时间就那么多。因此,要赚到与营业员等量齐观的利润,必须收取较高的服务手续费。要是推销员服务的对象是很多小客户,不是少数大客户,更需如此。在今天的经济环境中,小客户是大部份推销员必须服务的对象。

证券交易所的运作方式,主要仍须依赖股票营业员,而非股票推销员。他们的手续费率已经上升,但涨幅只和其他大部分的服务差不多。相对的,店头市场是以相当不同的原理运作。每一天,全国证券自营商协会(National Association of Security Dealers)指定的会员,在地区性报纸上刊登一长串交投较热络的未上市证券报价;当地持股人对这些股票的买卖有兴趣。他们和每一种证券交投最热络的店头自营商密切连系,而编纂出这些报价。这些报价和证券交易所提供的数字不同,不是实际交易的价格范围。他们做不到这一点,因为没有一个集中的结算所,可以向它报出交易价格。相反的,这些报价是进价(bid)和出价(ask),理论上是指任何有兴趣的金融自营商愿意买进每种股票的最高价格,和愿意卖出股票的最低价格。

仔细检查几乎总能发觉,进方或买方的报价很接近提供报价时的股票实际买价。出方或卖方的报价比买进报价高,差价通常是证券交易所中以同样价格出售股票所收收续费的数倍。这个差价经过计算,目的是让店头自营商以进价买入,支付推销员适当的收续费,以酬庸他们花时间销售证券,并在摊销一般间接费用后,仍有合理的利润。另一方面,如果客户,特别是大客户,找上同一家金融自营商,出价买进某支股票,这时就不需负担推销员的手续费,通常能以进价加上相当于证券交易所的手续费买得股票。一位店头自营商这么说过:「买进的一边,我们有个市场。卖出的一边,我们有两个市场。我们有零售和批发市场,部分取决于买进的数量,部分取决于卖出的数量和涉及的服务。」

这套制度碰到厚颜无耻的自营商,明显会遭到滥用。其他制度也是一样。但如投资人挑选店头自营商时,和选择其他任何专家为他效劳一样审慎,这套制度会运作得非常好。一般投资人没有时间,也没有能力自己挑选证券。自营商在严密的督导下,允许他们的推销员只能销售某些证券,这种做法等于让推销员获得投资咨询顾问。因此,值得投入相关的成本。

但对比较精明老练的投资人而言,这套制度的真正好处不在买进股票方面,而在于提高未上市股票的流动性或市场性。这些股票能给自营商够高的获利率,值得他们努力经营,所以很多店头自营商保有平常买卖的股票一定的库存量。市场上有五百股或一千股股票可买时,他们通常乐于买进。碰到他们喜欢的股票供售的数量较大时,他们往往会举行销售会议,特别花点心力去推动这些股票在市场上流通。正常情况下,他们做这件事会要求特别的销售手续费。不过这表示,如果某支店头股票经常有两家或更多家优良的店头自营商买卖,通常会有充分的市场性,满足大部分投资人的需求。视供售数量的多寡,自营商可能要求特别的销售手续费,好在市场上买卖大量股票。不过,由于手续费只占销售价格相当低的比率,投资人希望出售的股票,能在不使市场重跌的情况下,转换成现金。

这和证券交易所上市股票的流动性相比如何?答案主要要看哪种股票和在哪个证券交易所上市。在纽约证券交易所上市的较大型和交投较热络的股票,即使在今天的环境中,还是有个够大的标售市场存在,因此正常情况下,除非是非常大量的股票,否则所有的股票都能以低廉的一般收续费卖出,不致压低价格。至于在纽约证券交易所上市,交投较清淡的股票,市场性依然不错,但当巨量卖出委托单出现,并收取正常的手续费时,有时价格会重跌。在小型交易所上市的普通股,依我之见,市场性问题往往相当严重。

各证券交易所体认到这种状况,已经采取因应行动。今天,交易所认为某笔上市股票的卖出委托单太大,不能以平常的方式交易时,可能允许使用「特别卖出」(special offerings)的机制。这表示,这笔卖出委托单会让所有的会员知道,而且销售这些证券可以收取较高的手续费;手续费率已有明文规定。换句话说,卖出数量过大,营业员没办法以营业员的身分处理时,便以够高的手续费,酬庸他们以推销员的身分销售这些股票。

今天,所有这些做法,拉近了上市和未上市股票市场明显的差距。目前愈来愈多的买盘是由推销员处理,不经只收受委托单的营业员。这不表示从市场性的观点来说,纽约证券交易所交投热络的知名股票,相对于比较好的店头股票占有优势。但比较好的店头股票,流动性往往高于美国证券交易所和各区域性证券交易所上市的许多公司股票。我想,规模较小的证券交易所的相关人士一定十分不同意我的说法。不过,我相信,请个立场公正的人士研究实情,会指出这个看法正确。近年来许多中小型公司成长之后,不愿在规模较小的交易所上市股票,原因便在这里。相反的,它们先选择店头市场,再等公司规模成长到某个水平,适合在「大盘」(big board)──也就是纽约证券交易所──上市股票。

简言之,谈到店头证券,投资人应遵守的原则,和上市证券没有太大的不同。首先,你应十分肯定已选到正确的证券。接下来,确定你已找到能干和尽忠职守的营业员。如果这两件事都做得很好,就不需担心所买的股票是在「店头」交易,还是在交易所上市。
\\

\textbf{三、不要因为你喜欢某公司年报的「格调」,就去买该公司的股票。}


投资人没有经常仔细分析为什么他们买某支股票,却不买另一支股票。如仔细分析,可能很惊讶,因为他们常受公司致股东年报中的遣词用字和格式影响。年报的格调可能反映管理阶层的哲学,政策或目标,一如经过稽核的财务报表应准确反映一段期间的业绩。但年报也可能反映公司的公共关系部门在大众心目中塑造公司形象的能力。我们没办法判断总裁署名的文章真的由他亲笔执笔,或是公共关系部门员工代为捉刀。美不胜收的照片和色彩艳丽的图表,不一定反映管理团队才能出众、同心协力、士气高昂。

让年报的一般性遣词用字和格调影响买进普通股的决定,就像见到广告牌上的广告很吸引人,就去买某样产品。或许买回家的产品,用起来的确像广告说的那么好,但也有可能不然。对于低价产品,用这种方式购买也许说得过去,因为买了之后才晓得到底买得对不对,损失不大。但是对于普通股,很少人那么有钱,能凭一时冲动买股票。我们最好记住:今天的年报通常都经过精心设计,以争取股东的好感。不要光看外表,而应深入观察事实,十分重要。其他的销售工具也都倾向于展现公司「美好的一面」。它们很少平衡和完整地讨论企业经营上的真正问题和困难,而且往往过于乐观。

那么,如果投资人不让自己对年报格调的好感过度影响他的后续行动,则反其道而行好不好?他应让不良的感觉影响他吗?通常不可以,因为同样的,这就好像试着从外包装评断一只盒子的内容。不过这一点有个重要的例外,也就是当年报的内容未能适当揭露投资人觉得很重要的信息或问题时。这么做的公司,通常不可能提供投资成功需要的背景资料。
\\

\textbf{四、不要以为一公司的本益比高,便表示未来的盈余成长已大致反映在价格上。}


我们常见一种错误的投资推理会使投资人付出惨重的代价,因此值得特别注意。为了说明这件事,假设一家叫做XYZ的公司,多年来一直十分符合我们前面所说的十五要点。卅年来,它的营业额和盈余不断成长,而且一直有够多的新产品正在开发,强烈显示未来也能有等量齐观的成长。金融圈普遍赞赏这家公司卓越的表现。因此,多年来,XYZ公司的价格是当期盈余的廿到卅倍,约为道琼卅种工业股价指数采样股平均本益比的二倍。

今天这支股票的本益比正好是道琼指数采样股的二倍,也就是每一美元盈余的市场价格,为道琼指数采样股每一美元盈余平均市场价格的二倍。XYZ公司管理阶层刚发表预测,说他们预期未来五年盈余将增为二倍。以目前已有的证据来说,这个预测相当可信。

可是做成错误结论的投资人数目却多得叫人吃惊。他们说,由于XYZ公司的本益比是一般股票的二倍,而且因为需要五年的时间,XYZ公司的盈余才会增为二倍,XYZ目前的股价已经反映未来的盈余。他们十分肯定股价过高。

没人能反驳,一支股票的价格如果已经反映五年后的盈余,未免显得过高。这种错误的推理,起于假设五年后XYZ公司的本益比会和道琼指数采样股的平均本益比一样。卅年来,这家公司因为种种因素,经营十分出色,本益比一直是其他股票的二倍。这个纪录。令对它有信心的人蒙受利益。如果同样的政策持续下去,五年后管理阶层会推出另一组新产品,未来十年盈余的成长情形,将一如今天新产品提高公司的盈余,以及其他产品五年前、十年前、十五年前和廿年前对公司盈余的贡献。如果这样的事会发生,为什么这支股票五年后的本益比不能是其他普通股票的二倍,就像今天和以前那么多年的情形?果真如此,而且所有股票的本益比保持在原来的水平附近,XYZ公司五年后盈余倍增,也会使得它在这五年内市场价格增为二倍。准此,这支股票目前的价格保持在平常的本益比上,根本不能说已经反映未来的盈余!

这事不是很明显?可是看看你身边,有多少所谓精明老练的投资人把自己搞混,不晓得思考一支股票的价格反映未来的成长时,如何运用本益比做判断。如果公司的背景发生变化,情况更为严重。现在来谈另一家叫做ABC的公司。这两家公司各方面几乎完全相同,但ABC年轻得多。两年来,它的杰出基本面才受金融圈肯定,因此现在的本益比也是道琼卅种工业股票指数采样股平均值的二倍。许多投资人似乎不可能理解,一支股票过去没有那么高的本益比,而现在本益比那么高,反映的可能是它的内在价值,并非不合理地反映未来的成长。

关于这一点,重要的是彻底了解一家公司的经营特质,特别是考虑几年后的可能情形。如果未来盈余急剧成长只是昙花一现,而且公司的事业特质是目前的盈余成长来源用光之后,等量齐观的新来源无法开发出来,情况则很不一样。这时高本益比确实反映了未来的盈余。这是因为目前的冲劲结束之后,股价将回跌,本益比会下降到与其他普通股一样高的水平。但是如果这家公司高赡远瞩,不断开发新的获利来源,而且如果所处行业未来可望有相近的成长冲力,五或十年后的本益比肯定将远高于一般股票,就和今天的情形一样。这种股票的本益比反映未来的程度,往往远低于许多投资人相信者。这也是为什么有些股票乍看之下价格显得过高,仔细分析之后,却是非常便宜的股票。
\\

\textbf{五、不要锱铢必较。}


以上所用都是假设性的例子,为的是把一些事情说清楚。这里要用个真实的例子。廿多年前,有位绅士在大部分地方都展现高超的投资能力,想买纽约证券交易所上市的某支股票一百股。他决定买进的那一天,这支股票以三五.五美元收盘。隔天价格又是如此。但这位绅士不愿以三五.五美元买进。为了省五十美元,他下了三十五美元的买进委托单,拒绝提高价格。这支股票此后未曾跌到三十五美元。约廿五年后的今天,这支股票的前景似乎特别明亮。这些年来,把股利和股票分割算进去,目前的价格远高于五百美元。

换句话说,为了节省五十美元,这位投资人至少少赚了四万六千五百美元。不过,毫无疑问,这位投资人仍有可能赚了这四万六千五百美元,因为他还有这家公司的其他股票,而且是以更低的价格买进。由于四万六千五百美元是五十美元的九百三十倍,因此这位投资人必须省下五十美元九百三十次才能打平。很明显的,采取这种对自己十分不利的理财行动,无异于精神异常。

这个例子绝对不算极端。我故意选多年来涨势落后大盘的这支股票,而不选领先涨势的股票。如果上面所说那位投资人选了纽约证券交易所上市的其他五十支成长股里面的任何一支,则可能为了节省五十美元,失掉三千五百美元的价值,损失比四万六千五百更为惨重。

对于只想买几百股的小额投资人来说,原则很简单。如果想买的股票看起来是合适的股票,而且目前的价位似乎很吸引人,以「市价」买进便是。多花○.一二五、○.二五,或○.五美元,和没买到这支股票失之交臂的利润相比,实在微不足道。要是想买的股票没有这种长期的成长潜力,我相信投资人一开始就不应该去买它。

至于资金较多,想买数千股的投资人,问题没有那么简单。除了非常少数的股票,大部分股票的供应通常有其极限,即使按当时的市价只买想买数量的一半,也很可能导致报价大幅上扬。价格突然上涨,又会带来两项影响,使得买进股票更为困难。价格激涨本身,可能足以促使别人也产生兴趣,竞相买进。另外,本来计划卖出的一些人,惜售之心油然而生,期望涨势持续下去。这时,巨量买主该怎么面对这种情况?

他应去找营业员或证券自营商,把自己到底想买多少数量告诉他们。他应告诉营业员,尽可能买进股票,但授权他不理会小笔供售数量,以免买进之后引来很多人竞相叫进。最重要的是,他应让营业员完全放手在高于最近价格的某一价位以下买进。至于高出多少,必须和营业员或自营商磋商,并考虑想要买进的数量、那支股票平常的成交量、投资人多急着买进股票,以及其他任何有关的特别因素。

投资人可能觉得,他找不到具有良好判断力或做事谨慎的营业员或自营商,可以托付这种重责大任。果真如此,他应该马上去找一位足资信赖的营业员或自营商。毕竟,做这种事正是营业员或证券自营商交易部门的主要职能。

\section{投资人“另五不”原则}

\textbf{一、不要过度强调分散投资。}


没有一种投资原则比分散投资或多样化(diversification)更受人推崇。(有人挖苦说,这个观念很简单,所以连股票营业员都懂!)可能正因如此,一般投资人很少有机会练习多样化不足的做法。「把所有的蛋放在同一个篮子」的可怕下场,大家知之甚详。

但很少人充分想到过犹不及的坏处。把蛋放到太多篮子里,一定会有很多蛋没放进好篮子,而且我们不可能在蛋放进去之后,时时盯着所有篮子。比方说,持有普通股市值达二十五万或五十万美元的投资人中,投资股票种类达廿五种以上的比率高得吓人。吓人的不是廿五种以上的股票,而是绝大部分的例子中,只有少数持股是投资人或他的顾问十分了解的好股票。投资人被过分灌输分散投资的重要性,害怕一个篮子里有太多蛋,使得他们买进太少自己彻底了解的公司,买进太多自己根本不了解的公司。他们似乎从没想过,买进一家公司的股票时,如果对那家公司没有充分的了解,可能比分散投资做得不够充分还危险。他们的顾问更不懂这层道理。

分散投资真的需要做到什么程度?分散投资到什么程度又太危险?这就像步兵架枪。步兵架两支枪的稳定程度,一定不如架五、六支枪。但是五支枪的稳定程度不输给五十支枪。但谈到分散投资的问题,架枪和普通股间有一个很大的不同点。架枪的时候,必须架多少支枪才能稳定,和用哪一种步枪通常没有关系。至于股票,股票本身的特质和实际需要的分散投资程度有很大的关系。

有些公司,如大部分大型化学品制造商,公司内部就有很大的分散投资程度。虽然它们的所有产品都叫化学品,其中许多化学品可能具有完全不同行业中产品的大部分特性。有些可能有完全不同的制造问题。它们可能和不同的竞争对手一较长短,卖给不同类型的客户。此外,有些时候,单是一种化学品,客户群却分布很广的行业,产品本身就有很大的内部多样化效果。

一公司管理人员的广度和深度──指一公司脱离一人管理的程度──在决定需要多大的分散投资保护时,也是很重要的因素。最后,如果持有高度景气循环行业的股票──亦即股价会随着景气良窳而激烈起伏──和比较不受间歇性波动影响的股票比起来,需要用较高的分散投资程度来平衡。

由于股票内部多样化的程度不一,我们很难订定一套一成不变的铁则,说一般投资人分散投资最少需要到达什么程度,才能获得最佳的成果。各行业间的关系,也是应考虑的因素。比方说,一位投资人持有十种股票,每种股票的数量相同,但其中八种是银行股。这种分散投资的做法可能很不恰当。反之,同样这位投资人持有的每一支股票都处于完全不同的行业,则分散投资的程度可能远高于真正需要者。

有鉴于每种情况都不同,而且没有一套精确的准则可资遵循,以下的建议只当做粗略的指引,除了资金非常少的投资人,所有的人或可拿来当做最低分散投资需求的参考。

A.所有的投资或许可以只限于审慎选出来,根基稳固的大型成长股,前面已经提过的道氏、杜邦和IBM,便是典型的例子。这种情况中,投资人可以订定目标,至少拥有五支这样的股票。这表示,原始总投资金额中,任何一支股票不应超过二十%,但不表示万一某支股票成长得比其它股票快,十年后这支股票占他所有股票总市值的四十%时,必须为这么高的持股比率忧心。当然了,前提是他很了解手中的持股,而且这些股票的未来看起来至少和最近的过去一样明亮。

投资人如依原始投资金额的二十%去投资每家公司,应注意五家公司的产品线最多只能略微重迭。举例来说,如果道氏是这五家公司里面的一家,那么我看不出有什么理由,说杜邦不能是其中另一家。这两家公司的产品线很少重迭或相互竞争。如果他想买道氏,以及在它的活动领域中和它很像的另一家公司,则一定要有充分的理由,这种投资才算聪明。拥有营运活动类似的这两支股票,多年内获利可能很可观。但这种情况中,投资人应牢记在心:这样的多样化投资,本质上不适当,因此应时时提高警觉,注意可能影响整个行业的问题出现。

B.他投资的一些或全部股票,可能介于风险高的年轻成长公司和上面所说的机构型股票之间。这些公司有不错的管理团队,而非一人管理的公司,年营业额在一千五百万到一亿美元之间,在业内的根基相当稳固。至少应考虑两家这样的公司,才能平衡每一家A类公司。换句话说,如果只投资B类公司,则可动用的资金中,每一家一开始应只占十%。这一来,全部就有十家公司。不过,这个大类中的公司,风险相差很大。内在风险较高的股票,只占原始投资的八%,而非十%,可能比较慎重聪明。不管如何,属于这一类的每一支股票占原始总投资额的八%到十%──而非A类股票的二十%──应能提供适当的最低分散投资架构。

对投资人来说,B类公司通常比A类机构型公司较难确认。因此,这里值得简短说明我曾有机会密切观察,而且可视为典型例子的一两家这样的公司。

我们来看看本书第一版提到的这类公司,以及它们目前的状况。我提到的第一家B类公司叫梅勒里公司(P.R.Mallory &Co)。我说:

「梅勒里公司的内部多样化程度惊人。它的主要产品是电子和电机业的零组件、特殊金属、电池。比较重要的产品线方面,它都是各相关行业的重量级公司,在某些行业中则是最大的制造商。许多产品线,如电子零组件和特殊金属,供应美国一些成长最迅速的行业,显示梅勒里公司的成长应会持续下去。十年内,营业额成长约四倍,一九五七年达八千万美元左右,约三分之一的增幅来自审慎规划的外部收购行动。约三分之一来自内部成长。

「这段期间的获利率,略低于B类公司被视为令人满意的正常水平,但部分原因起于研究支出高于平均水平。更重要的是,公司已经采取措施,迹象开始显示这个因素将大幅改善。在精力充沛的总裁领导之下,管理阶层展现相当高程度的创意,最近几年在管理深度方面也取得大幅进展。一九四六到一九五六的十年内,梅勒里的股票价值增加约五倍,价格经常是当期盈余的十五倍左右。

「就投资层面来说,最重要的一个因素,或许不在梅勒里公司本身,而是预期它在梅勒里─谢尔隆金属公司(Mallory-Sharon Metals Corporation)将有三分之一的股份。这家公司计划由梅勒里─谢尔隆钛公司(Mallory-Sharon Titanium Corporation)──梅勒里公司拥有一半的股权,也是梅勒里赚钱的一个转投资事业──和国家蒸馏(National Distillers)同一行业中的原物料加工部门合并而成。迹象显示新公司将是成本最低的一贯作业钛制造商之一,因此在这个年轻行业可能出现的成长中,应会扮演重要的角色。在此同时,该公司一九五八年预料将推出第一个具有商业重要意义的锆产品,而且组织内部对其它具有商业用途的新「奇妙金属」,如钽和钶,已有相当丰富的技术知识。它拥有部分股权的公司不只在一种金属,而是在一系列金属上,可望成为全球领导厂商;这些金属势必在原子、化学和明天的导弹年代中,扮演愈来愈吃重的角色。因此,转投资公司可能是十分重要的资产,对梅勒里本身的成长会有重大贡献。」

约两年后的今天,如果再写这段文字,我会写得有点不同。我会稍微降低持股三分之一的梅勒里─谢尔隆金属公司可能做出的贡献。我想,两年前所说的每一件事,还是有可能发生。但是特别就钛来讲,我相信,这种金属寻找和开拓庞大的市场所花时间,可能比两年前所想要长。

另一方面,我倾向于强化对梅勒里本身说过的话,强化的程度和弱化它的关系企业的程度大致相同。这段期间内,我提过的管理阶层深度增加的速度很快。梅勒里虽然是耐久财工业的零组件供货商,经营的业务却会受景气普遍重挫的伤害,但管理阶层展现非凡的灵巧手腕,度过了一九五八年的经济情势,每股盈余虽不及前一年创历史新高纪录的二.○六美元,仍有一.八九美元。一九五九年盈余回升得很快,全年可望创每股二.七五美元左右的新高纪录。此外,这些盈余是在若干较新的事业部门成本负担虽然下降但仍沉重的情况中创下的。这表示,如果整体经济情势仍保持相当的荣面,一九六〇年盈余将进一步大幅成长。

本书所举的例子,到目前为止表现没有整体市场好,反而比整体市场差的股票很少,梅勒里是其中之一。但我认为,电子零组件业务方面,这家公司应付日本竞争对手,做得比若干同业成功;日本同业的威胁,可能是市场反应相对不佳的一个原因。另一个可能原因是金融圈大部分人士对不容易归入某一行业,而是横跨好几个行业的公司较无兴趣。或许随着时间的流逝,这个现象会改变,特别是如果投资人逐渐体认到它的微型电池产品线也有可能打进若干急剧成长的领域;电子产品日趋小型化,微型电池产品应能追随这股稳定的趋势成长。无论如何,本书出第一版时这支股票的价格是三十五美元,考虑之后发放两次二%的股票股利之后,目前的价格是三七.二五美元。


再来看第一版提到的另一支B类股票:



「铍公司(Beryllium Corporation)是B类股票投资的另一个好例子。这家公司的名称有年轻公司的意涵,不知情的人会以为这支股票带有相当高的风险,事实可能不然。它的生产成本很低,也是唯一的一贯作业公司,制造铍铜和铍铝的万能合金,也有一座加工厂,把这种万能合金制成杆、条、板和挤出成型产品,在工具方面,则生产最终产品。一九五七年止十年内,总营业额增加约六倍,达一千六百万美元。销售给电子、电算机器和未来几年可望迅速成长的其他行业的产品,占营业额的比率愈来愈高。铍铜模等重要新用途的产品销售额,开始占有一席之地,过去十年的高成长率可能正足以显示未来也将不错。五年来,它的本益比往往达廿倍左右,高得有其道理。

「新闻媒体引述空军所属一流研究单位蓝德公司(Rand Corporation)的预测,说几乎尚未存在的铍金属结构材料应用领域,一九六〇年代将有重大的突破。这种说法,显示铍公司有可能继续成长好几年。蓝德公司曾经正确地分析,战后不久钛将有重大的发展。」

「在铍成为结构材料的市场最后成型之前,更近的发展是一九五八年应可见到这家公司量产另一种全新的产品,也就是铍金属用在原子用途上。这种产品是利用较旧的万能合金产品线,在另一座工厂生产,和原子能管理委员会(Atomic Energy Commission)订有长期合约。迹象显示它在核能工业有很美好的未来,需求可能同时来自政府和民间工业。管理阶层正密切注意。其实,我们的十五个要点中,除了一点,这家公司的表现都很好。该公司了解自己的缺点,并已采取行动开始矫正。」

就梅勒里的例子来说,两年来我所描绘的画面有得有失。但是到目前为止,有利的发展似乎超过不利的发展,正是适合投资的公司。不利的发展方面,两年前所提的铍铜模美景,似乎失去不少光环,而且整个合金业务的长期成长曲线,可能略逊于当初所说者。在此同时,目前来看,未来几年铍金属的核能需求,似乎不如那时分析的那么乐观。不过,有稳定增强的迹象,显示多种航空用途的铍金属需求可能急剧成长,远超过上述不利发展的影响。这方面的需求已经存在,出现在很多地方和很多不同种类的产品上,难以预测它的极限在哪里。这事或许不如表面上看到的那么有利,因为这个领域可能变得太有吸引力,引来目前不在这一行的某家公司设法在技术上取得重大突破,而带来竞争威胁。幸好,这家公司在十五要点中唯一的弱点上,可能已经大幅自我强化。这个弱点是在研究活动上。

股价对这些事情的反应如何?考虑在那之后各次发放的股票股利,本书第一版撰稿期间,价格是一六.一六美元,今天则是二六.五美元,涨幅为六十四%。

其他几个好例子,虽然我没有那么熟悉,但相信以它们的管理阶层素质、业内地位、成长展望,以及其他特质,归入B类不成问题。它们是佛特矿业公司(Foote Minerals Company)、佛里登计算器器公司(Frien Calculating Machine Co, Inc.)、史普雷格电机公司(Sprague Electric Company)。持有这些公司的股票好几年,获利很高。一九四七到一九五七年,史普雷格电机的股价约增为四倍。佛里登的股票一九五四年公开发行,但不到三年的时间内,股价市值增加约二倍半。股票公开发行前约一年,据称有大量股票私底下换手,以那时的价格买进,到一九五七年增加四倍以上。这两支股票的价格涨幅,对大部分投资人来说,似乎令人满意,但和佛特矿业的涨幅比起来,则显小巫见大巫。佛特矿业的股票一九五七年在纽约证券交易所上市。这之前,股票在店头市场交易,一九四七年首次公开发行。那时股票交易价格约为每股四十美元。一九四七年首次公开发行时买进一百股的投资人,把股票股利和股票分割计算进去,持有到今天,将有二千四百股以上。最近这支股票的价格约为五十美元。

C.最后则是经营成功,获利会很可观,但经营失败将血本无归或损失大部份投资资金的小型公司。我曾在其他地方指出,为什么我相信投资清单上如有这类证券,数量多寡应视特定投资人的处境和目标而定。但是投资这类股票,有两个不错的准则,值得遵守。其中一个已经提过,也就是千万不要把赔不得的钱拿去投资。另一个是资金较多的投资人首次投资这种公司时,每一家的投资金额不要超过总投资资金的五%。我们在其他地方谈过,小额投资人所冒风险之一,是他的资金可能太少,没办法获得这类投资的大好美景,同时取得适当分散投资的好处。

本书第一版中,我提过一九五三年的安培斯公司(Ampex)和一九五六年的艾洛斯公司(Elox)是潜力很大但风险很高的公司,属于C类股票。这两支股票后来表现如何?本书第一版完稿时,艾洛斯的股价是十美元,今天是七.六二五美元。相形之下,安培斯在市场上的表现依然可圈可点,并具体而微地说明:为何出色的管理阶层一旦证明自己能力超群,而且基本面情况未变的情况下,绝对不要单因股价涨幅已大,价格看起来似乎暂时过高,而卖出持股。第三章谈到研究发展活动时,我提到这支股票一九五三年公开发行后四年内,股价涨了七百%。本书第一版完成时,它的价格是二十美元〔已考虑在那之后股票一股分割成二股半〕。今天,营业额和盈余年复一年急增,而且营业额中,八十%的产品四年前还不存在,股价高达一○七.五美元,两年的时间内,涨为四百三十七%,六年内更涨为三千五百%以上。换句话说,一九五三年投资一万美元买安培斯股票,今天的市值超过三十五万美元;这家公司已经证明它有能力在技术和经营上一再有所斩获。

其它一些股票,我没那么熟悉,但大可纳入这一类的有股票首次公开发行时的利顿实业公司(Litton Industries, Inc.)和氢化金属公司(Metal Hydrides)。但就分散投资的观点来说,这类公司有个特色,应谨记在心。它们带有很大的风险,也有那么美好的远景,最后,两件事情里面的一件通常会发生:不是经营失败,便是业内地位、管理深度、竞争优势成长到某个地步,可以从C类转入B类。

这种事情发生时,它们的股票市价通常大幅上扬。这时,要看这段期间内投资人的其他持股价值有了什么样的变化,B类股在投资组合中所占比率可能远高于以往。但是B类股的安全性远高于C类股,可以持有较多的数量,仍不损适当分散投资之效。因此,如果C类公司以这种方式换类,则几无卖出持股的理由──至少不能说,由于市价上涨,这家公司在总持股中所占比率太高,所以应该卖出。

一九五六到一九五七年间,安培斯便从C类公司转成B类公司。这家公司的规模成长为三倍,盈余增加得更快,而且随着它的磁性录音机和零组件市场扩及愈来愈多的成长性行业,这家公司的内在强势增加到可以划入B类。它不再带有极高的投资风险因子。到达这种地步后,安培斯持股在总投资中所占比率可能提高很多,却不违背审慎分散投资的原则。

以上提及的所有比率,只是最低或审慎的分散投资标准。低于这个限制,便像超过正常速度开车,驾驶人可以更快到达目的地。可是他应谨记在心,晓得以那种速度开车,必须格外提高警觉。忘记这一点,不但可能无法以更快的速度到达终点,甚至永远到不了目的地,也就是欲速则不达。

硬币的另一边又如何?是不是有任何理由,显示投资人不应超过如上所述的最低程度分散投资?只要在两件事情上,增加持有的股票看起来和最低的持股种类一样吸引人,就可以这么做。增加持有的证券,应在所涉风险水平上,能够达成和其他持股等量齐观的成长率。投资人增加持股之后,也必须有能力时时注意和追踪所有的投资。但是务实的投资人通常知道,他们的问题出在如何找到够多的出色投资,而不是在很多的投资中做选择。有时投资人的确找到多于自己真正需要的好公司,却很少有时间密切注意所有增加买进的公司动态。

证券投资清单很长,通常不是聪明投资人该有的做法,反而透露他对自己的看法很没把握。如果投资人持有的股票太多,无法直接或间接掌握相关公司管理阶层的动态,我们可以相当肯定,他的下场会比持有太少股票要惨。投资人应了解,投资难免犯错,所以应适当分散投资,以免偶尔犯错让自己一蹶不振。但是超过了这一点,应十分小心谨慎,不要只顾尽量持有很多股票;只有最好的股票才买。就普通股来说,多不见得好,但略多一些未尝比持有少数出色股票差。
\\

\textbf{二、不要担心在战争阴影笼罩下买进股票。}


富有想象力的人,对普通股通常很感兴趣。但现代战争的恐怖往往窒息我们的想象力。因此,我们的世界每次出现国际紧张形势,带来战争一触即发的威胁或实际爆发战争时,普通股总是有所反应。这是种心理现象,在理财投资上毫无意义。

战争大量杀戮人命和带来痛苦,令正常人惊骇不已。今天的原子时代中,我们更加担心亲人和本身的安危。未来可能爆发战争的这种忧虑、担心和厌恶,往往扭曲我们对纯经济因素的评估。财物大量毁于一旦、形同充公的高税率、政府干预企业等忧虑,主宰我们在投资理财事务上的思虑。在这种心态下投资理财的人,常会忽视一些更重要的基本面经济影响力量。

结果总是一样。整个廿世纪,除了一次例外,每次大规模的战争在世界任何地方爆发,或者每当美军卷入任何战斗,美国的股票市场总是应声重挫。例外的一次是一九三九年九月二次世界大战爆发。起初人们认为中立国将有获利可观的战争合约生意上门,股价因此弹升,但涨势终究无法持久,马上转为典型的下降走势。几个月后,德国胜利的消息日多,跌势变为恐慌性杀盘涌出。不过,所有实际的战斗尘埃落定之后──不管是一次世界大战、二次世界大战,或者韩战──大部分股票的价格都远高于战前。此外,过去廿二年,至少十次曾有国际危机可能演变成大规模的战争。每一次,股价都先在战争忧虑下重跌,战争忧虑消散之后强劲反弹。

投资人在战争忧虑和战争实际爆发时抛出持股,但等到战争结束,股价总是涨得更高,而非下跌,他们到底忽视了什么?他们忘了股票的报价是以金钱表示。现代战争总是使政府在战争期间的支出,远高于从纳税人身上征得的收入。这使得货币发行量大增,每一单位的金钱,如一美元,价值不如从前。买进同样数量的股数,花的钱远多于从前。当然了,这是典型的通货膨胀形式。

换句话说,战争总是使金钱变薄。因此,在战争一触即发,或实际爆发时,卖出股票换成现金,可说极不懂投资理财之道。其实,该做的事恰巧相反。如果投资人本来就决定好要买进某支普通股,可是全面爆发战争的阴影突然来袭,导致股价重跌,那么他应该抛弃当时的恐惧心理,勇往直前开始买进。这是保留多余投资现金最不当,而非最理想的时刻。但是有个问题必须考虑。他应买得多快?股价会跌得多深?只要影响股价下跌的因素是战争的威胁,而非战争本身,我们就没办法知道这些事情。如果战事真的爆发,价格无疑会跌得更低,而且可能低很多。因此,我们应做的事是买进,但缓缓买进,而且面对的如只是战争的威胁,则应小规模买进。战争一发生,则大幅加快买进步调。所买公司生产的产品或提供的服务,必须在战时仍有需求,或者能转换设备,以因应战时之需。在今天全面战争和生产弹性的情况下,绝大部分公司都符合以上所说条件。

到底是股票在战时变得更值钱,或者只是因为钱变薄的缘故?这要看情况而定。邀天之幸,美国参与的战争,从没打过败仗。战时,特别是现代战争期间,战败一方的货币可能变得一文不值或一落千丈,普通股将丧失大部分的价值。当然了,要是美国遭共产苏联击败,则美国的货币和美国的股票会变得没有价值。这时,不管投资人怎么做,结果都没有差别。

相反的,赢得战争或者战事陷入胶着状态,则股票的真正价值要看个别战争和个别股票而定。第一次世界大战期间,英国和法国战前的庞大储蓄涌入美国,大部分股票的实质价值涨幅,可能比当时如是承平时期要高。但这只是仅此一次的现象,不会重演。以定值货币──也就是实质价值──来表示,第二次世界大战和韩战期间,美国的股票的确不如当时如是承平时期要好。除了沉重的税负,太多的心力从获利较多的承平时期生产线,转向获利率极低的国防业务。如果花在这些获利率低的国防项目上的大量研究活动,可以投入承平时期的生产线,持股人的获利应会高出很多──当然了,这得假定美国仍是个自由国家,持股人能享受企业创造的利润。趁战时或战争的恐惧弥漫之际买进股票,原因不在于战争本身可能再次让美国的持股人获得利润,而是因为相较之下,持有现金较不理想,但以货币单位表示的股价总会上涨。
\\

\textbf{三、不要忘了你的吉尔伯特和沙利文。}


吉尔伯特和沙利文(Gilbert and Sullivan)很难被视为股市的权威。不过,正如他们告诉我们的,我们或可谨记在心:「花朵在春天盛开,可是那没有关系」。有些肤浅的金融统计数字,许多投资人往往奉为至宝。拿它们与吉尔伯特和沙利文春天盛开的花朵相比拟,也许过分夸张。我们不说它们没有关系,而要说无关大局。(译注:吉尔伯特〔William Schwenck Gilbert〕是英京剧作家,沙利文〔Arthur Seymour Sullivan〕是英国作曲家,两人曾合写十四部以机智及嘲讽点缀成的喜歌剧。)

这些统计数字中,首先要提的是一支股票前几年的价格区间。基于某些理由,很多投资人考虑买进某支股票时,想看的第一样东西是一张表,列出过去五或十年中,每一年的最高价和最低价。他们在心里经过一番琢磨,终于得出一个很漂亮的整数,愿意以那个数字买进。

这是不合理性的做法?投资理财上,这么做很危险?这两个问题的答案是句响亮的「是」。这种做法危险,在于强调不怎么重要的事,注意力反而脱离重要的事。这往往导致投资人只顾贪图小利,坐失厚利。要了解何以致此,我们必须看看这种心理过程为什么不合理性。

股票为什么会以某种价格出售?这是所有对这支股票有兴趣的人,当时认为这支股票可能应有的正确价值的一个综合估计值。所有潜在的买方和卖方,对这家公司的前景做了综合评估,再经每位买主或卖主打算买进或卖出的股数加权,相对于同一时刻其他公司个别前景的类似评估过程,而得出一个特别价位。偶尔有些因素,如被迫脱售持股,会使价位温和偏离这个数字。大持股人由于某些理由──如缴纳遗产税或偿还贷款──在市场上抛售股票时,便会发生这种事,但可能和卖方对这支股票的实质价值的看法没有直接关系。这种压力通常只使价位温和偏离所有投资人对这支股票的综合评估,因为逢低承接者往往会进场买进,使得价位自行调整。

真正要紧的事,在于价格是以投资人对当时情况的评估为准。一公司事务发生变化为人所知之后,投资人的评估会随着转趋有利或不利。于是这支股票的价格相对于其他股票会上涨或下跌。如果有待评估的因素经正确分析,这支股票相对于其他股票的价值就会永远较高或较低,于是保持上涨走势或下跌走势。如果更多相同的因素继续出现,整个金融圈就会逐渐认清它的价值。这一来,股价就会波动,进一步上涨或下跌。

因此,四年前某支股票的价格,可能和目前的价格几无或根本没有实质关系。这家公司可能培养出很多能干的高阶主管新面孔、很多高获利的新产品,或者其他类似的特点,使得它的股票本质上相对于其他股票的价值,四倍于四年前,这家公司也有可能落入做事缺乏效率的管理阶层手中,相对于竞争对手严重退步,起死回生的唯一方法是筹措大量新资金。这一来,股权稀释,今天的股票价格可能不到四年前的四分之一。

根据以上所说,可以看出为什么投资人往往错失可能带来厚利的股票,只为了追求那些绳头小利。过分强调「还没上涨的股票」,无意中养成一种错觉,以为所有的股票都会上涨相似的幅度,已经上涨很多的股票不会再涨,而还没上涨的股票,则「该」上涨。没有什么事情能够违背真理。决定现在要不要买进某支股票时,它过去几年已经上涨或者没有上涨,根本无关紧要。真正要紧的是这些年来各项作业改善是否足够,或者将来会有足够的改善,使得价格高于目前的水平高得有道理。

同样的,许多投资人过分重视过去五年的每股盈余,藉以决定要不要买进某支股票。观察每股盈余本身,并重视四、五年前的盈余,就像一具引擎不再和它施力运作的机器联机后,还希望那具引擎发挥功效。仅仅知道一公司四、五年前的每股盈余是今年盈余的四倍或四分之一,很难藉以判断应该买进或卖出某支股票。同样的,重要的是对背景状况有所了解。了解未来几年可能发生什么事,十分要紧。

投资人不断收到一大堆报告和所谓的分析,内容主要围绕着过去五年的价格数字打转。他应该牢记在心:现在对他重要的事情,是未来五年的盈余,不是过去五年的盈余。他会拿到旧统计数字的原因,在于放入报告中的这种数据,肯定是正确的。如果放进更重要的数字,则将来情势的演变,可能使那份报告看起来很蠢。所以说,报告撰稿人有强烈的倾向,腾出尽可能多的篇幅,填入大家无争辩余地的事实资料,而不管那些事实资料到底重不重要。可是金融圈内很多人强调往年统计数字,理由不一而足。他们似乎无法理解,仅仅几年之内,若干现代公司的实质价值可能有多大的变化。他们因此强调过去的盈余记录,打从心底相信去年详尽的会计数字描述,能画出明年将发生事情的真正面貌。一些受到高度管制的公司,如公用事业,或许真的如此。至于想利用投资资金赚取最大报酬的投资人,所找的企业,则完全不然。

这方面有个绝佳的例子,事件的发展,我有幸相当熟悉。一九五六年夏,有个很好的机会,能够向德州仪器公司(Texas Instrument, Inc.)一些重要高阶主管买到不少股票;这些人也是德州仪器的最大股东。仔细研究这家公司之后,发现它不只通过我们的十五要点检定,而且表现非常突出。这些高阶主管出售持股的理由,似乎十分合理;真正的成长公司往往出现这样的事。他们的持股价值已增加很多,就他们持有自己公司的股票来说,其中几人早成百万富翁。相对的,他们的其他资产微不足道。因此,特别是因为他们只卖出一小部分持股,有必要分散投资。对这些重要企业高阶主管来说,不管公司前途如何,单是遗产税问题,便有必要做出明智的行动。

总之,买卖双方磋商完毕,同意以十四美元的价格成交。这是一九五六年预估每股盈余约七十美分的廿倍。在那些特别重视旧统计数字的人眼里,以这种价格买进实在很不聪明。一九五二到一九五五四年内,公司报告的每股盈余分别是三十九、四十、四十八、五十美分──这种纪录很难叫人击节赞赏。有些人甚至认为比较重要的管理阶层素质和眼前的经营趋势,反不如肤浅的统计数字重要,更令他们泄气的是,这家公司经由收购行动,取得若干损失递延的好处,这段期间大部分时候负担较轻的所得税。这使得以过去的统计数字为基础,计算出来的价格显得更高。最后,即使把一九五六年的盈余纳入评估,粗略研究当时的情势,还是得出悲观的结论。没错,这家公司目前在前途看好的晶体管业做得非常好,而且整个半导体业显然有很明亮的未来,但像它那种规模的公司,和更大、更悠久、财力远为雄厚的公司相比,目前的强势地位能维持多久?后者肯定会积极投入,在成长展望美好的晶体管业一竞长短。

透过正常的证券管理委员会(SEC)管道报告高阶主管出售持股的消息传出之后,德州仪器的股票成交量激增,价格几无波动。我猜想,大部分的卖盘是各经纪商的评论引发的。很多经纪商引用过去的统计纪录,并就向来偏高的股价、未来的竞争和内部人卖出持股发表评论。有份报告甚至提到德州仪器的管理阶层完全同意它的说法。报告中提及那些出售持股的高阶主管说:「我们同意他们的看法,并建议采取相同的行动!」据我了解,这段期间的大买盘来自一家消息灵通的大型机构。

接下来十二个月发生了什么事?不久前投资人议论纷纷时,没人注意的德州仪器地球物理和军事电子业务继续成长。半导体(晶体管)事业部成长得更快。比晶体管业务成长迅速还重要的事,是能干的管理阶层采取大动作,从事研究发展、推动机械化计划,并在这个十分重要的半导体领域建立配销组织。证据愈来愈多,显示一九五六年的业绩并非昙花一现。反之,这家规模相当小的公司,在可望是美国成长最迅速的工业中,将继续是最大和成本最低的制造商之一。金融圈因此开始向上修正这家公司合理的本益比,好把握机会,参与这家经营良好的公司。一九五七年夏,管理阶层公开发表谈话,估计那一年的每股盈余约为一.一○美元,成长五十四%,以致于仅仅十二个月,股价市值上涨约一百%。

本书第一版中,我继续说道:

「我怀疑,如果这家公司主要事业部的总部不是设在达拉斯和休斯顿,而是位于大西洋北半部沿岸,或洛杉矶大都会区──这里有比较多的金融分析师和重要基金的经理人,对这家公司能够多了解一点──那么这段期间内,它的本益比或许会更高些。由于未来几年德州仪器的营业额和盈余可能继续大幅提高,这种持续性的成长迟早会使本益比再往上升。如果这样的事情发生,它的股票价格会再次上扬,速度甚至比盈余还快。两者交互影响,总是带动股价激剧上涨。」

这个乐观的预测有没有获得证实?对那些仍坚持可以根据过去的盈余做肤浅的分析,进而分析某支股票有没有投资价值的人来说,德州仪器的纪录可能叫他们大吃一惊。每股盈余从一九五七年的一.一一美元上升到一九五八年的一.八四美元,一九五九年可望超过三.五美元。本书第一版完稿以来,这家公司获得无数荣耀,势必紧紧吸引金融圈的目光。一九五八年,面对一些普遍公认的电子和电机设备业巨擘的竞争,全世界最大的电子计算器器制造商国际商业机器公司(International Business Machines Corporation),选择德州仪器为合作伙伴,共同研究半导体在电子设备上的应用。一九五九年,德州仪器宣布技术上有所突破,利用和当时的晶体管几乎同样大小的半导体材料,不只可以取代一个晶体管,甚至能制造整块电路板!机器设备可望趋于小型化,超乎想象。随着公司的成长,能力十分突出的产品研究发展单位等比例扩大。今天,消息灵通的人士几乎不敢怀疑这家公司一长串的技术和经营上的「第一」,不能持续到未来好几年。

股票的市场价格对这些事情有什么样的反应?本益比是不是持续上升,一如廿二个月前我说过这件事有可能发生?纪录似乎是肯定的。一九五七年以来,每股盈余增为三倍多一点。股价从本书第一版完稿时的二六.五美元上涨超过五倍。顺便一提,本书第一版曾提到,不到三年半前,有人买进不少股票,进价十四美元,目前的价格则是此数的一千%以上。虽然股价已经激涨,但我们很感兴趣,想知道未来几年营业额和盈余进一步成长能不能使股价涨得更多。

这又让我们想起另一种推理方式,导致若干投资人过份注意过去的价格区间和每股盈余等不相干的统计数字。他们相信,过去几年发生的事,势必无限期延伸下去。换句话说,有些投资人会找到过去五年或十年中,每年每股盈余和市价都上涨的股票,并且做成结论说,这个趋势几乎肯定会无限期持续下去。我同意这种情形有可能发生。但为了取得成长,必须进行研究发展,何时能获得成果,时间难断,而且推出新产品很花钱,所以即使最杰出的成长型公司,盈余成长率也难免偶尔一到三年下挫。盈余成长率下挫会使股价重跌。所以说,强调过去的盈余记录,却忽视左右未来盈余曲线的背景状况,可能让投资人损失不赀。

这是不是表示,决定要不要买进某支股票时,过去的盈余和价格区间应完全忽视?不是的。我的意思只是说,投资人重视它们的程度,不应高到引来危险的地步。只要投资人了解它们只是辅助工具,适用于特殊目的,不是决定一支普通股有没有吸引力的主要因素,它们就很有用。比方说,研究前几年每股盈余的起伏情形,可以了解一支股票的周期性,也就是企业盈余受各景气阶段影响的程度。更重要的是,比较过去的每股盈余和价格区间,可以了解一支股票过去的本益比。这可以当做起步的基础,根据它们衡量未来可能的本益比。但是同样的,应谨记在心:主宰股价走势的是未来,不是过去。也许一支股票过去几年的价格一直稳定维持在只为盈余八倍的水平,但现在,管理阶层更迭、公司建立起一流的研究部门等事情发生,使得股票目前的价格是盈余的十五倍左右。如果有人估计未来的盈余,并算出这支股票的预期价值只有盈余的八倍,不是十五倍,那么未免过分仰赖过去的统计数字。

这一小节我下的标题是「不要忘了你的吉尔伯特和沙利文」。或许我应写成「不要受无关紧要事务的影响」。以前的盈余统计数字,特别是每股的价格区间,往往「无关宏旨」。
\\

\textbf{四、买进真正优秀的成长股时,除了考虑价格,不要忘了时机因素。}


我们来谈谈经常发生的一种投资情况。根据我们的十五要点建立的标准,有家公司非常符合。此外,约一年后,获利能力将有长足的进步,但将导致盈余激增的因素,金融圈还没有完全清楚。更重要的是,强烈的迹象显示,新的获利来源至少将大幅成长好几年。

正常情况下,这支股票显然应该买进。但有个因素令我们暂时却步。前些年其他事业经营得很成功,这支股票大受金融圈垂青,若不考虑那些普遍未知的新影响因素,这支股票的价格在二十美元左右,可能被视为相当合理,但目前竟涨到不合理的三十二美元。假设五年后这些新影响因素能使股价轻易涨到充分反映其价值的七十五美元,那么我们现在是不是应以三十二美元──比我们认为这支股票应有的价位高六十%──的价格买进?新情势总有可能不如原先想象得那么好。而且,股价也有可能掉回我们认为应有的价位二十美元。

面对这种情况,许多保守型的投资人会密切注意报价。要是股价跌到接近二十美元,他们会积极买进,否则便不理会这支股票。这种事常发生,值得进一步分析。

二十美元这个数字很神圣吗?不然,因为它没有考虑未来价值中一个重要的成分──我们知道,但其他大部分人不知道的因素,而且我们相信未来几年这些因素会使它的价格涨到七十五美元。这里真正重要的是,我们能找到一种方法,以接近低价的价位买到股票,看着它从那个价位往上爬。我们关心的是,如果以三十二美元买进,后来股价可能跌到二十美元左右。这不只让我们暂时出现损失,更重要的是,要是股价后来涨到七十五美元,则以同样的钱,我们在三十二美元买到的股数,比耐心等候它跌到二十美元再买进少约六十%。假设廿年后,其他新的事业推升股价上升到二百美元,而非七十五美元,则同样的钱能买到多少股数,便格外重要。

幸好,碰到这样的情况,有另一个路标可资依赖,保险业和银行业的一些朋友甚至觉得,即使如履薄冰,也一样安全。这个做法就是不在特定的价格买进股票,而在特定的日子买进。研究这家公司过去其它成功的经营计划,可以发现这些经营计划在发展阶段的某一点,便会反映在股价上,或许那是在这些经营计划到达试车阶段之前平均约一个月。假设股价仍为三十二美元左右,那么何不准备在五个月后,也就是试车工厂开始运转之前一个月,买进股票?当然了,那时买进股票后,股价仍有可能下跌。不过,即使我们在二十美元买进,还是不能保证股价不会下跌。如果我们有相当不错的机会,能在尽可能接近低点的价位买进,即使我们觉得,以大家都知道的因素为基础,股价可能再跌,我们不是仍能达成目标吗?这种情况下,在某个日期买进,不是比在某个价位买进安全?

基本上,这个方法一点也没忽视股票的价值,只是表面上看起来忽视而已。要不是未来价值有可能大幅上升,则金融圈一些朋友宣称应在某个日期买进,而不是在特定的价位买进,就显得不合逻辑。但如迹象强烈显示价值势将提高,则在某个日期买进,不在某个价位买进,就有可能在最低价位或接近最低价位的地方,买到将进一步大幅成长的股票。总之,买进任何股票时,这正是应试着去做的事。
\\

\textbf{五、不要随群众起舞。}


有个重要的投资观念,如果没有相当丰富的投资理财经验,往往不容易理解,因为这个观念很难用精确的文字解释清楚,也无法化繁为简,以数学公式表达。

本书一再谈到各种不同的因素,影响普通股价格上涨或下跌。企业纯益升降、公司管理阶层更易、新的发明或新的发现问世、利率或税法改变──这只是随便举出的一些例子,用以说明哪些情况会使特定普通股的报价上涨或下跌。所有这些影响因素有个共同点。它们是我们生存的世界中真正发生的事情。它们已经发生或即将发生。现在我们要谈一种很不一样的价格影响因素,纯粹起于心理面。外在或经济世界压根儿没有发生变化,但绝大多数的金融界人士从迥异于以往的观点,观察完全相同的情境。由于评估同样一组基本事实的方式改变,结果同样一支股票,他们愿意支付的价格或本益比跟着改变。

股市里面有流行,也有狂热,一如女装。这些流行或狂热可能几年一次,扭曲目前的价格相对于实质价值的关系,影响之大,不亚于商人某年碰到流行及地长裙,不得不舍弃一堆高质量及膝洋装。举个实际的例子:一九四八年,我曾和一位绅士聊天。我相信,担任纽约证券分析师学会(New York Society of Security Analysts)理事长的他,是很能干的投资专家。这个职位通常授予金融圈内比较出色的人才。闲话莫表,到纽约之前,我刚拜访过密执安州密德兰的道氏化学公司总部。我提到,快要结束的那个会计年度,道氏的盈余会创下新高水平,我认为这支股票很值得买进。他答道,他觉得像道氏这样一家公司,能赚那么高的每股盈余,可能只具历史意义,或者只能留待将来做统计研究之用。也就是说,这种盈余水平不会使道氏股票变得吸引人,因为明显可以看出,这家公司只是绽现战后一时的荣景,无法持久。他进一步指出,美国内战和一次世界大战几年后发生的类似战后景气萧条出现之前,不可能评断这样一支股票的真正价值。令人遗憾,他的推理完全漠视这家公司当时正在开发许多有趣的新产品,可能对股票的价值有进一步的帮助。

后来道氏的盈余不曾跌到接近这个若干人以为的异常高峰,股价也从若干人以为的当时高价,继续往上攀升好几倍。这些事实,不是此处探讨的重点。我们想知道的是,为什么那位能干的投资专家,面对相同的事实,却对道氏股票的实质价值得出和几年前不同的结论。

答案在于一九四七到一九四九三年内,几乎整个金融圈沉溺在集体错觉中。事后评断往事很容易,所以我们现在能够舒舒服服地坐着,笑看一四九二年哥伦布的随行船员极感恐怖的事情,其实有悖事实。圣马里亚号(Santa Maria)上的大部分船员夜夜无法安枕,因为他们极端恐怖,害怕船只随时可能从地表的边缘掉落,尸骨无存。一九四八年,投资圈不重视任何普通股的盈余价值,因为大家普遍相信,近期内出现严重的经济萧条,以及股市大崩盘在所难免,因为前面两次大战后几年内都有相同的情形。一九四九年的确出现轻微的萧条。程度没那么严重的事实为人认清之后,金融圈发现随后的趋势是往上,不是向上,于是心理因素丕变,对普通股的观感大不相同。接下来几年内,许多普通股的价格上涨一倍以上,原因不过是心理面因素改变。有些普通股受益于更有形的外在因素,基本面改善,涨幅不只一倍。

金融圈在不同的时间,对相同事实的评估方式大异其趣的现象,绝不限于整体股市。特定行业和这些行业中的个别公司,金融圈的喜恶不断变动,原因往往出在时移势转,相同的事实却有不一样的解读方式。

举例来说,某些期间内,投资圈认为国防工业有欠魅力。国防工业最突出的特性之一,是它只有一位客户,也就是政府。这位客户有些年头大量采购军事装备,有些年头则一路缩减支出。因此,这个行业无从得知下一年会不会有大合约取消,业务枯竭。

除此之外,还须考虑承包政府的工作,获利率一向很低,而且法律规定,如果计算错误,厂商已赚得的利润必须吐回,承受的损失则自行负担。另外,厂商必须不断竞售最新型的设备,而在这个领域中一直更改工程设计,所以高风险和混乱是业内常态。不管工程设计得多好,任何事情都不可能标准化,让你的公司相对于积极进取的竞争对手,长期占有优势。最后,总是有可能碰到和平降临的「危险」,以致于业绩每下愈况。过去廿年,投资人很多次盛行这种看法,使得国防类股的价格相对于盈余偏低。

不过最近金融圈有时从相同的事实数据得出其他的结论。依目前的全球情势,多年内必须大量支出,研制空防设备。虽然每年的总值可能有所变化,但工程设计变动的步调,使我们需要愈来愈昂贵的设备,因此长期是向上的趋势。这表示投资这些证券的快乐投资人,置身于不会感受到下一波景气萧条的少数行业之一,但其他大部分行业迟早会面临这种困境。虽然利润率受到法律的限制,经营良好的公司还是有很多业务可做,因此总净利并无上限。投资人普遍持有这种看法之后,同样的背景事实便有相当不同的评估结果,股价水平自然与以往大不相同。

我们可以举出很多实例,说明过去廿年,金融圈最初以某种方式观察某种行业,后来又改用另一种方式,股票价格随之起伏不定。一九五〇年,制药类股普遍被视为具有工业化学公司同样的理想特质。卓越的研究成果带来无止尽的成长潜力,加上生活水平稳定提高,最好的制药业股票和最好的化学业股票本益比一样高,似有其道理。后来,某家制造商以前销售得很好的一样产品出了问题,金融圈马上闻风色变,觉得这个行业中,今天居于龙头地位,不能保证明天仍是主要公司之一。于是他们重新评估整个行业,得出完全不同的合理本益比水平,原因不在事实背景改变,而是对相同的事实背景阐释不同。

一九五八年的情形恰好相反。这一年,景气不振,制药业是少数产品需求未跌反增的行业之一。业内大部分公司的盈余上升到新高水平。在此同时,化学制造商的盈余急剧下滑──主要因为刚完成的大规模扩张行动导致产能过剩。善变的金融圈再次开始大幅提高制药类股的本益比。同时,愈来愈多投资人开始觉得化学类股不如早先想象的那么吸引人。所有这些,不过反映金融圈看法上的转变,基本面或内在条件并未发生变化。

一年后,投资人一些新的看法已经反转。比较优秀的化学公司率先恢复失落的利润率,随着成长趋势带动盈余迅速升抵新高水平,他们很快重拾暂时失去的声望。在政府紧抓制药业不当的订价和专利政策的不利环境背景中,重要的新药品数目愈来愈多,将带来长远的效益,可望进一步支撑制药类股的地位。因此,我们很感兴趣,乐于观察制药类股最近重获的地位,未来几年会进一步成长或开始衰颓。

本书第一版中,我又提到一个当时最新的例子,说明金融圈看法改变的情形:

「现在又有一个展望改变的例子。多年来,工具机制造商的股价相对于盈余一直很低。大家几乎一致认为,工具机是景气大好大坏轮替不休的行业。不管盈余有多好,都没有太大意义,因为那只是眼前荣景的产物,无法持久。但最近有个新学说,虽然还没有取得主流地位,却有愈来愈多人相信。这派学说相信,二次世界大战以来,基本面已经发生变化,影响到这些公司。所有产业的资本支出计划已从短期规划转为长期规划。因此,导致工具机业景气波动激烈的原因已经消失。工资率偏高,而且还在上升,即使无法永久改变,也能使这个行业多年内不再出现大好大坏轮替的现象。工程设计稳定进步的速度已经加快,将进一步加速这个行业产品淘汰的步调。因此,工具机业可望摆脱战前周期性的趋势,最近的成长趋势会继续延伸到未来。自动化可能使这个成长趋势变得十分突出。

「受到这种想法的影响,相对于整体市场,较优秀的工具机业股票现在得到比几年前要好的评价。它们的价格相对于本益比仍然偏低,因为大好大坏轮替的想法依然挥之不去,但不如以往那么强烈。如果金融圈日益接受工具机类股不再是周期性行业,而且成长展望乐观的看法,它们的本益比会再上升。这一来,它们的表现就会远优于市场。要是大好大坏的旧观念再占上风,它们的本益比就会低于目前。

「普通股投资人如希望买进股票获得最大的利益,这个工具机业的例子,显然可以减轻负担。他必须根据事实面做分析,探讨当时金融圈对整个行业和他打算买进的特定公司盛行的看法。如果他能找到某种行业或者某家公司,当时金融圈的一般看法远比事实资料呈现的要差,则不随群众起舞,可能获得额外的报酬。买进的公司和行业,如是当时金融圈的最爱,则应格外小心谨慎,确定它们真的值得买进──有些时候的确很值得买进──以及自己不致于付出高价,买到金融圈对基本事实的解读过份有利,成为当时投资狂热的股票。」

当然了,关于工具机业本质上不再是大好大坏相生的行业一说,我们已知道答案为何。一九五七年的经济衰退,彻底粉碎了长期企业规画足以做为缓冲,使得工具机业不再受苦于景气循环中衰退压力的说法。但是今天科技变迁的脚步愈来愈快,这类问题每解决一个,又带来更多的问题,聪明的投资人如能自外于群众,独立思考,在绝大部分人的意见偏向另一边时,提出自己的正确答案,将获益匪浅。「奇特的」能源股和一些小型电子股,以它们的内在条件来说,今天得到那么高的评价,当之无愧吗?超音波设备制造商的前景真的那么明亮,普通的本益比可以弃之不顾?一家公司如有极高的盈余来自海外业务,对美国投资人是好是坏?这些问题,投资大众的想法目前可能过偏,或者还没过偏。聪明的投资人想要参与受影响的公司时,必须确定何者是会持续下去的基本面趋势,何者只是一时的狂热。

这些投资狂热和不正确地解读事实数据,可能持续好几个月或好几年。但长期而言,真实状况不只将终结这些现象,而且往往暂时导致受影响的股票往反方向走得太远。一个人如有能力透视绝大多数人的看法,发现事实真相到底如何,投资普通股将获得优渥的报酬。不过,我们身边的人不容易形成一种综合性的意见,强烈影响所有人的想法。但是有件事情,每个人都可看得出来,而且在不追随群众起舞上,能给我们很大的帮助。这件事就是金融圈通常很慢才看出基本面已经发生变化,除非有知名人物或喧腾一时的事件涉入其中。ABC公司从事的行业虽然吸引人,但它的股价一直很低,因为经营管理很差。要是有个知名人物当上新总裁,股价通常不只应声上涨,还可能涨过了头。这是因为一开始出现的激情,忽视了基本面需要时间才会改善的事实。相反的,如由默默无闻的人士接掌经营管理的重责大任,很有可能几个月、几年过去了,公司得到的金融圈评价还是很不好,本益比依然偏低。认清这些情况──在金融圈矫正它的评价,导致股价激涨之前──是初出茅庐的投资人首先应该练习的最简单方法,千万不要随群众起舞,人云亦云。

\section{如何找到成长股}

《怎样选择成长股》第一版出书之后,我开始收到全国各地读者一堆来函,数量多得吓人。最常见的一个要求,是希望提供更详细的资料,说明读者(或他的投资理财顾问)应该怎么做,才能找到市价涨幅可观的股票。由于有那么多人对这件事感兴趣,在这里就这件事发表一些意见或许有好处。

做这件事要花很多时间,以及技巧和注意力。小额投资人可能觉得所花工夫超过个人所能负担。如果有某种简单、快速的方法,可以选到厚利股,则对小额投资人和投资大户都是好事。我非常怀疑有这种方法存在。当然了,应花多少时间在这些事情上面,每位投资人必须自己决定,看自己有多少时间可用在投资上、兴趣有多浓厚、个人的能力到什么程度。

我没办法保证只有我的方法可以找到赚钱的投资对象。我也不十分肯定它是最好的方法,但很明显的,如果我认为其他的方法更好,就不会使用自己的方法。这些年来,我一直遵照以下将详细说明的步骤;对我来说,这个方法不但管用,而且运作得很好。特别是在很重要的初步阶段,有些人拥有较丰富的背景知识、较好的人际关系,或者更有能力变通使用这些方法,而取得更好的整体成果。

以下所说分两个阶段,每个阶段所做决策的质量,对于能够获得的理财成果影响至巨。任何人马上会看出,这两个极其要紧的重点中的第二个,有关的决定十分重要,那就是「我现在应该买进这支股票,或者不买?」投资人可能不容易理解的是,选择普通股一开始也必须做一些决定,这些决定攸关你有没有可能发现一支十年后会增值廿倍的股票,而不是只找到一支还没有涨一倍的股票。

任何人如果即将着手寻找高成长证券,都会碰到这个问题:数十种行业中,数千支股票都有可能值得花很大的工夫去研究。除非你投入很多心血,否则无法确定到底哪些股票值得投资。不过即使只研究其中一部份,也没人有那么多时间。你如何选出其中一支或者很少的股票,腾出时间去研究?

这个问题远比表面上看到的要复杂。你所做的决定,可能轻易汰除某些股票,没去研究,想不到几年后它们却能给持股人带来可观的财富。你做出的决定,可能害自己囿限于研究难以开花结果的股票,等到愈来愈多资料搜集到手之后,情况变得日益明朗,得到的答案,和绝大部分的研究迟早获得的结论一样,也就是那不过是家乏善可陈的公司,或者可能稍好一点,却不是会带来可观利润的公司。从投资理财的观点来说,这个十分重要的决定,攸关你能不能挖到丰富的矿脉,或者因为对事实所知无多,只能找到贫瘠的土地。这是因为你必须先决定自己的时间要花在哪里,以及不要花在哪里,才去做够多的事,以便有个合适的基础,好做成结论。如果你做的事够多,那么做决定时便有适当的背景资料,这一来,由于你对每种状况都花了很多时间,做第一个重要决定时,速度会很快。不知不觉间,你便已做好决定。

几年前,我会很热心地告诉你,我用了一种看起来简洁有力的方法,解决这个问题;可惜这个方法是错的。就我调查各公司的结果,特别是我管理的资金所熟悉的公司,我和很多相当能干的企业高阶主管、科学家关系良好,能和他们聊起别家公司的情形。我相信,这些消息非常灵通的人士提供的点子和线索,能给我很多值得调查的对象,而且可能有非常高比率的公司,具备我不断寻找的杰出特质。

但我试着利用改善自己事业的相同分析和严格方法,期望我所投资的公司也用它们去改善营运活动。因此,几年前我做了个研究,探讨两件事。我如何选定想要研究的公司?根据事后回顾,我探讨了某种来源和另一种性质完全不同的来源提供的原始「点火器」点子,引起研究调查兴趣,并带来有价值的成果(也就是之后买到投资报酬可观的股票)所占的比率,两者是不是有很大的差异?

我发现到的事实,令自己吓一跳,但分析起来,完全合乎逻辑。我相信,企业高阶主管和科学家这一类,是我的原始点子的主要来源,促使我研究调查某家公司,而对另一家公司不感兴趣。但他们提供的线索,实际上只有约五分之一引起我的兴趣,并着手进一步研究。更重要的是,这些线索没带来高于平均水平的好投资对象。全部的研究调查占所有线索的五分之一,而这五分之一,只带来约六分之一有价值的投资成果。

相对的,相当不同的另一群人,提供的原始点子引起约五分之四的研究调查兴趣,并带来约六分之五的最终报酬(以买到有价值的股票来衡量)。在全国各地,我慢慢认识和尊敬少数一些人,他们在挑选成长型普通股方面,做得非常之好。这些能力很好的投资专家分散在很多地方,如纽约、波士顿、费城、水牛城、芝加哥、旧金山、洛杉矶、圣地亚哥。在很多地方,其中任何一人对他特别喜欢的股票所下的结论,我可能完全不敢苟同,甚至于不觉得那支股票有什么好值得研究调查的。一两个例子中,他们的思虑是否周延,我也感怀疑。但由于我晓得他们每个人的理财心思敏锐,纪录不同凡响,我乐于倾听他们不厌其详地解说我有兴趣,而且他们认为增值潜力雄厚、非常吸引人的公司。

此外,由于他们是训练有素的投资专家,我通常能够很快听到他们的意见,触及我觉得最重要的事情,好让我决定一家公司是不是值得研究调查。这些重要事情是什么?基本上,它们涵盖一家公司符合前面所谈十五要点的程度,并在这个初步阶段特别强调两个特定主题。这家公司经营的行业,营业额将有高成长机会?或者正往这样的行业迈进?随着整个行业的成长,它所经营的业务,新兴公司是不是相当容易创立,并取代原来的领导公司?如果这些业务的特质很难防范新进公司踏入这个领域,则这种成长的投资价值很低。

听取成就较低或能力较差的投资人士的话,做为原始的线索,找到值得研究调查的对象,这种做法的成果好不好?如果找不到更合适的人,我无疑会找他们多谈些。我总是试着找时间至少听任何投资人士说一次,但也留意那些在业内逐渐崭露头角,满怀抱负的年轻人,以免错过某些人的意见。但是能够利用的时间十分有限。随着事实资料显现,我会把某位理财专家的投资判断或他的可靠性降级。我发现自己有个倾向,也就是研究调查他所提公司的时间会减少很多。

从印刷资料中寻找原始线索,找到值得研究调查的公司,这种做法好不好?最可靠的经纪商发表的特别报告,如果没有到处散发,只分给少数特定人士阅读时,偶尔我会受报告内容的影响。不过整体而言,我觉得经纪商向每个人公开发表的典型印刷文字,不会有很多数据线索。它们不准确的地方太多。更重要的是,大部份报告只是重复金融圈已知的事实。同样的,偶尔我会从最好的业界和金融期刊,得到有价值的点子(我发现它们在完全不同的目的上,用处相当大);但因为我相信它们先天上有一些限制,我最感兴趣的一些事情,没办法报导,因此我发现,在寻找值得调查的最好公司方面,它们不是很丰富的新点子来源。

还有一个找得到原始线索的可能来源,技术背景较佳或能力较强的人,可能受益匪浅,但我一直办不到。这个来源是大型顾问研究公司,如阿瑟李特(Arthur D.Little)、史丹福研究所(Stanford Research Institute)或贝特尔(Battelle)。我发现,这些组织的员工对于商业和技术发展有很深入的了解,从他们那里,应能得到有价值的原始投资点子。但我觉得,这些人有个倾向(十分值得赞许),也就是不愿讨论他们知道的大部分事情,因为可能伤害客户公司对他们的信心。这一来,他们对我的用处便大打折扣。要是比我聪明的人能找到一种方法,在不伤害这些客户公司的前提下,挖出我怀疑这些组织拥有的投资信息宝矿,便可能大幅改善我的方法,在寻找成长股的这个阶段发挥功效。

第一步该说的就这么多。只要花几个小时找人一谈,通常是找杰出的投资专家,偶尔是找企业高阶主管或科学家,我就能确定某家公司是不是有意思,并开始做研究调查。接下来我怎么做?

我要特别强调有三件事不能做。这个阶段,我不找(我想,理由马上会很清楚)公司任何管理人员。我不会花无数小时翻阅旧年报,并详细研究资产负债表每年的微小变动。我不会找我认识的每位股票营业员,问他对这支股票的看法如何。但我会浏览资产负债表,确定资本结构和财务的整体状况。如有证券管理委员会(SEC)的公开说明书,我会仔细阅读其中有关总营业额分项、竞争状况、高阶主管或其他大股东持有普通股的程度(通常这也能从代理投票说明书中获得),以及所有的盈余报表数字,从中了解折旧(或者耗竭)、利润率、研究发展活动的深入程度,以及前几年的营运活动中,某些异常或非经常性的成本。

现在,我做好准备,要真正上路了。我将尽可能利用前面说过的「闲聊」法。我认识的企业高阶主管和科学家,当做原始的投资点子来源有美中不足之处,在这里却有难以估计的价值。我会试着去见(或在电话中联络)我认识的每一位重要顾客、供货商、竞争对手、以前的员工,或相关领域中的科学家,或者透过共同的朋友和他们接触。但假使我认识的人还不够多,或者朋友的朋友认识的人不够多,没办法提供我需要的背景资料,我会怎么做?

坦白说,如果我没办法得到想要的很多信息,我会放弃研究调查,另找目标。投资要赚得大钱,你考虑的每项投资不需要都获得某些答案。反之,你需要的是少数实际购买的股票得到许多正确的答案。由于这个原因,如果取得的背景资料太少,而且将来获得大量背景资料的希望微乎其微,我相信最聪明的做法,是把这件事搁到一边去,另找对象。

不过,假设已经得到不少背景资料。你已和自己认识或者容易接触的人谈过,但另外找到一两个人,相信他们如肯和你畅谈,对整幅画面的完成将大有帮助。我不会在大街上追着他们问话。大部分人不管对自己从事的行业多有兴趣,都不愿意告诉完全陌生的人,他们觉得某位顾客、竞争对手或者供货商强在哪里、弱在哪里。如果我想见某个人,我会去查他和哪家商业银行往来。要是为了这件事,你找上认识你的某家商业银行,坦诚告诉他们,你想要见谁,以及为了什么事,你会很惊讶地发现,大部分投资银行人士都乐意帮忙──只要你不常麻烦他们。可能更叫人惊讶的是,大部分企业人士在经常往来的银行人士引介下,也很乐意帮忙。当然了,只有在银行业人士十分肯定你寻找的任何信息,真的只用做投资时的背景参考,而且不管在什么情况下,你绝对不会引用不利的信息,让任何人难堪,他们才会帮忙。如果你严守这些原则,那么银行提供的帮忙有时有助于研究调查阶段完成;如果没有他们的帮忙,你可能永远完成不了有价值的研究调查工作。

以「闲聊」法,向各种来源尽量打听十五要点一章指出的很多数据之后,才可以准备采取下一步,考虑接触管理阶层。我觉得,投资人彻底了解何以必须如此,相当重要。

一家公司有优秀的管理阶层,投资它的普通股才能获得可观的利润。问到公司的弱点时,优秀的管理阶层几乎会和回答公司的优点一样坦诚以告。但就这件事来说,不管管理阶层多坦白,他们基于自身的利益,绝不会在没发问的前提下,主动谈到你这位投资人最关心的事情。如果你向一位副总裁说:「我可能投资你们公司的股票,你觉得关于贵公司,还有别的事情我需要知道的吗?」你想,他会答说,其他高阶管理人员表现非常突出,但他当副总裁几年下来,营销工作没做好,害得公司的营业额开始出现颓势?他有可能进一步主动提及,这事可能没那么严重,因为某位年轻的营销干部能力高强,再等六个月便会接掌大权,振衰起敝?当然了,他不会不打自招。不过,我发现,如果他晓得你已经知道营销方面的弱点,便有可能以外交辞令应付你,但如你找对管理阶层,而且他们对你的判断有信心,他们会给你很实在的答案,谈到他们有没有采取什么行动,矫正这样的弱点。

换句话说,接触管理阶层前,根据「闲聊」法获得背景资料,才能知道拜访一家公司时,应该试着了解什么事情。没有这方面的背景资料,你可能没办法确定最基本的要点──高阶管理人员本身的能力。即使是中型的公司,重要管理人员的数目也可能高达五人。第一或第二次拜访时,通常不容易见到所有的人。硬要和所有的人见面,和某些人一谈的时间可能相当短暂,没办法确定他们的相对能力。五人中往往有一二个人的能力远比别人好或差。不靠「闲聊」获得背景资料指引你,你可能因为见到的人不同,而对整个管理阶层有过高或过低的评估。「闲聊」之后,你可能会有相当准确的了解,晓得谁特别强或特别弱,而能站在更好的地位,求见特定的人士,好对他有更深一层的认识,用以验证「闲聊」得到的印象有否正确。

依我的意见,几乎任何领域中,除非把事情做对有其价值,否则不值得去做那些事情。就选择成长股这件事来说,适当的行动获得的报酬很大,判断不良遭到的惩罚也很大,所以不应只凭粗浅的知识去选成长股。投资人或理财专家如想选到适当的成长股,我相信他务必时时遵守的一个准则是:投资某支股票需要的所有知识中,除非至少先取得五十%,否则绝不要去拜访任何公司的管理阶层。不先做到这一点而冒然去见管理阶层,那么他所处的地位很危险,因为他对自己应知道的事情,了解甚少,和管理阶层见面,能不能得到正确的答案,只能靠运气。

拜访一家公司前,至少应先取得必要知识的一半,这件事所以重要,还有另一个原因。当红行业中某些公司的杰出管理人员,必须腾出很多时间,应付投资业人士。公司股票价格的高低,在许多方面对他们很重要,所以重要管理人员通常会腾出时间接见这些访客。但是我听过很多公司相同的说法。他们不想以粗鲁无礼的态度对待任何人,可是接见金融业访客时,到底是派出重要的管理人员,还是不负行政决策的人员,主要取决于公司对访客个人能力的评估,访客所代表金融利益的多寡,并没那么要紧。更重要的是,公司提供信息的意愿高低──也就是,公司回答特定问题和讨论重要事务时,愿意谈得多深入──几乎完全取决于对每位访客能力的评估。只是临时顺道来访,根本没做事前准备工作的访客,还没开始访问,往往已获两好球,离三振出局不远。

你可以见到什么人(务必见到实际做决策的人,而不是财务公关之类的员工),这件事十分重要,所以不妨花点工夫,找合适的人引介你认识管理阶层。重要的客户,或管理阶层认识的大股东,是极佳的介绍人,能为你首次拜访公司铺路。公司往来的投资银行,也是一条路子。不管怎么做,很希望第一次拜访便收获丰硕的人,应确定介绍人对自己的评价很高,能在管理阶层面前美言几句。

写这些文字之前几个星期,恰巧碰到一件事,或可说明首次拜访管理阶层之前,应先做好多充分的准备。那时,我正和某大投资公司的两位代表共进午餐,其中一位是我管理的基金投资的少数几家公司里面,两家公司往来投资银行业务的负责人。一位绅士晓得我投资的公司不多,而且持有股票的时间通常很长,问到我拜访的新(对我而言)公司里面,后来实际买进股票的比率有多少。我请他猜一下。他估计每拜访二百五十家,买进一家公司的股票。另一位绅士大胆指出,可能每廿五家就买一家。实际上是每二家到二家半便买一家!这不是因为我每拜访二家半公司,便有一家符合我自认相当严格的买进标准。如果他讲的是「注意的公司」,而不是「拜访的公司」,那么每四十家或五十家买进一家可能是对的。如果他讲的是「可能考虑研究调查的公司」(不管我实际上有没有去研究调查),那么他原来估计的每二百五十家买进一家便很接近实情。他忽视了一件事,也就是我相信,除非先做大量的「闲聊」工作,否则拜访工厂无法获得太大的好处,而且我发现,「闲聊」得到的背景资料,常能准确预测一家公司符合我的十五要点到什么程度,因此在我准备拜访管理阶层时,买进该公司股票的机率已相当高。很多比较不理想的对象,在这个过程中已经汰除。

以上所说,总结了我寻找成长股的方法。我开始着手研究调查的公司中,可能有五分之一来自业内朋友提供的线索,五分之四来自少数能力较强的投资专家。有关的决定做得很快,我必须迅速判断哪些公司值得花时间做研究调查,哪些公司应置之不顾。接着简短审视证券管理委员会公开说明书中的几个要点,之后积极与人「闲聊」,不断了解一家公司多接近我们的十五要点标准。这个过程中,我舍弃一家又一家可能的投资对象。有些被淘汰的公司是因为证据愈来愈多,显示它们平凡无奇,有些则是因为我没办法取得足够的证据,确定它们是好是坏。只有少数例子中,我得到很多有利的数据,才准备走到最后一步,也就是接触管理阶层。和管理阶层见过面后,如果我发现先前的期望果然没错,而且他们提出的答案头头是道,消弭了我先前的疑虑,一种感觉便油然而生,相信先前的种种努力终有报偿。

一些人对以上的做法不以为然,我了解他们反对的理由,因为曾听过很多次。我们怎能期望一个人花那么多时间,只为了寻找一家值得投资的公司?在投资业找到的第一个人,问他应买什么,难道答案不能恰合所需?有这些反应的人,我想请他们看看身边的世界。其他活动中,你能投资一万美元,十年后(这段期间,只偶尔看看公司管理阶层是不是仍然那么优秀)资产价值高达四万美元到十五万美元?慎选成长股,会有这样的报酬。一个人如果一个星期只花一个晚上,躺在舒适的扶手椅里,浏览一些言简易赅的经纪商免费报告,便有这样的成果,你觉得合情合理吗?一个人找了第一位投资专家一谈之后,支付一百三十五美元的手续费,便能获得这样的利润,你认为说得过去吗?一百三十五美元是在纽约证券交易所以每股二十美元买进五百股股票必须支付的手续费。就我所知,没有其他的活动领域能如此容易获得这么可观的报酬。股票市场也同样办不到,除非你或你的投资顾问拥有在其他活动领域赚得大钱的同样特质,也就是投入很多心血,加上一定的能力、判断力和观察力。运用这些特质,并利用和本章所述类似的方法,寻找非常符合我们的十五要点标准,但还没受金融圈垂青的公司,我们的纪录十分清楚地指出,你很有可能找到创造财富的成长股。不过,不经一番苦功,无法找到它们,而且没办法每天找到它们。

\section{汇总与结论}

目前正值廿世纪下半叶的第二个年代,人类生活水平提升的速度,很可能超越前面五千年。最近的投资风险很大,但投资成功的金钱报酬更大。但在投资的领域中,过去几百年的风险和报酬,与未来五十年相比,可能小巫见大巫。

在这种情况下,可能有必要掂量目前的形势。我们当然还没克服景气循环周期的问题,甚至没有能力缓和它的波动。不过,我们已增添了几个新因素,显著影响普通股投资的艺术。其中之一是现代企业管理的兴起,强化了普通股的投资特性。另一个新因素是从经济性的角度,利用科技研究发展工程设计。

这些因素崛起,并没有改变普通股投资成功的基本原则,反而使它们比以前更为重要。本书试图说明这些基本原则是什么、应该买进什么样的股票、何时去买,特别是绝对不要卖出──只要发行普通股的公司仍然具备经营十分成功的各项特质。

但愿字里行间提到许多投资人最常犯错的部分,能引起若干兴趣;这些投资人本来能力很强。但务请记住:晓得这些准则和了解常犯的错误,不能帮助那些没什么耐心和自律精神的人。我认识一位能力非常强的投资专家,几年前告诉我:在股票市场,强健的神经系统比聪明的头脑还重要。莎士比亚可能无意间总结了普通股投资成功的历程:「凡人经历狂风巨浪才有财富。」
\newpage

\part{保守投资人夜夜安睡}


\textit{就我的事业生涯来说,我相信个人事业──或任何事业──的成功,有赖于遵循二个I和一个H的原则,也就是诚信正直(integrity)、聪明才智(ingenuity)、努力工作(hard work)。愿以本书献给我的三个儿子,因为我认为,阿瑟(Arthur)、肯恩(Ken)从事的事业和我很像,唐(Don)从事相当不同的事业,但他们都依循二I和一H的原则。}

\section{引言}

虽然这种事情很难精确衡量,但迹象强烈显示,撰写这段文字时,美国投资人士气之低落,本世纪只见过一次同样的情形。名闻远近的道琼卅种工业股价指数(Dow Jones industrial average)是每天股市价位变动的极佳指针。但如考虑较长的期间,这个指数可能掩饰,而非完全揭露最近许多普通股投资人蒙受的伤害。这个股价指数本应呈现所有公开交易的普通股到底发生了什么事,但它没有依每支股票发行在外的股数加权。如果这么做,可以看出,一九七四年年中的平均股价比一九六八年的高价低七十%。

面对这种损失,很多投资人的行为都在预料之中。一类投资人完全撤离市场。可是很多公司目前的表现好得出奇。我们所处的环境中,通货膨胀愈来愈高似乎难以避免,以审慎的态度选择性买进股票,风险性可能远比其他一些看起来较安全的资金去处要低。还有一群为数更多的投资人,特别叫人感兴趣:决定「从现在起,我们的行为举止要较为保守」的一群人。这里常见的说法是,他们将只买进最大型的公司,至少它们的名字几乎每个人都知道。宾州中央(Penn Central)和联合爱迪生(Consolidated Edison)的名字,或两家公司提供哪种服务,美国不知道的人很少,东北部可能几乎每个人都知道。依照传统的标准,几年前的宾州中央和最近的联合爱迪生,被视为保守型的投资。很遗憾的,行事保守和行事守旧两者,投资人往往混淆不清,对那些真的想要保存资产的人,整件事有必要花一番工夫加以厘清──这事得从两个定义说起:


\textbf{一、保守型投资工具(investment)很有可能在最低的风险下,保存(亦即维持)购买力。}

\textbf{二、保守地投资(investing)指了解保守型投资工具的构成内涵,接着针对特定的投资工具,依照一套合适的行动程序,以确定特定的投资工具到底是不是保守型投资工具。}


因此,要当保守型投资人,投资人或提供建议给他的人,需要具备的条件不只一个,而是两个。他们应了解保守型投资工具的理想特质。接着采取研究调查行动,观察特定的投资工具是否符合这些特质。两项条件没有同时具备,则普通股投资人可能只有运气好坏,或所用方法属传统或非传统之分,但称不上是保守型投资人。

对我来说,这件事十分重要:任何时候,绝对不能在这种事情上产生混淆。不只持股人本身,连美国整体经济也是一样,不能再让那些真心努力了解各项守则的人,蒙受这一代投资人最近经历的惨况──这次失血之严重,仅次于约四十年前经济大萧条时另一代投资人的不幸遭遇。美国今天有无与伦比的机会,能够改善所有人民的生活方式。它肯定具备技术知识和专业长才。但要以美国传统的方式做这些事情。许多投资人以及投资业本身很多从业人士,有必要接受一些再教育,了解基本知识。只有更多投资人因为财务真的十分安全而感放心之后,新股发行市场才会再开启,好让企业能够合法取得更多资金,更易于推动各项新计划。要是这样的事没出现,则不管在美国或外国,我们只好以高成本、浪费、无效率的方式,满足所需──由政府提供融资,并置管理阶层于官僚习气浓厚的黑手之下。

由于这些理由,我相信投资人今天的问题应该勇敢直接面对。本书为了处理这些问题,我向儿子肯恩请教良多。肯恩想出这本书的书名,并贡献其他很多事情,包括这里所提部分基本概念。短短数语,难以尽言谢忱。

本书分成四个截然不同的部分。第一部分解剖──如果可以这么说的话──如定义一所说的保守型股票。第二部分分析这次空头市场的形成,金融圈推波助澜,扮演的角色──如果你愿意接受的话,应说是它们犯下的错误。里面的批评,并非只顾炮轰,更希望指出类似的错误将来可以避免,而且研究过去的错误之后,一些基本投资原则变得十分清楚明白。第三部分谈应该采取什么样的行动,才够资格如定义二所说,称得上是保守地投资。最后一部分提及今天的世界中,甚嚣尘上的一些影响力量,引起很多人深深怀疑普通股是否适合当做保存资产的工具──换句话说,除了当做赌博的工具,还能考虑拿普通股做些什么?造成最近一波空头市场的种种问题,是不是创造了一种环境,使得持有股票成了不知情的人容易掉进的陷阱,或者正如美国历史上每次的大空头市场,这些问题带来大好机会,能力和自律精神强的人,可以为自己好好打算,自外于目前普遍弥漫的情绪反应之外,采取应有的行动。但愿本书能就这一点提供若干指引。

\rightline{Philip A.Fisher}

加州圣马特奥(San Mateo)


\section{保守型投资的第一个要素——生产、行销、研究和财务}

可做为保守型投资工具的公司,论其规模和类别,必然是家复杂的公司。要了解这样一家公司必须具备什么,或许可以从它肯定拥有的特性的第一个要素谈起。这个要素可以分成四大类:
\\

\textbf{生产成本低}


要成为真正保守型的投资工具,一公司──即使不是全部的产品线,也需要是绝大部分的产品线──必须是生产成本最低的制造商,或者和任何竞争同业差不多一样低,而且未来可望继续如此。只有如此,持股人才能享受成本和售价间够大的差距,并由此创造出两个重要的条件。其一是在大部分竞争对手的损益平衡点以下有充分的转圜空间。一旦不景气袭击整个业界,价格不可能长久处于这个损益平衡点以下。只要出现这种情形,不少成本较高的竞争对手会有很大的亏损,某些同业将被迫停止生产。存活下来的低成本公司,利润会自动提高,因为关闭的厂房以前供应的需求由它们接手,产量会增加。竞争对手的供应量减少,低成本公司享受的好处不只如此;它们不只能做更多的生意,而且由于多余的供应量停止压迫市场,它们还能提高价格。

第二种情况是,高于平均水平的利润率,应能让一家公司获得够多的盈余,从内部创造公司成长所需的大部分资金,或者全部的资金,这一来,就不需要筹措太多的长期资金,或者根本不需要额外另筹资金。筹措资金可能(a)必须发行新股,使得已经发行的股票价值稀释,或者(b)固定利息支出和固定还本(主要必须从未来的盈余提拨)加重债务负担,使得普通股持有人风险大增。

但是投资人应了解,一家公司是低成本制造商,虽可提高普通股投资的安全性和保守性,但在多头市场欣欣向荣的期间内,投机魅力将有所减损。高成本、高风险的边际公司,碰到这种时候,利润增幅总是远高于前者。简单的算术运算可以说明何以如此。假设两家公司的规模相同,景气正常时,产品的单位售价是十美分。A公司每单位产品的利润是四美分,B公司则是一美分。再假设成本维持相同,但产品的需求暂时增加,把价格推升到十二美分,而且两家公司的规模还是相同。利润较高的那家公司,单位利润从四美分增加到六美分,成长五十%,但成本较高的另一家公司,利润则成长二百%,或者增为三倍。这是为什么短期而言,高成本公司的盈余有时会在景气呈现荣面时上升得比较多,以及为什么几年后景气转差,产品价格掉到八美分时,体质较强的公司盈余虽然缩水,但仍令人安心。高成本的公司即使没有破产,还是可能制造一批受到严重伤害的投资人(或者自认是投资人的投机客),他们相信错在整套制度,本身没错。

撰写以上文字时,心里想到的只是制造业公司;所以我用的是生产一词。当然了,许多公司不是制造商,从事的是服务业,如批发、零售,或者金融业中很多不同的行业,如银行业或保险业。同样的原则可以适用,但以营运(operations)一词代替生产(productions),也就是以低成本或高成本营运商代替低成本或高成本制造商。
\\

\textbf{强大的营销组织}


强大的营销商必须时时留意客户不断变化中的愿望,以便供应今天恰合所需的产品或服务,而不是以前需要的产品或服务。比方说,本世纪交替之际,知名马车制造商如果坚持生产更好的马车,不肯转而生产汽车,或完全停止生产马车,那么它的营销努力一定出问题。拿现在的例子来说,或许早在阿拉伯禁运石油,使得美国每个家庭体会到大型车十分耗油之前,汽车业的大型车制造部门恐怕已经做错一些事,没有认清小型进口车日益受欢迎,露出明显的迹象,显示消费大众的需求转向低价、省油、比多年来受宠的大型豪华车容易停车的产品。

但是认清消费大众的品味改变,并立即采取因应行动还不够。正如人们说过的,商业世界中,顾客不会开出一条路,走到某人门口,去买更好的捕鼠器。在竞争激烈的商业世界中,设法让潜在客户晓得某样产品或服务的好处,是十分重要的工作。只有了解潜在买主真正想要的是什么(有时顾客本身不是很清楚为什么产品或服务的优点对他有好处),并以他的平常用语,而非卖方的语汇向他说明,才能使他认知到那些好处。

到底是打广告、由推销员亲自拜访、请独立的专业营销组织推荐,或者综合以上各种方法的效果最好,要看企业经营的性质而定。但是不管使用哪种方法,每一种做法都必须严密控制,管理阶层也必须不断衡量它们的成本效益。这方面若无杰出的管理阶层主其事,将导致(a)失去本来可得的大量业务;(b)成本大幅增加,以致于获得的业务,利润缩小;以及(c)由于产品线内的各个组成利润率不同,公司将无法在产品线内取得最高的利润组合。效率高的制造商或营运商,如果营销和销售能力薄弱,可能就像力量强大的引擎,由于驱动皮带松弛或差动调整不良,产生的成果将大打折扣。
\\

\textbf{杰出的研究和技术努力}


没多久以前,出色的技术能力似乎只对少数高度科技取向的行业十分重要,如电子、航空、药品和化学品制造。随着这些行业成长,它们日益扩大的技术渗透到几乎所有的制造业和服务业。今天,拥有出色的研究和技术人才,对制鞋厂、银行、零售商或保险公司来说,重要性不亚于维持大量研究人员,曾被视为很奇特的科技业。科技努力目前导入两个方向:生产更好的新产品(当然了,关于这一点,研究科学家对化学公司的贡献,可能多于对连锁杂货店的贡献),以及用比以前更好的方法和更低的成本,提供服务。关于后一项目标,杰出的技术人才对两者来说都很宝贵。其实,若干服务业中,技术人员一直努力开拓新的产品线,而且正在做铺路工作,希望用更好的方式提供旧服务。银行就是个好例子。低价电子输入设备和小型计算机,有助于它们对客户提供会计和记账服务,创造出一种新的产品线。

在研究和科技领域,各公司的效率差异很大,一如在营销领域。开发新产品的任务十分复杂,效率难免有很大的出入。一公司研究人员相对于另一公司研究人员的能力和才智固然很重要,却只是影响研究成果的许多因素之一。开发新产品往往必须集合许多研究人员之力;每个人都专精于不同的技术领域。这些人能不能携手合作(或在一位领导人诱导下,群策群力,彼此激励),通常和参与其事者个人能力的高低一样重要。此外,为了尽量提高利润,不能随便开发任何产品,只能开发顾客需求大的产品,(几乎总是)能由公司目前的营销组织销售出去,而且售价可以创造不错的利润。所有这些事情,有赖研究和营销、生产单位间密切地连系。这个世界上,最好的企业研究团队如果只开发卖不出去的产品,将成公司的负债而非资产。要成为优异的投资对象,一家公司必须有高于平均水平的能力,控制所有这些复杂的关系,但同时又不致于控制过度,使得研究人员失去冲力和创意,一开始便注定不可能有杰出的表现。
\\

\textbf{财务能力}


讨论生产、营销和研究时,一再使用到利润(profit)和利润率(profit margin)二个名词。拥有多种产品线的大公司中,要确定一种产品相对于其余产品的成本,不是那么容易的事,因为除了原物料和直接人工,大部分的成本都由很多产品分摊,而且可能是由所有的产品分摊。财务人才比平均水平优秀的公司,占有几项重要的优势。确切晓得每样产品能赚多少钱,他们就能尽最大的努力,产生最大的利益。深切了解每项成本因素所占的份量,而且不只知道生产作业的成本,同时晓得销售和研究的成本,这么一来,即使微不足道的公司营运活动,也能看出哪些地方值得特别努力去降低成本,方法可能是藉由技术创新,也可能是调整派任的人员。最重要的是,真正杰出的公司透过精湛巧妙的预算和会计作业,能够创造早期警报系统,很快察觉威胁利润计划的不利影响因素,并采取矫正行动,避免许多公司的投资人出乎意料惨遭痛苦的事情发生。公司财务能力优秀带给投资人的「好处」不止如此。有这种能力的公司,通常能够善选资本投资计划,从公司投资的资金中产生最高的报酬。它们也更能控制应收帐款和存货;利率居高不下的期间内,这件事愈来愈重要。

总结而言:符合保守型投资工具第一个要素的公司,是在自己的领域中成本很低的制造商或营运商,它们有出色的营销和财务能力,并且展现优于一般水平的技能,善于处理相关的管理问题,从它的研究或技术单位获得有价值的成果。我们置身的世界中,变迁的脚步愈来愈快,符合第一个要素的是(一)公司有能力源源不断开发获有利润的新产品或产品线,弥补旧产品线因为其他公司技术创新而淘汰落伍仍有余;(二)公司现在和未来都有能力以相当低的成本生产这些产品线,使得利润成长的速度至少和营业额一样快,即使在百业景气最糟的年头,利润也不致于萎缩到危及投资的安全性;以及(三)公司有能力销售更新的产品以及未来可能开发出来的产品,获利力至少和目前的产品一样。

这是精明投资的一个要素──这个要素如不被其它要素破坏,则投资人选择的投资对象不可能让他梦想破灭。但在检视其他要素之前,还有一点应充分了解。如果投资人的目标是保存资金,也就是追求安全性,那么我们为什么要谈成长性和开发额外的新产品线?为什么不能只求维持企业规模和获利水平于现状,避免开创新活动必须承受的所有风险?讨论通货膨胀对投资的影响时,追求成长的其他理由便会清楚地浮现。但基本上,我们不应忘记,在世界变迁的脚步愈来愈快之际,没有什么事情能够长久维持不变。我们不可能文风不动。公司不是成长,便是萎缩。强力攻击是最好的防守。只有变得更好,公司才能确保不致变得更糟。不向上爬的公司,肯定会走下坡──如果过去的确如此,将来更将如此。这是因为除了技术创新的脚步愈来愈快,社会习俗和购买习惯的改变,以及政府的新规定日益加速,连最守旧的行业也不能置身事外,必须跟着改变。

\section{第二个要素——人的因素}

简言之,保守型投资工具的第一个要素,由管理阶层在生产、营销、研究和财务控制等基本领域中突出的能力构成。第一个要素谈的是企业的现状,基本上是个结果。第二个要素则谈导致这些结果的成因,而更重要的是将来会继续产生这些结果的成因。一家公司所以在各个基本领域都有突出的表现,并成为业内出色的投资工具,而其他公司表现普通、乏善可陈,或者更糟,根本原因在人。

早期的创业资本家爱德华·赫勒(Edward H.Heller),事业生涯中发表的谈话,影响本书一些概念至巨。他用到「生龙活虎」(vivid spirit)一词,描述他愿意提供大量资金支持事业发展的个人典型。他说,每一家经营成功,不同凡响的公司背后,都有这么一位刚毅果决的创业家,具备冲劲、原创力、必要的技能,让公司成为真正值得投资的对象。

非常小的公司已成长为业务蒸蒸日上的大公司时(他最感兴趣,也是大获成功的一个领域),爱德华·赫勒无疑是对的。但在这些规模较小的公司一路成长,可望成为合适的保守型投资工具之际,如有另一位才华洋溢的企业家大表怀疑,觉得某家公司的总裁是他的私人好友,投资那家公司有待商榷时,爱德华·赫勒的看法可能有所保留。这个人不觉得这样的投资值得考虑的理由是:「我的朋友是我认识最聪明的人之一。他老是把事情做得很好。在小公司,这种情形或许不错。但随着公司成长,你有时也必须用到正确的人。」

真正的保守型投资工具第二个要素的核心是:企业执行长致力于长期的成长,身边有一群能力很强的团队,并大量授予职权,要他们主持公司的各个部门和职能。这些人不能只顾内部永无止尽地争权夺利,相反的,应携手合作,共同迈向明定的企业目标。投资这家公司要获得可观的利润,则它的目标之一必须是高阶管理人员应腾出时间,寻找和训练合格和士气高昂的资浅人员,在需要轮换新血时,接替资深的管理人员。同样的,在指挥体系的每一个层级,管理人员应注意这个层级的人所做的事,是不是和下一个层级的人做的事完全一样。

这是不是表示适合做为保守型投资工具的公司,除了最基层的员工,或者刚踏入职场的人员,只应从内部擢升人才,不假外求?一家公司成长非常迅速时,可能需要增添人力,但没有时间训练内部员工担当所有的职位。此外,即使经营管理最好的公司,有时也需要专业技能十分独特的人才,但从公司内部日常一般事务,根本无法培养这样的人。钻研法律、保险、科技某一领域特别技术知识的人才,可能不是公司主要经营业务范围内能够拥有的人才。此外,偶尔从外面找人有个好处:能够从外面引进崭新的观点,注入新的理念,向公司既有的成规挑战。

但是一般来说,具有高投资价值的公司,通常从内部擢升人才。这是因为最适合投资的所有公司(不一定是最大和最知名的公司),已发展出一套政策和做事方法,很适合自己所需。如果这些特别的方法真的管用,那就很难重新训练早已习惯于旧方法的人。新进人员在组织中任职的层级愈高,灌输新文化的成本愈高。虽然我无法引用统计数字证明这一点,但据我观察,经营管理较好的公司中,从外面聘用的高阶主管,几年后消声匿迹的人数很高。

投资人可以肯定一件事:大公司如需从外面聘用执行长,一定是很不好的兆头,表示目前的管理阶层基本上出了问题──不管最近的盈余报表有多好,因为那只是表面文章。新上任的总裁很有可能把事情做得非常好,迟早会在身边建立起名符其实的管理团队,以后就不需要从外面找空降部队,震撼整个组织。因此,这样一支股票迟早会成为聪明投资人的宠儿。但是重建管理团队很花时间,而且风险很高,投资人如发现自己的持股中发生这样的事,最好能够检视他所有的投资活动,确定他过去是不是真的根据良好的基础采取行动。

管理阶层由一人主控大局,或者真的是个运作顺畅的团队,有个线索能让所有的投资人看出蛛丝马迹(但这个线索没办法指出那个管理阶层有多好)。所有股票公开上市的公司,高阶管理人员的年薪,都在代理投票说明书中公开发表。要是第一号人物的薪水远高于第二号或第三号人物,警讯便响起。如果薪酬缓慢递减,则无关紧要。

投资人如想获得最好的成果,管理人员群策群力,而且有能力填补上一个职缺还不够。爱德华·赫勒所说,「生龙活虎」的人数必须尽可能多──这些人有脑筋,也有决心,不想让事情维持原状,差强人意,而是渴望在现有的成果上百尺竿头。这样的人不容易找到。摩托罗拉公司(Motorola Inc.)有件事做了一段时间,但金融圈乏人注意,不晓得它在这方面竟能做得那么好。

一九六七年,摩托罗拉公司的管理阶层体认到,未来几年成长会非常迅速,高层管理人事稳定扩增在所难免,它决定挺身面对这个问题。那一年,摩托罗拉在亚利桑纳州欧勒科(Oracle)设立高阶主管学院(Executive Institute),目的是在远离公司办公室和厂房日常琐事的环境里,实现两件事:摩托罗拉的明日之星接受的训练,可以超越目前所负职责的范畴,以便将来有能力付予重任;高阶管理人员可以获得更多重要的证据,晓得同一批人适不适合予以擢升。

高阶主管学院创立时,公司一些管理人员质疑花那么多钱做这件事是否值得,主要因为他们相信,整个摩托罗拉公司,具有足够才华的人找不到一百个,从公司的观点来说,不值得为他们提供这种特别的训练。后来事实证明这些人的疑虑大错特错。该学院一年开五到六班,每班十四人。到了一九七四年年中,约四百位摩托罗拉员工受过训练;而且有很多人,包括一些现任的副总裁,发现他们具备的能力,远高于当初获准入学时所以为者。此外,参与这项工作的人觉得,从公司的观点来说,最近的班级获得的成果,比以前的班级要好。随着摩托罗拉的成长,公司的总用人数继续扩增,前景看好的员工人数够多,这项活动可以无限期延续下去。在投资人眼里,所有这些事情显示,多用点脑筋,连成长率已经远高于平均水平的公司,也能从内部培养所需的杰出人才,维持公司的优异表现,而不必像很多迅速成长的公司那样,到外面找来几位能力突出的人才,往往造成内部摩擦并证明此路行不通。

每个人都有个性──一种人格特质的组合,使得他或她有别于他人。同样的,每家公司都有它自己做事情的方式──有些把它们变得很正式,成为说明十分清楚的政策,其他公司则不然──至少和其他公司略有不同。公司经营愈成功,若干政策愈有可能显得很独特。经营成功期间已经相当长的公司,尤其如此。个人的基本人格特质会变化,但一旦成熟,便很少再改变。公司做事情的方式则与此不同,不只受外部事件影响,而且受一群个性不相同的人对那些事件的反应影响;这些人随着时间的流逝,一个接一个出任高层职位。

不管各公司的政策差异多大,三个要素必须存在,一公司的股票才能当做保守型的长期投资工具持有。
\\

\textbf{一、这家公司必须体认它置身的世界,变迁的速度愈来愈快。}


公司所有的思虑和规划,都必须导引于向目前正在做的事挑战──不是偶一为之,而是一而再,再而三提出挑战。每个被视为理所当然的做事方式,都必须检视再检视,好在人非圣贤,孰能无过的心态下,确定所用的方法的确是最好的。为因应不断变迁的环境,以新方法来替代时,应接受一些风险。不管目前的方法用起来多叫人安心,都不能因为它们过去管用,而且传统上一向如此,使得那种方法显得十分神圣,而永远墨守成规。食古不化,行为僵固,而且没有不断向自己挑战的公司,只有一条路好走,也就是下坡路。相形之下,一些大公司的管理阶层,刻意建立起一种组织结构,好让自己有能力变化,因而为股东创造出非常可观的报酬。道氏化学公司(Dow Chemical Company)便是个好例子,过去十年的成就如果不算全世界最突出的话,也常被视为超越美国其他任何大型化学公司。道氏摆脱过去的做法最显著的地方,或许在于把管理阶层按地域别分成五个不同的管理单位(美国道氏、欧洲道氏、加拿大道氏等)。他们相信,只有这么做,各地的问题才能以最适合当地情况的方式迅速处理,消除了公司规模变大往往随之而来的低效率官僚作风。欧洲道氏总裁指出净效果是:「结果,今天向我们挑战的是全球各地的(道氏)姊妹公司。它们不是我们的直接竞争对手,却不断进步,鞭策我们要成为第一。」从投资人的观点来说,这次变革最重要的特色,或许不在于变革本身,而在于道氏的总营业额远低于其他许多跨国公司之际,便采取改革行动;后者这时还以既有的方式经营得相当成功。换句话说,道氏的变革和改善来自追求创新的想法,好把已经行得通的制度做得更好──不是源于经营发生危机,被迫采取行动。

这家凡事走在前端的公司,采取很多方法,摆脱过去,达成惊人的竞争纪录,以上所说,不过其中一种方法。另一种方法是这家制造业公司采取前所未有的行动,在瑞士从头做起,成功经营一家独资银行,以便在这个出口市场针对客户的需求提供融资。同样的,这家公司的管理阶层面对早期的风险,仍毅然决然摆脱过去,最后终能强化公司的内在优势。

关于这家公司的事迹,还有很多例子,不过这里只再举其一,用以说明这类行动的多样性。道氏远在其他大部分公司之前,不只体认到应花钱避免污染,更做成结论说,要获得重大的成果,不能光靠高阶管理人员谆谆善诱,说一动做一动。公司有必要取得中阶经理人持续不辍的合作。公司认为,要做到这一点,最能收效的方式,是唤起直接参与最多的人的利润动机。公司鼓励他们去寻找能够赚钱的方法,把污染物转换成可以出售的产品。其余现在已成历史。在高阶管理人员、工厂经理人和高技能的化学工程师通力合作下,道氏在防治污染的计划上,取得好几个第一,赢得反商情结通常浓得化不开的许多环境保护团体的赞誉。或许更重要的是,工厂所在地的大部分小区,都不敌视该公司。他们以很低的总成本做到这一点,有时甚至还有营业利益。
\\

\textbf{二、公司必须持续不断努力,让每个阶层的员工,从新进蓝领或白领劳工到最高阶管理人员,觉得公司是工作的好场所,而且那是真实的感觉,不是宣传之词。}


这个世界中,我们大部分人每个星期得投入很多时数,做别人要求我们做的事,才能领得薪资,虽然我们宁可把这些时间用在休闲玩乐上。大部分人能够体认并接受这件事。管理阶层如能让员工普遍觉得(不只少数高阶人员有这种感觉),他们已做了每一件合情合理的事,创造出良好的工作环境,并照顾员工的利益,则公司从生产力提高和成本降低等方面得到的报偿,将远超过这种政策负担的成本。

这种政策的第一步,是确定(不只是口头上讲讲而已,而是真正确定)每位员工都获得合理的尊重和关怀。约一年前,我在报上看到一位工会职员表示,全美最大的一家公司迫使生产线员工满手油污吃午餐,因为公司的洗手间数目不够,大部分人没有充裕的时间饭前洗手。我对这家公司的股票不感兴趣,理由不在于此,因此,我不晓得这样的指控是根据事实,还是劳资双方磋商时的情绪性对阵说词。不过假使此事为真,则依我之见,单是这件事,这家公司的股票便不适合小心谨慎的投资人持有。

除了尊重和善待员工,取得员工效忠输诚的方法有许多。养老金和利润分享计划可以扮演很重要的角色。各阶层员工间沟通良好也是。至于普遍关切的事务,不只让每个人确切晓得公司正在做些什么事,更让他们知道为什么要做那些事,往往可以消除不必要的摩擦。确实晓得公司各个阶层的员工在想些什么,特别是那些想法有害无利时,可能更为重要。让全公司员工觉得,任何人都可以向上级表达自己的不满,不用担心遭到惩处。这种门户开放政策对公司有好处,但不是那么容易维持,因为必须浪费很多时间在稀奇古怪的小问题上。员工有所抱怨时,应如何处理,有关的决定必须迅速做成。不满长期郁积,对公司造成的伤害通常很大。

德州仪器公司(Texas Instuments)实施的「人员效能」(people-effectiveness)计划,和员工达成一致的目标,劳资双方获益匪浅,便是绝佳的例子。从这个计划的历史,可以说明杰出的管理阶层如何坚忍不拔,即使面对外界的新影响力量迫使这类政策转向,仍力求完美。在这家公司的初创阶段,高阶管理人员就深信,如能建立一套制度,让所有的员工参与管理决策,以改善绩效,则每个人将同蒙其利,不过,要维系员工的兴趣于不坠,所有的参与者必须从他们的贡献得到的成果中真正获得实益。一九五〇年代,半导体生产主要仍靠手工组装,员工有很多机会提出非常好的绩效改善建议。公司召开会议,甚至开设正式的课程,告诉生产工人如何个别或集体提出改善作业的方法。在此同时,透过利润分享计划和奖金、奖励,参与者既获得物质奖励,同时又觉得自己是整幅画面的一部分。接着,以前的手工作业开始走向机械化。随着这股趋势成长,若干类型的个人贡献机会减少了,因为在某些地方,机器控制了以前所做的事。组织中的一些领班开始觉得,低阶员工不再能够参与贡献管理决策。高阶管理人员的看法恰好相反:员工参与所扮演的角色将甚于以往。不过,现在要靠群体或团队的努力,由员工形成一个群体,研究可以做哪些事情,并且设定自己的绩效目标。

由于员工开始觉得他们(一)真正参与决策,不是听令于他人做事,以及(二)获得物质上的奖赏和精神上的荣耀与认可,所以这个计划的成效非凡。无数的例子显示,员工团队为自己设定的目标,比管理阶层考虑建议的要高出许多。有些时候,设定的目标看起来可能无法达成,或者各团队间的竞争造成敌对状态时,员工会提出建议,并就前所未闻的事情(就那个时代来讲)自动投票表决,如减少咖啡休息时间,或缩短午餐时间,好把事情解决。动作迟缓或态度懒散的员工,危及团体自定义目标达成时,受到的同侪团体压力,远高于透过传统管理方法给予惩处的鞭策力量。美国员工长久以来享有政治民主,可是这些成果不限于美国员工才能实现。不管员工是什么肤色,以及来自哪个经济背景相当不同的出生国,所用的方法似乎同样有效,大家也都同蒙其利。虽然绩效目标计划在美国首先推动,同样出色的成果不只出现在所谓已开发工业国家的德仪工厂,如法国和日本,也出现在员工为亚洲人的新加坡,以及员工绝大多数为黑人的库拉索岛(Curacao,译注:位于委内瑞拉东北方)。在所有的国家,由于员工团队不只可以直接向高阶管理人员报告,而且晓得他们报告的事情会受到重视,成就会得到认可和鼓励,所以士气大受鼓舞。

公司总裁马克·谢柏德(Mark Shepherd, Jr.)在一九七四年的年会上对股东发表演讲时,把所有这些事情对投资人的涵义讲得很清楚。他指出,公司设计了一个人员效能指数,根据净销售额除以总员工数来计算。该公司最大的产品线是半导体,可是在今天通货膨胀率偏高的世界中,很少产品像半导体那样,单位价格跌个不停,而且该公司的美国工厂,工资每年上涨七%,意大利和日本则上涨二十%,所以我们能够合理地预期,虽然人员效能改善,这个指数一定会下跌。事实不然,它竟从一九六九年的二.二五%左右,上升为一九七三年底的二.五%。此外,虽然为了这些改善计划,必须再推动其他计划,而且利润分享资金必须进一步提高,公司仍宣布,目标是一九八〇年前把指数提高到二.一%──这个目标如果达成,这家公司会是很赚钱的工作场所。多年来,德仪经常公开发表相当有野心的一些长期目标,而且迄今多能实现。

从投资的观点来说,前面所举以人员为导向的计划,三个例子有一些非常重要的类似性;我们所以选这三个例子,是为了说明保守型投资工具第二个要素的一些层面。提及摩托罗拉公司设立学院,甄选与训练出色人才,以因应公司的成长需求,并作一般性的描述,说起来相当简单。提到道氏公司找到一种方法,激励员工携手合作,共同处理环境问题,并为公司赚到利润,或者列举一些事实,阐述德仪的人员效能计划办得有声有色,说起来也一样简单。但是,如果另一家公司决定从头开办类似的计划,可能碰到的问题,或许远比单纯说服董事会批准必要的经费复杂得多。这类计划很容易拟定,实施起来是另一回事。犯错可能带来昂贵的代价。要是类似摩托罗拉的训练学校选错擢升的人才,致优秀的资浅人员失望之余离开公司,我们不难想象可能发生什么事。同样的,假设一家公司试着大体上仿效德仪的人员效能计划,却未能创造一种气氛,让员工真正觉得参与其事,或者未能适当地奖励员工,结果可能大失所望。这种计划如果运用不当,有可能毁了一家公司。在此同时,企业本来就着有成效的人员导向政策和技巧,如能精益求精,通常能找到更多的方法,从中获益。对这些公司来说,这样的政策和技巧──面对问题和解决问题的特殊方式──带有只此一家,别无分号的味道。正由于这个理由,对长期投资人来说,它们十分重要。
\\

\textbf{三、管理阶层必须愿意以身作则,遵守公司成长所需的戒律。}


我们已经指出,在这个变迁迅速的世界中,公司不能有如一滩死水。它们不进则退,不是变得更好,就是变得更差;不是往上走,便是走下坡。真正值得投资的成长股对象,不只有所斩获,还要能避免损失。很少管理阶层不宣称自己的公司正在成长。不过,管理阶层说自己以成长为导向,实际上不见得如此。很多公司似乎有一股难以抗拒的驱力,想在每个会计期间结束的时候,拿出可能得到的最大利润给大家看──把可能赚到的每一分钱都算进盈余里。真正成长导向的公司,绝不会做这样的事。它念兹在兹的,应是赚取充分的当期利润,以挹注事业扩张所需的成本。所需额外资金经调整后,值得长期投资的公司,遇有发展新产品、制程、创设新产品线或其他的大好机会,优先要务是仰制立即获取最高的利润,希望今天花下的一块钱,未来赚回好几块钱。这样的行动,包括雇用和训练企业成长所需的新人,以及放弃从客户的订单赚取最高的利润,而在客户亟需之际,紧急配合,供应所需,建立起客户长期的忠诚。对保守型投资人来说,检验这些行动的好方法,是看管理阶层有没有真正为公司长期的利润奋斗,而非只是表面看起来如此。一家公司不管多有名,如果只是嘴巴上说说有这些政策,投资人把钱投资到它的股票,不可能得到快乐。想要见贤思齐,但半途而废的公司,也是一样。比方说,一家公司花了庞大的研究支出,但努力方向不对,可能毫无所获。

\section{第三个要素——若干企业的投资特征}

保守型股票投资的第一个要素,是对目前和未来获利力最重要的公司活动,卓越到什么程度。第二个要素是控制这些活动和相关政策的人员素质。第三个要素谈的东西有点不一样:企业本身的特性是不是带有若干与生俱来的特质,在可预见的将来,有可能长久维持高于平均水平的获利力。

检视这些特质之前,我们必须指出为什么高于平均水平的获利力对投资人那么重要,因为那不只是进一步利得的来源,更能保障已经获有的利益。关于这一点,公司成长扮演极其重要的角色,前面已经讨论过。公司成长,必须在很多方面花钱。原本可做为利润的一部分资金,必须挪用于实验、发明、试销、新产品营销,以及挹注扩张所需的其他所有营运成本,包括这些扩张行动难以避免的失败带来的损失。更花钱的是厂房、店面、设备需要增添。在此同时,随着企业成长,为了应付流通通路所需,存货势必增加。最后,除了极少数企业能够银货两讫,大部分公司的应收帐款会提高,而耗损公司的资源。为了应付所有这些事情,获利极其重要。

通货膨胀期间,获利力的问题更为重要。物价上扬并导致成本普遍上涨时,企业能够适时把成本转嫁出去,提高本身产品的价格。但这事往往不能立即做到。在这段前后青黄不接的期间内,利润率高的公司,利润受损的程度显然远低于高成本竞争对手,因为高成本公司的成本可能增加得较多。

获利力可以用两种方式表示。投资资产报酬率是最基本的方法,大部分管理阶层以之为衡量准绳。企业会根据这个因素,决定要不要推出某种新产品或新制程。这笔钱以这种方式投资,相对于同样的资金以另一种方式运用,预期报酬率相较如何?这个衡量准绳,投资人运用起来远比企业高阶主管困难。投资人注意的,通常不是一家企业特定部门运用一定资金所能获得的报酬率,而是这家企业的总盈余相对于总资产的比率。值此资本财成本上升的速度和四十年来一样高之际,比较各公司的总投资资金报酬率,可能因为不同的公司做重大支出时,价格水平有所变动而遭扭曲,因此这个数字极易产生误导作用。由于这个原因,只要牢记一个要点,比较每一美元销售额的利润率,可能比较有帮助。这个要点是:销售额相对于资产比率较高的公司,和销售利润率较高,但销售额周转率较低的公司比起来,获利可能较高。举例来说,某公司的年销售额是资产的三倍,利润率较低,但和另一家公司必须运用一美元的资产才能取得一美元年销售额比起来,获利高出许多。不过,虽然从获利力(profitability)的观点来说,投资报酬率应和销售利润率一并考虑,但从投资安全性(safety of investment)的观点而言,所有的重点应放在销售利润率。因此,如果两家公司的营运成本都上升二%,而且没办法提高价格,则利润率为一%的公司将发生亏损而遭淘汰,但利润率为十%的另一家公司,成本增加只会使利润少掉五分之一。

要正确观察保守型投资的这个要素,应谨记在心的最后一件事情是:在今天变动不居和竞争激烈的企业环境中,大家都希望获得高于平均水平的利润率或高资产报酬率,因此,如果有家公司在一段相当长的期间内达成这个目标,势必面对一大堆潜在的竞争对手。要是潜在竞争对手真的踏进同一个领域,将侵占原来的公司现在拥有的市场。正常情况下,潜在竞争对手成为实际的竞争对手之后,继之而来的销售额争夺战,将使原有公司迄今保有的高利润率略微或大幅下降。原有公司的高利润率有如一罐没有封口的蜂蜜,难免吸引一群饥饿的昆虫竞相前来争食。企业世界中,一公司只有两种方法,可以保护蜂蜜罐,免遭竞争同业吞食。一种方法是独占,通常属非法行为,但源于专利权的独占,也许不然。无论如何,独占地位可能相当快便告结束,不能靠它成为最安全的投资对象。拥有蜂蜜罐的公司,驱离昆虫的另一种方法,是在营运上远比其他公司有效率,使得现有或潜在竞争对手找不到采取行动的动机,打乱现有的情势。

现在,我们结束相对获利力的背景讨论,探讨保守型投资第三个要素的核心──也就是,管理良好的公司大致上能够无限期维持高于平均水平利润率的一些特质。最常见的特质可能是企业人士所说的「规模经济」(economies of scale)。举一个规模经济的简单例子:一家经营良好的公司,每个月生产一百万单位的产品,和另一家公司同期内只生产十万单位比起来,单位生产成本通常较低。产量相差十倍的这两家公司,单位生产成本的差异,可能因为不同的营业项目而有很大的不同。有些营业项目的单位成本可能几无差距。此外,我们绝不能忘记,任何行业中,规模较大的公司,只有经营极为出色,才能发挥最大的优势。公司规模愈大,愈难用很高的效率管理。我们经常看到,公司内部有太多官僚习气浓厚的中阶经理人,使得效率低落,结果造成决策延误,而且有些时候,大公司的高阶主管似乎无法迅速知道鞭长莫及的各个部门,哪里需要特别注意,规模经济的内在利益因此大打折扣,坏处甚至超过它所带来的好处。

另一方面,一公司显然成为某一行业的领导者时,只要管理阶层能力高强如昔,不管是营业额,还是获利力,很少会被其他公司取代龙头地位。我们探讨保守型投资的第二个要素时谈过,这样的管理阶层必须能够配合不断变迁的外在环境,调整公司原来的营运方式。有一种投资思想学派,主张买进某一行业中排名第二或第三的公司股票,因为「它们能成为第一,领导公司却再也爬不上去,只有掉落的份」。有些行业中,规模最大的公司没有取得明显的领先地位;但如取得领先地位,我们要特别强调,不同意这种观点。我们观察到,历经多年的尝试,西屋公司(Westinghouse)一直未能超越奇异公司(General Electric),蒙哥马利华德公司(Montgomery Ward)没有取代施乐百公司(Sears),而且──国际商业机器公司(IBM)一在计算机市场建立早期的主宰地位──奇异等美国一些大公司再怎么使出浑身解数,都无法成功取代IBM压倒性的市场占有率。许多靠削价战竞争的小型外围设备供货商,也未能取代IBM的主流地位;IBM仍是计算机业中获利最高的主要营运商。

一家公司如何能够一开始便取得这种规模优势?通常是抢先推出能够满足殷切需求的某种新产品或服务,取得这种优势,然后辅以够好的营销、售后服务、产品开发,有些时候,必须做广告,让现有的客户满意,回来再买更多东西。这往往能够塑造一种气势,吸引新客户投向领导厂商怀抱,主要的理由是领导厂商已建立良好的声誉,(或很好的价值),选它不可能遭致批评。在其他公司力图争食IBM的计算机业务时,没人晓得有多少打算首次使用计算机的企业员工,建议使用IBM的计算机,而不购买规模较小的竞争对手的产品,虽然他们私底下认为后者的设备比较好或比较便宜。这种情况中,主要的动机可能是他们觉得,万一日后设备运作出了问题,建议买IBM计算机的人不会遭致批评,因为他们选了业界领导厂商。但如所选是尚未建立声誉的小公司产品,运作之后出问题,很可能被骂得狗血淋头。

制药业有种说法:一种优秀的新药品问世后,率先踏入的公司会占有并维持六十%的市场,获得很高的利润。跟进推出同类竞争产品的第二家公司,可能占有二十五%的市场,利润普通。第三家公司分走十%到十五%的市场,利润微薄。再有其他公司跟进,会发现自己处于相当不愉快的境地。一般品牌取代自有品牌的趋势,可能不会改变这个比率,但这个公式不见得恰好适用于其他行业。不过,投资人评估哪家公司的获利力具有天生的优势,哪家公司没有这种优势时,背后的观念应谨记在心。

规模大能够不断带给一家公司的竞争优势,不只在于生产成本较低,以及因为品牌名气响亮,更能吸引新客户上门。金宝浓汤公司(Campbell Soup Company)快餐汤部门具有投资优势,检视背后的一些因素,可以明了这一点。首先,到目前无止,它是美国最大的罐装快餐汤公司,能以向后整合的方式,减低总成本,但规模较小的其他公司做不到。该公司生产很多汤罐,满足本身的需求,便是其中一例。更重要的是,金宝的业务量够大,能把装罐厂分散设立在全国各地具有策略性意义的地方,因此带来双重大优势:供货商把产品运到装罐厂的距离较短,装罐厂送货到超级市场的平均距离也较短。由于罐装汤品的重量相对于价值偏高,运货成本相当可观。因此,只有一两座厂房、规模较小的装罐厂,想在全国市场竞争,便居于十分不利的劣势。其次,可能最重要的是,金宝的品牌十分响亮,顾客到超级市场时,晓得有这个牌子,也想买这个牌子,零售商自然而然会在十分宝贵的橱架空间,腾出显目和相当大地方,摆放金宝的产品。相对的,零售商通常不愿对名气较差或默默无闻的竞争同业产品,做那么多事。引人注目的橱架空间,有助于金宝浓汤的销售,使得排名第一的宝座更形稳固,这个因素对潜在竞争对手具有很大的吓阻力量。潜在竞争对手不敢妄动的另一个原因,是金宝的正常广告预算换算为单罐成本后,远低于产量很小的竞争对手。基于以上所述种种理由,这家公司具有很强的内在力量,足以保护既得的利润率。不过,为了呈现完整的画面,我们必须指出,已有一些影响力量正往反方向拉扯。金宝本身的成本上升时(通货膨胀期间,成本有可能大幅攀高),卖给消费者的涨价幅度不能高于其他食品的平均幅度,否则需求可能从汤食转向其他主食。远为重要的是,金宝有个很强大的竞争对手,而这个竞争对手,大部分公司都不必与之相抗。随着生产成本上升,导致消费者负担的价格提高,金宝的市场可能大受影响。这个主要竞争对手就是美国的家庭主妇,为了撙节开销,亲自下厨做羹汤。我们要提这一点,只是为了说明:即使规模大能带来显著的竞争优势,而且公司经营得很好,这些因素虽然重要,但不能保证获有很高的利润。

一公司相较于其他公司,能够长期维持很高的获利力和投资吸引力,规模绝非唯一的投资因素。我们相信特别有意思的另一个因素,是在科技领域中与经营十分成功、根基相当稳固的公司竞争很不容易,尤其是有关的科技不是靠一种学科,而是两种,甚至多种相当不同的学科交互作用的结果。要解释上面这句话的意思,我们假设某人发展出一种电子产品,可望在计算机或仪具市场开启广大的新市场。两个领域都有不少能力很强的公司,拥有内部专家,能够仿制这种产品所需的电子硬件和软件程序设计,因此如果新市场显得够大,激烈的竞争可能马上出现,使得规模较小的发明人的利润变得相当微薄。这种领域中,经营成功的大公司拥有进一步的内在优势。许多这类产品很难销售,除非设有服务网,能在顾客所在地迅速提供维修服务。基础稳固的大公司通常已有这样的组织。推出优良新产品的小型新创公司,想建立这样的服务网极其困难,也必须花很多钱。新创公司可能更难以说服潜在买主,相信它的财力充足,不只能在产品出售后建立服务网,而且将来会继续维持下去。而且,在所有这些因素影响下,虽然有少数公司做到,但以前推出优良产品的新公司,很难在电子业的大部分领域建立起真正的领导地位,未来可能更加困难。这是因为半导体在愈来愈多产品的总成分和总技术知识中,占有愈来愈高的比重。生产这些产品的领导公司,如果想在大多为电子产品的许多新产品领域,和首屈一指的老牌计算机公司、仪具公司争锋,现在内部拥有的知识,也至少和它们一样多。德州仪器公司(Texas Instruments)在风靡一时、急剧成长的掌上型电子计算器方面做得极其成功,但早期一些开路先锋在这个领域经营得相当辛苦,便是很好的例子。

但如果生产出来的产品,不是依赖和电子硬件及软件有关的一种技术,而必须结合这些技术和相当不同的领域中的特别知识,如原子核物理学或某些高度专业的化学,情况就变得很不一样。大型电子公司内部根本没有这种技能,踏入跨学科的技术。这一来,经营出色的创新公司,有很好的机会,在自己的特殊产品线取得领先地位,并赚得很高的利润率,而且只要管理阶层的能力没有减弱,整个情况可能继续维持下去。我相信,在电子技术不是扮演吃重角色的领域中,一些跨学科的技术性公司,最近已证明是高赡远瞩的投资人可以把握的大好机会。我想,将来还会有更多这种机会。比方说,我预期将来某个时点,会有新的领导公司崛起,因为它们结合运用其他学科和生物学,推出新的产品和制程,不过目前我还没看到这个领域有那么出色的公司。但这不是说没有这样的公司存在。

一公司的活动中,与众不同的条件可能带来机会,使得高利润率维持长久,并不是只靠技术开发和规模经济这两个层面。有些情况中,营销或销售等领域也会出现这样的事。一个例子是,某家公司已让客户养成习惯,在再订购单中几乎自动标明产品的规格,竞争对手想要依样画葫芦,取而代之,成本效益很不划算。这样的事情要发生,需要两组条件存在。第一,这家公司必须在产品质量和可靠性上建立起声誉,而且(a)客户认为这些产品非常重要,才能执行本身的业务,(b)低劣或功能不全的产品会带来严重的问题,(c)竞争同业只供应一小部分市场,在大众心目中,居于主宰地位的公司几乎就是供应来源的同义词,以及(d)在客户的总营运成本中,这项产品的成本只占相当低的一部分。所以说,价格略微调低,能够节省的钱不多,但找不知名供货商的风险很大。不过,即使一家公司很幸运,挤身于这种地位,光是这一点还不足以确保它年复一年享有高于平均水平的利润率。第二,它必须有某种产品,卖给很多小客户,不是只卖给少数大客户。这些客户的特性必须相当专业化,潜在竞争对手才不会觉得,透过杂志或电视等广告媒体,有可能接触他们。他们所构成的市场,只要居于主宰地位的公司维持产品的质量和良好的服务,便只有消息灵通的业务员个别拜访,才有可能取代这家公司的位置。可是每位客户的订单都那么小,这样的销售努力根本不划算!拥有所有这些优势的公司,可以透过营销,几乎无限期维持高于平均水平的利润率,除非有重大的技术变迁(或者,如前面提过的,它本身的效率减退),取代它的地位。这类公司最常见于中高科技供应领域。它们的特质之一,是经常举办技术研讨会,讨论如何使用它们的产品,维持领导厂商的形象。公司一旦到达这种地位,研讨会是很有效的营销工具。

我们要特别指出,「高于平均水平」的利润率或「高于正常水平」的投资报酬率,不必──实际上是不该──高达业界一般水平的好几倍,这家公司的股票才有很大的投资吸引力。事实上,如果利润或投资报酬率太惹人觊觎,反而可能成为危险之源,因为各式各样的公司可能禁不起诱惑,想来一竞长短,争食蜂蜜。相对的,只要利润率一直比次佳竞争同业高出二%或三%,就足以做为相当出色的投资对象。

出色的保守型投资的第三个要素,总结而言是:不能只有第二个要素讨论的人员素质,而且还要那些人员(或他们的前任)引导公司踏进某些活动领域;那种特殊事业的性质具有某些内在经济因素,高于平均水平的利润率将不是短期现象。简单的说,关于第三个要素,我们要问的问题是:「这家公司能做些什么事,其他公司没办法做得那么好?」如果答案是没有这样的事,这一来,随着那家公司业务欣欣向荣,其他公司也能抢进,和该公司平起平坐,分享同样的荣景。因此,我们即可据以作成结论:该公司的股票或许很便宜,却不符合第三个要素的理想投资条件。

\section{第四个要素——保守型投资的价格}

任何股票投资的第四个要素,涉及本益比(price-earnings ratio)──也就是当时的价格除以每股盈余。要评估某支股票的本益比是否符合它的正确价值,问题便来了。大部分投资人,包括许多应该懂得更多的专业人士,在这一点经常搞混,因为到底什么原因导致某支股票的价格显著上涨或下跌,他们了解得不清楚。这种误解使得投资人损失数百十亿美元,因为后来他们才发现,当初不应该用那么高的价格去买股票。投资人在错误的时间,基于错误的理由而卖出持股,又损失数百十亿美元,因为不管如何,他们都该抱牢那些股票,长期投资,获利非凡。如果这样的事一再发生,另一个后果是本来值得投资的公司,筹募适当资金的能力会严重受损,可能导致每个人的生活水平都下降。每次个别股票跌得令人作呕时,总有另一群严重烧伤的投资人怪罪到整个制度上,而不反躬自省本身和他们的顾问犯下的错误。他们做成结论说,任何类型的普通股都不适合他们拿储蓄去投资。

硬币的另一面是,其他许多投资人因为多年来持有正确的股票很长的时间,获利十分可观。他们所以成功,可能是因为了解基本的投资原则,或者只是运气不错。但是共同的成功因素,是他们拒绝单因某支股票急涨之后,本益比相较于投资圈习以为常的水平,突然显得偏高,而卖出好得出奇的股票。

这件事那么重要,却很少有人深入表层,确切了解是什么原因导致价格急涨,实在叫人不解。可是有关的原理,说穿了相当简单:任何个别普通股相对于整体股市,每次价格大幅波动,都是因为金融圈对那支股票的评价发生变化。

我们来看看实务上这种事情是怎么发生的。两年前G公司被视为相当普通。每股盈余为一美元,价格是盈余的十倍,也就是十美元。两年来,大部分同业的盈余都是每下愈况。相对的,G公司由于一连推出不少优良的新产品,加上老产品的利润率比以前好,去年报告每股盈余是一.四○美元,今年则为一.八二美元,未来几年还可望进一步成长。很明显的,G公司近来的业绩和业内其他公司形成强烈的对比,公司内部所采取的行动,绝对不是只起于两年前,一定已经持续了一段相当长的时间,否则营运上的经济效益和优良的新产品不可能出现。但是G公司符合我们所说的前三个要素,这个事实,金融圈迟迟才给予肯定(评价),并导致本益比上升到二十二倍。其他股票也有高于平均水平的类似经营特质,成长前景等量齐观,和它们比起来,G公司二十二倍的本益比看起来并没有那么高。二十二乘以一.八二美元,得出股价为四十美元,因此两年来股价非常合理地涨了三百%。同样重要的是,像G公司那样的纪录,往往显示公司现在拥有能力高强的管理团队,可以引导公司未来好几年继续成长。这样的成长率即使偏低一点,比方说,未来十年或廿年每年成长十五%,到时盈余还是很容易便成长几十倍,而非几倍而已。

「评价」(appraisal)是了解变幻无常的本益比的关键。我们绝不能忘记,评价是很主观的事。它不一定和现实世界中发生的事有关系。相反的,这要看做评价的人,相信正在发生什么事,不管他的判断和事实有多大的出入。换句话说,任何个股不会因为该公司实际发生的事情,或将发生的事情,而在任何时点上涨或下跌。它是因为金融圈对正在发生的事以及将发生的事,目前持有的共同看法而上涨或下跌,不管这种共同的看法和真正发生或将发生的事差多远。

走笔至此,许多务实的人会对以上所说嗤之以鼻。如果个股价格大幅波动,只是因为金融圈的评价改变,而这些评价有时和一公司在现实世界中发生的事完全脱节,那么另外三个要素岂非不重要?我们何必熟悉企业的经营管理、科技和会计?为什么不靠心理学家就好?

答案和时机有关系。由于金融圈的评价和事实不符,一支股票的价格可能长期远高于或远低于它的实质价值。此外,金融圈内很多人习惯玩「追随领导人」的游戏,特别是领导人为纽约市某大银行时。因此,有时金融圈对某支股票的评价有欠务实,导致它的价格远高于事实能够支撑的水平,却仍能维持过高的价位很长的时间。其实,从这个已经太高的水平,价格还有可能再涨。

金融圈对一支股票的评价,和影响价格的真实状况间的差距过大,可能持续好几年的时间。不过泡沫总会破掉──有时几个月,有时则在很久以后。如果因为不切实际的预期,一支股票的价格太高,迟早会有愈来愈多的持股人厌倦于久候。他们的卖盘马上压过仍对旧评价有信心的少数新买盘。于是股价重跌。有时继之而来的新评价相当踏实。但是在价格下跌的情绪性压力下重新做出的检讨,往往过度强调负面因素,使得金融圈的新评价远比实情不利,而且这种看法可能持续一段时间。这种现象发生时,出现的事情会和评价太有利时很像。唯一的不同点是整个状况倒反过来。可能需要几个月或几年的时间,较为有利的印象才会取代目前的印象。不过,随着令人愉快的盈余数字节节上涨,迟早会发生这样的事。

幸运的持股人──不因为股价开始上涨就卖出持股的投资人──接下来便能从这个现象中获得利益,也就是相对于股市有关的风险,能够获得最大的报酬。在每股盈余稳定攀升,以及本益比同时急剧上扬两者共同作用下,股价将大幅上涨。随着金融圈从这家公司的基本面(现在是新的印象),正确地发现它的投资价值远高于旧印象发挥作用时体认到的价值,导致股价上涨的因素中,本益比上升往往比每股盈余实际上升更重要。G公司的情形正是如此。

我们现在开始能够真正观察保守的程度──也就是任何投资的基本风险。风险量尺的最底端,也就是最适合聪明人投资的公司,前面所说三个要素符合的程度甚高,但根据目前金融圈的评价,价值没有基本面事实应有的水平那么高,因此本益比较低。风险次低,通常也适合聪明人投资的公司,前面所说三个要素符合的程度相当高,而且留下应有的印象,本益比大致上吻合基本面。这是因为如果这些公司真的拥有这些特质,未来将继续成长。风险再次低者,依我之见,通常适合保守型投资人持有,但不适合新资金首次购买。这样的公司,前面所说三个要素也一样好,但因为这些特质几已成为金融圈人尽皆知的事实,受到的评价或本益比高于基本面应有的水平。

依我的看法,虽然价格看起来很高,这些股票通常应予保留,理由很重要:如果基本面真的很强,这些公司的盈余迟早会上升到不止足以让目前的价格看起来合理,而且还能支撑价格涨得更高。在此同时,以前面三个要素的标准来说,真正吸引人的公司,数目很少。价格低估的股票不容易找到。对一般投资人来说,犯下错误,转而买进那些看起来符合前面所说全部三个要素,但其实不然的股票,这样的风险,远高于抱牢本质绝对良好,但目前价值高估的股票所带来的暂时性风险,因为真正的价值迟早会赶上目前的价格。同意我所说这一点的投资人,应做好心理准备,忍受这些价值暂时高估的股票市价偶尔重跌。另一方面,根据我的观察,有人卖出这种股票后,希望等到更适当的时机再买回同样的股票,很少能够如愿以偿。他们等候的跌幅,通常实际上从没出现。结果是,几年后,这些基本面很强的股票升上的高价,远高于当初卖出的价格,他们错失了后来出现的全部涨幅,而且可能转而买进本质上差很多的股票。

风险再高一点的股票,前面所说的三个要素表现普通,或者素质相对偏低,但是金融圈的评价低于这些不怎么吸引人的基本面,或者大致上吻合。评价低于基本面条件的股票,或许是很好的投机对象,但不适合明智审慎的投资人。今天变动迅速的世界,充满太多危险,不利的情势发展可能严重影响这些股票。

最后是到目前为止最危险的股票:金融圈目前给某些公司的评价或留下的印象,远高于眼前的情势所能支撑者。买进这些股票可能损失惨重,并驱使投资人大量脱售持股,而动摇投资业的根本。如果投资人想要逐件研究金融圈在某个时点对一些当红公司盛行的评价,以及此后出现的基本面状况两者间的对比,他会发现商业图书馆和或华尔街大型经纪商有很丰富的材料。经纪商的报告,总是列举一些理由,建议买进这些股票,接着我们比较字里行间所提未来远景和实际发生的事情,读之令人不禁蹙眉。随便举一张这些公司的例子,可能包括:记忆器材(Memorex)高价一七三.八七五美元、安培斯(Ampex)高价四九.八七五美元、利瓦伊兹家具(Levitz Furniture)高价六○.五美元、摩霍克资料科学(Mohawk Data Sciences)高价111美元、利顿实业(Litton Industries)高价一○一.七五美元、卡尔瓦(Kalvar)高价一七六.五美元。

这张清单还可以拉得很长。但是更多的例子不过使同一个论点再三浮现。由于我们可以很清楚地看出,评估金融圈目前对某家公司的评价,以及这家公司实际的基本面两者间可能存在的差距,这样的习惯很重要,所以把时间花在进一步检视金融圈所做评价的特质,应会更具建设性。但是首先,为免产生不必要的误解,最好把前面所说,导致普通股价格大幅变动的背后理由一句话中,两个名词定义清楚,以免造成语义上的混淆:任何个别普通股相对于整体股市,每次价格大幅波动,都是因为金融圈对那支股票的评价发生变化。

我们使用「价格大幅波动」一词,而不用「价格波动」,排除了技巧拙笨的营业员在市场中急抛二万股股票,使得股价下跌一、二美元,等到抛售行动结束,价位通常恢复原来的水平等价格小幅波动的情形。同样的,有些时候,机构投资人可能决定至少买进某支股票一定的数量,结果往往导致价格瞬间小幅上扬,等到买盘结束,又回软到原来的价位。这样的行动,不代表整个金融圈对某家公司的评价真的有所改变,因此对股价不重要或者没有长期性的影响效果。特殊的买盘或卖压一结束,这种小幅的价格波动往往随之消失。

我们使用的「金融圈」(financial community)一词,包括全部有能力和有兴趣的人,他们可能准备以某个价格买进或卖出某支股票。每位潜在买主和卖方对价格可能造成的冲击,重要性需看他们行使的买进或卖出力量有多强而定。

\section{再论第四个要素}

关于金融圈对一支股票的评价,走笔至此,可能让人以为,这种评价不过针对特定股票本身而进行。这种想法未免过度简化。其实,最后的评价是由三个不同的评价综合而成:目前金融圈对整体普通股投资吸引力的评价、对某公司所处行业的评价,最后则是对该公司本身的评价。

我们先来讨论金融圈对整个行业的评价。大家都知道,长期而言,一个行业从市场潜力雄厚的早期发展阶段,到后期可能受到新科技的威胁,整个过程中,金融圈希望参与该行业,愿意支付的本益比可能大幅下降。所以说,在电子业的早期阶段,业者生产的是所有电子产品使用的基本零组件电子管,股票的本益比很高。接下来,随着半导体的发展,电子管市场逐渐被取代,业者的本益比急剧下降。最近磁性内存装置也因为同样的理由,受害于同样的命运。这些事情都很明显,大家也知之甚详。但金融圈对某种行业的印象可能有起有落,原因不是出在这些沛然莫之能御的影响力量,而是在某个特定的时点,金融圈强调某些业界背景影响力量甚于其他影响力量。可是这事没那么明显,大家也不是那么能够理解。不过两组背景条件可能在一段时间内都有效,而且从任何一个角度看,两者的影响力都有可能再持续一段时间。

化学业或可做为例子。从经济大萧条的谷底一直到一九五〇年代中期,美国最大几家化学公司的股票本益比,和其他大部分股票相比,显得相当高。金融圈对这些公司的想法,也许能从一幅漫画看出端倪。这幅漫画把化学业描绘成毫无尽头的输送带,一边有科学家正在试管中做新的化合物,叫人屏息以待。经过神秘和难以模仿的工厂,这些材料在另一边以妙不可言的新产品面貌出现,如尼龙、滴滴涕(DDT)、合成橡胶、快干漆,而且不胜枚举的其他新材料,对幸运的制造商来说,似乎肯定将是日益扩增的财富来源。到了一九六〇年代,这样的印象改变了。在投资圈眼里,化学业变得很像钢铁、水泥、造纸业,以某种技术规格为基础,销售大宗商品,结果张三的产品和李四的产品大致相同。资本密集产业通常承受很大的压力,必须以很高的设备利用率运转,才能摊销庞大的固定投资。结果往往导致激烈的价格竞争,以及利润率萎缩。印象改变之后,一九七二年结束的十年内,主要化学公司相对于整体股市的本益比,远不如从前。化学业的本益比虽然仍显著高于许多行业,却开始愈来愈像钢铁、造纸和水泥。

所有这些变化中,叫人称奇的是,除了一件重要的事例外,一九六〇年代这个行业的基本背景和前面卅年相比,几无不同,或者根本没有不同。没错,一九六〇年代后面五年,某些产品的产能严重过剩,如大部分合成纺织品。一些领导化学公司的盈余暂时大受影响,特别是杜邦公司。但这个行业的基本特质没有那么大的变化,足以令它在金融圈心目中的地位丕变。化学品生产一直都是资本密集工业。大部分产品都按照某种技术规格出售,所以张三很少能够把价格提高到李四之上。另一方面,由于一大堆质量大幅改善的新杀虫剂、包装材料、纺织品、药品,以及其他无数产品问世,一九六〇年代和一九七〇年代这个行业的市场日益扩大。机会似无止尽,因为聪明的人类能够重新排列分子,创造出大自然找不到的产品,具备特殊的性质,更能迎合人类的需求,或者比以前使用的天然材料便宜。

最后,不管在化学股以前地位较为崇高的时期,还是最近地位较为低落的时期,还有另一个因素一直没什么改变。较老旧和数量较大的化学产品,是把盐或碳氢化合物的基本分子来源特别做成的材料进行「第一步」加工处理,不可避免地,主要必须依规格出售,而且价格要很有竞争力。但是惊觉性高的公司一直有机会,把这些第一步产品加工成远为复杂和价格高出许多的产品。这些产品至少在一段时间内是自己特有的产品,免除了激烈竞争之苦。等到这些产品又必须具有价格竞争力的时候,警觉性高的公司还是能够不断找到更新的产品,加进高利润率的产品线中。

换句话说,化学股是金融圈最爱的时候,所有有利的因素,在后来化学股失宠时依然存在。但一九六〇年代为人强调的不利因素,以前也存在,只是被人忽视。改变的是强调的重点,不是事实真相。

但是事实真相也会改变。一九七三年年中左右,化学股再获金融圈青睐。对这个行业的新看法,开始居于主流地位。现代历史中,主要工业国家首次(大规模战争除外)体验到经济资源匮乏的痛苦,制造业产能只能缓慢增加。因此,激烈的价格竞争可能要等好几年以后才会再度发生。这个印象为化学股投资人开启了全新的局面。投资人现在的问题是:确定背景事实是否真能支撑新印象;如果能够,则依新情势来看,化学股相对于整体股市的涨幅,已经太高还是仍嫌不足。

最近的金融史提供了本益比变动比化学业高出许多的例子,因为金融圈对某行业背景事实的评价大幅改变,但该行业几与从前毫无两样。一九六九年,计算机周边股甚受市场垂青。这些公司生产各种特殊设备,可以附加到中央运算单位或计算机的主机,以增进用户从中央单位获得的利益。高速打印机、额外的记忆单位、键盘装置,是其中一些主要产品。有了它们,就不需要打孔作业员把数据输入计算机。当时一般人的印象中,这些公司的未来无可限量。虽然中央计算机本身大致已发展完成,而且市场由少数强大、地位稳固的公司主宰,小型独立公司还是能够在这些周边产品上,卖得比大公司便宜。但是小型公司的产品通常采出租方式,而非卖断,同时大型计算机主机制造商决心进军「挂在」它们设备上的产品市场,今天投资人有了新的看法,发现小型公司的财务有压力。到底是基本面改变了,还是针对基本面的评价发生改变?

评价改变的一个极端例子,是一九六九年相对于一九七二年,金融圈对连锁加盟事业和加盟股票基本面的观感大为不同。同样的,就计算机周边股票来说,人们以很高的本益比买进这些股票时,这个行业的所有问题本质上已经存在,只是当时一般的印象是眼前表现突出的公司,成长不会中断,因此对问题视而不见。

对一个行业持有这样的印象,投资人的问题还是没有改变。目前居于主流地位的评价,和基本经济事实比起来,比较有利?比较不利?或者大致相同?有时对最精明的投资人来说,这个问题也很难棘手。一九五八年十二月就有这样一个例子。那时候,作风一向保守的投资银行美邦公司(Smith, Barney & Co.)率先采取的一项行动,在今天看来没什么,当时则与人不同:公开发行尼尔逊公司(A.C.Nielsen Co.)的股票。这家公司没有厂房,没有有形产品,因此没有存货。它从事的是「服务业」,收取费用,供应市场研究信息给客户。没错,一九五八年,银行和保险公司长久以来,一直获得很高的评价,被市场视为优良的保守型投资对象。但是这些行业很难相互比较。一家银行或保险公司的账面价值是现金、流动性投资或应收帐款,所以买进银行或保险股的投资人,似乎有坚实的价值足资依赖,但新引进金融圈的那家服务公司,没有这样的东西。可是仔细研究尼尔逊公司的情况,发现它的基本面好得出奇。管理阶层诚信正直、能力高强,公司竞争地位坚强且独特,未来好几年可望进一步成长。不过,在证据显示金融圈首次对这样一种行业有什么反应之前,似乎仍宜暂缓买进。是不是需要好几年的时间,才能对这样一家公司的投资价值,做出切合实际的评价,袪除因为缺乏熟悉的价值量尺可能引起的疑虑?像尼尔逊这样一家公司,多年来的高本益比表示金融圈给它很高的评价,但有些人决定接纳已获认可的基本面,买进这些股票时,却带着如履薄冰的心情,宛如跳下悬崖,看看底下的空气能否支撑我们。这样的行为,今天看来实在荒谬可笑,但当时服务公司的观念还很新颖,与我们已经习惯的观念不同。事实上,几年内钟摆摆向了另一边。随着尼尔逊公司的盈余成长再成长,华尔街兴起一种新观念。很多公司不分青红皂白,被混在一起,形成金融圈的一种印象,视为极具吸引力的服务业。这些公司从事的虽然都是服务业,不是生产商品,但经济基本面相差甚远。有些公司的本益比开始高于应有的水平。一如以往,基本面终将主宰一切。把本质上不同的公司凑成一种类股,形成的错误印象,慢慢褪色。

这一点十分重要:保守型投资人对某支股票感兴趣时,一定要了解金融圈目前对该公司所处行业的评价性质。他必须不断研究调查,观察这种评价是否远比基本面事实有利或不利。只有正确分析这一点,则主宰该行业个股市场价格长期趋势的三个变量中的一个,实情才能相当确定。

\section{三论第四个要素}

就影响本益比的因素来说,金融圈对一家公司本身特质的评价,比金融圈对该公司所处行业的评价还重要。关于个别公司最理想的投资特质,前面讨论保守型投资的三个要素时已经给予定义。大致来说,金融圈对特定股票的评价愈接近这些特质,它的本益比愈高。评价如低于这些标准,则视低落的程度如何,本益比倾向于下降。投资人如能运用聪明的头脑,确定值得投资的特定公司的事实真相,远优于或远劣于目前金融圈对该公司的印象,便很有可能知道哪支股票的价格显著低估或显著高估。

确定两支或多支股票的相对吸引力时,投资人往往试图利用过于简单的数学方法处理这种问题,而把自己搞混。假设他们比较的公司有两家,经审慎研究后,发现它们未来每年的成长率可能都有十%。如果其中一家的价格是盈余的十倍,另一家是廿倍,则本益比十倍的股票看起来比较便宜。或许是吧,但也不见得必然如此。理由有许多。表面上看起来较为便宜的公司,可能利用相当多的杠杆资金(必须先支付利息和优先股股利,盈余才归普通股持有人),则本益比较低的股票预期中的成长率无法实现的风险可能高出许多。同样的,纯就企业经营层面来说,由于两支股票的成长率都只是最有可能实现的估计值,如果有出乎意料的事件发生,则对某支股票估计值的影响,可能远高于对另一支股票估计值的影响。

过份依赖单纯的比较方法,观察成长机会类似的几支股票的相对本益比,可能做成错误的结论。这种犯错方式远为严重,却很少人了解。为了说明这一点,假设两支股票未来四年盈余都很有可能倍增,而且目前的价格都是盈余的廿倍,可是同一市场中,其他没有成长展望,但体质良好的公司,价格为盈余的十倍。再假设四年后,整体股市的本益比没有变化,体质大致良好,但没有成长展望的公司,价格仍为盈余的十倍。同时,四年后,两支股票中的一支,未年的成长展望和四年前大致相同,因此金融圈的评价是:未来四年这支股票的盈余应该还会再增加一倍。这表示,过去四年它的盈余已经倍增,但价格仍会是这个盈余数字的廿倍,换句话说,那段期间内,它的价格也升高了一倍。相对的,四年后第二支股票的盈余正如预期,也增加一倍,但这时金融圈对它的评价,是未来四年盈余将显平疲,不过体质仍然良好。这表示,虽然四年内盈余增加一倍的预期已经实现,第二支股票的持有人将大失所望。由于金融圈对第二支股票的印象是「未来四年盈余不会成长」,现在会预期它的本益比只有十倍。所以说,尽管盈余已经增加一倍,这支股票的价格还是和以前一样。所有这些,可以总结成一条基本的投资原则:未来的盈余继续成长的可能性愈高,投资人负担得起的本益比愈高。

但是应用这个原则时必须非常小心。投资人千万不要忘记,本益比实际上的变化,不是受实际发生的事情影响,而是导因于金融圈目前相信什么事情会发生。市场人气普遍乐观的期间,一支股票可能因为金融圈准确预测到未来很多年成长率很高,而有极高的本益比。但是我们必须等很多年过去之后,这个成长率才会完全实现。准确反映在本益比的高成长率,可能有一阵子「没有反映」,特别是如果公司经营上出现暂时性的退步;即使最优良的公司,也会发生这样的事。市场人气普遍悲观的期间,一些很好的投资对象这种「没有反映」的情况,可能相当严重。这种情形发生时,耐性够的投资人,如能区辨目前的市场印象和事实真相间的不同,便能以相当低的风险,找到一些长期获利十分可观的普通股大好投资机会。

一九七四年三月十三日有个精彩的例子,可用以说明精明的投资人如何分析投资圈对一家公司的评价将发生变化。前一天,纽约证券交易所摩托罗拉公司(Motorola)股票收盘价是四八.六二五美元。三月十三日收盘价为六十美元,涨了约二十五%!原来十二日交易所收盘后,摩托罗拉公司宣布将退出电视机事业,美国的电视机工厂和存货要以接近账面价值的价格,卖给日本大厂松下公司。

投资人普遍知道摩托罗拉的电视机事业有小幅亏损,并因此拖累其他事业的盈余。这件事本身就足以使股票价格升高若干,但不应如实际的涨幅那么大。买盘背后的主要动机,源于更复杂的推理。一段期间以来,相当多投资人相信摩托罗拉赚钱的事业部门,特别是通讯事业部,使得这家公司成为美国极少数非常值得投资的电子公司之一。例如,史宾塞崔斯克公司(Spencer Trask and Co.)曾经发表证券分析师欧提斯·布雷德里(Otis Bradley)撰稿的一份报告,十分详细地讨论摩托罗拉公司通讯事业部的投资价值。这份报告使用不同寻常的方法,只计算一个事业部目前和未来估计的本益比,不谈摩托罗拉整体的盈余。报告中比较这个事业部和惠普公司(Hewlett Packard)、普宜公司(Perkin-Elmer)的估计营业额和本益比;就投资观点而言,后两家公司普遍被视为非常优秀的电子公司。根据这份报告,我们很容易推论出(文章中没有明确提及)摩托罗拉通讯事业部的投资质量十分优良,单是它的价值,便值回摩托罗拉当时的股价。所以说,以那个价格买摩托罗拉公司的股票,恰好等于买到整个通讯事业部,其他所有的事业部则免费奉送。

市场上对摩托罗拉一些高复杂性的产品线有这样的看法,那么,到底是什么因素,在电视机事业卖给松下的消息宣布之后,促使买盘大量拥现?大买摩托罗拉股票的人,长久以来都知道,金融圈内很多人对这支股票不屑一顾,因为它给人留下的印象是电视机制造公司。金融圈内大部分人一听到摩托罗拉,马上想到电视机,接着才想到半导体。电视机事业卖给松下的消息宣布时,标准普尔公司(Standard&Poor′s)的股票指南中,用以列出每家公司主要事业的一小块地方,说摩托罗拉生产的是「收音机及电视机:半导体」。这种看法虽然没有不对,却有误导作用,因为和摩托罗拉实际的状况不符,而且完全忽视了非常重要的通讯事业部;那时通讯事业部占整个公司的一半左右。

在电视机事业卖给松下的消息出现后,抢进摩托罗拉股票的一些人,无疑只是因为这是利多消息,可望导致股价上扬。但是大量买盘背后有个原因,在于投资人相信金融圈对这家公司的评价一直远比事实真相为差。从历史纪录可以看出,在电视机事业,摩托罗拉被视为只是「在后头苦苦追赶者」,不如业界领导厂商增你智(Zenith)。既然电视机业务不再模糊投资人对该公司其他事业部门的看法,一种新的印象将形成,本益比会升高很多。

以高价抢进摩托罗拉的人做得很聪明?不尽然。接下来几个星期,股价吐回当时应声上涨的涨幅,因此稍有耐性等候还是对的。市场下跌时,金融圈对一家公司的印象变坏,比变好更快为人接受。市场上涨时,情况则恰好相反。因为松下的消息而抢购摩托罗拉股票的人很不幸,因为接下来几个星期,短期利率急剧回升,对整体股市造成下跌压力,并使当时弥漫的空头心理更为恶化。

或许还有另一股力量,也对抢进摩托罗拉股票的人不利。这股力量是整个投资领域中最微妙和最危险的一种,即使最老练的投资人也必须时时注意防范。当一支股票长期停留在某个价位区间时,例如低价三十八美元,高价四十三美元,投资人便会不知不觉中,把这个价位视为真正的价值。等到金融圈把这个价位当做那支股票的「价值」,并且根深蒂固,习以为常之后,如果评价改变,股价跌到二十四美元,则各式各样的投资人会一拥而出,大举抢进。他们遽然做成结论说,这支股票现在真的十分便宜。可是如果基本面很坏,二十四美元的价位可能还是太高。相反的,如果股价涨到五十、六十、七十美元,卖盘会倾巢而出,获利了结,因为很多人无法抗拒,认为这样的价格「偏高」。屈服于这样的冲动之下,可能损失惨重。投资股票想要赚到可观的利润,必须大量持有上涨好几倍的股票。一支股票的价格到底「便宜」,还是「偏高」,唯一真正的检定标准,不是目前的价格相对于以前的价格是高是低(不管我们多么习惯于以前的价格),而是这家公司的基本面显著高于或低于目前金融圈对那支股票的评价。

前面提过,除了投资圈对整个行业和特定公司的评价,我们还必须考虑第三个评价要素。只有这三个要素全部融合在一起,我们才能合理地判断某个时点一支股票到底便宜还是昂贵。第三个评价是针对整体股市的展望。为了观察某些时期中,整体股市的评价可以产生多么极端的影响,以及这些看法可以和事实相差多远,我们最好回头检视本世纪两次最为极端的评价。今天我们看起来或许觉得荒谬可笑,但一九二七到一九二九年间,金融圈绝大部分人真的相信我们处于「新纪元」。多年来,美国大部分公司的盈余只升不降。不只严重的景气萧条已成过眼烟云,连伟大的工程师兼企业家胡佛(Herbert Hoover)都被选为总统。他那杰出的能力,可望确保经济更为繁荣。这种环境中。许多人似乎觉得,持有股票几乎不可能赔钱。很多人想要尽可能从这件稳赚不赔的事中获利,于是融资买进本来不可能买到那么多的股票。事实真相终于粉碎那种评价,我们都知道后来发生了什么事。「大萧条」的苦痛,以及一九二九到一九三二年的空头市场,难以从人们的记忆中消除。

一九四六年年中到一九四九年年中三年内,投资圈对普通股做为投资工具的评价,展望与上面的情节恰巧相反,但错得一样离谱。当时大部分公司的盈余非常令人愉悦。但在当时的评价下,股票的本益比为多年来最低。金融圈的说法是:「这些盈余不算什么」,「它们只是昙花一现,在即将到来的经济萧条中,将急剧萎缩或消失不见」。金融圈记得内战之后是一八七三年的恐慌,开启了极其严重的经济萧条,一直延续到一八七九年。一次世界大战后,一九二九年的股市崩盘加上六年的大萧条,情况更糟。由于二次世界大战耗费的人力物力远甚于一次世界大战,因此经济遭到更严重的扭曲,人们认为,更为凄惨的空头市场和更为严重的萧条迫在眉睫。只要这种评价持续存在,大部分股票就卖得很便宜,等到金融圈终于看清楚,晓得这样的印象不对,没有严重的萧条等在眼前之后,美国历史上为期最久之一的股价涨势于焉展开。

一九七二到一九七四年的空头市场笼罩下,大部分股票的本益比和一九四六到一九四九年一样低,为本世纪两次仅见的纪录。可是很明显的,我们不禁要问,导致这种事情出现的金融圈评价正确吗?促使本益比掉到历史低点的恐惧心理,合乎道理吗?一九四六到一九四九年的历史会不会重演?本书稍后会试着解答这些问题。

影响所有股价整体水平的因素,和影响一支股票相对于另一支股票本益比的因素,两者间有基本上的差异。由于前面已经讨论过的原因,任何时点影响一支股票相对于另一支股票本益比的因素,只是投资圈对某家公司以及该公司所处行业的目前印象。但整体股价水平不只是印象问题,而有部分来自金融圈目前对普通股吸引力的评价,部分来自现实世界一些纯金融因素。

现实世界的因素主要和利率有关。长期或短期货币市场利率升高时,尤其是两者同时升高时,会有更多的投资资金流向这些市场,股票的需求因此降低。人们可能卖出股票,把资金转移到这些市场。相反的,利率偏低时,资金会流出这些市场,进入股市。所以说,利率升高通常会使所有的股票的价格下跌,利率下降则会使股价上扬。同样的,如果民众愿意把所得中更高比率的资金储蓄起来,则会有更多的资金成为投资资金,和投资资金上升缓慢比起来,股价后市以荣面居多。不过和利率水平比起来,这个因素的影响力量小得多。新股发行量的起伏变化,影响力量更小。新股发行会吸走原本可用于投资股票市场的部分资金。新股发行量对整体股价影响较小的原因,在于其他因素使得股票受人垂青时,新股发行量总会增加,以把握有利机会。普通股价格掉到低点时,新股发行量通常大减。这么一来,新股发行量的起伏变化,主要受其他因素影响,本身比较不像能够发挥影响力的因素。

股票投资的第四个要素,或可总结如下:任何个股在特定时点的价格,由当时金融圈对该公司、该公司所处行业,以及在某种程度内,股价水平的评价决定。要确定某支股票在特定时点的价格是否具有吸引力、不具吸引力,或者介于两者之间,主要得看金融圈的评价偏离事实真相的程度。不过由于整体股价水平也会在某种程度内影响整个画面,我们也必须准确估计若干纯金融因素即将出现的变化;这方面,仍以利率最重要。
\newpage

\part{发展投资哲学}

\textbf{献给法兰克·布洛克}
\\

\textit{本书最早是应特许财务分析师协会(Institute of Chartered Financial Analysts)的史都华·谢柏德奖(C.Stewart Shepard Award)之请而发表的。这个奖颁发给法兰克·布洛克(Frank E.Block C.F.A),表彰他的杰出贡献,热心努力,以富于启发作用的开导性精神,促进特许财务分析师学会成为一股重要的力量,培育财务分析师、建立高超的道德伦理标准,并且发展各种训练课程及出版品,鼓励财务分析师继续接受教育。}

\section{哲学的起源}

要了解任何严守戒律的投资方法,有必要先知道这种方法设计的目的。除了暂时以现金或相当于现金的形式持有,等候更合适机会的资金外,费雪公司(Fisher&Co.)管理的任何资金,目标都在于投资非常少数的公司。这些公司因为管理阶层的素质优异,营业额以及更重要的盈余成长率,应该都会远高于业界整体的水平。和成长率比起来,它们所承受的风险也相当小。要合乎费雪公司的标准,管理阶层必须有一套可行的政策,愿意牺牲短期的利润,追求更高的长远利益,以达成这些目标。此外,还需要两个特质。其一是他们有能力在企业经营的所有例行性任务上,每天都有出色的表现,用以执行长期的政策。另一是重大的错误发生时,能够认清这些错误,并采取矫正行动。管理阶层提出创新性的观念或推出新产品时,有时难免犯错。另一方面,经营成功也可能使管理阶层因骄矜自满而发生错误。

由于我相信自己对制造业公司的特性十分了解,所以费雪公司的主要活动,限于兼容并蓄,采用尖端科技和运用卓越的经营判断,以达成这些目标的制造业公司。近几年来,我限制费雪公司只投资这类公司,因为我偶尔投资别的领域,总是对所获结果不满意。不过,在零售、运输、金融等领域,具有必备长才的人,运用相同的原则之后,我看不出有任何理由,不能获得等量齐观的利润。

没有一种投资哲学能在一天或一年之内发展完全,除非抄袭别人的方法。就我的情形来说,它是在很长的时间内发展出来的,其中一部份可能来自所谓合乎逻辑的推理,部分来自观察别人的成败,但大部分来自比较痛苦的方法,也就是从自己的错误中学习。向别人解释我的投资方法时,最好的方法可能是回顾历史,细说从前。因此,我将回到早年我的方法慢慢成形的时候,试着一点一滴把这个投资哲学的发展历程交代清楚。
\\

\textbf{兴趣诞生}


很小的时候,我就晓得有股票市场存在,以及股价变动可能带来机会。家父在五个兄弟姊妹中排行最小,家母也是八位兄弟姊妹中的老么,所以我出生时,祖父母只剩下一位。这可能是我和祖母特别亲近的原因。一天下午,小学一下课,我便去看她。有位伯伯刚好也来,和她谈到他对未来一年工商业景气的看法,以及她的股票可能受到什么影响。一个全新的世界展开在我眼前。存了一点钱之后,我便有权在这个国家最重要的几百家企业当中,任选一家买它的股票,分享它未来的利润。如果选对,利润会叫人雀跃不已。我发觉,判断是什么因素促使企业成长,整件事很有意思,而这里正好有一种游戏,用适当的方式去玩,相形之下,我熟悉的其他事情都显得单调乏味、沉闷无聊、毫无意义。伯伯走后,祖母转向我说,我在那边时,碰巧他也来了,不得不和他谈一些我可能不感兴趣的事,实在很抱歉。我说,恰好相反,他们两人似乎只谈了十分钟,我却听到很有意思的事情。几年后,我才发现祖母持有的股票很少,而且那天的谈话内容十分肤浅,可是他们两人的谈话激起的兴趣,终我一生持续不坠。

由于有这么浓厚的兴趣,而且那时大部分企业不像今天那么担心和未成年人往来的法律危险性,所以我能在一九二〇年代中期狂飙的多头市场中,赚了一点钱。但是当医生的父亲很不赞成我做这件事,他觉得这事只会教我养成赌博的习惯。这是不可能的,因为我的本性不会光为了碰运气就去尝试;赌博本质上如此。另一方面,回顾前尘往事时,就投资政策来说,我发现那段期间小规模的股票投资活动,几乎没教到我什么很有价值的东西。
\\

\textbf{养成经验}


但在一九二〇年代的大多头市场轧然而止并崩跌之前,有一段经验,教了我非常重要的事情,可供来年使用。一九二七─二八学年,我被史丹福大学那时刚成立的商学研究所录取为一年级学生。那年的课程中,百分之廿是每个星期抽出一天,参观旧金山湾区一些最大的企业。主持这项活动的波利斯·艾梅特(Boris Emmett)教授,由于一般的学术背景,被赋予这个责任。那时候,大型邮购公司很多商品都是和供货商签约购得的,而这些供货商唯一的客户是邮购业者里面的一家公司。合约条件往往对制造商很不利,利润率很低,所以每过一段时间,就有一家制造商陷入严重的财务困境。眼睁睁看着供货商倒闭,不合邮购公司的利益。多年来,艾梅特教授因为是专家的缘故,受雇于一家邮购公司,负责在供货商被压榨得太厉害,摇摇欲坠时,担任救援任务。因此,他对企业的经营管理懂得很多。这个课程的举行,有个原则,就是我们绝不拜访只让我们看工厂的公司。「看过轮子转动」之后,管理阶层必须愿意和我们坐下来讨论,在教授非常犀利的问题下,我们可以获悉一家企业实际经营上的优点和弱点。我发觉这正是我想要的学习机会,而且能够利用个人特有的东西,掌握这个特别的机会。半个世纪前,汽车相对于人口的比率远低于今日,艾梅特教授没有车子,但是我有。于是我主动提议,搭载他前往各个工厂。前往工厂途中,我没学到什么东西。不过每个星期回史丹福时,可以聆听他对某家公司的真正看法。我享有这样一种特权,给了我十分宝贵的学习机会。

这些行程中,我也养成了一种明确的信念,后来证明很有价值,更奠立了我的事业基础。有个星期,我们拜访两家制造工厂,而不是只有一家。圣荷西(San Jose)这两家工厂恰好在隔壁。其中一家叫约翰毕恩喷洒泵浦公司(John Bean Spray Pump Company),是生产这种泵浦的全球领导厂商,用在果树上喷洒杀虫剂,驱逐天然害虫。另一家叫安德生─巴恩葛罗佛制造公司(Anderson-Barngrover Manufacturing Company),也是全球领导厂商,但生产水果罐头工厂使用的设备。一九二〇年代,金融圈还没有喊出「成长型公司」(growth company)的观念。我用诘屈聱牙的字句,向艾梅特教授说:「我觉得这两家公司有可能成长得远比目前的规模大,而且大到我们拜访过的公司,没有一家能比。」他同意我的看法。

同时,车内聊天时,有时谈到艾梅特教授以前的企业经验,从里面学到其他一些东西,对我的未来帮助很大。我们谈到,一家企业经营要健全,销售能力极其重要。一家公司可能是非常有效率的制造商,或者一位发明人发明的产品有令人叹为观止的用途,但对企业经营健全而言,这些事情绝对不够。除非企业内部有一些人才,能够说服别人相信他们的产品很有价值,否则绝对没办法真正控制自己的命运。后来我根据这个观念,进一步发扬光大,做成结论,认为即使强大的销售团队还不够。一家公司要成为真正有价值的投资对象,不只必须有能力销售产品,还要能够评估客户需求和欲望上的变化;换句话说,它必须娴熟所有的营销观念。
\\

\textbf{经验学校的第一堂课}


一九二八年夏天接近,我在商学研究所的第一年即将结束,有个大好机会到来,我觉得不容错过。今天这所商学研究所每年招生数百人,但当年我们那一班,是商学研究所的第三届,只有十九个学生。比我们早一年的毕业班只有九个人,其中两人主修财务。在股票市场发烧的那段期间,这两人都被纽约的投资信托业者网罗过去。最后一刻,旧金山一家独立银行(几年后被旧金山市的国安银行〔Crocker National Bank〕收购)向商学研究所招聘一位主修投资的研究生。商学研究所非常不希望错过这个机会,因为如果派出去的代表获得该行核可任用,以后可能有更多的机会,把毕业生安插到那家银行工作。但是他们没人可派。这事不容易,但我听到这个机会,最后终于说服研究所派我前去。我的想法是,如果应征成功,我会留在那里,要是不适任,我会回到学校,再念第二年的课程,同时让那家银行晓得,研究所并没有欺骗他们,派一位没有受过完整训练的学生去他们那里做事。

崩盘前的那段日子,证券分析师(security analysts)叫做统计员(statisticians)。连续三年股价跌得七荤八素,使得华尔街统计员的工作信用扫地,于是改名为证券分析师。

我发现自己成了那家银行投资银行单位的统计员。那时候,法律没有限制银行不能经营经纪或投资银行业务。我奉命做的事非常简单,而且依我的看法,那件事必须运用智慧做出不够诚实的事来。那家银行的投资单位主要销售新发行的高利率债券,做为承销集团的一员,可以收取相关可观的手续费。他们没有试着去评估所卖债券或股票的质量。相反的,在卖方市场的那段期间内,纽约同业或大型投资银行给他们承销任何证券,他们都乐于接受。接下来,银行的证券业务员会向客户说,他们有个统计部门,能够研究那些客户持有的证券,并针对他们经手的每一种证券,发给客户一份报告。事实上,所谓的「证券分析」,不过是查阅当时已有的手册,如《穆迪》(Moody's)或《标准统计》(Standard Statistics)里面特定公司的资料。再下来,像我这样的人,只要抄袭手册的遣词用字,写成报告即可。报告中,凡是营业额很高的公司,总是一成不变地把它们说成「经营管理良好」,原因只是它们的规模很大。没有人直接指示我,向客户建议把我「分析」的一些证券,转成本行当时希望出售的证券,但是整个气氛鼓励做这种分析。
\\

\textbf{建立基础}


没多久,整个作业过程的肤浅,令我觉得,一定还有更好的方式可以做这件事。我非常幸运,因为直属上司充分理解我关心的事,并给我时间,去做我向他提出的实验。一九二八年秋,收音机股票的投机风气甚盛。我向旧金山一些零售商的收音机部门消费者自我介绍,说是那家银行投资单位的代表。我请他们依自己的看法,说出这一行的三大业者。每个人给我的看法,雷同程度叫人惊讶。有个人是工程师,在其中一家公司工作,我从他身上获得很多东西。一家叫菲尔公司(Philco)的业者,从我的观点来说,很不幸的是私人持有的公司,因此在股市没有投资获利机会,却开发出具有特殊市场吸引力的产品。由于生产效率高,所以能在市场取得一席之地,并获有很可观的利润。RCA能够维持既有的市场占有率,但当时受股市垂青的另一家公司,市场占有率急剧下滑,而且迹象显示会陷入困境。这些公司和我们的银行没有直接关系,因为我们不经手收音机类股。不过写一份评估报告,似乎对我在那家银行里面的地位有很大的帮助,因为阅读这份报告的很多高阶主管,个人都有投机买卖这些股票。华尔街上的公司有谈到这些「热门」收音机类股,但我没办法从他们的资料,找到一言半语,讨论这些投机宠儿明显正在浮现的麻烦。

接下来十二个月,股市继续马不停蹄上涨,大部分股票攀升到新高点,我愈看愈有意思,因为我挑出来的那些问题股,却在涨势市场中一跌再跌。这是我所上的第一堂课,后来成为我的基本哲学的一部份:看一家公司的书面财务纪录,不足以分析那家公司是否值得投资。精明谨慎的投资人必须做一件很要紧的事,也就是从直接熟悉某家公司的人口中,了解那家公司的经营情形。

但早年的时候,在这种推理中,我还没有到达下一个合乎逻辑的步骤:我们也有必要尽可能了解经营公司的人员,方法是自己去认识那些人,或者透过第三者,你对他有信心,他又对他们知之甚详。

随着一九二九年展开,我愈来愈相信,似乎将涨个不停的股市狂飙荣面,本质上不健全。股价继续涨到更高的价位,依据的理由,是叫人惊异不置的理论:我们正置身于「新纪元」。因此,每股盈余年复一年上升是理所当然之事。可是在我试着评估美国基本产业的前景时,我见到许多产业出现供需问题,在我看来,它们的前景变得相当不稳定。

一九二九年八月,我对银行的高阶主管发出另一份特别报告,预测六个月内,廿五年来最严重的大空头市场将展开。要是那个时候,我能够急剧改变所发生的事,并留下我的预测完全正确的印象,不但很能满足自尊,也能从这样的智慧赚到大钱。事实恰好相反。

虽然我强烈觉得,一九二九年那些危险的日子里,整个股市实在太高,还是免不了被股市的魅力所惑。于是我到处寻找一些「还算便宜」的股票,以及值得投资的对象,因为它们「还没涨到」。由于几年来小规模交易股票赚了一点利润,加上我薪水中一大部分的储蓄,以及大学赚到的钱,一九二九年我凑到了几千美元。我把这笔钱大致等分,买三支股票。由于一时不察,我觉得在整体股市过高之际,它们的价格仍然低估。其中之一是一家首屈一指的火车头公司,本益比仍然相当低。铁路设备是受经济景气周期影响最大的行业之一,所以不必有太丰富的想象力,就知道在即将席卷我们的景气萧条中,该公司的营业额和盈余到底如何。另两家公司是地方性的广告广告牌公司和地方性的出租汽车公司,本益比也非常低。尽管我成功地看出收音机类股将发生何事,却没有想到向熟悉这两家地方性企业的人士,询问类似的问题,虽然取得这样的信息,或者去见经营这些企业的人士相当简单,因为他们就近在眼前。随着景气萧条日益严重,我终于十分清楚为什么这些公司的本益比那么低。到了一九三二年,我持有的这些股票市值,只及原始投资金额很低的百分率。
\\

\textbf{大空头市场}


我十分讨厌赔钱,这件事对我未来的财富而言,是很幸运的一件事。我一直相信,愚者和智者的主要差别,在于智者能从错误中学习,愚者则不会。由此可以想见,我应该十分小心谨慎地探讨自己所犯的错误,不要重蹈覆辙。

从一九二九年所犯的错误中学习之后,我的投资方法更上层楼。从这次经验中,我晓得,虽然一支股票的本益比偏低可能很有吸引力,但低本益比本身不能保证什么,反而可能是个警讯,指出一家公司有它的弱点存在。我开始了解,决定一支股票价格便宜或昂贵的真正要素,不是它的价格相对于当年盈余的比率,而是价格相对于未来数年盈余的比率。这一点,和华尔街的想法恰好相反。如果我能培养自己的能力,在合理的上下限内,确定几年内可能的盈余数字,就能找到一把钥匙,不但能避免亏损,更能赚到厚利!

大空头市场期间,个人惨不忍睹的投资表现,除了让我晓得一支股票价格便宜,带来的低本益比很可能只是投资陷阱外,更深刻体会到另一件可能更重要的事。这次多头市场泡沫破灭的时间,我预测得极其准确,在判断整体即将发生的事情方面,也几乎正确。不过除了在一小群人里面,自己的名声可能稍有提升,这事其实对我一点好处都没有。在那之后,我了解到,不管投资政策或一支股票适合买进或卖出,推理得如何正确,除非付诸实施,完成特定的交易,否则一点价值也没有。
\\

\textbf{自行其是的大好机会}


一九三〇年春,我换了老板。提这件事,只是为了说明此后发生的种种事情,形成一种投资政策,引导着我日后的行为。当时一家区域性经纪公司来找我,提出的薪水待遇,对廿二岁的我,以及在那个时间和地点,都很难抗拒。此外,和在前述那家银行的投资银行单位当「统计员」、难以令人满意的经验比起来,他们打算让我做的事,非常有吸引力。他们没有指派特定的职务给我。我可以自由运用时间,根据每支股票的特性,找出我认为特别适合买进或卖出的个股,接着把结论写成报告,传发给该公司的营业员参考,帮助他们推广可以让客户赚钱的业务。

这件工作找上门来,刚好在胡佛总统发表有名的「荣景就在眼前」(Prosperity is just around the corner)的声明之后不久。该公司几位合伙人私底下信之不疑。由于一九二九年的崩盘,他们公司的员工总数从一百廿五人减为七十五人。他们告诉我,如果我加入,就是第七十六个人。那时我看空的程度,不亚于他们看涨的程度。我相当肯定空头市场还有一段漫漫长路。我告诉他们,要过去可以,不过有个条件。如果他们不满意我的工作质量,随时可以炒我鱿鱼,但如果金融市场情势恶劣,他们不得不再次裁员时,绝不能以我的资历浅就要我优先走路。他们同意这个条件。
\\

\textbf{祸兮福所倚}


那样的老板,再也找不到更好的了。接下来八个月,我有了一段毕生最珍贵的企业教育经验之一。我接触到第一手资料,亲眼看到一个又一个例子,晓得投资业不应该怎么经营。随着一九三〇年展开,以及股票又持续似无止尽的跌势,我的雇主处境岌岌可危。就在一九三〇年圣诞节前,从经济大屠杀中幸存下来的我们,眼睁睁看着一幅悲惨的画面:整家公司因为资金周转失灵,被旧金山证券交易所暂停交易。

同事眼中的坏消息,后来却证明是我一生很幸运的事业发展点,如果不能说是最幸运的话。一段时间以来,我一直有个模糊的计划,希望在荣景再临时,创立自己的事业,向客户收取费用,管理他们的投资事务。我故意拐弯抹角描述投资顾问(investment counselor或investment advisor)的活动,因为那时还没人使用这个词汇。但一九三一年一月惨淡的岁月中,金融业几乎每个人都在节衣缩食,我能找到的唯一一件证券业工作,是纯当文书作业员,对我来说,很没意思。审情度势之后,我体会到这正是开创心中所想新事业的正确时机。理由有二。其一是经过约两年美国前所未见的严重空头市场之后,几乎每个人都对既有的经纪商关系不满意,愿意聆听我这种年轻人的看法,主张采用极端不同的方法,处理他们的投资事务。而且,一九三二年正当经济跌到谷底,很多重要的企业家没什么事情好做,所以有时间接见一些人。正常时候,我绝对过不了秘书那一关。我的整个事业生涯中,有位很重要的客户,家里的投资事务,今天我还在帮他们处理,便是个典型的例子。几年后,他告诉我,我过去拜访那一天,他刚看完报纸的运动版,几乎无事可做。因此当秘书把我的名字和目的告诉他之后,他心里想着:「听听这位小伙子怎么说,至少可以打发一点时间。」他坦承:「如果约一年后你才来见我,绝对进不了我的办公室。」
\\

\textbf{奠定基础}


就这样,好几年内,我在一间很小的办公室内非常努力地工作,经常性开销压得很低。办公室没有窗户,只有玻璃隔间当做两面墙,整个楼板空间刚好够挤进一张桌子、我的椅子和另一张椅子。这些加上房东先生的秘书员兼接待员提供的免费当地电话服务,以及数量合理的秘书事务工作,每个月支出金额高达廿五美元。其他支出包括文具用品、邮资,还有很久才打一次的长途电话。从目前仍在我手中的账本,可以看出一九三二年开创新事业多么辛苦。极长时间的工作之后,扣除这些经常性开销,那一年每个月的净利平均二.九九美元。一九三三年仍然很艰苦,但我做得稍好一点,盈余增加将近一千%,平均每个月略高于二十九美元。或许当个报童,沿街叫卖报纸,我也可以赚到这么多钱。可是因为有这些年头,我才有后来的日子,所以它们可说是我一生中获利最丰富的两年。它们让我奠下基础,到了一九三五年,便建立起获利极其可观的事业,以及一批非常忠诚的客户。如果能够这么说,那一定很棒:由于我自己聪明的头脑,才会想到创立事业,而不是等到好时光来临。其实,那是因为我能找到的唯一一件工作索然无味,才把我推进自创事业的路上。

\section{从经验中学习}

我在银行做事的时候,带着很浓厚的兴趣注意到一则新闻报导,提及圣荷西相邻的两家公司最新的动态;这两家公司,我在史丹福商学院学生时代,就觉得很有意思。一九二八年,约翰毕恩喷洒泵浦公司(John Bean Spray Pump Company)、安德生─巴恩葛罗佛制造公司(Anderson-Barngrover Manufacturing Company)和伊利诺伊州胡伯斯顿(Hoopeston)的领导性蔬菜罐头制造公司史普雷格谢尔斯公司(Sprague Sells Corporation)合并,组成一家叫做食品机械公司(Food Machinery Corporatuon)的新公司。

和其他投机风气甚炽的时期一样,美国正处于买进股票的狂热当中,食品机械公司的上市股票价格需求殷切,因此提高价格以为因应。同一年,也就是一九二八年,旧金山证券交易所会员公司销售新股给需股孔急的弯区买者,至少还有其他廿种,或者可能多达此数的两倍。有些新股体质有欠健全,到达骇人听闻的地步。证券交易所一家会员公司的高阶主管告诉我,有家公司卖的瓶装水来自太平洋对岸,股票交由该公司承销,可是他们并没有拿到完整的一套财务报表,只有一张照片,里面是生产瓶装水取用的一处温泉景像,而且他们和出售持股的股东没见过几次面!在大众心目中,食品机械公司的股票只是那一年另一支叫人兴奋的新上市股,没比其他新股好很多,也没差很多。上市价格是二一.五美元。

那时候,股友社集资炒作股票完全合法。有个地方性团体,经营股友社几无经验,却由一个人带头。他对食品机械公司很感兴趣,决定「经营业务」,买卖该公司的股票。所有这些股友社使用的方法本质上雷同。会员本身不断来回相互销售股票,价格愈卖愈高。股票交易信息纸带上这些活动会引起别人注意,于是开始买进,股票从股友社流出,价格更高。有些技巧十分高超的炒手,赚了几百万美元,其中有个人过了约一年,提议让我当资浅合伙人。这个人经验相当丰富,擅长于运用这种很有问题的艺术。但是买卖食品机械公司的股友社,目标不在炒作价格。一九二九年秋即将来到时,股价濒临崩跌边缘,该股友社终于买到食品机械公司公开上市的大部分股票。虽然食品机械公司股票的最高报价达五十美元以上,流入大众手中的股数却很少。

接下来几年,每年的整体景气状况都比前一年恶劣,一九二八年新股狂热期间公开上市、体质欠佳的小型公司股票,发生了什么事,不问可知。这些公司一家接一家宣告破产,其余很多公司不但没有盈余,还发生亏损。这些公司的股票市场枯竭殆尽。

公开上市新股中,除了食品机械公司,还有一两家本质上相当健全,也很有吸引力。但是投资大众不分青红皂白,认为所有这类股票都是投机性垃圾。一九三二年股市终于跌抵谷底,以及一九三三年三月四日罗斯福(Franklin D.Roosevelt)总统就职那天,全国银行体系关闭,股市再跌到相同的谷底时,食品机械公司的股价跌到四美元到五美元间,之前的历史性最低价是每一百股三.七五美元。
\\

\textbf{食品机械公司是投资良机}


随着一九三一年展开,我到处为我的新创事业寻找投资机会,看到食品机械公司的情况,愈看愈对眼。我晓得几年前,由于我没花工夫去见两家地方性公司的管理阶层,并评估他们的素质,结果投资损失惨重,所以我决定不再犯同样的错误。我对食品机械公司的人认识得愈清楚,对他们愈加敬重。多年来,大萧条谷底期间,这家公司许多方面的表现,正是我寻找的未来机会的缩影。花点篇幅,说明为什么近半个世纪以前,我在这家公司身上看到这样的未来,可能有帮助。

顺便一提,很遗憾的,在这之后几年内,我并没有把我的深度现场分析政策订为合乎逻辑的结论。对于较远地方的公司,我做得不够勤快,去认识和评估它们的管理阶层。

首先,虽然食品机械公司的规模相当小,但它从事的三种业务,每一种业务的产品线论规模和质量,我相信都是全球首屈一指。这家公司因此拥有规模优势。也就是说,由于这家公司的规模大且有效率,有可能也是低成本制造商。

其次,从竞争的观点来说,它的营销地位极强。它的产品获得客户很高的评价。它有自己的销售组织。此外,它的装罐机械已有很多公司安装使用,在某种程度内「锁住」了市场。这个市场包括设备的备用和替换零组件。

在这个坚稳的基础上,又有一个最叫人振奋的单位。对于像它那种规模的公司来说,它拥有创意十足的工程设计或研究部门。这家公司在前景看好的新产品设备上力求精进。其中包括业内第一部机械式梨子去皮机、第一部机械式桃子去核机,还有一种合成制程,为柳橙上色。有些地方生产的柳橙可口多汁,但和其他质量没有比较好的柳橙比起来,因为外观引不起家庭主妇喜爱,竞争上屈居下风。我的事业生涯中,前景极佳的新产品相对于现有产品,依我的判断可望带来的营业额一样多者,除了一九三二到一九三四年的食品机械公司,只看过另一家公司有同样的情形。

这时候,我已学得够多,晓得这些事情本身不管多吸引人,都不足以确保经营十分成功。这家公司的人员素质一样要紧。我使用的素质(quality)一词,涵盖两种相当不同的特质。其一是企业经营能力。企业经营能力又可细分成两种很不相同的技能。一种是以高于一般水平的效率,处理日常事务。所谓日常事务,包括很多事情,从不断寻找更好的方法以提高生产效率,到密切注意应收帐款催收的情形。换句话说,他们必须有高于一般水平的经营技能,处理公司近期内营运活动有关的很多事情。

但在商业世界,一流的管理能力也需要另一种相当不同的技能。那就是前赡未来和拟定长期计划的能力,促使公司将来能够大幅成长,同时避免可能招致灾难的财务风险。许多公司的管理阶层很擅长于其中一种技能,但要成功,两者必须兼备。

我相信,真正值得投资的公司,缺之不可的两种「人员」特质中,企业经营能力只是其中一种。另一种特质是诚信正直(integrity),也就是公司经营者必须诚实和拥有高尚的人格。凡是一九二九年股市崩盘前首次踏进投资世界的人,都会看到不少鲜活的例子,并了解诚信正直极其重要。一家公司的所有人和经理人,总是比股东更接近公司的事务。如果经理人不觉得对股东负有受托责任,股东迟早没办法从中获得应得的利益。只顾追求私利的经理人,不可能在身边培养起一群热情洋溢、忠诚奉献的得力左右手──一家公司要成长到无法再靠一两个人控制的地步,绝对必须这么做。

不管是在景气严重萧条的那些黑暗日子里观察,还是隔了这么多年再来回顾,从「人」的观点而言,股票刚上市的食品机械公司都不同凡响。约翰毕恩制造公司的总裁及原创办人的女婿约翰·科拉米(John D.Crummey),不只是效率极高的营运首长,很受客户和员工敬重,而且宗教信仰虔诚,一丝不苟地遵守高道德标准。公司的工程长是才华洋溢的概念设计师。他潜心研究,使得产品获得专利保护。这件事也很重要。最后,为了补强这家相当小的组织的优势,约翰·科拉米说服女婿保罗·戴维斯(Paul L.Davies)加入,强化财务能力和保守作风。保罗·戴维斯本来不想放弃前途看好的银行业事业生涯。其实,保罗·戴维斯起初同意的很勉强,只答应向银行请一年的长假,在家族事业合并后艰苦的第一年帮点忙。那一年内,他对未来一片美景大感振奋,决定永远留在公司。后来,他当上总裁,带领公司壮大规模和欣欣向荣,程度远超过接下来几年令人愉快的成就。

那时这家公司本质上便有这些理想的特质。值得投资的公司,偶尔才能看到这些特质。食品机械公司的人员素质十分杰出。公司规模虽小,却不是只有一人做出重大的贡献。和竞争对手相较,这家公司显得很强,企业经营得很好,而且即将推出的新产品线够多,相对于公司当时的规模,潜力很大。即使一些新产品最后无法推出,其他新产品的未来还是非常明亮。
\\

\textbf{别人左转你右转}


但是除了所有这些事情,投资一家公司的股票要获得可观的利润,还得加上同等重要的一些东西。金融圈都向左转时,有能力正确地转向右边的人,才能在投资领域获得最大的利润。如果那时候食品机械公司的未来能得到适当的评价,则在一九三二到一九三四年间买进该公司股票的人,获得的利润会减少很多。正因为这家公司的真正价值没有普遍为人体认,而且人们认为食品机械公司只是许多「脆皮」公司里面的另一家,趁股票投机热到极点时卖给大众,等到后来价格惨跌,却能大量买进这些股票。依我的看法,训练自己不要盲从群众,而要能够在群众左转时右转,是投资成功最重要的基本工夫之一。

当我看到还没受人垂青的食品机械公司,可能对我个人小小的财富,以及我想要创立的新事业有多大的帮助时,内心、情绪和理性上的激情,但愿能透过文字适当地表达。我似乎抓对了时机。就像弹簧被压得太紧,开始弹开,一九三三到一九三七年整个股市先是缓慢上升,继之爆发成全面开展的多头市场,但一九三八年跌得不轻,隔一年又全面回升。由于我深信食品机械公司的表现会远超过整体股市,所以劝服我的客户尽量买进并抱牢。和任何潜在客户一谈时,我总是拿这个可能性,开门见山谈我使用的方法。我觉得一生几乎难得一见的独特机会就在眼前,正如莎士比亚说得很好的:「凡人经历狂风巨浪才有财富。」那段激昂的年头里,我的期望很高,但我的资金和名声,在金融圈几乎不存在,于是我一再引述这些振奋人心的字句,以坚定自己的决心。
\\

\textbf{反其道而行,但走对路}


反向意见(contrary opinion)的重要性,其他投资著作着墨已多。但是光有反向意见还不够。我看过不少投资人士鼓吹背离一般思想潮流,抱持反向意见,却完全忽视了个中的涵义是:背离一般投资思想潮流时,你必须非常肯定自己是对的。比方说,当情势变得很明显,汽车将取代电车,而且曾经受宠的都会铁路股票本益比愈来愈低时,如果以每个人都相信它们正在走下坡,所以一定极具吸引力为由,反其道而行,买进电车公司的证券,一定损失惨重。金融圈大部分人都左转时,适时右转的人,如有强烈的迹象,显示自己转对方向,往往能获得庞大的利润。

如果在这件事情上面,莎士比亚的名言对我所采政策的形成,有很大影响力的话,说起来奇怪,一次世界大战期间一首流行歌曲也很重要。一九一八年扰攘不安的时期,仍记得后方反应的人愈来愈少,或许我是其中一人。当时美国民众对战争的激情和热情十分天真浪漫,和二次世界大战期间阴郁的态度相当不同,因为人们比较能够清楚地了解战争的可怕。一九一八年,前线战场伤亡、恶行、恐怖的一手消息,还没有渗透到美国大陆。因此,那时的流行歌曲充斥轻快和幽默的战争主题,二次世界大战期间,这种现象则很少,越战期间更是完全绝迹。大部分这些歌曲都来自钢琴的单张乐谱。一首歌曲上面画着骄傲的母亲俯视行军中的阿兵哥,曲名是「除了吉姆,大家的步伐都乱了」。

我从一开始就晓得,我的确冒险「和别人的步伐不一样」。我很早就买进食品机械之后,后来又买了其他很多「不合时宜」的公司,因为它们的实质价值没有被金融圈完全认清。我自己的想法很可能完全错误,金融圈可能是对的。果真如此,则对我的客户和我自己来说,金融圈左转,而我右转转错了;由于我对某种状况的坚定信心,结果庞大的资金无限期套牢,无利可赚,再没有比这件事更糟的了。

但是我知道得很清楚,要从前面所说的别人左转而你右转的做法中,获得可观的利润,有件事很重要,也就是我一定要有某种计量检测方法,确定我右转转对。
\\

\textbf{耐心和绩效}


心里记得这点,所以订下了我自称的三年守则。我一再向客户表示,我为他们买进某样东西时,不要以一个月或一年为期评估成果,必须容许我有三年的时间。这段期间内,我如果没有为他们带来可观的成果,他们应该炒我鱿鱼。第一年不管是赚是赔,恐怕运气的成分居多。这么多年来,我管理个别股票,都遵守相同的原则,只有一次例外。如果我深信某支股票到了三年结束的时候,不会有好成绩,我会提前卖掉。要是过了一两年,这支股票的表现一直没有比市场好,而是比市场糟,我不会喜欢它。但如果没什么事情改变我对那家公司的原始看法,我会继续持有三年。

一九五五年下半年,我大量买进两家公司的股票;以前从没买过它们的股票。和当时金融圈普遍接受的看法背道而驰,反向投资,这正是典型的例子,可以看出其中的优势和问题。事后来看,一九五五年或可视为前后约十五年的「电子类股第一个黄金时代」的开端。我使用「第一个」一词,以免和其他人心目中后来的半导体类股黄金时代混淆。我猜后者就在眼前,一九八〇年代将降临。总之,一九五五年和之后不久,金融圈即将被一堆电子公司弄得目眩神迷。这些股票的价格扶摇直上,到了一九六九年,涨幅十分可观。国际商业机器公司(IBM)、德州仪器(Texas Instrument)、维利安(Varian)、利顿实业(Litton Industries)、安培斯(Ampex),只是少数例子。但是一九五五年,这些事情都还没发生。那时候,除了IBM,所有这些股票都被视为投机色彩非常浓厚,保守型投资人或大型机构不屑一顾。但是察觉到未来将发生的一部分事情,我在一九五五年下半年买进德州仪器和摩托罗拉(Motorola)的股票,而且对我来说,买进的数量相当庞大。

今天,德州仪器是最大的全球性半导体制造商,摩托罗拉紧跟在后,排名第二。那时候,摩托罗拉在半导体内的地位微不足道。根本没什么因素促使我去买这些股票。我只是对摩托罗拉公司的人,以及这家公司在行动通讯事业上居于主宰地位,留下深刻的印象;通讯事业看起来潜力雄厚。可是金融圈视它为另一家电视机和收音机制造商。摩托罗拉后来崛起于半导体领域,部分原因是取得丹尼尔·诺伯(Daniel Noble)博士的服务,给我带来锦上添花的效果,当初买进股票时始料未及。至于德州仪器,除了同样喜欢和尊敬那家公司的人,我受到另一组相当不同的信念影响。我和其他人一样,看到以人类的聪明才智,半导体的前途无可限量,德州仪器的晶体管事业可能有异常明丽的未来。虽然华尔街大部分人的看法不一样,我却觉得该公司的管理阶层面对奇异(General Eleectric)、RCA、西屋(Westtinghouse)和其他巨型公司的竞争,至少能打成平手,甚至可能更占上风。很多人批评我把资金拿去冒险,投入一家小型的「投机公司」。他们认为,面对巨型公司的竞争,德州仪器势将身受其害。

买进这两支股票之后,短期内的表现殊不相同。一年内,德州仪器增值相当可观,摩托罗拉则比我的买进成本低五%到十%,表现很差,一位大客户对它的市场走势非常恼火,不愿直呼其名,只称它是「你帮我买的那只火鸡」。令人不悦的价位持续了一年多。等到金融圈慢慢了解摩托罗拉通讯事业部门的投资重要性之后,在加上首次有迹象显示半导体即将当红,这支股票开始有不俗的表现。

买进摩托罗拉的时候,我是和一家大型保险公司连手。这家保险公司让摩托罗拉的管理阶层晓得,他们也对我首次拜访获得的结论感兴趣。这家保险公司也大量买进摩托罗拉的股票之后,不久便把整个投资组合送到纽约一家银行做评估。除了摩托罗拉,该行把他们的投资组合分成三类:最具魅力、魅力较差、魅力最差。但他们不肯把摩托罗拉放到任一类中,理由是这种公司不值得他们花时间伤脑筋。因此,他们对摩托罗拉不予置评。可是那家保险公司一位高阶主管三年后告诉我,虽然华尔街的评价那么不好,在那之前,摩托罗拉在他们的投资组合中,表现比其他每一支股票要好!如果我没有那套「三年守则」,则在市场走势不佳及一些客户的批评下,恐怕也没有那么坚强的信心,抱牢自己的摩托罗拉股票不放。
\\

\textbf{每个原则都有例外的时候……但不多}


我是不是曾经根据三年守则卖出股票,后来因为那支股票大涨而悔不当初,但愿没把股票卖掉?其实,我只根据三年守则而卖出持股的次数不多。这倒不是因为我买进的股票很少出现当初买进时所期待的高涨幅。绝大部分时候,由于我继续研究调查整个情况的其他层面,而有了进一步的了解,使得我改变对某支股票的看法。但是在相当少数只根据三年守则而卖出持股的例子中,我记不起来曾有一次因为后来的股市走势,而悔不当初,但愿没有卖出股票。

我是否曾经违反我自己的三年守则?答案是有,但只有一次,而且是在很多年后才这么做,时间接近一九七〇年代中期。三年前,我买了数量不多也不少的罗杰斯公司(Rogers Corporation)股票。罗杰斯专长于某些聚合体化学产品,我相信他们即将开发出各种半特有的系列产品,营业额将增加不少,但不是只昙花一现一两年便消失,而是会持续很多年。但三年结束的时候,这支股票的价格下跌,盈余也减少。可是当时有几股力量发挥影响力,我觉得这一次应该忽视自己的标准,使它成为「守则的例外」。其中一股力量是我对该公司总裁诺曼·葛林曼(Norman Greenman)的强烈感觉。我深信他具有非凡的能力,有决心度过难关,而且还具备另一种东西,我觉得对聪明的投资人来说,很有价值:他非常诚实,绝不隐瞒一再出现,不会使公司倒闭,但会叫他尴尬的坏消息。对他的公司有兴趣的人,他总是设法让他们除了晓得有利的潜力,也知道正在发生的所有不利事情。

还有另一个因素影响我很大:该公司盈余那么差的主要原因,是罗杰斯花了不成比例的资金在单一新产品的开发上。这种产品似乎有非常明亮的前景。结果公司的资金和人力都从其他也很有潜力的新产品抽调出来,投注在它们上面的资金和人力较少。该公司最后终于做出痛苦的决定,放弃对那项产品的所有努力,之后不久,情况便变得相当明朗,其他几样前途很好的创新开始开花结果。但是这得花点时间。在此同时,该公司未能达成许多持股人的期望,导致股价大跌。和营业额、资产或任何正常的获利能力比起来,股价跌到很荒谬的水平。这似乎是金融圈左转,我该右转的典型例子。因此,我不管有没有三年守则,大幅提高了我自己和客户的持股,但有些客户不耐多年来的苦候以及对该公司的业绩不满,在某种程度内,对我的做法很感忧虑。像这样的例子常常出现的情形是,一旦情势逆转,会变化得很快。随着整个情况明朗化,大家知道盈余改善不只是一两年的事情,相反的,强烈的迹象显示这将是未来好几年高成长的基础之后,股价继续扬升。
\\

\textbf{试验性抓取市场进出时机}


但以上的事情,把我的故事讲得快了一点,因为在一九三〇年代,我还得从错误中尝试摸索,学习成长,投资哲学才慢慢成形。在我四处寻找各种方法,希望从普通股赚到钱之际,我开始看到,研究食品机械公司,可能带来一样很有价值的副产品。这家公司很多业务依赖水果蔬菜罐头业,为了确定我买进食品机械公司的股票买得对,我无意中了解很多影响水果蔬菜罐头公司兴衰的因素。这个行业的周期性很强,因为整体景气状况变动不居,而且异常的天候也会影响特定的农作物。

总之,随着我慢慢熟悉罐头业的特性,不免想到为什么不好好利用这方面的知识?但我不想象食品机械公司那样做长线投资,只想短线进出加州罐头制造公司(California Packing Corporation)的股票。加州罐头制造公司那时是家独立公司,也是规模最大的水果蔬菜罐头生产商。从大萧条的谷底到那个年代结束,我曾经三次买进这家公司的股票。每次卖出时都获有利润。

表面上看起来,我好像做得蛮不错的。但是几年后,我试着分析我在事业上做过的聪明和愚蠢举动之后,这些行为愈看愈蠢,原因稍后解释。它们花了我很多时间和精力,而这些时间和精力大可用在别的事情上面。全部赚到的钱相较于所冒的风险,和我买进食品机械公司为客户赚到的钱,以及其他场合中,采取长期投资的做法,抱牢很长的时间获得的利润比起来,实在是小巫见大巫。此外,我看了很多短线进出操作,包括一些非常聪明的人在内,晓得连续成功三次之后,只会使得第四次发生灾难的可能性大为提高。和我看到一些前景不错的公司,买进同等数量的股票,并抱牢很多年比起来,短线进出的风险高出很多。因此,二次世界大战结束后,我目前的投资哲学已经大致成形,于是我做了事业生涯中相当珍贵的一个决定:把所有精力用到长期投资赚取厚利。
\\

\textbf{锱铢必较,机会飞掉}


一九三〇年代我学到,或者至少部分学到另一件很重要的事情。前面提过,我曾准确预测到一九二九年开始的大空头市场,但一点好处也没得到。这个世界上,所有正确的推理,如果没有付诸行动,则投资股票毫无利益可言。我第一次经营事业时,恰好碰到大萧条的谷底,即使一点点的钱,也非常重要。可能因为这段经验,或者因为我的个性,创立事业后,我发现自己每每为了「八分之一美元或四分之一美元」计较不已。懂得远比我多的营业员一再告诉我,如果我相信一支股票几年内会上涨成目前价格的好几倍,那么我是用十美元或一○.二五美元买进,根本无关紧要。可是我还是继续下限价单,纯粹根据个人随便做出的决定,不管其他理由,绝对不肯以一○.一二五美元以上的价格买进。这当然是很荒谬的行为。我观察到,除了我之外,还有很多人的这个投资坏习惯根深蒂固,可是有些人不会这么做。

随便限定价格的潜在危险,因为另一个人犯下的错误,终于叫我看清楚。记忆中宛如昨天的事情:当时我恰好走过旧金山一家银行前面的人行道,于是顺道去拜访一位相当重要的客户。我告诉他,我刚拜访过食品机械公司。前景实在好得不得了,我觉得他应该再买一些股票。他十分同意我的看法,并询问那天下午的收盘价。我说是三四.五美元。他给了我一张数量很大的委托单,说要以三三.七五美元买进,绝对不能高于此数。接下来一两天,股价在他的买价之上不远处波动起伏,后来一直没跌下来。我打过两次电话给他,促请他提高买价○.五美元或一美元,这样我才能买到股票。很遗憾地,他答道:「不,我要买的价格就是那样。」几个星期内,股价涨了五十%以上,而且股票分割之后,该公司的股价一直没有掉到接近他可能买到的价位。

这位绅士的行为让我留下深刻的印象,我自己同样愚蠢的行为却没有这种效果。我慢慢克服自己的这个缺点。我十分清楚,一个人如果想买很多股票,绝不能完全忽视八分之一美元或四分之一美元的差异,因为只要买进一点,其余买到的股票价格就会上涨很多。但对绝大部分的交易来说,坚持一点小小的价格差异不肯放弃,可能得不偿失。以我自己的例子来说,买进股票时已完全克服这个坏习惯,但卖出股票时只能部分克服。去年我不肯以市价卖出,而限价下了一张小卖单,结果因为差四分之一美元,无法成交。写这段文字时,股价已比我下单时低三十五%。在我的限价和目前的价格之间一半的地方,我只卖出部分持股。

\section{哲学成熟}

就我的投资哲学成形的历程来说,二次世界大战不能说完全没有作用。早在一九四二年,我便发现自己扮演一个很不习惯的角色,在陆军航空兵团当地勤官,处理各种商业相关工作。前后三年半内,我替山姆大叔做那不是很有价值的服务,只好把自己的事业「拉到岸边」。最近几年,我常说我为国家做得很不错。不管是希特勒,还是日本天皇裕仁,都没成功地派遣一人攻进我的防地。我的防地在阿肯色州、德州、堪萨斯州和内布拉斯加州!总之,穿着山姆大叔的制服,做各种事务性工作的这段期间,我发现,几乎在没有预警的情况下,我得轮流经历两种截然不同的处境。有时,该做的事很多,根本无暇想起我的承平时期事业。有时则闲得发慌。事情没那么忙的时候,我发现,详细规划快活的日子来临,我不用再穿制服时,要如何壮大我的事业,和从短期的观点思考个人的生活及不得不面对的军队问题比起来,不会那么无趣。这段时间内,我目前的投资哲学慢慢变得更为具体。这个时候,我终于想清楚,前一章所说的加州罐头制造公司股票的短线进出操作做法,不会有太大的前途。

同时,我做成另外两个结论,对我未来的事业相当重要。战前我服务各式各样的客户,不管大小,目标各异。我的业务大部分集中在寻找与众不同的公司,未来几年将享有高于一般水平的大幅成长。战后,我要限制自己经营一小群大户,目标只集中在高成长的投资对象。从税负的角度来说,高成长股对这些客户比较有利。

另一个重要的结论是:战后化学工业会有一段高成长期。因此,复员后我的优先要务,是从大型化学公司里面找出最具吸引力的公司,而且我管理的资金,主要的持股将是化学公司。我当然没有把全部的时间投入做这件事,但事业重新开张后的第一年内,确实花了很多时间,凡是对这个复杂行业非常了解的人,能找到的都找了他们一谈。这些人有:经销一家或多家大公司产品线的经销商;大学化学系的教授,他们和化学业内人士很熟;甚至找到一些大建设公司,因为他们替各式各样的化学制造商兴建厂房。这些人后来都证明是极有价值的背景信息来源。有了这些信息,再加上分析一般的财务数据,只花了约三个月的时间,便把选择范围缩小到三家公司里面的一家。此后,步伐放慢,决策做起来较为困难。但是到了一九四七年春,我确定我要的是道氏化学公司(Dow Chemical Company)。
\\

\textbf{合众为一}


从许多前景不错的化学公司里面,选上道氏化学,理由有许多。我相信一一列举有好处,因为它们可以清楚地说明我在少数公司里面,到底找些什么样的东西,才肯把资金投入。开始认识道氏公司的人员之后,我发现公司已有的成长,又反过来在很多管理阶层创造出一股激昂的热情。整个组织弥漫着将来还会成长得更快的信念。和任何企业主管第一次见面时,我最爱问的问题是,他认为公司面对的最重要长期问题是什么。向道氏公司的总裁问这个问题时,他的答复令我印象极为深刻:「那就是在我们成长得很大时,必须抗拒强大的压力,不要成为军队一样的组织,而且维持非正式的关系,让不同阶层和不同部门的人员,继续以完全非结构性的方式彼此沟通,同时不致造成行政管理上的混乱。」

我完全同意该公司其他一些基本政策。道氏限制自己生产的化学产品线,必须有合理的机会,在这个领域成为最有效率的制造商,原因是数量较多、化学工程设计较好、对产品有更深入的了解,或者其他原因。道氏深深了解,富有创意的研究,不只是跑在前头,也是维持在前头所必需。道氏也强烈体认到「人的因素」之重要。特别是,公司有一种感觉,觉得有必要尽早找出才能不俗的人才,把道氏特有的政策和作业程序灌输给他们。如果表面上看起来才华洋溢的人才,在本职上做得不好,则尽一切努力,给他们合理的机会,尝试去做可能比较适合他们个性的其他事情。

我发现,虽然道氏的创办人赫伯特·道(Herbert Dow)博士已于约十七年前去世,他的信念仍深受员工敬重,经常有人向我引述他讲过的一两句话。尽管他说的话主要是针对道氏公司的内部事务,我却认为至少有两句话同样适用于我的事业;除了适用于道氏化学公司的内部事务,也适用于找到最理想的投资对象。其一是「绝不要擢升没犯过一些错误的人,因为这么一来,等于擢升从没做过什么事情的人」。投资圈内很多人未能见到这一点,结果一再在股票市场创造出绝佳的投资机会。

想在商业世界拥有一番不俗的成就,几乎总是需要某种程度的开路先锋精神,创新和实用兼容并蓄。利用尖端科技研究以取得领先优势时,尤需如此。不管一个人多能干,也不管他们大部分的构想多美好,难免有些时候,努力终归失败,而且失败得很惨。一旦这样的事情发生,把失败的成本加上去,当年的盈余可能远低于原先的估计值。投资圈对管理阶层素质的评价通常应声调低。这一来,除了当年的盈余下降,本益比更降到历年的水平以下,使得盈余降低的影响更为扩大。股价往往跌到十分便宜的地步。可是如果同一批管理阶层以前曾有非凡的表现,则平均成功相对于平均失败的比率,将来可望维持相同的水平。由于这个原因,能力极强的人经营的公司,特别严重的错误曝光之后,股价可能跌得很低。相对的,不做开路先锋,不肯冒险犯难,只随群众起舞的公司,在这个高度竞争的时代中,表现将乏善可陈。

道氏另一句话,我也试着用到投资选择过程中。他说:「如果某件事情你没办法做得比别人好,那就不要去做。」今天,政府强力干预很多企业活动、税率高、工会活跃,而且大众对产品的喜好瞬息万变。我觉得,除非只找竞争精神高昂,不断尝试且往往能以优于业界一般水平的方式把事情做好的公司,否则不值得冒险持有普通股。不这么做,利润率很难高到足以应付公司的成长需要。当然了,通货膨胀率偏高的时期尤应如此,因为高通货膨胀率会使公司发表的盈余数字缩水。
\\

\textbf{历史与机会}


我在经济大萧条谷底创立事业,以及服役三年半后,一九四七年到一九五〇年代初事业重新开张,两段期间有若干雷同之处。两者都碰到悲观情绪浓厚得化不开,很难为客户立即创造利润。但是这两次,凡是有耐性的人,获得的报酬都相当可观。前面一次,股票价格相对于实质价值,跌到可能是廿世纪仅见的最低水平,原因不只出在大萧条蹂躏整体经济,也因为价格反映了许多投资人忧虑美国人私人企业能否存续。这套制度不但存活了下来,而且接下来几年,能够且愿意投资正确股票的人,获得极优渥的报酬。

二次世界大战结束后几年,另一种忧虑导致股票相对于实值价值,与大萧条谷底时期所见几乎一样低。这一次,企业业绩很好,盈余稳定攀升。但是几乎整个投资圈深陷在一种简单的比较中,无法自拔。内战后几年,短期的荣景昙花一现,接着出现一八七三年的恐慌,以及约六年的严重萧条。一次世界大战后,荣面维持类似的长度,接着是一九二九年的股市崩盘,以及更为严重的萧条,为期与上一次也相近。二次世界大战期间,每天的战争成本约为一次世界大战的十倍。当时居于主流地位的投资观点说:「因此,目前出色的盈余没有任何意义。」不久他们会面临十分可怕的崩盘,以及一段极为艰困的时期,每个人都将身受其害。

年复一年,更多公司的每股盈余上升。一九四九年左右,这段期间被称作「美国企业生不如死」的年代,因为一有传言散布,说某家股票公开上市公司即将结束营业,股价一定大涨。许多公司的清算价值远高于当时的市值。一年过了又一年,投资大众慢慢晓得,或许股票是因为子虚乌有的神话而涨不起来。预期中的景气衰退一直没来,而且,除了一九五〇年代两次相当轻微的衰退,基础已经奠好,长线投资人将大有斩获。

一九八〇年代开始前几个星期,我写这段文字时,我很惊讶地发现,投资人并没有多花点心思重新研究一九四六年下半年起几年来的股市走势,探讨以前的战后时期和目前之间是否真的有雷同之处。现在,我这一生第三次看到以历史标准而言,很多股票的价格低得不象话。相对于企业发表的账面价值,股价或许不像二次世界大战后那么便宜,但如以扣除通货膨胀因素后的实质重置价值(replacement value),调整账面价值,它们可能比前两次低价期都便宜。问题是:目前这段期间,能源成本居高不下、政治上有极左派的威胁、信用过度扩张,在流动性恢复的此刻,企业活动水平难免受到压抑,这些忧虑使得股价欲涨不易,但和前两段期间导致股价低迷的忧虑比起来,目前的忧虑有比较严重,更容易阻碍未来的成长吗?如果不会,则一旦信用过度扩张的问题解决,我们便可以合理地假设:一九八〇年代和之后的期间,提供的高报酬机会,可能不亚于股价极低的前两段期间。
\\

\textbf{丰收年得到的教训}


从个人事业经营的角度来说,一千九百五十四到一九六九年的十五年内,是我辉煌腾达的一段期间,因为我持有的股票虽然不多,涨幅却远高于整体股市。即使如此,我还是犯下一些错误。成功来自我勤于运用前面所述的方法。比较值得一提的是错误的部分。每次犯错都带来新的教训。

好运易生怠惰。现在最困扰我的事情,不是损失最惨重的事情,而是良好的原则运用时漫不经心。

一九六〇年代初,我在电子、化学、冶金、机械业的科技投资相当如意,但在前景看好的制药业,没有等量齐观的投资,于是开始寻找投资对象。这个过程中,我找了这个领域一位著名的医学专家一谈。那时他对中西部一家小型制造商即将推出的一种新型药品系列极为看好。他觉得这些药品可能为这家公司相对于同业的未来盈余带来相当有利的影响。潜在市场似乎非常雄厚。

我只找了那家公司一位高阶主管和少数投资人士一谈。他们都对这种新药品的潜力有同样美好的看法。很不幸,我并没有做标准的检查动作,向其他药品公司或熟悉这门专业的其他专家请教,探讨是否有相反的证据。很遗憾,后来我才晓得,那些非常看好未来美景的人,也没彻底研究调查。

还没考虑新系列药品的利益之前,这家公司的股价远高于实质价值,但如新药品真如那些看好远景的人想象的那么美好,则当时的价格可能远低于潜在价值。我买进这支股票之后,价格竟然节节下跌,起初跌掉二十%,接着跌幅超过五十%。最后整家公司以这么低的现金价格,卖给想要踏进药品业的一家非制药业大公司。即使以这个价格,也就是比我当初买价的一半还低,我后来才晓得,收购公司在这宗交易上赔了钱。新的药品系列不只没能达成我的医学专家朋友寄以的厚望,而且从事后痛苦的「验尸报告」,我发现这家小型制药公司的管理阶层有问题。我相信,只要研究调查做得再彻底些,我一定能看清这两件不利的事情。

这次尴尬的经验之后,我总是在诸事顺遂之际,研究调查工作做得特别彻底。这次愚蠢的投资行为,没有损失更多钱的唯一理由,源于我个人小心谨慎的态度。由于我和管理阶层接触不多,所以初步的投资金额很低,计划对该公司认识更清楚之后,再多买一点。可是在我有机会让原来的愚行更为严重之前,公司便出了问题。

为期甚长的多头市场终于在一九六九年到达最高点,这时我又犯了另一次错误。为了了解这件事,有必要说明那时大部分投资人对科技股着迷的情况。这些公司的股票,特别是许多规模较小的公司,涨幅远高于大盘。一九六八到一九六九年期间,似乎只有想象力才能冷却许多这类公司即将一步登天的梦想。没错,一些公司的确有很大的潜力,可是投资人不分青红皂白,买红了眼。比方说,许多人相信,任何公司不管以什么方式对计算机业提供服务,前途都无可限量。这股热潮也蔓延到仪器和其他科技公司。

在这之前,我都极力克制自己,不受诱去买前一两年才以很高价格「公开上市」的类似公司股票。可是由于我经常和一些人接触,而他们对这些公司赞不绝口,因此我也不断寻找可能值得投资的对象。一九六九年,我终于找到一家设备公司,从事于极有趣的新科技领域;这个领域很有存在的价值。这家公司的总裁才华洋溢,为人诚实。我现在还记得,和这个人吃过冗长的午餐之后,在机场等候搭机回家时,我一直在想,要不要以当时的市价买进这家公司的股票。几经思索,我终于决定放手去做。

关于这家公司的潜力,我的分析可能是对的,因为接下来几年,该公司的确有成长。可是这却是一次差劲的投资。我所犯的错误,在于为了参与这家公司美好的未来,付出的价格不对。几年后,这家公司成长相当大,我卖出股票,但价格和原始成本几无不同。我认为这家公司未来的成长已变得远不如以往确定,决定卖出持股,我相信这么做是对的,可是持有那么多年后才赚到微薄的利润,不是资本成长之道,更别提保障老本不受通货膨胀侵蚀。这个例子中,绩效所以令人大失所望,在于忍受不住当时弥漫的激情诱惑,付出不切实际的价格买进股票。
\\

\textbf{把少数事情做好}


政策判断错误引起另一种相当不同的犯错方式,害我损失不赀。这次错误,是把个人的技能用到曾有的经验以外的地方。我开始投资到个人非常了解的行业以外,进入到活动完全不同的领域。那些行业,我并没有等量齐观的背景知识。

谈到供应工业市场的制造公司,或者拥有尖端科技,服务制造商的公司,我相信自己晓得要看什么──强在何处,以及哪里可能有陷阱。但是评估产销消费性产品的公司时,不同的技能很重要。相互竞争的公司生产的产品本质上很相近,以及市场占有率的变动主要取决于消费大众不断变迁的品味,或受广告效果影响很大的流行热时,我终于晓得个人挑选杰出科技公司的能力,无法延伸应用,找到不同凡响的不动产公司。

其他人可能在相当不同的投资领域做得很好。和我的事业生涯中犯下的其他错误比起来,其他人或许可以忽略这一点。不过,分析师应该了解自己的能力极限,把身边的羊儿照顾好。
\\

\textbf{市场可能转而下跌时,该留该卖?}


面对市况可能转差时,投资人应不应该卖出手中的好股票?关于这个问题,在今天盛行的投资心理中,恐怕我的看法属于少数。目前这个国家持有大量股票的人,所做所为似乎反映了一种信念,也就是投资人的持股已有不错的利润,但担心股价可能下跌时,则应该脱售持股,获利了结。这种现象比以前更为明显。即使某家公司的股票价格似乎已到或接近暂时性的头部,而且近期内可能大幅下跌,只要我相信长期内仍值得投资,便不会出售这家公司的股票。如果我分析几年内这些股票的价格会涨得远高于目前的水平,我宁可抱牢不放。我的信念源于投资过程中本质上的一些基本面考虑因素。增值潜力雄厚的公司很难找,因为这样的公司不多。但是如能了解和运用良好的基本面原则,我相信真正出色的公司和平凡普通的公司一定有差异,而且准确度可能高达九十%。

预测特定股票未来六个月内的表现,则困难得多。要分析个股短期内的表现,必须从分析整体工商业景气近期内的经济展望做起。预测专家预测工商业景气变动的纪录,大致来说惨不忍睹。他们可能严重误判景气会不会衰退,以及何时衰退,至于严重程度和时期长短,准确度更糟。此外,整体股市和任何个股的走势,不见得和景气环境亦步亦趋。群众心理的改变,以及金融圈对整体工商业景气或特定股票的评价,重要性高出许多,而且变动不居,难以预测。由于这些理由,我相信不管如何精研预测技巧,准确预测到股价短期内走势的机率很难超过六十%。而且这个估计值可能还太过乐观。既然如此,在正确机率顶多可能只有六十%的情形下,便决定卖出正确机率高达九十%的股票,这样的做法实在不合理。

而且,对长线投资追求厚利的投资人来说,赢的胜算只是考虑因素之一。如果投资的对象是经营管理良好的公司,财务力够强,即使最严重的空头市场,也不会使持股价值化为乌有。相对的,真正不同凡响的股票,后来创下的高价往往是先前高价的好几倍。所以说,从风险/报酬的角度来看,长期投资比较有利。

因此,用最简单的数学公式可以看出,机率和风险/报酬都有利于抱牢持股。和预测一支好股票长期雄厚的价格成长潜力比起来,分析好股票短期走势不利,但分析错误的机率要高得多。抱牢正确的股票,即使市场暂时大幅回调,最糟的时候,股价顶多暂时比前一个最高价下跌四十%,最后总有一天赚回来;但如果你卖出去而且没有补回,则和预期价格短期内会反转而卖出持股所获得的短期利益比起来,错失交臂的长期利润可能是好几倍。根据我的观察,我们很难抓准近期内好股票的价格波动时机,即使有几次卖出持股,后来以低很多的价格补回,获得利润,但和时机抓错失去的利润比起来,实在微不足道。许多人太早卖出股票,后来不是补不回来,便是转投资的时间拖得太久,没有尽可能重新掌握利润。

以下用说明这一点的例子,是我的经验中最弱的一环。一九六二年,我大量投资两家电子公司的股票价格涨到很高的价位,近期内的价格走势展望十分危险。德州仪器(Texas Instruments)的价格是我七年前进价的十五倍以上。另一家公司,我在约一年后买进,应该用个假名称呼它,姑且叫做「中央加州电子公司」,涨幅和德州仪器相近。两者的价格都涨得太高。于是我通知每位客户,说两支股票的价格已经高得离谱,不鼓励他们用这些价格衡量自己目前的财富净值。我很少这么做,除非有异常强烈的信心,觉得下一波重要的走势中,我持有的一支或多支股票价格将大幅下跌。不过,虽然有那么强烈的信心,我还是敦促客户抱牢持股,相信几年后两支股票会涨到高出许多的价格。两支股票的回调幅度比我预期的要大。德州仪器最低跌到比一九六二年的高价低八十%,中央加州电子公司没有那么糟,但还是跌了约六十%。我的信心受到极为严峻的考验!

但是几年内,德州仪器再涨到新高价,比一九六二年的高价高一倍以上。耐心终有所获。中央加州电子公司的表现则令人不快。在股市大盘开始回升之际,中央加州电子公司管理阶层的问题浮现出来,人事有所异动。我变得相当担心,做了彻底的研究调查,得到两个结论,而且这两个结论都难令人欣喜。我应该更留意管理阶层的缺点才是,却没这么做。新的管理阶层上任,我也不觉得特别振奋,认为值得抱牢它的股票。于是接下来十二个月内,我卖掉该公司的股票,价格只比一九六二年高价的一半稍高一点。即使如此,我的客户因为进价不同,利润仍是原始成本的七到十倍。

前面已经指出,我故意引用比较差的例子,而不用戏剧性的例子,来说明为什么我相信值得忽视前景亮丽的好股票的短期波动。我投资中央加州电子公司所犯的错误,不在持有股票度过暂时性的跌势,而在另一件远为重要的事情上。由于投资这家公司已有可观的获利,我变得志得意满而松懈下来。我开始过分听信高阶管理人员告诉我的话,向较低阶员工、客户查证的工作做得不够充分。当我认清形势,并采取行动之后,我就能把资金移转出来,投资其他电子公司,赚到原本预期能从中央加州电子公司股票上面赚到的钱。移转出来的资金,主要投资于摩托罗拉(Motorola)。很幸运的,几年内摩托罗拉的股价涨为中央加州电子公司上一个高价的好几倍。
\\

\textbf{进出不停可能蚀掉老本}


从德州仪器和中央加州电子的例子,还可以学到更多东西。一九五五年夏天买进德州仪器公司的股票时,本来就打算做最长期的投资。在我看来,这家公司的表现值得我对它那么有信心。约一年后,股价涨为两倍。我管理的资金有各式各样的投资人,除了一群人,其他人都很熟悉我的操作方法,而且和我一样,都不喜欢获利了结。不过,那时候我有个相当新的账户,由一群人持有。他们有自己的做法,习惯在市场跌到低点时买进,涨到高点时则大量卖出。德州仪器股价涨为两倍时,他们给我很大的压力,说要卖出,不过我还能抗拒他们的要求。等到股价再涨二十五%,他们的利润为成本的一百二十五%时,卖出的压力更强大。他们解释说:「我们同意你的看法。我们喜欢这家公司,但我们一定能够趁下跌时用更好的价格买回。」我终于和他们达成妥协,说服他们保留一部分,卖出其余的部分。几年后大跌时,价格从最高点下挫八十%,但新的低点还是比他们急着卖出的价位高约四十%。

一支股票经过一段很急的涨势之后,对没有受过理财投资训练的人来说,价格总是显得太高。那些客户表现了另一种冒险的做法,也就是只因已实现不错的利益,股价看起来暂时显得过高,便急着卖出成长前景仍然很好的股票。这些投资人犯了错之后,很少以更高的价格买回股票,又错失另一段可观的涨幅。

虽有一提再提之嫌,我还是要再说一遍:我相信短期的价格波动本质上难以捉摸,不易预测,因此玩抢进杀出的游戏,不可能像长期抱牢正确的股票那样,一而再,再而三,获得庞大的利润。
\\

\textbf{只够浅尝的股利}


经过这么多年之后,前面所说种种,是为了说明各式各样的经验,有助于我的投资哲学缓慢成形。但是回顾过去,我找不到特定的事件,不管是犯错或者有利的机会,引导我就股利一事做成结论。经年累月观察很多事情,我的看法逐渐具体。四十年前和今天一样,大家普遍认为股利对持股人很有利,应该张开双手欢迎。起初我的看法也一样。接下来,我开始看到一些公司有很多令人振奋的新观念从研究部门流出,根本无法全部善加利用,付诸实行。资源太少,也太昂贵。我开始想到,要是没有派发股利,而把公司大部分的资源保留下来,投资在更多的创新性产品上,对某些持股人来说,利益应该更高。

我逐渐体认到,所有持股人的利益不尽相同。有些投资人需要股利收益,以应日常生活之需。这些持股人无疑喜欢当期股利甚于未来更多的盈余,并因为公司增加投资前途看好的产品和技术,使得他们的持股价值提高。这些投资人可以找到一些公司投资,因为它们的资金需求不大,而且没有很多机会,能将资金用在生产性用途上。

但如持股人的赚钱能力或其他收入来源超过日常需求,而且经常能够储蓄,则情况如何?公司不派发股利,把资金拿去再投资,追求未来更高的成长,不是比较好吗?股利往往适用相当高的所得税率,但盈余转投资不须课税。

二次世界大战结束后没多久,我开始把投资活动几乎完全集中追求长期的大幅资本增值,股利配发问题的另一个层面变得更为明显。成长前景极佳的公司,受到很大的压力,要求不要配发股利。它们十分需要资金,善用资金的能力也很强。开发新产品的成本,只是动用资金追求成长的第一笔大开销。接着需要庞大的营销费用,把新产品引介给客户。上市成功的话,公司又必须扩厂,以因应日益增加的需求量。新产品一上市,增加存货和应收帐款的资金需求会进一步提高。大部分情况下,存货和应收帐款大致随着营业量等比例升高。

投资机会很多的公司,和希望在一定的风险下,获得最大利润,而且不需要额外收入或不想多缴不必要税负的投资人,两者的利益似乎可以搭配得天衣无缝。我相信,这样的投资人应该把主要的投资局限于不派发股利,但获利能力很强,而且有好地方转投资盈余的公司。我想服务的是这样的客户。

但是最近,情况变得没有那么明显。机构持股人在每天的股票交易活动上,主宰力量愈来愈大。养老金和利润分享基金领得的股利不必缴纳所得税。因此许多机构持股人订定政策,除非派发一些股利,不管多低,否则不投资若干公司。为了吸引和留住这些买主,许多前景非常明亮的公司,开始适量派发股利,但只占每年总盈余的很低比率。在此同时,一些准成长型公司的经理人大幅降低股利派发金额。今天,要区分哪些公司的确不同凡响,善于投资保留盈余的技能,是比以前更要紧的因素。

由于这些原因,我相信,关于股利这个问题,能够说的话是:凡是不需要这笔收入的人,应该把这个因素的重要性大幅降低。大体来说,派发股利低或者根本不派发股利的公司,可以找到更多值得投资的机会。但是由于决定股利政策的人普遍觉得,派发股利对投资人有好处(至少对某些投资人有利),因此偶尔我能在高股利的公司中找到真的不错的投资机会,但这种事不常有。

\section{市场真的有效?}

到了一九七〇年代,在四十年来的经验塑造下,我的投资哲学几乎全部成形。前面提过一些例子,用以说明我的投资哲学成形的背景。这些例子中,不管是聪明的行为,还是愚蠢的行为,除了一个,其余全部发生在前面四十年,并非巧合。这不表示我在一九七〇年代没有犯下错误。很遗憾的,不管我如何尝试,似乎总要以同样的方式摔倒一次以上,才能真正学到教训。但是我用到的例子,通常是某一事件首次发生的情形,藉之说明我的论点。这可以说明为什么除了一个例子,其余的例子全部发生在以前四十年。

过去四十年,每一年代之间的雷同之处令人惊讶。指出这一点,或许有帮助。或许除了一九六〇年代,每个年代都有一段期间,人们盛行的看法是外部影响力量很大,而且超出个别公司管理阶层的控制能力,因此即使最聪明的普通股投资也属有勇无谋的行为,可能不适合精明谨慎的人去做。一九三〇年代,受到经济大萧条的影响,有几年这样的看法甚嚣尘上,但和一九四〇年代人们对德国战争机器及二次世界大战的恐惧,或一九五〇年代另一次景气严重萧条势必来临的忧虑,或一九七〇年代人们担心通货膨胀、政府过度干预等事情比起来,则有所不及。但是事后来看,每个年代都创造出不可思议的投资机会。这五个年代,每个年代投资人都有很多机会(不是只有少数机会),买进普通股并抱牢,十年后获得好几倍的利润。有些时候,利润甚至高达几十倍。同样的,这五个年代,每个年代都有一些当时的投机性热门股,后来证明是盲从群众的人最危险的陷阱,而真正晓得自己在做什么的人,不受波及。所有的年代本质上都很像,也就是获利最丰的机会来自非常值得的投资,但当时金融圈严重错估情势,使得它们的价格低估。当我回顾五十年来冲击证券市场的各种力量,以及这段期间内群众悲喜轮替的现象时,「事情变得愈多,愈是保持原状」这句法国谚语便会浮上心头。我一点都不怀疑,在我们踏进一九八〇年代之际,面对它带来的所有问题和美景,同样的事情会继续维持下去。
\\

\textbf{有效市场的谬误}


过去几年,人们把太多的注意力放在我相信不对的一种观念上。我指的是市场效率十分完美的观念。但其他时期的其他错误信念一样,明察秋毫的人如持反向意见,可能有获利良机。

对不了解「有效」市场理论("efficient"market theory)的人来说,「有效」这个形容词,不是指市场显而易见的机械效率。想要买卖股票的人,可以把委托单下到市场里,几分钟内,交易便很有效率地执行完毕。「有效」也不是指敏锐的调整机制,能够反映买方和卖方相对压力的变化,而使股价上涨或下跌几分之一美元。相反的,这个观念说,任何时点,市场的「有效」价格应已充分和务实地反映一公司所有已知的事情。除非某人拥有重要、非法的内线情报,否则没办法找到真正便宜的股票,因为能够让潜在买主相信值得投资的情况存在的有利影响力量,已经反映在股票价格上!

如果市场真如一般人相信的那么有效率,而且重要的买进机会或重要的卖出原因没有不断出现,股票的报酬率就不应该有那么大的变动。所谓变动,我不是指整体市场价格的变化,而是指一支股票相对于另一支股票价格变化的分散情形。如果市场有可能那么有效率,则进行分析以获得这种效率的因果关系,整体来说一定很差。

有效市场理论源于随机漫步学派(School of Random Walkers)的学界人士。这些人发现,很难找到一种技术操作策略运作得够好,扣除交易成本后,相对于所冒的风险,能够提供优渥的利润。我不反对这种说法。前面已说过,我们极难根据短期的市场预测,从短线进出操作中赚到钱。所谓市场效率很高,或许只是这个狭隘的意思。

我们大部分是投资人,也应该是投资人(investors),不是操作者(traders)。我们应该寻找长期前景非常美好的投资机会,避开前景较差的投资机会。这一直是任何状况中,我的投资方法的中心支柱。我不相信对勤奋、知识丰富的长期投资人来说,价格很有效率。

一九六一年我的经验可以直接应用在这件事上面。那年秋天,以及一九六三年春天,我接受一件富有挑战性的工作,代理专职财务学教授,在史丹福大学商学研究所教高级投资课程。「高效率」市场的观念未能见及未来数年的情形,和我即将描述的作业练习的动机无关。相反的,我只是要让学生用一种难忘的方式,了解整体市场的波动,相较于个股间价格变动的差异,实在微不足道。

我把全班学生分成两组。第一组拿到纽约证券交易所(New York Stock Exchange)依英文字母顺序排列的股票名单,从A开始;第二组则从T开始。每一支股票都依字母顺序(优先股和公用事业股除外,我认为它们是不同的类别)。每位学生分到四支股票,必须去查一九五六年封关日的收盘价,并经股票股利和股票分割调整(忽略权值,因为影响不大,不值得再加计算),并拿这个价格和十月十三日星期五(没其他用意,只是这一天的收盘价很精彩)的价格比较。将近五年的这段期间,每支股票的涨跌幅度都标注出来。道琼卅种工业股价指数从四百九十九点涨到七百零三点,涨幅为四十一%。总计整个样本有一百四十支股票。结果如下表:

\begin{tabular}{lllll}
    \textbf{}              & \textbf{资本利得或损失百分率} & \textbf{同类股票数目} & \textbf{占全部股票的百分率} & \textbf{} \\
    \hline
    \multicolumn{1}{r}{利得} & 200\% 到 1020\%      & 15支股票           & 11\%               & \textbf{} \\
    \multicolumn{1}{r}{利得} & 100\% 到 199\%       & 18支股票           & 13\%               & \textbf{} \\
    \multicolumn{1}{r}{利得} & 50\% 到 99\%         & 14支股票           & 10\%               &           \\
    \multicolumn{1}{r}{利得} & 25\% 到 49\%         & 21支股票           & 15\%               &           \\
    \multicolumn{1}{r}{利得} & 1\% 到 24\%          & 31支股票           & 22\%               &           \\
    不变                     &                     & 3支股票            & 2\%                &           \\
    损失                     & 1\% 到 49\%          & 32支股票           & 23\%               &           \\
    损失                     & 50\% 到 74\%         & 6支股票            & 4\%                &           \\
    \hline
                           &                     & 140支股票          & 100\%              &          
    \end{tabular}


从这些数据可以看出很多东西。在道琼卅种工业股价指数上涨四十一%的期间内,三十八支股票有资本损失,占总数的二十七%。其中六支股票,占全部股票的四%,总值损失超过五十%。相对的,约四分之一的股票有非常不错的资本利得。

言归正传,我注意到,如果一个人投资一万美元,等分于这张清单上最好的五支股票,则四年又三季以后,他的资本现在值七万零二百六十美元。相反的,如果他投资一万美元买到五支最差的股票,资本现在萎缩为三千一百八十美元。这种极端的结果很不可能发生。要出现这种大好或大坏的结果,除了技能,得靠运气,不管好运或坏运。投资判断力真正不错的人,能从十支最好的股票中挑到五支,投资一万美元,并非难以想象。这种情况中,十三日星期五他的资产净值将是五万二千零七十美元。同样的,一些投资人老是基于错误的理由选择股票,而且总是选到烂股。对他们来说,选到十支最差的股票里面的五支,也不是完全不可能。这种情况中,一万美元的投资会缩水为四千二百七十美元。拿这两个数字来比较,可以发现,不到五年内,聪明的投资行为和愚昧的投资行为,两者的财富差距可以高达四万八千美元。

一年半后,我又教同一个课程,再做完全相同的练习,只是没用字母A和T,改用另两个字母,构成股票样本。同样的,在五年的时间架构内,虽然起迄时间不同,变异的程度几乎完全一样。

我相信,回顾大部分为期五年的市场,我们可以发现股价涨跌有相似的差异。股价表现相差很大,有些可能来自出乎意料的事件──关于一支股票前景的重要新信息,在一段期间之初始料未及。但是大部分的差异,至少在涨跌方向和相对于市场的涨跌幅度,大致上事先预料得到。
\\

\textbf{雷伊化学公司}


有鉴于这些证据,我很难想象有人认为股市有高效率。这里又用到「高效率」一词,因为主张这个理论的人是这么使用的。但在进一步详细讨论之前,容我先谈仅仅几年之前的一种股市情况。一九七〇年初,雷伊化学公司(Raychem Corporation)的股票在市场上享有很高的盛名,因此本益比相当高。其中一些原因,从该公司执行副总裁罗伯·海尔普林(Robert M.Halperin)讲的一些话,可能看出端倪。谈到雷伊化学公司经营哲学的四大要点时,他说:


一、雷伊化学不做技术上简单的任何事情(也就是潜在竞争对手容易模仿的东西)。

二、除非能够垂直整合,否则雷伊化学不做任何事情;也就是,雷伊化学必须研究发展、生产制造、销售产品给客户。

三、除非有很大的机会可以获得特有的保障,通常是指专利权保护,否则雷伊化学不做任何事情。除非能有这样的事情,否则研究发展精力不会用在某个项目上,即使符合雷伊化学的专长,也不惜牺牲。

四、不管在什么产品目标市场立基,不论市场规模大小,雷伊化学只产制它相信能成为市场领导者的新产品。


到了一九七〇年代中期,控制大型机构基金的人普遍留意到这些异常突出的优势,相信雷伊化学具有非凡的竞争优势和吸引力,因此在市场上大量买进该公司的股票。但是最能吸引这些持股人,以及那时本益比偏高的可能原因,在于这家公司的另一个层面。许多人认为雷伊化学花在新项目开发上的资金,占营业额的比率高于平均水平,拥有非常优秀的研究组织,能生产出够多的新产品,公司可以仰赖它们,营业额和盈余将持续上扬,毫不中断。这些研究中的产品的确对金融圈有一股特别的吸引力,因为许多较新的产品只是和其他公司的老产品间接竞争。最重要的是,这些新产品能让工资很高的劳工,做同样的事所花的时间,远低于从前。公司节省下来的成本够多,除了回馈最终客户,还能有相当不错的利润率。所有这些事情,推动股价上扬,一九七五年底涨到四二.五美元以上的高价(价格经后来的股票分割调整)──是一九七六年六月三十日结束的会计年度估计盈余的廿五倍左右。
\\

\textbf{雷伊化学期望破灭,股价崩跌}


一九七六年六月三十日止的会计年度快结束的时候,雷伊化学惨遭双重打击,不但股价重挫,在金融圈的声誉也一落千丈。金融圈本来对一种叫做Stilan的特有聚合物前景十分看好,因为相对于航空业用于涂布线路的其他化合物,它有独特的优点,而且处于最后研发阶段。同时这种聚合物是雷伊化学走向基本化学品的第一种产品,也就是雷伊化学开始在自己的工厂生产上游化学品,而不向别人购买原物料再加工。由于这种产品很有吸引力,雷伊化学拨给这种研发产品的经费,远高于其他任何产品,相差之巨,历年仅见。金融圈认为这种产品业已迈向成功之路,经历所有新产品必经的「学习曲线」之后,它将有很高的利润率。

事实上,后来的发展完全相反。照雷伊化学公司管理阶层的说法,Stilan「科学上成功,但商业上失败」。一家能力很强的竞争对手推出改良型产品,虽然技术上不如Stilan那么好,功能却恰合所需,而且价格便宜许多。雷伊化学的管理阶层承认这一点。几个星期内,管理阶层做成痛苦的决定,放弃这种产品,冲销在它上面投下的庞大投资。冲销金额使那一会计年度的盈余减少约九百三十万美元。不计一些特别利益,每股盈余从上一会计年度的七.九五美元降低到○.○八美元。

除了盈余急降,金融圈对该公司研发能力的强烈信心也动摇,令人大感不安。金融圈普遍忽视了一个基本原则,也就是所有公司的一些新产品开发难免终归失败。所有的工业研究活动,以及经营管理良好的公司,都无法摆脱这一点,但长期而言,其他成功的新产品将能弥补失败的产品仍有余。花最多钱的特定项目,最后竟然失败,有时可能只是运气欠佳。无论如何,股价受到很大的影响。一九七六年第四季,股价跌到约一四.七五美元的低点(同样经过后来一股分割成六股的调整),是前一个高点的三分之一左右。当然了,只有少量股票能以当年的低价买进或卖出。更重要的是,此后好几个月内,股价只略高于这个低价。

另一件事也影响该公司此时的利润,并导致雷伊化学失宠。任何成长顺利的公司,主其事者最艰难的任务之一,是适时调整管理结构,好在公司成长时,兼容并蓄小公司适当控制和大公司最佳控制间的差异。一九七六会计年度结束之前,雷伊化学的管理阶层主要是依制造技术划分事业部门;也就是,根据生产的产品设立事业部门。公司规模还小时,这套方法运作得很好,但随着公司逐步成长,不利于以最高的效率服务客户。因此,一九七五会计年度结束之际,雷伊化学的高阶管理人员开始着手于「大公司」的管理理念。公司依所服务的行业重新架构事业部门,而不是依产品的物理和化学组成。调整组织结构的目标日期订在一九七六会计年度底。当初订定这个日期时,管理阶层压根儿没想到竟和放弃Stilan产生的庞大冲销金额碰在一起。

雷伊化学公司每个人都知道,组织结构调整后,至少一季,可能两季以上,盈余会大幅减低。虽然组织结构调整之后,雷伊化学的管理阶层人事几乎没有变动,但是很多人现在有了不同的上级主管,不同的下级部属,以及必须相互协调作业的不同同事,所以会有一段效率较低的适应期,直到雷伊化学的员工学会如何以最好的方式,和往来的新面孔协调作业。公司管理阶层决定按照原定日期实施结构调整计划,而不把势必对雷伊化学目前的盈余造成二次打击的事情延到以后再说。从这点,正可以强烈看出公司对长期的未来有信心,不担心短期的绩效。

事实上,大幅调整组织结构遭遇的困难,远比当初想象的要低。正如预期,新会计年度第一季的盈余远低于如果不调整组织结构时的水平。但是这次变革运作得很好,随着第二季展开,这方面的短期成本多已消除。从基本面看,在分析师眼里,这些发展应属利多。雷伊化学现在所处地位,更能应付未来的成长,而照以前的方式,根本做不到。它已成功地跨越一道障碍,这道障碍很可能使原本吸引人的成长公司的光采黯淡下来。但是大体来说,金融圈似乎未能体认到这一点,相反的,盈余暂时进一步萎缩,使股价跌到低点后欲振乏力。

对潜在投资人而言,这个价位更吸引人的地方,在于另一股影响力量。其他公司放弃已证明不成功的重大研究计划后不久,我也曾看到这股力量。放弃Stilan的一个财务效果,是投入那个计划的大量资金现在释放出来,可用于别的地方。更要紧的是,重要的研究人员同样能够抽身而出,从事其他的计划。一两年内,就像久旱逢甘霖,百花齐放,公司开始见到前景美好的研究计划相对于它的规模来说,比例之高,前所未见。
\\

\textbf{雷伊化学公司和高效率市场理论}


那么,雷伊化学的故事,和最近在金融圈若干地方引起许多人支持的「高效率市场」理论,有什么关系?那个理论说,股价会自动和立即反映任何该公司已知的事情,因此只有拥有他人不知的非法「内线情报」的人,才会可能从一支股票未来的走势中获利。就这件事来说,雷伊化学的管理阶层十分坦白,愿意向任何有兴趣的人说明前面提过的所有事实,并解释为什么盈余不佳的期间会相当短暂。

其实,以上所有的事情发生之后,盈余终于攀升到新高水平,雷伊化学公司的管理阶层更进一步,做了一件事。一九七八年一月二十六日,他们在总公司举办长达一天的会议,我有幸参加。雷伊化学公司的管理阶层邀请所有机构、经纪商和投资顾问参加这次会议。这些人不是对雷伊化学有兴趣,就是可能有兴趣。会议中,公司最高十位主管十分坦诚和不厌其详地说明解释公司的前景、问题、个人负责事务的现状。我只偶尔在其他公司的类似会议见到那么诚恳和详细的报告。

会议召开后一两年,雷伊化学的盈余成长状况,和根据会议上所说的事情去推测一模一样。那段期间,股价从会议当天的二三.二五美元上涨一倍以上。可是在会议结束后几个星期内,股价没有受到特别的影响。若干与会者显然对会议上描述的美景留下深刻的印象,但太多人仍无法摆脱一两年前双重震撼的阴影。他们显然不相信会议上听到的话。高效率市场理论不过如此。

从雷伊化学等公司的经验,投资人或投资专业人士能够得出什么样的结论?接受和受「高效率市场」理论影响的人,大致可分成两类。一类是学生,他们的实务经验很少。另一类很奇怪,似乎是大型机构基金的许多经理人。大体而言,投资散户很少注意这个理论。

从我实际运用个人的投资哲学获得的经验,可以做出结论:一九七〇年代即将结束的此刻,在我专长的科技股,大型公司的有利投资机会多于小型科技公司;前者的市场为机构所主宰,投资散户则在后者扮演远为吃重的角色。约十年前,一些人能看出当时盛行的两级市场观念的谬误,并因为认清这个观念特别荒唐之处而从中获利。同样的,每个年代中,一有错误的观念出现,就会为那些具有投资敏锐观察力的人创造机会。
\\

\textbf{结论}


以上所说,就是我从半个世纪的商业经验中发展出来的投资哲学。它的核心或许可以总结成如下八点:


一、买进的公司,应有按部就班的计划,以使盈余长期大幅成长,而且内在的特质很难让新加入者分享那么高的成长。选择这样一家公司,还有很多有利或不利的细节应该考虑,显然不可能用这么短的篇幅说明清楚。有兴趣的读者,不妨参考《保守型投资人夜夜安枕》一篇前三章。我在那里以尽可能简洁的文字陈述这个主题。

二、集中全力购买那些失宠的公司;也就是说,由于整体市况或当时金融圈误判一家公司的真正价值,使得股票的价格远低于真正的价值更为人了解时将有的价位,则应该断然买进。

三、抱牢股票直到(a)公司的特质从根本发生改变(如人事异动后,管理阶层的能力减弱),或者(b)公司成长到某个地步后,成长率不再能够高于整体经济。除非有非常例外的情形,否则不因经济或股市走向的预测而卖出持股,因为这方面的变动太难预测。绝对不要因为短期原因,就卖出最具魅力的持股。但是随着公司的成长,不要忘了许多公司规模还小时,经营得相当有效率,却无法改变管理风格,以大公司所需的不同技能来经营公司。

四、主要目标在追求资本大幅成长的投资人,应淡化股利的重要性。获利高,但股利低或根本不发股利的公司中,最有可能找到十分理想的投资对象。配发给持股人的股利占盈余百分率很高的公司,找到理想投资对象的机率小得多。

五、为了赚到厚利而投资,犯下若干错误是无法避免的成本,一如经营管理最好和最赚钱的金融贷款机构,也无法避免一些呆账损失。重要的是尽快承认错误、了解它们的成因,并学会避免重蹈覆辙。良好的投资管理态度,是愿意承受若干股票的小额损失,并让前途较为看好的股票,利润愈增愈多。好的投资一有绳头小利便获利了结,却任令坏的投资带来的损失愈滚愈大,是不良的投资习惯。绝对不要只为了实现获利就获利了结。

六、真正出色的公司,数量相当少。它们的股票往往没办法以低廉的价格买到。因此,有利的价格存在时,应充分掌握当时的情势。资金应该集中在最有利的机会上。那些介入创业资金和小型公司(如年营业额不到二千五百万美元)的人,可能需要较高程度的分散投资。至于规模较大的公司,如要适当分散投资,则必须投资经济特性各异的各种行业。对投资散户(可能和机构投资人以及若干基金类别不同)来说,持有廿种以上的不同股票,是投资理财能力薄弱的迹象。通常十或十二种是比较理想的数目。有些时候,基于资本利得税成本的考虑,可能值得花数年的时间,慢慢集中投资到少数几家公司。投资散户的持股往廿种增加时,淘汰一些最没吸引力的公司,转而持有较具吸引力的公司,是理想的做法。务请记住:ERISA的意思是「徒劳无功:行动时思虑欠周」(Emasculated Results:Insufficient Sophisticated Action)。

七、卓越的普通股管理,一个基本要素是能够不盲从当时的金融圈主流意见,也不会只为了反其道而行便排斥当时盛行的看法。相反的,投资人应该拥有更多的知识,应用更好的判断力,彻底评估特定的情境,并有道德勇气,在你的判断结果告诉你,你是对的时候,「虽千万人而吾往矣」。

八、投资普通股和人类其他大部分活动领域一样,想要成功,必须努力工作、勤奋不懈、诚信正直。


以上所说每项特质,我们有些人可能与生俱来优于或劣于他人。但是我相信,只要严以律己和投入心血,所有的人都可以在上述每个领域,「养大」自己的能力。

普通股投资组合的管理,有时难免有些地方需要靠勇气,但长期而言,好运、坏运会相抵。想要持续成功,必须靠技能和继续运用良好的原则。根据我的八点指导原则架构,我相信未来主要属于那些能够自律且肯付出心血的人。

\appendix
\section{评估好公司的重要因素}

根据我的哲学,我只投资少数公司,而且这些公司的前景必须非常好。很明显的,我调查研究公司时,会注意它们有没有成长潜力的蛛丝马迹。同样重要的,我试着透过研究,避开风险。我希望确定公司的管理阶层有能力善用潜力,并在这个过程中让我的投资风险降到最低。我做财务分析、访问企业管理阶层、和熟悉某行业的人士讨论时,总会观察研究中的企业是否符合我的杰出标准。它们应具备的一些守势型特质,总结如下。
\\

\textbf{功能因素}


一、这家公司生产的产品或提供的服务,相较于竞争对手,必须是成本最低的公司之一,而且可望继续保持。

 a.损益平衡点相对偏低,可以在市况低迷时,让这家公司存活下去,并在体质较弱的竞争对手退出市场之后,强化它的市场和订价地位。

 b.由于利润率高于平均水平,这家公司能从内部创造出更多的资金,维系公司的成长,不必发行新股,使得股权稀释,或因为过度依赖固定收益融资工具,造成财务上的压力。

二、一家公司必须时时以顾客为念,体察客户需求和兴趣的变化,接着以适当的方式因应这些变化。要有这个能力,公司必须源源不绝推出新产品,弥补趋于成熟或落伍的老旧产品仍有余。

三、效果卓越的营销工作,不只必须了解客户需要什么,也要用客户能懂的语汇,向他们说明解释(透过广告、推销,或其他方法)。公司应该密切控制和不断监视市场努力的成本/效益。

四、今天即使非科技公司也需要强大和目标正确的研究能力,藉以(a)生产更新和更好的产品,以及(b)以效果更好或更有效率的方式提供服务。

五、研究的成效有很大的差异。研究要更有成效,两个要素不可或缺,它们是(a)市场/利润意识,以及(b)有能力汇聚必要的人才,组成成效卓著的工作的团队。

六、拥有强大财务团队的公司,享有几项重要的优势:

 a.良好的成本信息有助于管理阶层把精力导向利润贡献潜力最高的产品。

 b.成本制度应能明确指出哪些次营运单位的生产、营销和研究成本的运用效率欠佳。

 c.严格控制固定和营运资金的投资,而能保存资本。

七、至关紧要的财务功能可以做为预警系统,提前找出可能危及利润计划的影响力量,让公司有充裕的时间及早拟定矫正计划,把不利的意外事件减到最低。
\\

\textbf{人的因素}


一、公司要经营得更为成功,需要的领导者具有刚毅果决的创业家个性,结合必要的驱力、原创性构想和技能,以建立公司的财富。

二、成长导向的执行长身边必须围绕一个能力很强的团队,充分授权给他们,负责公司事务的营运。和具有破坏力的争权夺利不同,团队合作精神极其要紧。

三、公司必须用心吸引较低阶层的优秀经理人,并训练他们负更大的责任。遇有人事升迁,应优先考虑内部人才。从外面聘任执行长,尤其是企业衰弱的危险讯号。

四、整个组织必须弥漫创业家精神。

五、经营较为成功的公司,通常有一些独特的人格特质──做事情有一些特别的方法,对管理阶层来说,效果特别好。这是正面迹象,不是负面迹象。

六、管理阶层必须体认并适应一个事实,也就是公司运作其中的这个世界,变化的脚步愈来愈快。

 a.每一种已被接受的做事方式,必须定期重新检讨,寻找更好的新方法。

 b.改变管理方法难免有风险,应认清这一点,把风险降到最低后,冒险去做。

七.公司必须真诚、脚踏实地、念兹在兹、持续不断努力,让每个阶层的员工,包括蓝领阶级在内,相信公司真的是工作的好地方。

 a.对待员工的方式,必须让他们觉得受到尊重,建立起合理的尊严。

 b.公司的工作环境和福利计划,应能激励工作士气。

 c.员工应能在不必心怀恐惧的情况下表达不满,而且能够合理期待公司给予适当的注意和采取行动。

 d.参与式计划似乎运作得很好,而且是好构想的重要来源。

八、管理阶层必须愿意严守戒律,促使公司成长。要追求成长,必须牺牲若干目前的利润,为更美好的未来奠定基础。
\\

\textbf{企业特质}


一、虽然考虑新投资的时候,管理人员很重视资产报酬率,投资人却必须认清:以历史成本列示的历史资产,使得各公司间绩效的比较遭到扭曲。即使周转率有差异,盈余相对于营业额的比率高,可能表示投资的安全性较高,特别是通货膨胀率上扬时。

二、高利润率会引来竞争,竞争则会侵蚀获利机会。仰制竞争的最好方式,是以很高的效率营运,使得潜在竞争对手没有加入的诱因,知难而退。

三、规模效率往往被官僚习气浓厚的中阶管理人员低落的效率抵消掉。但对经营良好的公司来说,业界领袖地位可以创造出很强的竞争优势,而对投资人构成吸引力。

四、抢先踏进新产品市场,夺得第一,并非一蹴可几。有些公司抢到第一的条件比别人好。

五、各种产品并非孤岛。比方说,每种产品都在间接竞争,希望赚到消费者的钱。价格一有变动,即使经营良好的低成本公司,一些产品也可能失去吸引力。

六、在根基稳固的竞争对手已有强大地位的市场领域,很难推出优异的新产品。虽然新加入者可以努力加强生产、营销力量、商誉,以提高竞争力,既有的竞争对手还是能够采取强大的防卫行动,夺回受到威胁的市场。创新者如能相对于目前的竞争对手,以崭新的方式结合不同的科技学门,如电子和原子核物理学,则成功的机率较高。

七、要取得业界领导地位,科技只是其中一条管道。培养消费者的「忠诚」是另一条管道,卓越的服务也是。不管如何,一家公司必须有强大的能力,对抗新竞争对手,保卫既有的市场。理想的投资对象,这样的能力缺之不可。

\end{document}