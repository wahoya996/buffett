%\documentclass[UTF8,a4paper,zihao=-4]{ctexart} %设置了A4纸张和小四字号,这个mac也可以用
\documentclass[UTF8,a4paper,zihao=-4,fontset = ubuntu]{ctexart} %设置了A4纸张和小四字号和windows字体
%\usepackage{graphicx}
\setmainfont{Times New Roman} %设置英文字体为Times New Roman
\setlength{\parindent}{20pt}
\title{战胜市场 逆向投资} %标题加粗
\author{吉姆•罗杰斯}
%\date{2024年4月4日}
%\maketitle
\begin{document}
\section{我热衷于投资是因为投资可以使人获得自由}

罗杰斯:我年轻时之所以想赚钱,只是因为我渴望自由,而不是因为想开豪车或住豪宅。实际上,我家里也没有多余的车,豪车更是一辆也没有。住宅也是一样,只要能满足自己的居住需求、安保设施齐全的房子我认为就足够了。总之,我想要的是个人的自由、按照自己的想法生活。

为了实现这个愿望,我把从20到30多岁的时间和精力几乎全都用到了投资方面。虽然也有过痛苦的经历,但现在回想起来,我觉得那段时光十分宝贵。后来,我在投资上取得了成功,获得了自由。获得自由后我开始长期周游世界。这是我获得自由后最想实现的梦想。

我这么说或许有些苦口婆心了,现在的年轻人应该想想,自己是为了什么在学习。不少人二三十岁了还在研究生院学习MBA课程,我觉得他们应该重新去认识这是为了什么。

我对读MBA持否定态度。它不仅要花费大量的金钱和时间,而且所学的内容是否在社会上行得通,我也表示怀疑。况且,MBA的学费越来越高,很多人还没踏入社会就背上了沉重的债务。

与其花这么多时间和金钱在MBA上,不如用这些来投资股票,或者试着买一些商品期货。这样才能学到更多东西,让你的人生变得丰富多彩。或者也可以创业,哪怕规模很小。即使失败了,对于你来说也是一笔宝贵的人生财富。

俗话说“失败是成功之母”,我认为的确如此。我年轻时也经历过人生低谷,也曾因投资失败而身无分文。有些东西或许能在大学或研究生阶段学到,但在投资或事业上失败才能对你的人生大有裨益。

\textbf{——您从投资失败中得到的最大的教训是什么呢?}

罗杰斯:就是找到自己的风格。刚进入投资领域的新手,很容易受媒体和周围人的影响。“这只股票会涨到1000美元”“大豆在一个月后会下跌”,我们总是会相信这样的话。但是如果能相信自己,学会自己做判断,几年下来就会形成自己的投资风格。这样一来,投资成功的机会自然就会多起来。

投资的方式和风格有很多。比如,100多岁还活跃在投资领域的罗伊·纽伯格就是一位优秀的短线交易者。

他的投资风格就是以1小时、1天或几天为单位进行短线交易来获利的。不过,短线交易并非我的强项。

在我漫长的投资生涯中,我学会的最佳投资法就是先找到一个被过低评价的投资产品,然后长线持有。虽然这种投资风格会被人称为“逆向投资”,但我一直喜欢当大家都感到绝望的时候,寻找下一个投资机会,也乐意了解大家都厌恶的国家或股票,我的投资风格就是在这一过程中逐渐形成的。

\section{投资的基本原则就是“低买高卖”}

罗杰斯:投资并非什么难事。虽说没有“绝对成功”的方法,但基本原则就是“低买高卖”。

不过,能真正做到这一点的人并不多见。因为大部分人都只在意牛市的行情,而不愿意关注熊市的行情。相信不少人都会认同这一点,尤其是日本人,在看到行情走高,股价上涨后,不少人就会抱着一种自己也不容错过的心理进入市场。

而我则恰恰相反,我时刻关注熊市行情,寻找股价的最低点。当人们狂热的时候,我会静观其变;当人们陷入困境的时候,我反而会紧盯市场,寻找那些无人在意的、被市场低估的股票。

1973年,我与乔治·索罗斯共同创立了一个对冲基金,用于投资其他人不看好的、被市场低估了的股票和商品。这些投资都很成功,10年来回报率高达4200\%。我们投资了东日本大地震后因恐慌而被抛售的日本股票,也为我们带来了丰厚的回报。

正如我以前多次提到的那样,日本农地的价格被严重低估,现在已经基本触底。如果可以,我想购买日本的农地,投资日本的农业。不仅是日本,世界其他地方的农业潜力也很巨大,充满了商机。特别是现在日本的农业从业人员老龄化严重,缺乏足够的竞争,如果能够把充满干劲的年轻人聚到一起的话,那么,等待我们的将会是美好的未来。

当人们对市场怀有戒心且避之唯恐不及的时候正是一次机遇,我们应该果断行动起来。即使不买,你也应该密切关注它,观察行情走势,把行情的走势、周围的人是如何想的统统印在脑子里。因为这些将有助于你赢得接下来的投资。当所有人都害怕的时候,你应该鼓起勇气大胆地买入,像这样的成功案例比比皆是。

\textbf{——除了“低买高卖”以外,您还有其他建议吗?}

罗杰斯:在上一本书中我已经说过自己的看法,那就再来谈谈这个问题吧。

我们要学会通过储蓄来积累资产。并非每个人一开始就拥有资产,尤其是年轻人,一开始不会拥有太多资产。所以,我们要做的就是事业有成之后积累资产。
尽管每个人都想一夜暴富成为亿万富翁,但这绝非易事。所以,我的建议就是学会耐心等待。要成为一个成功的投资者,很重要的一点就是大多数时候能够沉下心来静观其变。机会到来时,要毫不犹豫地出手,然后继续等待。

我的下一条建议是,充分研究自己将要投资的领域,成为这个领域的专家后再开始投资。大多数人会盲目地根据电视、网络上散布的信息,比如看到“那只股票很便宜”“能涨到3万美元”就开始了投资,但赚钱并不是那么简单的事。

投资领域可以根据自己的喜好而定,时尚、汽车、运动、美食什么都行。如果你对时尚感兴趣,可以通过书籍或网络调查相关信息。然后,就是切莫只有三分钟热度,要长年累月地坚持下去。若能做到这一点,你就会自然拥有投资家的视角与思路。

当然,你可能会忍不住向亲朋好友炫耀你的想法或发现,但刚开始时先不要透露给任何人,而是要默默地调查相关领域里有望成功的商机或企业。当然,也不能三天打鱼两天晒网,而是要持之以恒。这样你就能先于华尔街的分析家发现有望成功的商机或企业。

另外,不要购买别人建议的投资项目,投资应该基于自己的调查。我总是一个人默默地做这件事,今后应该也会这样做下去。因为过去的经验让我懂得对别人言听计从是要吃亏的。

\section{投资收益并不是“不劳而获”}

\textbf{——在日本越来越多的人开始关注投资,但也有很多人认为这是不劳而获,对投资敬而远之。}

罗杰斯:如果我们刚刚谈到的那些都是投资所必需的,你还会认为这是不劳而获吗?买入前需要花费大量的时间和精力,同样在卖出时,也需要花费这些时间和精力。当你要投资一个领域时,首先要花上几年的时间去研究它。因此,投资并不是一个轻轻松松就能赚钱的行业。

我曾多次说过,若你一生仅有20次投资的机会,你一定会对投资更加谨慎,不会四处盲目投资。投资之前,我想你一定会充分调查到自己认为万无一失为止。这才是所谓成功的投资,即投资前充分调查,然后谨慎行事,除此以外,再无其他。

还有一条建议非常重要。那就是卖掉股票后要学会及时收手。通常情况下,投资结束后,人们会立刻开始新的投资。特别是在大赚一笔、骄傲自满时,更要学会见好就收。

越是在这种时候就越要谨慎,你可以利用这个时间重新学习,等待新的时机到来再谨慎投资。但很多人因缺少足够的耐心而急于出手,所以投资失败。无论如何都等不及的话,我建议你不如看看电影,去海滩悠闲地喝点儿啤酒。

实际上,“学会等待”也是投资者成功的重要因素之一。不少人因为缺乏足够的耐心而投资失败,这样的案例我知道很多。对投资者而言,很重要的一点就是大多数时候要能够沉下心来静观其变。根据我多年的经验,刚获利后,下一个好的投资机会绝不会立刻出现,所以要静下心来耐心等待。这就是我为什么要说你的人生中仅有20次投资机会。因为这样你就能心平气和地等待下一次投资机会的到来。

不过,以上都是我擅长的长线投资的成功法则。我常常以10年或20年为单位,去寻找可以进行长线投资的项目。但若你是一个优秀的短线投资者,你也可以好好研究一下短线投资的法则。总之,要想在投资上获得成功,最佳的方法还是找到并形成自己的投资风格。

\end{document}
